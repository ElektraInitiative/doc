This tutorial assumes that you are already familiar with \hyperlink{doc_tutorials_namespaces_md}{namespaces}. This tutorial will only explain \hyperlink{doc_help_elektra-cascading_md}{cascading lookup}.

When Elektra looks up a {\itshape cascading key} (i.\+e. key names without a namespace and a leading slash {\ttfamily /}, the namespaces are searched in the following order\+:


\begin{DoxyItemize}
\item \href{https://github.com/ElektraInitiative/libelektra/blob/master/doc/help/elektra-namespaces.md#spec}{\tt spec} (contains metadata, e.\+g. to modify Elektra\textquotesingle{}s lookup behavior)
\item \href{https://github.com/ElektraInitiative/libelektra/blob/master/doc/help/elektra-namespaces.md#proc}{\tt proc} (process-\/related information)
\item \href{https://github.com/ElektraInitiative/libelektra/blob/master/doc/help/elektra-namespaces.md#dir}{\tt dir} (directory-\/related information, e.\+g. {\ttfamily .git} or {\ttfamily .htaccess})
\item \href{https://github.com/ElektraInitiative/libelektra/blob/master/doc/help/elektra-namespaces.md#user}{\tt user} (user configuration)
\item \href{https://github.com/ElektraInitiative/libelektra/blob/master/doc/help/elektra-namespaces.md#system}{\tt system} (system configuration)
\end{DoxyItemize}

If a key, for example, exists in both {\bfseries user} and {\bfseries system} namespace, the key in the {\bfseries user} namespace takes precedence over the one in the {\bfseries system} namespace. If there is no such key in the {\bfseries user} namespace the key in the {\bfseries system} namespace acts as a fallback.

But lets demonstrate this with an example\+:

Configuration in the {\bfseries system} namespace is the same for all users. Therefore this namespace provides a default or fallback configuration.

With the default Elektra installation only an administrator can update configuration settings within the {\bfseries system} namespace.


\begin{DoxyCode}
# Backup-and-Restore:/tests/tutorial

# Backup old override specification
kdb set user:/tests/overrides $(mktemp)
kdb export system:/tests/overrides dump > $(kdb get user:/tests/overrides)

kdb get /tests/tutorial/cascading/#0/current/test
# RET: 11
# STDERR: Did not find key '/tests/tutorial/cascading/#0/current/test'

# Now add the key ...
sudo kdb set system:/tests/tutorial/cascading/#0/current/test "hello world"
#> Create a new key system:/tests/tutorial/cascading/#0/current/test with string "hello world"

# ... and verify that it exists
kdb get /tests/tutorial/cascading/#0/current/test
#> hello world
\end{DoxyCode}


A user may now want to override the configuration in {\bfseries system}, so he/she sets a key in the {\bfseries user} namespace\+:


\begin{DoxyCode}
kdb set user:/tests/tutorial/cascading/#0/current/test "hello galaxy"
#> Create a new key user:/tests/tutorial/cascading/#0/current/test with string "hello galaxy"

# This key masks the key in the system namespace
kdb get /tests/tutorial/cascading/#0/current/test
#> hello galaxy
\end{DoxyCode}


Note that configuration in the {\bfseries user} namespace only affects {\itshape this} user. Other users would still get the key from the {\bfseries system} namespace.

The {\bfseries dir} namespace is associated with a directory. The configuration in the {\bfseries dir} namespace applies to the {\bfseries current working directory} and all its subdirectories. This is useful if you have project specific settings (e.\+g. your git configuration or a .htaccess file).

As {\bfseries dir} precedes the {\bfseries user} namespace, configuration in {\bfseries dir} can overwrite user configuration\+:


\begin{DoxyCode}
# create and change to a new directory ...
mkdir kdbtutorial
cd kdbtutorial

# ... and create a key in this directories dir-namespace
# By default this data will be saved in the directory `.dir`.
kdb set dir:/tests/tutorial/cascading/#0/current/test "hello universe"
#> Create a new key dir:/tests/tutorial/cascading/#0/current/test with string "hello universe"

# This key masks the key in the system namespace
kdb get /tests/tutorial/cascading/#0/current/test
#> hello universe

# But is only present in the associated directory
cd ..
kdb get /tests/tutorial/cascading/#0/current/test
# hello galaxy
\end{DoxyCode}


The {\bfseries proc} namespace is not accessible by the command line tool {\bfseries kdb}, as it is unique for each running process using Elektra. So we have to omit an example for this namespace at this point. \hyperlink{doc_help_elektra-glossary_md}{Elektrified} applications can use this namespace to override configuration from other namespaces internally.

The {\bfseries spec} namespace is used to store metadata about keys and therefore Elektra handles the {\bfseries spec} namespace differently to other namespaces. The following part of the tutorial is dedicated to the impact of the {\bfseries spec} namespace on cascading lookups.

The {\ttfamily spec} namespace is special as it can completely change how the cascading lookup works.

During a cascading lookup for a specific key, the default Elektra behavior can be changed by a corresponding {\ttfamily spec-\/key}, i.\+e. a key in the {\bfseries spec} namespace {\bfseries with the same name}.

For example, the metadata {\ttfamily override/\#0} of the respective {\ttfamily spec-\/key} can be specified to use a different key in favor of the key itself. This way, we can implement a redirect or symlink like behavior and therefore even config from current folder ({\ttfamily dir} namespace) can be overwritten.

The cascading lookup will consider the following {\bfseries metadata keys} of {\ttfamily spec-\/key}\+:


\begin{DoxyEnumerate}
\item {\ttfamily override/\#n} redirect to one of the specified keys
\item {\ttfamily namespaces/\#n} specifies which namespaces will be considered and in which order
\item {\ttfamily fallback/\#n} if no key was found these keys will act as a fallback
\item {\ttfamily default} defines a default value for the key if none of the keys was found
\end{DoxyEnumerate}

{\bfseries Note\+:} {\ttfamily override/\#n}, {\ttfamily namespaces/\#n} and {\ttfamily fallback/\#n} are Elektra {\bfseries array keys}. This means, such keys can exist several times, each with a different number for {\ttfamily n}, e.\+g. {\ttfamily override/\#0}, {\ttfamily override/\#1}... This way, we can define multiple values for a specific key with a defined order.

As you can see, override links are considered before everything else, which makes them really powerful.

Consider the following example\+:

First, we create a target key to demonstrate the override link mechanism\+:


\begin{DoxyCode}
sudo kdb set system:/tests/overrides/test "hello override"
#> Create a new key system:/tests/overrides/test with string "hello override"
\end{DoxyCode}


Override links can be defined by adding them to the {\ttfamily override/\#} metadata array key of the corresponding {\ttfamily spec-\/key}\+:


\begin{DoxyCode}
sudo kdb meta-set spec:/tests/tutorial/cascading/#0/current/test override/#0 /tests/overrides/test
\end{DoxyCode}


Now when doing a cascading lookup, we get the value of our target key instead of the specified one\+:


\begin{DoxyCode}
kdb get /tests/tutorial/cascading/#0/current/test
#> hello override
\end{DoxyCode}


As we used a cascading key for our override link ({\ttfamily /tests/overrides/test}) we can use this to allow users to provide their own {\ttfamily tests/overrides/test} keys. If a user sets the {\ttfamily /tests/overrides/test} key, the {\bfseries user} namespace will be used (for a non-\/root user) and therefore the new target for our {\ttfamily /tests/tutorial/cascading/\#0/current/test} key will be {\ttfamily user\+:/tests/overrides/test} instead of {\ttfamily system\+:/tests/overrides/test}.


\begin{DoxyCode}
kdb set /tests/overrides/test "hello user"
# STDOUT-REGEX: .+(Create a new key.+|Set string to) "hello user"
kdb get /tests/tutorial/cascading/#0/current/test
#> hello user
\end{DoxyCode}


Furthermore, you can specify a custom order for the namespaces, set fallback keys and more. For more information, read the \hyperlink{doc_help_elektra-spec_md}{`elektra-\/spec` help page}.

As last part in this tutorial we remove the modifications to the database we made previously.


\begin{DoxyCode}
kdb rm -r user:/tests/tutorial/
kdb rm -r system:/tests/tutorial
kdb rm -r system:/tests/overrides
kdb import system:/tests/overrides dump < $(kdb get user:/tests/overrides)
rm $(kdb get user:/tests/overrides)
kdb rm user:/tests/overrides

kdb rm -r spec:/tests/tutorial/

rm -r .dir/
rmdir kdbtutorial
\end{DoxyCode}
 