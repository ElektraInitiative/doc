
\begin{DoxyItemize}
\item infos = Information about xmltool plugin is in keys below
\item infos/author = Markus Raab \href{mailto:elektra@libelektra.org}{\tt elektra@libelektra.\+org}
\item infos/licence = B\+SD
\item infos/provides = storage/xml
\item infos/needs =
\item infos/placements = getstorage setstorage
\item infos/status = unittest final memleak unfinished old nodoc discouraged
\item infos/description = Storage using legacy X\+ML format.
\end{DoxyItemize}

This plugin is a storage plugin allowing Elektra to read and write X\+ML formatted files. It uses the legacy Elektra 0.\+7 X\+ML format.

This plugin can be used for migration of Key Databases from 0.\+7 -\/$>$ 0.\+8. It should not be used otherwise.

See \hyperlink{doc_INSTALL_md}{installation}. The package is called {\ttfamily libelektra5-\/xmltool}.


\begin{DoxyItemize}
\item {\ttfamily libxml2-\/dev}
\end{DoxyItemize}


\begin{DoxyItemize}
\item only supports metadata as defined in Elektra 0.\+7
\item null and empty values are not distinguished
\item exported relative to first key found, not to parent key (ks\+Get\+Common\+Parent\+Name)
\item error messages vague (no difference between error opening file and validation errors)
\end{DoxyItemize}

After you have upgraded Elektra, you can import X\+ML files from Elektra 0.\+7\+:


\begin{DoxyCode}
kdb import system:/ xmltool < system.xml
kdb import user:/ xmltool < user.xml
\end{DoxyCode}


Or you can also mount an X\+ML file using {\ttfamily xmltool} (not recommended)\+:


\begin{DoxyCode}
kdb mount /etc/example.xml system:/example xmltool
\end{DoxyCode}
 