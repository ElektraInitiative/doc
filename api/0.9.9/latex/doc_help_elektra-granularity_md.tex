\label{doc_help_elektra-granularity_md_md_doc_help_elektra_granularity}%
\Hypertarget{doc_help_elektra-granularity_md_md_doc_help_elektra_granularity}%
 Keys of a backend can only be retrieved as a full key set. Currently it is not possible to fetch a part of the keys of a backend. So the user needs to cut out the interesting keys with {\ttfamily \mbox{\hyperlink{group__keyset_ga6b00cf82b59af4d883a9bad6cf4a4a4a}{ks\+Cut()}}} afterwards. If the keys should be committed again, the whole key set must be preserved. Otherwise, the clipped keys will be removed permanently.

This restriction simplifies storage plugins while it does not limit the user. Other plugins would not be able to fulfil their purpose without the full {\ttfamily Key\+Set}. For example, a plugin that checks the availability and structure of all keys cannot work with a partial key set.

It is problematic to have too many keys in one backend. The applications would need memory for unnecessary configuration data. Instead we recommend introducing several mount points to split up the keys into different backends. Splitting up key sets makes sense if any application requests only a part of the configuration. No benefits arise if every application requests all keys anyway.

Let us assume that many keys reside in {\ttfamily user\+:/sw} and an application only needs the keys in {\ttfamily user\+:/sw/org/app}. To save memory and get better startup-\/times for the application, a new backend can be mounted at {\ttfamily user\+:/sw/org/app}. On the other hand, every mounted backend causes a small run-\/time overhead in the overall configuration system.

The solution in Elektra is flexible, because the user decides the granularity. It is possible to mount a backend on every single key, so that every key can be requested for itself. If no backends are mounted, all keys reside in the default backend.

To sum up, Elektra’s core searches for the nearest mount point and gets the configuration from there. It is possible that the user gets more configuration than requested. The user can decide by means of mounting how much configuration on specific requests are returned. 