\label{doc_news_2014-10-22_augeas_md_md_doc_news_2014_10_22_augeas}%
\Hypertarget{doc_news_2014-10-22_augeas_md_md_doc_news_2014_10_22_augeas}%

\begin{DoxyItemize}
\item author\+: Markus Raab
\item guid\+: eca69e19-\/5ddb-\/438c-\/ac06-\/57c20b1a9160
\item pub\+Date\+: Mon, 22 Oct 2014 17\+:31\+:42 +0200
\end{DoxyItemize}

A common question is\+: now we have Augeas for editing config files, why do we need Elektra, Config\+::\+Model or something else?

First, it is clear that Augeas is a huge step forward and improved configuration of Linux systems, especially when used with centralized configuration management tools.

Augeas, in short, introduces a special-\/purpose programming language that allows to transform configuration files into configuration trees and back. This transformation is it strength (so it is easy to add support for a particular legacy configuration files), but also it weakness (the mapping is implementation-\/defined by a language that is limited to a bit more than regular expressions). Augeas is not able nor is it intended to provide more abstraction over the configuration files. Instead Augeas mirrors the structure of the configuration as closely as possible.

Elektra’s goal, instead, is not only to provide access to legacy configuration files, but to provide access to the configuration exactly as the programs itself uses it. So with Elektra, the developers of applications are part of Elektra’s ecosystem by providing specifications how their configuration should look like and by writing plugins that define how the configuration is accessed and checked. Ideally, after some time of legacy issues and migration, developers will also not have to care about writing plugins anymore, but just use any available ones (and users of their application can choose any other compatible plugin). What is about to stay is a specification that defines the application\textquotesingle{}s configuration, e.\+g. in INI (could be any syntax)\+:


\begin{DoxyCode}{0}
\DoxyCodeLine{[/yourapp/file\_dialog/show\_hidden\_files]}
\DoxyCodeLine{type=Boolean}
\DoxyCodeLine{default=true}

\end{DoxyCode}


allows other applications to reuse your setting show\+\_\+hidden\+\_\+files by referring to above specification. So Elektra not only abstracts from cross-\/platform-\/related issues with an consistent API, but also allows us to be aware of other applications\textquotesingle{} configurations, leveraging easy application integration.

Config\+::\+Model shares most of Elektra’s goals, especially those regarding validation (you saw the type=Boolean above) and having a unified interface for all programs (this feature is unavoidable with any such approach). The projects mainly differs that Elektra is supposed to be used by the programs themselves (and not only by GUIs and validation tools) and that Elektra uses self-\/describing data\+: the specification itself is also in Elektra’s key database, stored in metadata and e.\+g. below system/elektra/mountpoints. In Elektra validators can be written in any language (because the specification is just data) and can enforce constraints on any access (because plugins define the behaviour of the key database). 