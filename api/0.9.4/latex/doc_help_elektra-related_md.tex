Settings and preferences are ubiquitous in every software. For this reason, a vast amount of software has emerged on this topic. Most of the other projects, however, do not devote their work exclusively to configuration.

We have already mentioned that Elektra provides the features of a configuration library. There are countless libraries out there that parse and some even generate configuration files. But there is a big catch\+: These libraries do not provide an abstract view of configuration. They require programmers to open the files themselves or at least to remember the path to them. Such information is itself configuration and inherently operating system dependent. Therefore, it is not possible to give default values that just work.

Another feature mostly missing in these libraries is cascading. It is not difficult to implement, but disturbing if it is not available at all or does not work correctly.

Augeas is not intended for applications themselves, so it only creates a new view to configuration files which is not necessarily the same as applications see it. Additionally, Augeas has a rather poor level of abstraction and many syntactical details must be reflected in the configuration tree. Nevertheless, Elektra provides an \hyperlink{autotoc_md43_src_plugins_augeas_README_md}{augeas plugin} which allows parts of Elektra’s global configuration tree to be implemented using lenses. Lenses are a promising technology, which allow mapping from and to configuration files to be specified with a single program.

A project that shares some of the goals with Elektra, but uses a different approach is {\itshape Uniconf}. Besides a stand-\/alone library it supports a daemon mode. With the daemon running, it has the disadvantage of protocol overhead and a single point of failure. On the other hand, Uniconf can also notify applications when configuration files change and can work in a distributed mode. The project does not solve the problem of how to configure the configuration system. So we still have to pass a so-\/called {\itshape moniker string} that contains the knowledge where to find the configuration. The plugin system is flexible, but does not depend on the key name. So it is not possible to use different plugins for different applications and still have a global key database.

Debconf is a set of Debian-\/tools that separates user interaction from the configuration process for the system\textquotesingle{}s configuration when packages are installed or upgraded. Debconf itself does not change the configuration files. Instead, the administrator is asked questions depending on template files using different frontends. These answers are then cached. The configuration changes themselves are applied by post-\/install scripts of the particular package. Exactly for this last step Elektra could yield much better results, because it can compare configuration on a per-\/key basis and thus handle conflicts in a much more fine-\/grained way.

The \href{https://freedesktop.org}{\tt freedesktop.\+org} initiative unifies different desktops using the X Window System in various ways. The main focus, however, lies in K\+DE and Gnome integration. One of the most disturbing and still unresolved problems is configuration. Elektra intends to fill this gap in the future.

Already available is the so-\/called X\+DG Base Directory Specification. Perhaps it will define file formats in the future, but currently only the paths are specified. To do so, it introduces some environment variables that help to find the actual configuration. Only if they are unset or empty, a built-\/in fallback should be used. If desired, elektrified applications behave in a X\+DG conforming way. (See the configuration x for \hyperlink{autotoc_md579_src_plugins_resolver_README_md}{the resolver}). 