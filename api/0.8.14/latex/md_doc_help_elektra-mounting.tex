{\bfseries Mounting} allegorises a common technique for file systems. File systems on different partitions or devices can be added to the currently accessible file system. Mounting is typically used to access data from external media. A more advanced use case presents mounting a file system that is optimised for specific purposes, for example, one that can handle many small files well. Mounting also allows us to access data via network storage. As a result, mounting of file systems has proved to be extremely successful.

Mounting in Elektra specifically allows us to map a part of the global key database to be handled by a different storage. A difference to file systems is that key names express what file names express in a file system. And instead of file systems writing to block devices, backends writing to key databases are mounted into the global key database. Mounting allows multiple backends to deal with configuration at the same time. Each of them is responsible for its own subtree of the global key database.

Mounting works for file systems only if the file system below is accessible and a directory exists at the mount point. Elektra does not enforce such restrictions.


\begin{DoxyItemize}
\item More information about \hyperlink{md_doc_help_elektra-backends_doc_help_elektra-backends_md}{elektra-\/backends(7)}
\item More information about elektra-\/plugins(7)
\item The tool for mounting plugins is \hyperlink{md_doc_help_kdb-mount_doc_help_kdb-mount_md}{kdb-\/mount(1)}
\item The plugins are ordered as described in \hyperlink{md_doc_help_elektra-plugins-ordering_doc_help_elektra-plugins-ordering_md}{elektra-\/plugins-\/ordering(7)} 
\end{DoxyItemize}