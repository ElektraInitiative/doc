
\begin{DoxyItemize}
\item infos = Information about the csvstorage plugin is in keys below
\item infos/author = Thomas Waser \href{mailto:thomas.waser@libelektra.org}{\tt thomas.\+waser@libelektra.\+org}
\item infos/licence = B\+SD
\item infos/provides = storage/csv
\item infos/needs =
\item infos/recommends =
\item infos/placements = getstorage setstorage
\item infos/status = unittest nodep libc configurable limited
\item infos/description = parses C\+SV files
\end{DoxyItemize}

This plugin allows you to read and write C\+SV files using Elektra. It aims to be compatible with R\+FC 4180. Rows and columns are written using Elektra\textquotesingle{}s arrays ({\ttfamily \#0}, {\ttfamily \#1},..). By configuring the plugin you can give columns a name.

{\ttfamily delimiter} Tells the plugin what delimiter is used in the file. The default delimiter is {\ttfamily ,} and will be used if {\ttfamily delimiter} is not set.

{\ttfamily header} Tells the plugin to use the first line as a header if it is set to {\ttfamily colname}. The columns will get the corresponding names. Skip the first line if it is set to {\ttfamily skip} or treat the first line as a record if it is set to {\ttfamily record}. If {\ttfamily header} is not set, or set to {\ttfamily record}, the columns get named \#0,\#1,... (array key naming)

{\ttfamily columns} If this key is set the plugin will yield an error for every file that doesn\textquotesingle{}t have exactly the amount of columns as specified in {\ttfamily columns}.

{\ttfamily columns/names} Sets the column names. Only usable in combination with the {\ttfamily columns} key. The number of subkeys must match the number of columns. Conflicts with usage of {\ttfamily header}.

{\ttfamily columns/index} Specifies which column should be used to index records instead of the record number.

{\ttfamily export=,export/$<$column name$>$=} Export column {\ttfamily column name}\+:


\begin{DoxyItemize}
\item The key {\ttfamily export} must be present, additionally to {\ttfamily export/$<$column name$>$}
\item Also multiple column names can be given for different columns to export.
\begin{DoxyItemize}
\item Then {\ttfamily delimiter} will be used as delimiter ({\ttfamily ,} as default).
\item The order depends on the alphabetic order of the column names. Use `awk -\/F\textquotesingle{},\textquotesingle{} \textquotesingle{}B\+E\+G\+IN\{O\+FS=\char`\"{},\char`\"{}\} \{print \$2, \$1, \$3\}\textquotesingle{}` or similar to reorder.
\item Unknown column names are ignored.
\end{DoxyItemize}
\end{DoxyItemize}

First line should determine the headers\+:


\begin{DoxyCode}
kdb mount test.csv /csv csvstorage \(\backslash\)
  "delimiter=;,header=colname,columns=2,columns/names,columns/names/#0=col0Name,columns/names/#1=col1Name"
\end{DoxyCode}


The example below shows how you can use this plugin to read and write C\+SV files.

``{\ttfamily  @section autotoc\+\_\+md128 Mount plugin to}user/tests/csv` \hypertarget{autotoc_md124_autotoc_md129}{}\section{We use the column names from the first line of the}\label{autotoc_md124_autotoc_md129}
\hypertarget{autotoc_md124_autotoc_md130}{}\section{config file as key names}\label{autotoc_md124_autotoc_md130}
sudo kdb mount config.\+csv user/tests/csv csvstorage \char`\"{}header=colname,columns/names/\#0=col0\+Name,columns/names/\#1=col1\+Name\char`\"{}\hypertarget{autotoc_md124_autotoc_md131}{}\section{Add some data}\label{autotoc_md124_autotoc_md131}
printf \textquotesingle{}band,album~\newline
\textquotesingle{} $>$$>$ {\ttfamily kdb file user/tests/csv} printf \textquotesingle{}Converge,All We Love We Leave Behind~\newline
\textquotesingle{} $>$$>$ {\ttfamily kdb file user/tests/csv} printf \textquotesingle{}mewithout\+You,Pale Horses~\newline
\textquotesingle{} $>$$>$ {\ttfamily kdb file user/tests/csv} printf \textquotesingle{}Kate Tempest,Everybody Down~\newline
\textquotesingle{} $>$$>$ {\ttfamily kdb file user/tests/csv}

kdb ls user/tests/csv \#$>$ user/tests/csv/\#0 \#$>$ user/tests/csv/\#0/album \#$>$ user/tests/csv/\#0/band \#$>$ user/tests/csv/\#1 \#$>$ user/tests/csv/\#1/album \#$>$ user/tests/csv/\#1/band \#$>$ user/tests/csv/\#2 \#$>$ user/tests/csv/\#2/album \#$>$ user/tests/csv/\#2/band \#$>$ user/tests/csv/\#3 \#$>$ user/tests/csv/\#3/album \#$>$ user/tests/csv/\#3/band\hypertarget{autotoc_md124_autotoc_md132}{}\section{The first array element contains the column names}\label{autotoc_md124_autotoc_md132}
kdb get user/tests/csv/\#0/band \#$>$ band kdb get user/tests/csv/\#0/album \#$>$ album\hypertarget{autotoc_md124_autotoc_md133}{}\section{Retrieve data from the last entry}\label{autotoc_md124_autotoc_md133}
kdb get user/tests/csv/\#3/album \#$>$ Everybody Down kdb get user/tests/csv/\#3/band \#$>$ Kate Tempest\hypertarget{autotoc_md124_autotoc_md134}{}\section{Change an existing item}\label{autotoc_md124_autotoc_md134}
kdb set user/tests/csv/\#1/album \textquotesingle{}You Fail Me\textquotesingle{} \hypertarget{autotoc_md124_autotoc_md135}{}\section{Retrieve the new item}\label{autotoc_md124_autotoc_md135}
kdb get user/tests/csv/\#1/album \#$>$ You Fail Me\hypertarget{autotoc_md124_autotoc_md136}{}\section{The plugin stores the index of the last column}\label{autotoc_md124_autotoc_md136}
\hypertarget{autotoc_md124_autotoc_md137}{}\section{in all of the parent keys.}\label{autotoc_md124_autotoc_md137}
kdb get user/tests/csv/\#0 \#$>$ \#1 kdb get user/tests/csv/\#1 \#$>$ \#1 kdb get user/tests/csv/\#2 \#$>$ \#1 kdb get user/tests/csv/\#3 \#$>$ \#1\hypertarget{autotoc_md124_autotoc_md138}{}\section{The configuration file reflects the changes}\label{autotoc_md124_autotoc_md138}
kdb file user/tests/csv $\vert$ xargs cat \#$>$ album,band \#$>$ You Fail Me,Converge \#$>$ Pale Horses,mewithout\+You \#$>$ Everybody Down,Kate Tempest\hypertarget{autotoc_md124_autotoc_md139}{}\section{Undo changes to the key database}\label{autotoc_md124_autotoc_md139}
kdb rm -\/r user/tests/csv sudo kdb umount user/tests/csv 
\begin{DoxyCode}
# Column as index
\end{DoxyCode}
 kdb mount config.\+csv /tests/csv csvstorage \char`\"{}delimiter=;,header=colname,columns/index=\+I\+M\+D\+B\char`\"{}

printf \textquotesingle{}I\+M\+DB;Title;Year~\newline
\textquotesingle{} $>$$>$ {\ttfamily kdb file /tests/csv} printf \textquotesingle{}tt0108052;Schindler´s List;1993~\newline
\textquotesingle{} $>$$>$ {\ttfamily kdb file /tests/csv} printf \textquotesingle{}tt0110413;Léon\+: The Professional;1994~\newline
\textquotesingle{} $>$$>$ {\ttfamily kdb file /tests/csv}

kdb ls /tests/csv \#$>$ user/tests/csv/tt0108052 \#$>$ user/tests/csv/tt0108052/\+I\+M\+DB \#$>$ user/tests/csv/tt0108052/\+Title \#$>$ user/tests/csv/tt0108052/\+Year \#$>$ user/tests/csv/tt0110413 \#$>$ user/tests/csv/tt0110413/\+I\+M\+DB \#$>$ user/tests/csv/tt0110413/\+Title \#$>$ user/tests/csv/tt0110413/\+Year

kdb get /tests/csv/tt0108052/\+Title \#$>$ Schindler´s List

kdb rm -\/r /tests/csv sudo kdb umount /tests/csv


\begin{DoxyCode}
# Export filter
\end{DoxyCode}
 kdb mount config.\+csv /tests/csv csvstorage \char`\"{}delimiter=;,header=colname,columns/index=\+I\+M\+D\+B\char`\"{}

printf \textquotesingle{}I\+M\+DB;Title;Year~\newline
\textquotesingle{} $>$$>$ {\ttfamily kdb file /tests/csv} printf \textquotesingle{}tt0108052;Schindler´s List;1993~\newline
\textquotesingle{} $>$$>$ {\ttfamily kdb file /tests/csv} printf \textquotesingle{}tt0110413;Léon\+: The Professional;1994~\newline
\textquotesingle{} $>$$>$ {\ttfamily kdb file /tests/csv}

kdb export /tests/csv csvstorage -\/c \char`\"{}delimiter=;,header=colname,columns/index=\+I\+M\+D\+B,export=,export/\+I\+M\+D\+B=,export/\+Title=\char`\"{} \#$>$ I\+M\+DB;Title \#$>$ tt0108052;Schindler´s List \#$>$ tt0110413;Léon\+: The Professional

kdb export /tests/csv csvstorage -\/c \char`\"{}delimiter=;,header=colname,columns/index=\+I\+M\+D\+B,export=,export/\+I\+M\+D\+B=,export/\+Year=\char`\"{} \#$>$ I\+M\+DB;Year \#$>$ tt0108052;1993 \#$>$ tt0110413;1994

kdb rm -\/r /tests/csv sudo kdb umount /tests/csv

```\hypertarget{autotoc_md124_autotoc_md140}{}\subsection{Limitations}\label{autotoc_md124_autotoc_md140}

\begin{DoxyItemize}
\item Does not work on file streams (e.\+g. {\ttfamily kdb import} without file)
\item When using csvstorage for exporting, all parent keys must be present (see \href{https://issues.libelektra.org/2304}{\tt https\+://issues.\+libelektra.\+org/2304}) 
\end{DoxyItemize}