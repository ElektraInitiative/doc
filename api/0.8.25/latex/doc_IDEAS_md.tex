This page is the first point for anyone interested in contributing to Elektra!

\subsection*{How can I get started?}

We prepared \href{https://github.com/ElektraInitiative/libelektra/issues?q=is%3Aissue+is%3Aopen+label%3A%22good+first+issue%22}{\tt good first issues} for you.


\begin{DoxyItemize}
\item We encourage you to improve documentation.
\item In the source code, you should look into \hyperlink{md_src_libs_README_src_libs_README_md}{libs} and \hyperlink{md_src_plugins_README_src_plugins_README_md}{plugins}.
\item You can always peek into the T\+O\+D\+Os, if you don\textquotesingle{}t know what to do.
\item An ideal project is to elektrify some free software.
\end{DoxyItemize}

Elektrify means\+:


\begin{DoxyItemize}
\item Patch the application so that it uses Elektra as its configuration system.
\item Write a specification that describes how the configuration of the application looks like.
\end{DoxyItemize}

If you are interested in directly improving Elektra, other projects might be of more interest to you\+:


\begin{DoxyItemize}
\item Further configuration file formats could be supported.
\item The qt-\/gui could be improved.
\item Your own ideas?
\end{DoxyItemize}

\subsection*{What are the requirements to participate?}

Anyone can join!

First you should familiarize with the Elektra Initiative\+:


\begin{DoxyItemize}
\item If you did not yet look at the \href{https://www.libelektra.org/}{\tt home page} please do so!
\item Look into the contributing guidelines
\item Look into the \href{https://issues.libelektra.org/}{\tt issue tracker} and pick an easy task.
\item Say hello in the issue you are interested in participating, or create a new issue.
\item Create a PR which solves the task as good as you can.
\item We will give you further feedback how you can improve. 
\end{DoxyItemize}