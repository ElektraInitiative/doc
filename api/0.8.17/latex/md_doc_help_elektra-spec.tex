spec is a special namespace that describes via meta data the semantics of individual keys.

Most importantly it\+:

0. describes which keys are of interest to the application
\begin{DoxyEnumerate}
\item describes the meta data to be copied to every key
\item describes how the cascading lookup works
\item describes the mountpoints including the plugins needed for them
\end{DoxyEnumerate}

It is, however, not limited to this but can express any other key database semantics (new plugins might be necessary, though).

\subsection*{Application}

The most simple use is to enlist all keys that will be used by an application and maybe give a description for them (we use ini syntax in this document)\+:

``` \mbox{[}mykey\mbox{]}

\mbox{[}folder/anotherkey\mbox{]} description = set this key if you want another behaviour ```

So Keys in {\ttfamily spec} allow us to specify which keys are read by the application. The description key will be copied (referred to) to {\ttfamily folder/anotherkey} of any namespace, so that it can easily be accessed.

\subsection*{Cascading Lookup}

Other features are directly implemented in {\ttfamily ks\+Lookup}. When cascading keys (those starting with {\ttfamily /}) are used following features are now available (in the meta data of respective {\ttfamily spec}-\/keys)\+:


\begin{DoxyItemize}
\item {\ttfamily override/\#}\+: use these keys {\itshape in favour} of the key itself (note that {\ttfamily \#} is the syntax for arrays, e.\+g. {\ttfamily \#0} for the first element, {\ttfamily \#\+\_\+10} for the 11th and so on)
\item {\ttfamily namespace/\#}\+: instead of using all namespaces in the predefined order, one can specify which namespaces should be searched in which order
\item {\ttfamily fallback/\#}\+: when no key was found in any of the (specified) namespaces the {\ttfamily fallback}-\/keys will be searched
\item {\ttfamily default}\+: this value will be used if nothing else was found
\end{DoxyItemize}

E.\+g.

``` \mbox{[}promise\mbox{]} default=20 fallback/\#0=/somewhere/else namespace/\#0=user ```


\begin{DoxyEnumerate}
\item When this file is mounted to {\ttfamily spec/sw/app/\#0} we specify, that for the key {\ttfamily /sw/app/\#0/promise} only the namespace {\ttfamily user} should be used.
\item If this key was not found, but {\ttfamily /somewhere/else} is present, we will use this key instead. The {\ttfamily fallback} technique is very powerful\+: it allows us to have (recursive) links between applications. In the example above, the application is tricked in receiving e.\+g. the key {\ttfamily user/somewhere/else} when {\ttfamily promise} was not available.
\item The value {\ttfamily 20} will be used as default, even if no configuration file is found.
\end{DoxyEnumerate}

Note that the fallback, override and cascading works on {\itshape key level}, and not like most other systems have implemented, on configuration {\itshape file level}.

\subsection*{Validation}

You can tag any key using the {\ttfamily check} meta data so that it will be validated.

For example\+:

``` \mbox{[}folder/anotherkey\mbox{]} check/validation = abc.$\ast$ check/validation/message = def does not start with abc ```

\subsection*{Mounting}

In the spec namespace you can also specify mountpoints. First you need the meta key {\ttfamily mountpoint} and a configuration file name. Otherwise, it basically works in the same way as the contracts in plugins using {\ttfamily infos} and {\ttfamily config}\+:

``` \mbox{[}\mbox{]} mountpoint=file.\+abc config/plugin/code/escape = 40 config/plugin/lua\#abc/script = abc\+\_\+storage.\+lua infos/author = Markus Raab infos/needs = resolver\+\_\+abc rename code lua\+::abc infos/recommends = hexcode ```

\subsection*{S\+E\+E A\+L\+S\+O}


\begin{DoxyItemize}
\item \hyperlink{doc_tutorials_application-integration_md}{see application integration tutorial (towards end)}
\item \hyperlink{doc_tutorials_namespaces_md}{see namespaces tutorial}
\item \hyperlink{md_doc_help_elektra-namespaces_doc_help_elektra-namespaces_md}{elektra-\/namespaces(7)}
\item \hyperlink{md_doc_help_elektra-cascading_doc_help_elektra-cascading_md}{elektra-\/cascading(7)}
\item \hyperlink{md_doc_help_elektra-plugins-ordering_doc_help_elektra-plugins-ordering_md}{elektra-\/plugins-\/ordering(7)} 
\end{DoxyItemize}