
\begin{DoxyItemize}
\item infos = Information about F\+S\+T\+AB plugin is in keys below
\item infos/author = Markus Raab \href{mailto:elektra@libelektra.org}{\tt elektra@libelektra.\+org}
\item infos/licence = B\+SD
\item infos/provides = storage/fstab
\item infos/needs =
\item infos/recommends = struct type path
\item infos/placements = getstorage setstorage
\item infos/status = unittest nodep difficult limited unfinished old
\item infos/description = Parses files in a syntax like /etc/fstab file
\end{DoxyItemize}

This plugin is an implementation of a parser and generator of the /etc/fstab file.

\subsection*{Old fstab Entries}

(Deprecated, remove this section after it is reimplemented in the new way)

For each device in fstab elektra will store the following keys\+:


\begin{DoxyCode}
pseudoname/device
pseudoname/mpoint
pseudoname/type
pseudoname/options
pseudoname/dumpfreq
pseudoname/passno
\end{DoxyCode}


Each represents a column in fstab.

The pseudoname can be any name for setting keys, the will be generated when getting keys, so don\textquotesingle{}t expect the same name.

the directory {\ttfamily /} will be called {\ttfamily rootfs}

all swap devices will be called {\ttfamily swap\+XX} with a number from 00 on for XX

otherwise the mount point without the \textquotesingle{}/\textquotesingle{} character will be used.

At the other point there is the issue with the pseudonames, you can\textquotesingle{}t rely on the pseudoname you have set.

The biggest issue is that you can\textquotesingle{}t change or delete existing entries. All entries you set will be appended to the other filesystems.

So if you get the filesystems and change the type of the file system of the rootfs and set it again the resulting fstab will be like\+:


\begin{DoxyCode}
/dev/sda6       /               ext3>----   >----defaults,errors=remount-ro 0 1
/dev/sda6       /               jfs>----   >----defaults,errors=remount-ro 0 1
\end{DoxyCode}


which will be not like you desired!

setmntent is used, so it is only conforming to B\+SD 4.\+3 and linux and you can\textquotesingle{}t use any comments.

\subsection*{New fstab Entries}

Specification\+:


\begin{DoxyCode}
[/\_]
type = array
explanation = the name of the key is the mount point (so the former
  mpoint is not needed); the value is the number of entries in the
  array
[/\_/#]
explanation = an entry in the array
[/\_/#/device]
[/\_/#/type]
[/\_/#/options]
[/\_/#/dumpfreq]
[/\_/#/passno]
\end{DoxyCode}


Example\+: A fstab that looks like\+:


\begin{DoxyCode}
/dev/sr0        /media/cdrom   udf,iso9660 user,noauto     0       0
\end{DoxyCode}


would have a key name that is an array (so the value is the number of children, in this case 1)\+:


\begin{DoxyCode}
system/filesystems/\(\backslash\)/media\(\backslash\)/cdrom
\end{DoxyCode}


with the array entry\+:


\begin{DoxyCode}
system/filesystems/\(\backslash\)/media\(\backslash\)/cdrom0/#0/
\end{DoxyCode}


So when following line is added


\begin{DoxyCode}
/dev/sr0        /media/cdrom   ramfs user,noauto     0       0
\end{DoxyCode}


Implementation hint\+: use {\ttfamily \hyperlink{group__keyname_gaa942091fc4bd5c2699e49ddc50829524}{key\+Add\+Base\+Name()}} to get escaping of {\ttfamily /}, then add array items below it

If a mount point exists more than once (that could be proc, swap or overlay mount points) the array below gets incremented (otherwise \#0 is used for every unique entry).

The order of the array must, of course, be preserved. Other lines may be reordered for now, a proper \char`\"{}order\char`\"{} could be done later.

Spaces in the names are replaced by \textbackslash{}040 in the fstab.

\subsection*{Example}

Mount the plugin\+:


\begin{DoxyCode}
kdb mount /etc/fstab system/filesystems fstab struct type path
\end{DoxyCode}
 