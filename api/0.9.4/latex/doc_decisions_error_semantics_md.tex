While we have a classification of errors and warnings, it remains unclear when plugins actually should emit errors and warnings.


\begin{DoxyItemize}
\item Should not be contradicting to specified behavior in \hyperlink{doc_tutorials_storage-plugins_md}{storage plugin tutorial}.
\end{DoxyItemize}


\begin{DoxyItemize}
\item Users want a uniform behavior within Elektra, so plugins must behave uniformly.
\end{DoxyItemize}


\begin{DoxyItemize}
\item freedom to plugin writers
\item strict rules and conformance tests for plugins
\end{DoxyItemize}

Provide guidelines in the form as tutorials, covering\+:


\begin{DoxyItemize}
\item prefer errors to warnings
\item that any not understood metadata (e.\+g. types), should lead to an error
\item that wrong specifications, like {\ttfamily kdb meta-\/set /tests/ipaddr/ipv4 check/ipaddr ipv8} should be rejected
\item if the value does not confirm {\bfseries exactly} to the specified type, an error should be emitted (e.\+g. only {\ttfamily 0} or {\ttfamily 1} as boolean)
\item anything else that is beyond the capabilities of a plugin (not implemented), should lead to an error
\end{DoxyItemize}

Violations against these guidelines can be reported as bug and then either\+:


\begin{DoxyItemize}
\item the bug gets fixed
\item the plugin get a worse {\ttfamily infos/status} but still get shipped with 1.\+0
\item the plugin gets removed
\end{DoxyItemize}

It is easier for developers if there are clear expectations on how a plugin should behave. And it is much easier for overall Elektra if there is more consistency.


\begin{DoxyItemize}
\item more checks\&errors in storage plugins are needed
\end{DoxyItemize}


\begin{DoxyItemize}
\item \hyperlink{doc_decisions_spec_metadata_md}{Metadata in Spec Namespace}
\item \hyperlink{doc_decisions_capabilities_md}{Capabilities}
\item \hyperlink{doc_decisions_boolean_md}{Boolean}
\end{DoxyItemize}


\begin{DoxyItemize}
\item \href{https://issues.libelektra.org/1511}{\tt Issue \#1511} 
\end{DoxyItemize}