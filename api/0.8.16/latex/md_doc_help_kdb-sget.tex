{\ttfamily kdb sget $<$path$>$ $<$default-\/value$>$}

Where {\ttfamily path} is the full path to the key and {\ttfamily default-\/value} is the value that should be printed if no value can be retrieved.

\subsection*{D\+E\+S\+C\+R\+I\+P\+T\+I\+O\+N}

This command is used to retrieve the value of a key from within a script. When using the kdb tool in a script, the user should use the {\ttfamily sget} command in place of the kdb-\/get(1) command. The kdb-\/get(1) command should not be used in scripts because it may return an error instead of printing a value in certain circumstances. The {\ttfamily sget} command guarantees that a value will be printed (unless the user passes faulty arugments). This command will either print the value of the key it retrives or a default value that the user specifies.

\subsection*{O\+P\+T\+I\+O\+N\+S}


\begin{DoxyItemize}
\item {\ttfamily -\/\+H}, {\ttfamily -\/-\/help}\+: Show the man page.
\item {\ttfamily -\/\+V}, {\ttfamily -\/-\/version}\+: Print version info.
\item {\ttfamily -\/p}, {\ttfamily -\/-\/profile}=$<$profile$>$\+: Use a different kdb profile.
\end{DoxyItemize}

\subsection*{E\+X\+A\+M\+P\+L\+E\+S}

To get the value of a key from a script or return the value {\ttfamily 0}\+: {\ttfamily kdb sget user/example/key 0}

To get the value of a key using a cascading lookup or return the value {\ttfamily notfound}\+: {\ttfamily kdb sget /example/key \char`\"{}notfound\char`\"{}}

\subsection*{S\+E\+E A\+L\+S\+O}


\begin{DoxyItemize}
\item \hyperlink{md_doc_help_kdb-get_doc_help_kdb-get_md}{kdb-\/get(1)} 
\end{DoxyItemize}