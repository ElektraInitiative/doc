{\ttfamily kdb export $<$source$>$ \mbox{[}$<$format$>$\mbox{]}}~\newline


This command allows a user to export keys from the key database.~\newline
 Keys are exported to {\ttfamily stdout} in whichever format is specified.~\newline
 This command can also be used to view full key(s) including their values.~\newline


Where {\ttfamily source} is the path of the key(s) you want to export. Additionally, the user can specify a format to use by passing it as the option argument {\ttfamily format}.~\newline
 The {\ttfamily format} attribute relies on Elektra’s plugin system to export the keys in the desired format.\+The user can view all plugins available for use by running the kdb-\/plugin-\/list(1) command. To learn about any plugin, the user can simply use the kdb-\/plugin-\/info(1) command.~\newline
 The {\ttfamily storage} plugin can be configured at compile-\/time or changed by the link {\ttfamily libelektra-\/storage.\+so}.


\begin{DoxyItemize}
\item {\ttfamily -\/H}, {\ttfamily -\/-\/help}\+: Show the man page.
\item {\ttfamily -\/V}, {\ttfamily -\/-\/version}\+: Print version info.
\item {\ttfamily -\/p}, {\ttfamily -\/-\/profile $<$profile$>$}\+: Use a different kdb profile.
\item {\ttfamily -\/C}, {\ttfamily -\/-\/color $<$when$>$}\+: Print never/auto(default)/always colored output.
\item {\ttfamily -\/E}, {\ttfamily -\/-\/without-\/elektra}\+: Omit the {\ttfamily system/elektra} directory.
\item {\ttfamily -\/c}, {\ttfamily -\/-\/plugins-\/config $<$plugins-\/config$>$}\+: Add a configuration to the format plugin.
\item {\ttfamily -\/v}, {\ttfamily -\/-\/verbose}\+: Explain what is happening. Prints additional information in case of errors/warnings.
\item {\ttfamily -\/d}, {\ttfamily -\/-\/debug}\+: Give debug information. Prints additional debug information in case of errors/warnings.
\end{DoxyItemize}


\begin{DoxyItemize}
\item {\ttfamily /sw/elektra/kdb/\#0/current/format} Change default format (if none is given at commandline) and built-\/in default {\ttfamily storage} is not your preferred format.
\end{DoxyItemize}

To view your full key database in Elektra’s {\ttfamily storage} format\+:~\newline
 {\ttfamily kdb export /}~\newline


To backup your full key database in Elektra’s {\ttfamily storage} format to a file called {\ttfamily full-\/backup.\+ecf}\+:~\newline
 {\ttfamily kdb export / $>$ full-\/backup.\+ecf}~\newline


To backup a keyset stored in {\ttfamily user/keyset} in the {\ttfamily ini} format to a file called {\ttfamily keyset.\+ini}\+:~\newline
 {\ttfamily kdb export user/keyset ini $>$ keyset.\+ini}~\newline


Change default format to {\ttfamily simpleini}\+:~\newline
 {\ttfamily kdb set /sw/elektra/kdb/\#0/current/format simpleini}

Create two key values and export them as {\ttfamily xml}\+:


\begin{DoxyCode}
kdb set user/tests/kdb-export/one one
kdb set user/tests/kdb-export/two two

kdb export user/tests/kdb-export/ xml
#> <?xml version="1.0" encoding="UTF-8" standalone="no" ?>
#> <kdb-export>
#>
#>   <one>one</one>
#>
#>   <two>two</two>
#>
#> </kdb-export>


kdb rm -r user/tests
# cleanup
\end{DoxyCode}


Create two key values and export them with the {\ttfamily xerces} plugin\+:


\begin{DoxyCode}
kdb set user/tests/kdb-export/one one
kdb set user/tests/kdb-export/two two

kdb export user/tests/kdb-export/ xerces
#> <?xml version="1.0" encoding="UTF-8" standalone="no" ?>
#> <kdb-export>
#>
#>   <one>one</one>
#>
#>   <two>two</two>
#>
#> </kdb-export>

kdb rm -r user/tests
# cleanup
\end{DoxyCode}



\begin{DoxyItemize}
\item Only storage plugins can be used for formatting. 
\end{DoxyItemize}