
\begin{DoxyItemize}
\item infos = Information about the yamlcpp plugin is in keys below
\item infos/author = René Schwaiger \href{mailto:sanssecours@me.com}{\tt sanssecours@me.\+com}
\item infos/licence = B\+SD
\item infos/needs = base64 directoryvalue
\item infos/provides = storage/yaml
\item infos/recommends =
\item infos/placements = getstorage setstorage
\item infos/status = maintained unittest preview unfinished concept discouraged
\item infos/metadata =
\item infos/description = This storage plugin reads and writes data in the Y\+A\+ML format
\end{DoxyItemize}\hypertarget{md_src_plugins_yamlcpp_README_src_plugins_yamlcpp_README_md}{}\section{Y\+A\+M\+L C\+PP}\label{md_src_plugins_yamlcpp_README_src_plugins_yamlcpp_README_md}
\subsection*{Introduction}

The Y\+A\+ML C\+PP plugin reads and writes configuration data via the \href{https://github.com/jbeder/yaml-cpp}{\tt yaml-\/cpp} library.

\subsection*{Usage}

You can mount this plugin via {\ttfamily kdb mount}\+:


\begin{DoxyCode}
sudo kdb mount config.yaml /examples/yamlcpp yamlcpp
\end{DoxyCode}


. To unmount the plugin use {\ttfamily kdb umount}\+:


\begin{DoxyCode}
sudo kdb umount /examples/yamlcpp
\end{DoxyCode}


. The following examples show how you can store and retrieve data via {\ttfamily yamlcpp}.

``{\ttfamily  $<$h1$>$Mount yamlcpp plugin to cascading namespace}/examples/yamlcpp` sudo kdb mount config.\+yaml /examples/yamlcpp yamlcpp

\section*{Manually add a mapping to the database}

echo \char`\"{}🔑 \+: 🐳\char`\"{} $>$ {\ttfamily kdb file /examples/yamlcpp} \section*{Retrieve the value of the manually added key}

kdb get /examples/yamlcpp/🔑 \#$>$ 🐳

\section*{Manually add syntactically incorrect data}

echo \char`\"{}some key\+: @some  value\char`\"{} $>$$>$ {\ttfamily kdb file /examples/yamlcpp} kdb get \char`\"{}/examples/yamlcpp/some key\char`\"{} \section*{S\+T\+D\+E\+RR\+: .$\ast$yaml-\/cpp\+: error at line 2, column 11\+: unknown token.$\ast$}

\section*{E\+R\+R\+OR\+: 10}

\section*{R\+ET\+: 5}

\section*{Overwrite incorrect data}

echo \char`\"{}🔑\+: value\char`\"{} $>$ {\ttfamily kdb file /examples/yamlcpp}

\section*{Add some values via {\ttfamily kdb set}}

kdb set /examples/yamlcpp/fleetwood mac kdb set /examples/yamlcpp/the chain

\section*{Retrieve the new values}

kdb get /examples/yamlcpp/the \#$>$ chain kdb get /examples/yamlcpp/fleetwood \#$>$ mac

\section*{Undo modifications}

kdb rm -\/r /examples/yamlcpp sudo kdb umount /examples/yamlcpp 
\begin{DoxyCode}
## Arrays

YAML CPP provides basic support for Elektra’s array data type.
\end{DoxyCode}
 \section*{Mount yamlcpp plugin to cascading namespace {\ttfamily /examples/yamlcpp}}

sudo kdb mount config.\+yaml /examples/yamlcpp yamlcpp

\section*{Manually add an array to the database}

echo \textquotesingle{}sunny\+:\textquotesingle{} $>$ {\ttfamily kdb file /examples/yamlcpp} echo \textquotesingle{} -\/ Charlie\textquotesingle{} $>$$>$ {\ttfamily kdb file /examples/yamlcpp} echo \textquotesingle{} -\/ Dee\textquotesingle{} $>$$>$ {\ttfamily kdb file /examples/yamlcpp}

\section*{List the array entries}

kdb ls /examples/yamlcpp \#$>$ user/examples/yamlcpp/sunny \#$>$ user/examples/yamlcpp/sunny/\#0 \#$>$ user/examples/yamlcpp/sunny/\#1

\section*{Read an array entry}

kdb get user/examples/yamlcpp/sunny/\#1 \#$>$ Dee

\section*{You can retrieve the last index of an array by reading the metakey {\ttfamily array}}

kdb getmeta /examples/yamlcpp/sunny array \section*{1}

\section*{Extend the array}

kdb set user/examples/yamlcpp/sunny/\#2 Dennis kdb set user/examples/yamlcpp/sunny/\#3 Frank kdb set user/examples/yamlcpp/sunny/\#4 Mac

\section*{Retrieve the last array entry}

kdb get user/examples/yamlcpp/sunny/\$(kdb getmeta user/examples/yamlcpp/sunny array) \#$>$ Mac

\section*{Undo modifications to the key database}

kdb rm -\/r /examples/yamlcpp sudo kdb umount /examples/yamlcpp 
\begin{DoxyCode}
The plugin also supports nested arrays.
\end{DoxyCode}
 \section*{Mount yamlcpp plugin to cascading namespace {\ttfamily /examples/yamlcpp}}

sudo kdb mount config.\+yaml /examples/yamlcpp yamlcpp

\section*{Add some key value pairs}

kdb set /examples/yamlcpp/key value kdb set /examples/yamlcpp/array/\#0 scalar kdb set /examples/yamlcpp/array/\#1/key value kdb set /examples/yamlcpp/array/\#1/🔑 🙈

kdb ls /examples/yamlcpp \#$>$ user/examples/yamlcpp/array \#$>$ user/examples/yamlcpp/array/\#0 \#$>$ user/examples/yamlcpp/array/\#1 \#$>$ user/examples/yamlcpp/array/\#1/key \#$>$ user/examples/yamlcpp/array/\#1/🔑 \#$>$ user/examples/yamlcpp/key

\section*{Retrieve part of an array value}

kdb get /examples/yamlcpp/array/\#1/key \#$>$ value

\section*{Since an array saves a list of values, an array parent}

\section*{-\/ which represent the array -\/ does not store a value!}

echo \char`\"{}/examples/yamlcpp/array\+: “`kdb get /examples/yamlcpp/array`”\char`\"{} \#$>$ /examples/yamlcpp/array\+: “” kdb get /examples/yamlcpp/array/\#1 echo \char`\"{}/examples/yamlcpp/array/\#1\+: “`kdb get /examples/yamlcpp/array/\#1`”\char`\"{} \#$>$ /examples/yamlcpp/array/\#1\+: “”

\section*{Remove part of an array value}

kdb rm /examples/yamlcpp/array/\#1/key

kdb ls /examples/yamlcpp \#$>$ user/examples/yamlcpp/array \#$>$ user/examples/yamlcpp/array/\#0 \#$>$ user/examples/yamlcpp/array/\#1 \#$>$ user/examples/yamlcpp/array/\#1/🔑 \#$>$ user/examples/yamlcpp/key

\section*{The plugin stores array keys using Y\+A\+ML sequences.}

\section*{Since yaml-\/cpp stores keys in arbitrary order -\/}

\section*{either {\ttfamily key} or {\ttfamily array} could be the “first” key -\/}

\section*{we remove {\ttfamily key} before we retrieve the data. This way}

\section*{we make sure that the output below will always look}

\section*{the same.}

kdb rm /examples/yamlcpp/key kdb file /examples/yamlcpp $\vert$ xargs cat \#$>$ array\+: \#$>$ -\/ scalar \#$>$ -\/ 🔑\+: 🙈

\section*{Undo modifications to the key database}

kdb rm -\/r /examples/yamlcpp sudo kdb umount /examples/yamlcpp 
\begin{DoxyCode}
## Metadata

The plugin supports metadata. The example below shows how a basic `Key` including some metadata, looks
       inside the YAML configuration file:

```yaml
key without metadata: value
key with metadata:
  !elektra/meta
    - value2
    - metakey: metavalue
      empty metakey:
      another metakey: another metavalue
\end{DoxyCode}


. As we can see above the value containing metadata is marked by the tag handle {\ttfamily !elektra/meta}. The data type contains a list with two elements. The first element of this list specifies the value of the key, while the second element contains a map saving the metadata for the key. The data above represents the following key set in Elektra if we mount the file directly to the namespace {\ttfamily user}\+:

\tabulinesep=1mm
\begin{longtabu} spread 0pt [c]{*{4}{|X[-1]}|}
\hline
\rowcolor{\tableheadbgcolor}\PBS\centering \textbf{ Name }&\PBS\centering \textbf{ Value }&\PBS\centering \textbf{ Metaname }&\PBS\centering \textbf{ Metavalue  }\\\cline{1-4}
\endfirsthead
\hline
\endfoot
\hline
\rowcolor{\tableheadbgcolor}\PBS\centering \textbf{ Name }&\PBS\centering \textbf{ Value }&\PBS\centering \textbf{ Metaname }&\PBS\centering \textbf{ Metavalue  }\\\cline{1-4}
\endhead
\PBS\centering user/key without metadata &\PBS\centering value1 &\PBS\centering — &\PBS\centering — \\\cline{1-4}
\PBS\centering user/key with metadata &\PBS\centering value2 &\PBS\centering metakey &\PBS\centering metavalue \\\cline{1-4}
\PBS\centering &\PBS\centering &\PBS\centering empty metakey &\PBS\centering — \\\cline{1-4}
\PBS\centering &\PBS\centering &\PBS\centering another metakey &\PBS\centering another metavalue \\\cline{1-4}
\end{longtabu}
. The example below shows how we can read and write metadata using the {\ttfamily yamlcpp} plugin via {\ttfamily kdb}.

``{\ttfamily  $<$h1$>$Mount yamlcpp plugin to cascading namespace}/examples/yamlcpp` sudo kdb mount config.\+yaml user/examples/yamlcpp yamlcpp

\section*{Manually add a key including metadata to the database}

echo \char`\"{}🔑\+: !elektra/meta \mbox{[}🦄, \{comment\+: Unicorn\}\mbox{]}\char`\"{} $>$ {\ttfamily kdb file user/examples/yamlcpp} kdb lsmeta user/examples/yamlcpp/🔑 \#$>$ comment kdb getmeta user/examples/yamlcpp/🔑 comment \#$>$ Unicorn

\section*{Add a new key and add some metadata to the new key}

kdb set user/examples/yamlcpp/brand new kdb setmeta user/examples/yamlcpp/brand comment \char`\"{}\+The Devil And God Are Raging Inside Me\char`\"{} kdb setmeta user/examples/yamlcpp/brand rationale \char`\"{}\+Because I Love It\char`\"{}

\section*{Retrieve metadata}

kdb lsmeta user/examples/yamlcpp/brand \#$>$ comment \#$>$ rationale kdb getmeta user/examples/yamlcpp/brand rationale \#$>$ Because I Love It

\section*{Undo modifications to the key database}

kdb rm -\/r user/examples/yamlcpp sudo kdb umount user/examples/yamlcpp 
\begin{DoxyCode}
We can also invoke additional plugins that use metadata like `type`.
\end{DoxyCode}
 sudo kdb mount config.\+yaml /examples/yamlcpp yamlcpp type kdb set /examples/yamlcpp/typetest/number 21 kdb setmeta /examples/yamlcpp/typetest/number check/type short

kdb set /examples/yamlcpp/typetest/number \char`\"{}\+One\char`\"{} \section*{R\+ET\+: 5}

\section*{S\+T\+D\+E\+RR\+: .$\ast$\+The type short failed to match for .$\ast$/number with string\+: One.$\ast$}

\section*{E\+R\+R\+OR\+: 52}

kdb get /examples/yamlcpp/typetest/number \#$>$ 21

\section*{Undo modifications to the key database}

kdb rm -\/r /examples/yamlcpp sudo kdb umount /examples/yamlcpp 
\begin{DoxyCode}
## Binary Data

YAML CPP also supports [base64](https://tools.ietf.org/html/rfc4648) encoded data via the [Base64](@ref
       src\_plugins\_base64\_README\_md) plugin.
\end{DoxyCode}
 \section*{Mount Y\+A\+ML C\+PP plugin at cascading namespace {\ttfamily /examples/binary}}

sudo kdb mount test.\+yaml /examples/binary yamlcpp \section*{Manually add binary data}

echo \textquotesingle{}bin\+: !!binary a\+Gk=\textquotesingle{} $>$ {\ttfamily kdb file /examples/binary}

\section*{Base 64 decodes the data {\ttfamily a\+Gk=} to {\ttfamily hi} and stores the value in binary form.}

\section*{The command {\ttfamily kdb get} prints the data as hexadecimal byte values.}

kdb get /examples/binary/bin \#$>$ 

\section*{We can use {\ttfamily ruby} to convert the hexadecimal value returned by {\ttfamily kdb get}}

\section*{to its A\+S\+C\+II representation. If you use {\ttfamily bash} or {\ttfamily fish} as shell then}

\section*{printf {\ttfamily kdb get /examples/binary/bin} \# Bash}

\section*{or}

\section*{printf (kdb get /examples/binary/bin) \# fish}

\section*{should work too.}

ruby -\/e \char`\"{}print A\+R\+G\+V\mbox{[}0\mbox{]}.\+split(\textquotesingle{}\textbackslash{}x\textquotesingle{})\mbox{[}1..-\/1\mbox{]}.\+map \{ $\vert$byte$\vert$ byte.\+to\+\_\+i(16).\+chr \}.\+join\char`\"{} {\ttfamily kdb get /examples/binary/bin} \#$>$ hi

\section*{Add a string value to the database}

kdb set /examples/binary/text mate \section*{Base 64 does not modify textual values}

kdb get /examples/binary/text \#$>$ mate

\section*{The Base 64 plugin re-\/encodes binary data before Y\+A\+ML C\+PP stores the key set. Hence the}

\section*{configuration file contains the value {\ttfamily a\+Gk=} even after Y\+A\+ML C\+PP wrote a new configuration.}

grep -\/q \textquotesingle{}bin\+: !.$\ast$ a\+Gk=\textquotesingle{} {\ttfamily kdb file user/examples/binary} \section*{R\+ET\+: 0}

\section*{Undo modifications to the database}

kdb rm -\/r /examples/binary sudo kdb umount /examples/binary 
\begin{DoxyCode}
## Dependencies

This plugin requires [yaml-cpp][]. On a Debian based OS the package for the library is called
       `libyaml-cpp-dev` . On macOS you can install the package `yaml-cpp` via [HomeBrew](https://brew.sh).

## Limitations

### Leaf Values

One of the limitations of this plugin is, that it only supports values inside [leaf
       nodes](https://github.com/ElektraInitiative/libelektra/issues/106). Let us look at an example to show what that means. The YAML
       file below:

```yaml
root:
  subtree:    🍂
  below root: leaf
level 1:
  level 2:
    level 3:  🍁
\end{DoxyCode}


stores all of the values ({\ttfamily 🍂}, {\ttfamily leaf} and {\ttfamily 🍁}) in the leaves of the mapping. The drawing below makes this situation a little bit clearer.



The key set that this plugin creates using the data above looks like this (assuming we mount the plugin to {\ttfamily user/examples/yamlcpp})\+:

\tabulinesep=1mm
\begin{longtabu} spread 0pt [c]{*{2}{|X[-1]}|}
\hline
\rowcolor{\tableheadbgcolor}\textbf{ Name }&\textbf{ Value  }\\\cline{1-2}
\endfirsthead
\hline
\endfoot
\hline
\rowcolor{\tableheadbgcolor}\textbf{ Name }&\textbf{ Value  }\\\cline{1-2}
\endhead
user/examples/yamlcpp/level &\\\cline{1-2}
user/examples/yamlcpp/level 1/level 2 &\\\cline{1-2}
user/examples/yamlcpp/level 1/level 2/level 3 &🍁 \\\cline{1-2}
user/examples/yamlcpp/root &\\\cline{1-2}
user/examples/yamlcpp/root/below root &leaf \\\cline{1-2}
user/examples/yamlcpp/root/subtree &🍂 \\\cline{1-2}
\end{longtabu}
. Now why is this plugin unable to store values outside leaf nodes? For example, why can we not store a value inside {\ttfamily user/examples/yamlcpp/level 1/level 2}? To answer this question we need to look at the Y\+A\+ML representation\+:


\begin{DoxyCode}
level 1:
  level 2:
    level 3:  🍁
\end{DoxyCode}


. In a naive approach we might just try to add a value e.\+g. {\ttfamily 🙈} right next to level 2\+:


\begin{DoxyCode}
level 1:
  level 2: 🙈
    level 3:  🍁
\end{DoxyCode}


. This however would be not correct, since then the Y\+A\+ML node {\ttfamily level 2} would contain both a scalar value ({\ttfamily 🙈}) and a mapping ({\ttfamily \{ level 3\+: 🍁 \}}). We could solve this dilemma using a list\+:


\begin{DoxyCode}
level 1:
  level 2:
    - 🙈
    - level 3:  🍁
\end{DoxyCode}


. However, if we use this approach we are not able to support Elektra’s array type properly.

\paragraph*{Directory Values}

To overcome the limitation described above, the Y\+A\+ML C\+PP plugin requires the \hyperlink{md_src_plugins_directoryvalue_README_src_plugins_directoryvalue_README_md}{Directory Value} plugin. This plugin converts the value of a non-\/leaf node to a leaf node with the name {\ttfamily \+\_\+\+\_\+\+\_\+dirdata}. For example, let us assume we have the following key set\+:


\begin{DoxyCode}
directory      = Directory Data
directory/file = Leaf Data
\end{DoxyCode}


. The Directory Value plugin will convert the key set in the set (write) direction to


\begin{DoxyCode}
directory            =
directory/\_\_\_dirdata = Directory Data
directory/file       = Leaf Data
\end{DoxyCode}


. Consequently the Y\+A\+ML plugin will store the key set as


\begin{DoxyCode}
directory:
  \_\_\_dirdata = Directory Data
  file       = Leaf Data
\end{DoxyCode}


. A user of the Y\+A\+ML plugin will not notice this feature unless he edits the configuration file by hand, as the following example shows\+:

``{\ttfamily  $<$h1$>$Mount Y\+A\+ML C\+PP plugin at}user/examples/yamlcpp` sudo kdb mount test.\+yaml user/examples/yamlcpp yamlcpp

kdb set user/examples/yamlcpp/directory \textquotesingle{}Directory Data\textquotesingle{} kdb setmeta user/examples/yamlcpp/directory comment \textquotesingle{}Directory Metadata\textquotesingle{} kdb set user/examples/yamlcpp/directory/file \textquotesingle{}Leaf Data\textquotesingle{}

kdb ls user/examples/yamlcpp/directory \#$>$ user/examples/yamlcpp/directory \#$>$ user/examples/yamlcpp/directory/file

kdb get user/examples/yamlcpp/directory \#$>$ Directory Data kdb getmeta user/examples/yamlcpp/directory comment \#$>$ Directory Metadata kdb get user/examples/yamlcpp/directory/file \#$>$ Leaf Data

\section*{Undo modifications to the database}

kdb rm -\/r user/examples/yamlcpp sudo kdb umount user/examples/yamlcpp 
\begin{DoxyCode}
.

### Other Limitations

- Saving a **single scalar value** directly below the mountpoint does not work
- Adding and removing keys does remove **comments** inside the configuration file
- The plugin currently lacks proper **type support** for scalars.
- If Elektra uses YAML CPP as **default storage** plugin, multiple tests of the test suite fail. However,
       if you mount YAML CPP at `/`:
\end{DoxyCode}
 kdb mount default.\+yaml / yamlcpp ```

all tests should work correctly. The problem here is that Elektra does not load additional required plugins ({\ttfamily infos/needs}) for a default storage plugin. 