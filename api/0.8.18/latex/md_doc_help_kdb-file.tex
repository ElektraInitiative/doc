{\ttfamily kdb file $<$path$>$}

Where {\ttfamily path} is the path of a key.

\subsection*{D\+E\+S\+C\+R\+I\+P\+T\+I\+O\+N}

This command prints which file a given key is stored in. While many keys are stored in a default key database file, many others are stored in any number of configuration files located all over the system. This tool is made to allow users to find out the file that a key is actually stored in. This command makes use of Elektra's {\ttfamily resolver} plugin which the uer can learn more about by running the command {\ttfamily kdb info resolver}.

\subsection*{O\+P\+T\+I\+O\+N\+S}


\begin{DoxyItemize}
\item {\ttfamily -\/\+H}, {\ttfamily -\/-\/help}\+: Show the man page.
\item {\ttfamily -\/\+V}, {\ttfamily -\/-\/version}\+: Print version info.
\item {\ttfamily -\/p}, {\ttfamily -\/-\/profile}=$<$profile$>$\+: Use a different kdb profile.
\item {\ttfamily -\/n}, {\ttfamily -\/-\/no-\/newline}\+: Suppress the newline at the end of the output.
\item {\ttfamily -\/\+N}, {\ttfamily -\/-\/namespace}=$<$ns$>$\+: Specify the namespace to use when writing cascading keys.
\item {\ttfamily -\/\+C}, {\ttfamily -\/-\/color}=\mbox{[}when\mbox{]}\+: Print never/auto(default)/always colored output.
\end{DoxyItemize}

\subsection*{K\+D\+B}


\begin{DoxyItemize}
\item {\ttfamily /sw/elektra/kdb/\#0/current/namespace}\+: Specifies which default namespace should be used when setting a cascading name. By default it is {\ttfamily user}, except if you are root, then it is {\ttfamily system}.
\end{DoxyItemize}

\subsection*{E\+X\+A\+M\+P\+L\+E\+S}

To find which file a key is stored in\+: {\ttfamily kdb file user/example/key}

\subsection*{S\+E\+E A\+L\+S\+O}


\begin{DoxyItemize}
\item \hyperlink{md_doc_help_elektra-mounting_doc_help_elektra-mounting_md}{elektra-\/mounting(7)}
\item \hyperlink{md_doc_help_elektra-namespaces_doc_help_elektra-namespaces_md}{elektra-\/namespaces(7)} 
\end{DoxyItemize}