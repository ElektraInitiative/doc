\section*{elektra-\/key-\/names(7) -- the names of keys }

Every {\ttfamily Key} object with the same name will receive the very same information from the global key database. The name locates a {\bfseries unique key} in the key database. Key names are always absolute; so no parent or other information is needed. That makes a {\ttfamily Key} self-\/contained and independent both in memory and storage.

Every key name starts with a \hyperlink{md_doc_help_elektra-namespaces_doc_help_elektra-namespaces_md}{namespace}, for example {\ttfamily user} or {\ttfamily system}. These prefixes spawn key hierarchies each.

The shared {\itshape system configuration} is identical for every user. It contains, for example, information about system daemons, network related preferences and default settings for software. These keys are created when software is installed, and removed when software is purged. Only the administrator can change system configuration.

Examples of valid system key names\+: \begin{DoxyVerb}    system
    system/hosts/hostname
    system/sw/apache/httpd/num_processes
    system/sw/apps/abc/current/default-setting
\end{DoxyVerb}


user configuration is empty until the user changes some preferences. User configuration affects only a single user. The user's settings can contain information about the user's environment, preferred applications and anything not useful for the rest of the system.

Examples of valid user key names\+: \begin{DoxyVerb}    user
    user/env/#1/LD_LIBRARY_PATH
    user/sw/apps/abc/current/default-setting
    user/sw/kde/kicker/preferred_applications/#1
\end{DoxyVerb}


The slash ({\ttfamily /}) separates key names and structures them hierarchically. If two keys start with the same key names, but one key name continues after a slash, this key is {\bfseries below} the other and is called a {\itshape subkey}. For example {\ttfamily user/sw/apps/abc/current} is a subkey of the key {\ttfamily user/sw/apps}. The key is not directly below but, for example, {\ttfamily user/sw/apps/abc} is. {\ttfamily \hyperlink{group__keytest_ga6bb0f95ac34ce9c42d61bb35a76139d0}{key\+Rel()}} implements a way to decide the relation between two keys. 