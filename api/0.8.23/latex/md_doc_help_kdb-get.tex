{\ttfamily kdb get $<$key name$>$}

Where {\ttfamily key name} is the name of the key.

\subsection*{D\+E\+S\+C\+R\+I\+P\+T\+I\+ON}

This command is used to retrieve the value of a key.

If you enter a {\ttfamily key name} starting with a leading {\ttfamily /}, then a cascading lookup will be performed in order to attempt to locate the key. In this case, using the {\ttfamily -\/v} option allows the user to see the full key name of the key if it is found.

\subsection*{L\+I\+M\+I\+T\+A\+T\+I\+O\+NS}

Only keys within the mountpoint or below the {\ttfamily $<$key name$>$} will be considered during a cascading lookup. A workaround is to pass the {\ttfamily -\/a} option. Use the command {\ttfamily kdb get -\/v $<$key name$>$} to see if an override or a fallback was considered by the lookup.

\subsection*{R\+E\+T\+U\+RN V\+A\+L\+U\+ES}

This command will return the following values as an exit status\+:


\begin{DoxyItemize}
\item 0\+: No errors.
\item 1-\/10\+: standard exit codes, see \hyperlink{md_doc_help_kdb_doc_help_kdb_md}{kdb(1)}
\item 11\+: No key found.
\end{DoxyItemize}

\subsection*{O\+P\+T\+I\+O\+NS}


\begin{DoxyItemize}
\item {\ttfamily -\/H}, {\ttfamily -\/-\/help}\+: Show the man page.
\item {\ttfamily -\/V}, {\ttfamily -\/-\/version}\+: Print version info.
\item {\ttfamily -\/p}, {\ttfamily -\/-\/profile $<$profile$>$}\+: Use a different kdb profile.
\item {\ttfamily -\/C}, {\ttfamily -\/-\/color $<$when$>$}\+: Print never/auto(default)/always colored output.
\item {\ttfamily -\/a}, {\ttfamily -\/-\/all}\+: Consider all of the keys.
\item {\ttfamily -\/n}, {\ttfamily -\/-\/no-\/newline}\+: Suppress the newline at the end of the output.
\item {\ttfamily -\/v}, {\ttfamily -\/-\/verbose}\+: Explain what is happening. Gives a complete trace of all tried keys. Very useful to debug fallback and overrides.
\end{DoxyItemize}

\subsection*{E\+X\+A\+M\+P\+L\+ES}

To get the value of a key\+: {\ttfamily kdb get user/example/key}

To get the value of a key using a cascading lookup\+: {\ttfamily kdb get /example/key}

To get the value of a key without adding a newline to the end of it\+: {\ttfamily kdb get -\/n /example/key}

To explain why a specific key was used (for cascading keys)\+: {\ttfamily kdb get -\/v /example/key}

To use bookmarks\+: {\ttfamily kdb get +kdb/format}

This command will actually get {\ttfamily user/sw/elektra/kdb/\#0/current/format} if the bookmarks commands from \hyperlink{md_doc_help_kdb-set_doc_help_kdb-set_md}{kdb-\/set(1)} man pages are executed before.

\subsection*{S\+EE A\+L\+SO}


\begin{DoxyItemize}
\item \hyperlink{md_doc_help_kdb_doc_help_kdb_md}{kdb(1)} for how to configure the kdb utility and use the bookmarks.
\item For more about cascading keys see \hyperlink{md_doc_help_elektra-cascading_doc_help_elektra-cascading_md}{elektra-\/cascading(7)}
\item To get keys in shell scripts, you also can use \hyperlink{md_doc_help_kdb-sget_doc_help_kdb-sget_md}{kdb-\/sget(1)}
\item \hyperlink{md_doc_help_elektra-key-names_doc_help_elektra-key-names_md}{elektra-\/key-\/names(7)} for an explanation of key names. 
\end{DoxyItemize}