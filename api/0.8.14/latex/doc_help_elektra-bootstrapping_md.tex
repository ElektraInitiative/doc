\section*{elektra-\/bootstrapping(7) -- default backend }

One important aspect of a configuration library is the out-\/of-\/the-\/box experience. How does the system work before anything is configured? The optimal situation is that everything works fully, and applications, that just want to load and store configuration, do not see any difference in a well-\/configured, fine-\/tuned system.

To support that experience, a so-\/called {\bfseries default backend} is responsible in the case that nothing was configured so far. It must have a storage that is able to store full Elektra semantics including every type and arbitrary metadata. To avoid reimplementation of storage plugins, for default storage plugins a resolver plugin additionally takes care of the inevitable portability issues.

The default backend is guaranteed to stay mounted at {\ttfamily system/elektra} where the configuration for Elektra itself is stored. After mounting all backends, Elektra checks if {\ttfamily system/elektra} still resides at the default backend. If not, it will be mounted there.

To summarise, this approach delivers a good out-\/of-\/the-\/box experience capable of storing configuration. For special use cases, applications and administrators can mount specific backends anywhere except at, and below, {\ttfamily system/elektra}. On {\ttfamily \hyperlink{group__kdb_ga6808defe5870f328dd17910aacbdc6ca}{kdb\+Open()}}, the system bootstraps itself starting with the default backend.

The default backend consists of a default storage plugin and default resolver plugin. The default resolver has no specific requirements, but the default storage plugin must be able to handle full Elektra semantics. The backend is not mounted anywhere in particular, so any keys can be stored in it. The implementation of the core guarantees that user and system keys always stay separated. 