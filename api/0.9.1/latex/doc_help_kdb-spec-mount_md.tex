{\ttfamily kdb spec-\/mount \mbox{[}/$<$mountpoint$>$\mbox{]} \mbox{[}$<$plugin$>$ \mbox{[}$<$config$>$\mbox{]} \mbox{[}..\mbox{]}\mbox{]}}


\begin{DoxyItemize}
\item {\ttfamily mountpoint} is where in the key database the new backend should be mounted to. It must be a cascading mount point, i.\+e., {\ttfamily mountpoint} must start with {\ttfamily /}.
\item {\ttfamily plugin} are extra Elektra plugins to be used (next to the one specified in {\ttfamily spec/$<$mountpoint$>$}).
\item Plugins may be followed by a {\ttfamily ,} separated list of {\ttfamily keys=values} pairs which will be used as plugin configuration.
\end{DoxyItemize}

{\ttfamily kdb smount} is an alias and can be used in the same way as {\ttfamily kdb spec-\/mount}.

This command allows a user to mount a new {\itshape backend} described by a previously mounted specification. To mount a specification file to {\ttfamily spec}-\/\mbox{\hyperlink{doc_help_elektra-namespaces_md}{namespace}} first use \mbox{\hyperlink{doc_help_kdb-mount_md}{kdb-\/mount(7)}}\+:


\begin{DoxyCode}{0}
\DoxyCodeLine{sudo kdb mount /path/to/some-spec-file.ini spec/example/mountpoint ni}
\end{DoxyCode}


The idea of mounting is explained \mbox{\hyperlink{doc_help_elektra-mounting_md}{in elektra-\/mounting(7)}}. The {\ttfamily spec} \mbox{\hyperlink{doc_help_elektra-namespaces_md}{namespace}} contains metaconfiguration that describes the configuration in all other namespaces. The metadata used for the specification is described in \href{/home/mpranj/workspace/libelektra/doc/METADATA.ini}{\texttt{ M\+E\+T\+A\+D\+A\+T\+A.\+ini}}.

During {\ttfamily spec-\/mount} the {\ttfamily spec} keys are searched for relevant metadata\+:


\begin{DoxyItemize}
\item For every metadata {\ttfamily mountpoint} an additional cascading mount point will be mounted. The metadata {\ttfamily mountpoint} is usually at the parent key (the top-\/level key of the specification).
\item The {\ttfamily infos/$\ast$} and {\ttfamily config/needs} from \href{/home/mpranj/workspace/libelektra/doc/CONTRACT.ini}{\texttt{ C\+O\+N\+T\+R\+A\+C\+T.\+ini}}, that are tagged by {\ttfamily usedby = spec}, will work as described there.
\item For other metadata suitable plugins are searched and mounted additionally.
\end{DoxyItemize}

For example\+:


\begin{DoxyCode}{0}
\DoxyCodeLine{kdb meta-get spec/example/mountpoint mountpoint  \# verify that we have a mountpoint here}
\DoxyCodeLine{sudo kdb spec-mount /example/mountpoint  \# mounts /example/mountpoint according to}
\DoxyCodeLine{                                         \# the specification as found at}
\DoxyCodeLine{                                         \# spec/example/mountpoint}
\end{DoxyCode}


This command writes into the {\ttfamily /etc} directory and as such it requires root permissions. Use {\ttfamily kdb file system/elektra/mountpoints} to find out where exactly it will write to.

Note that many specifications have globs like {\ttfamily \+\_\+} and {\ttfamily \#}. They will only work if the {\ttfamily spec} plugin is present\+:


\begin{DoxyCode}{0}
\DoxyCodeLine{kdb global-mount}
\end{DoxyCode}



\begin{DoxyItemize}
\item {\ttfamily -\/H}, {\ttfamily -\/-\/help}\+: Show the man page.
\item {\ttfamily -\/V}, {\ttfamily -\/-\/version}\+: Print version info.
\item {\ttfamily -\/p}, {\ttfamily -\/-\/profile $<$profile$>$}\+: Use a different kdb profile.
\item {\ttfamily -\/C}, {\ttfamily -\/-\/color $<$when$>$}\+: Print never/auto(default)/always colored output.
\item {\ttfamily -\/v}, {\ttfamily -\/-\/verbose}\+: Explain what is happening.
\item {\ttfamily -\/q}, {\ttfamily -\/-\/quiet}\+: Suppress non-\/error messages.
\item {\ttfamily -\/R}, {\ttfamily -\/-\/resolver $<$resolver$>$}\+: Specify the resolver plugin to use if no resolver is given, the default resolver is used. See also \href{\#KDB}{\texttt{ below in K\+DB}}. Note that the resolver will only added as dependency, but not directly added.
\item {\ttfamily -\/c}, {\ttfamily -\/-\/plugins-\/config $<$plugins-\/config$>$}\+: Add a plugin configuration for all plugins.
\item {\ttfamily -\/W}, {\ttfamily -\/-\/with-\/recommends}\+: Also add recommended plugins and warn if they are not available.
\end{DoxyItemize}


\begin{DoxyItemize}
\item {\ttfamily /sw/elektra/kdb/\#0/current/verbose}\+: Same as {\ttfamily -\/v}\+: Explain what is happening.
\item {\ttfamily /sw/elektra/kdb/\#0/current/quiet}\+: Same as {\ttfamily -\/q}\+: Suppress default messages.
\item {\ttfamily /sw/elektra/kdb/\#0/current/resolver}\+: The resolver that will be added automatically, if {\ttfamily -\/R} is not given.
\item {\ttfamily /sw/elektra/kdb/\#0/current/plugins}\+: It contains a space-\/separated list of plugins and their configs which are added automatically (by default sync). The plugin-\/configuration syntax is as described above in the \href{\#SYNOPSIS}{\texttt{ synopsis}}.
\end{DoxyItemize}

To mount /example as described in {\ttfamily spec/example}\+:~\newline
 {\ttfamily sudo kdb spec-\/mount /example}

Additionally, add {\ttfamily ini} plugin (instead of some default resolver) with a config for I\+NI\+:~\newline
 {\ttfamily sudo kdb spec-\/mount /example ini section=N\+U\+LL}


\begin{DoxyItemize}
\item \mbox{\hyperlink{doc_help_elektra-glossary_md}{elektra-\/glossary(7)}}.
\item \mbox{\hyperlink{doc_help_kdb-umount_md}{kdb-\/umount(7)}}.
\item \mbox{\hyperlink{doc_help_elektra-mounting_md}{elektra-\/mounting(7)}}.
\item \mbox{\hyperlink{doc_tutorials_application-integration_md}{see application integration tutorial}}
\item \mbox{\hyperlink{validation.md_doc_tutorials_validation_md}{see validation tutorial}} 
\end{DoxyItemize}