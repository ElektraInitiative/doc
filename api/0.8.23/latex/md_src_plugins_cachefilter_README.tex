
\begin{DoxyItemize}
\item infos = Information about the cachefilter plugin is in keys below
\item infos/author = Marvin Mall \href{mailto:namoshek@libelektra.org}{\tt namoshek@libelektra.\+org}
\item infos/licence = B\+SD
\item infos/needs =
\item infos/provides = filter
\item infos/recommends =
\item infos/placements = postgetstorage presetstorage
\item infos/status = maintained unittest libc nodep global experimental nodoc
\item infos/metadata =
\item infos/description = ensures result keyset only contains requested keys
\end{DoxyItemize}

A global plugin that steps in during {\ttfamily \hyperlink{group__kdb_ga28e385fd9cb7ccfe0b2f1ed2f62453a1}{kdb\+Get()}} process to filter the results in a way, so that no other keys than the requested one or descendants of it are returned. During {\ttfamily \hyperlink{group__kdb_ga11436b058408f83d303ca5e996832bcf}{kdb\+Set()}} the filtered keys are added back to the output, so that they don\textquotesingle{}t get lost during the storage process. In other words, the plugin caches filtered keys to simplify the use of the A\+PI.

\subsection*{Usage}

There is not much to do to use the plugin. Just mount is as global plugin and you are done\+:


\begin{DoxyCode}
# Backup-and-Restore: system/elektra/globalplugins
kdb global-mount cachefilter
\end{DoxyCode}


To remove the plugin just use {\ttfamily kdb global-\/umount}\+:


\begin{DoxyCode}
kdb global-umount cachefilter
\end{DoxyCode}
 