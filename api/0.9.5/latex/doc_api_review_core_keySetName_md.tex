
\begin{DoxyItemize}
\item start = 2021-\/02-\/14 03\+:30
\item end = 2021-\/02-\/14 03\+:50
\item reviewer = Stefan Hanreich \href{mailto:stefanhani@gmail.com}{\tt stefanhani@gmail.\+com}
\end{DoxyItemize}

{\ttfamily ssize\+\_\+t \hyperlink{group__keyname_ga7699091610e7f3f43d2949514a4b35d9}{key\+Set\+Name(\+Key $\ast$key, const char $\ast$newname)}}

(bullet points are in order of appearance)


\begin{DoxyItemize}
\item \mbox{[}x\mbox{]} First line explains briefly what the function does
\item \mbox{[} \mbox{]} Simple example or snippet how to use the function
\begin{DoxyItemize}
\item \mbox{[} \mbox{]} add
\end{DoxyItemize}
\item \mbox{[} \mbox{]} Longer description of function containing common use cases
\item \mbox{[} \mbox{]} Description of functions reads nicely
\item \mbox{[} \mbox{]} {\ttfamily @pre}
\begin{DoxyItemize}
\item \mbox{[} \mbox{]} \begin{DoxyPrecond}{Precondition}
new\+Name must be a valid name
\end{DoxyPrecond}

\item \mbox{[} \mbox{]} \begin{DoxyPrecond}{Precondition}
must not be a read-\/only key
\end{DoxyPrecond}

\item \mbox{[} \mbox{]} \begin{DoxyPrecond}{Precondition}
must not have been inserted before
\end{DoxyPrecond}

\end{DoxyItemize}
\item \mbox{[} \mbox{]} {\ttfamily @post}
\begin{DoxyItemize}
\item \mbox{[} \mbox{]} \begin{DoxyPostcond}{Postcondition}
Key has (possibly modified) new\+Name as name
\end{DoxyPostcond}

\end{DoxyItemize}
\item \mbox{[} \mbox{]} {\ttfamily @invariant}
\begin{DoxyItemize}
\item \mbox{[} \mbox{]} add
\end{DoxyItemize}
\item \mbox{[}x\mbox{]} {\ttfamily @param} for every parameter
\item \mbox{[} \mbox{]} {\ttfamily @return} / {\ttfamily @retval}
\begin{DoxyItemize}
\item \mbox{[} \mbox{]} add {\ttfamily @retval} -\/1 if key is read-\/only
\end{DoxyItemize}
\item \mbox{[} \mbox{]} {\ttfamily @since}
\begin{DoxyItemize}
\item \mbox{[} \mbox{]} add
\end{DoxyItemize}
\item \mbox{[}x\mbox{]} `{\ttfamily }
\item {\ttfamily \mbox{[} \mbox{]}}\begin{DoxySeeAlso}{See also}
{\ttfamily 
\begin{DoxyItemize}
\item \mbox{[} \mbox{]} addkey\+Set\+Name\+Space()`
\end{DoxyItemize}}
\end{DoxySeeAlso}

\end{DoxyItemize}

{\ttfamily 
\begin{DoxyItemize}
\item \mbox{[}x\mbox{]} Abbreviations used in function names must be defined in the \hyperlink{doc_help_elektra-glossary_md}{Glossary}
\item \mbox{[}x\mbox{]} Function names should neither be too long, nor too short
\item \mbox{[}x\mbox{]} Function name should be clear and unambiguous
\item \mbox{[}x\mbox{]} Abbreviations used in parameter names must be defined in the \hyperlink{doc_help_elektra-glossary_md}{Glossary}
\item \mbox{[}x\mbox{]} Parameter names should neither be too long, nor too short
\item \mbox{[}x\mbox{]} Parameter names should be clear and unambiguous
\end{DoxyItemize}}

{\ttfamily }

{\ttfamily  (only in P\+Rs)}

{\ttfamily 
\begin{DoxyItemize}
\item \hyperlink{doc_dev_symbol-versioning_md}{Symbol versioning} is correct for breaking changes
\item A\+B\+I/\+A\+PI changes are forward-\/compatible (breaking backwards-\/compatibility to add additional symbols is fine)
\end{DoxyItemize}}

{\ttfamily }

{\ttfamily 
\begin{DoxyItemize}
\item \mbox{[}x\mbox{]} Function parameters should use enum types instead of boolean types wherever sensible
\item \mbox{[}x\mbox{]} Wherever possible, function parameters should be {\ttfamily const}
\item \mbox{[} \mbox{]} Wherever possible, return types should be {\ttfamily const}
\begin{DoxyItemize}
\item \mbox{[} \mbox{]} might be possible to make it {\ttfamily const}
\end{DoxyItemize}
\item \mbox{[}x\mbox{]} Functions should have the least amount of parameters feasible
\end{DoxyItemize}}

{\ttfamily }

{\ttfamily 
\begin{DoxyItemize}
\item \mbox{[}x\mbox{]} Functions should do exactly one thing
\item \mbox{[}x\mbox{]} Function name has the appropriate prefix
\item \mbox{[} \mbox{]} Order of signatures in kdb.\+h.\+in is the same as Doxygen
\begin{DoxyItemize}
\item \mbox{[} \mbox{]} swapped with functions for unescaped
\end{DoxyItemize}
\item \mbox{[}x\mbox{]} No functions with similar purpose exist
\end{DoxyItemize}}

{\ttfamily }

{\ttfamily 
\begin{DoxyItemize}
\item \mbox{[}x\mbox{]} Memory Management should be handled by the function wherever possible
\end{DoxyItemize}}

{\ttfamily }

{\ttfamily 
\begin{DoxyItemize}
\item \mbox{[}x\mbox{]} Function is easily extensible, e.\+g., with flags
\item \mbox{[} \mbox{]} Documentation does not impose limits, that would hinder further extensions
\begin{DoxyItemize}
\item \mbox{[} \mbox{]} behavior on invalid names
\end{DoxyItemize}
\end{DoxyItemize}}

{\ttfamily }

{\ttfamily 
\begin{DoxyItemize}
\item \mbox{[} \mbox{]} Function code is fully covered by tests
\begin{DoxyItemize}
\item \mbox{[} \mbox{]} test\+\_\+bit \char`\"{}\+C\+A\+N\+N\+O\+T\char`\"{} fail, so might not be necessary to cover this for now should be made more resistant to future changes
\end{DoxyItemize}
\item \mbox{[} \mbox{]} All possible error states are covered by tests
\begin{DoxyItemize}
\item \mbox{[} \mbox{]} test read-\/only keys
\end{DoxyItemize}
\item All possible enum values are covered by tests
\item \mbox{[} \mbox{]} No inconsistencies between tests and documentation
\begin{DoxyItemize}
\item \mbox{[} \mbox{]} \href{https://github.com/ElektraInitiative/libelektra/blob/master/tests/abi/testabi_key.c#L357}{\tt https\+://github.\+com/\+Elektra\+Initiative/libelektra/blob/master/tests/abi/testabi\+\_\+key.\+c\#\+L357} checks for -\/1 if null pointer is provided documentation says 0 will be returned
\item \mbox{[} \mbox{]} Documentation says name will be {\ttfamily \char`\"{}\char`\"{}} after an invalid name Tests show that name stays unchanged \href{https://github.com/ElektraInitiative/libelektra/blob/master/tests/abi/testabi_key.c#L601}{\tt https\+://github.\+com/\+Elektra\+Initiative/libelektra/blob/master/tests/abi/testabi\+\_\+key.\+c\#\+L601}
\item \mbox{[} \mbox{]} Documentations should include stripping trailing {\ttfamily /}
\end{DoxyItemize}
\end{DoxyItemize}}

{\ttfamily }

{\ttfamily  }

{\ttfamily }