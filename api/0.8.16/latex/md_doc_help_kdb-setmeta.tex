{\ttfamily kdb setmeta $<$key-\/name$>$ $<$meta-\/name$>$ $<$meta-\/value$>$}

Where {\ttfamily key-\/name} is the path to the key that the meta key is associated with, {\ttfamily meta-\/name} is the name of the meta key the user would like to set the value of (or create), and {\ttfamily meta-\/value} is the value the user wishes to set the meta key to.

\subsection*{D\+E\+S\+C\+R\+I\+P\+T\+I\+O\+N}

This command allows the user to set the value of an individual meta key. If a key does not already exist and the user tries setting a meta key associated with it, the key will be created with a null value. There is some special handling for the meta data atime, mtime and ctime. They will be converted to time\+\_\+t.

For cascading keys, the namespace will default to {\ttfamily spec}, because that is the place where you usually want to set meta data.

\subsection*{O\+P\+T\+I\+O\+N\+S}


\begin{DoxyItemize}
\item {\ttfamily -\/\+H}, {\ttfamily -\/-\/help}\+: Show the man page.
\item {\ttfamily -\/\+V}, {\ttfamily -\/-\/version}\+: Print version info.
\item {\ttfamily -\/p}, {\ttfamily -\/-\/profile}=$<$profile$>$\+: Use a different kdb profile.
\item {\ttfamily -\/v}, {\ttfamily -\/-\/verbose}\+: Explain what is happening.
\end{DoxyItemize}

\subsection*{E\+X\+A\+M\+P\+L\+E\+S}

To set a meta key called {\ttfamily description} associated with the key {\ttfamily user/example/key} to the value {\ttfamily Hello World!}\+: {\ttfamily kdb setmeta spec/example/key description \char`\"{}\+Hello World!\char`\"{}}

To create a new key {\ttfamily spec/example/newkey} with a null value (if it did not exist before) and a meta key {\ttfamily namespace/\#0} associated with it to the value {\ttfamily system}\+: {\ttfamily kdb setmeta /example/newkey \char`\"{}namespace/\#0\char`\"{} system}

To create an override link for a {\ttfamily /test} key\+: \begin{DoxyVerb}    kdb set /overrides/test "example override"
    sudo kdb setmeta spec/test override/#0 /overrides/test
\end{DoxyVerb}


\subsection*{S\+E\+E A\+L\+S\+O}


\begin{DoxyItemize}
\item How to get meta data\+: \hyperlink{md_doc_help_kdb-getmeta_doc_help_kdb-getmeta_md}{kdb-\/getmeta(1)}
\item \hyperlink{md_doc_help_elektra-meta-data_doc_help_elektra-meta-data_md}{elektra-\/meta-\/data(7)} 
\end{DoxyItemize}