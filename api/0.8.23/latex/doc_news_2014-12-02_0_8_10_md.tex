
\begin{DoxyItemize}
\item guid\+: 6ce57ecf-\/420a-\/4a31-\/821e-\/1c5fe5532eb4
\item author\+: Markus Raab
\item pub\+Date\+: Tue, 02 Dec 2014 18\+:37\+:51 +0100
\item short\+Desc\+: adds X\+D\+G/\+Open\+I\+CC compatibility \& java binding
\end{DoxyItemize}

Hello,

we are delighted to announce our latest feature release providing major updates in\+:


\begin{DoxyItemize}
\item compatibility with standards,
\item tooling,
\item plugins (hosts, rename),
\item Qt-\/\+Gui and
\item a new Java binding
\end{DoxyItemize}

\subsection*{X\+DG Compatibility}

Elektra now is \href{http://standards.freedesktop.org/basedir-spec/basedir-spec-0.8.html}{\tt fully X\+DG 0.\+8} compliant. Following changes were necessary\+:


\begin{DoxyItemize}
\item newly created configuration files for user/ now have the permissions 0600
\item newly created configuration directories for user/ now have the permissions 0700
\item existing configuration files will retain their permissions.
\item The default path to store user configuration is now $\sim$/.config
\item A new resolver variant x (for user and system) is introduced
\begin{DoxyItemize}
\item implements handling of X\+DG environment variables
\item ignores empty dirs and absolute paths in envvar
\end{DoxyItemize}
\item add new shell based test suite for (xdg)-\/resolver
\end{DoxyItemize}

For example, we could use resolver\+\_\+fm\+\_\+xhp\+\_\+x\+: \begin{DoxyVerb}kdb mount --resolver=resolver_fm_xhp_x file.dump /example dump
kdb file user/example
kdb file system/example
\end{DoxyVerb}


Will show you that for both user+system the resolver respects X\+DG environment variables, e.\+g. above lines will print\+: \begin{DoxyVerb}/home/m/.config/file.dump
/etc/xdg/file.dump
\end{DoxyVerb}


Of course, any attempts to get and set keys below user/example and system/example will also be in these files.

The letters after \+\_\+ describe the variant of the resolver\+:


\begin{DoxyItemize}
\item f .. file based locking
\item m .. mutex based locking (for multiple K\+DB per process)
\item for user configuration (after next \+\_\+)
\begin{DoxyItemize}
\item x .. first check X\+D\+G\+\_\+\+C\+O\+N\+F\+I\+G\+\_\+\+H\+O\+ME environment
\item h .. then check H\+O\+ME environment
\item p .. then fall back to passwd
\end{DoxyItemize}
\item for system configuration (after next \+\_\+)
\begin{DoxyItemize}
\item x .. check all paths in X\+D\+G\+\_\+\+C\+O\+N\+F\+I\+G\+\_\+\+D\+I\+RS and falls back to /etc/xdg
\end{DoxyItemize}
\end{DoxyItemize}

A lot of such resolver variants are added when -\/\+D\+P\+L\+U\+G\+I\+NS=A\+LL is used. Of course you can create new variants with different behaviour by adding them to P\+L\+U\+G\+I\+NS.

To make your application (that uses Elektra) X\+DG aware, you have nothing to do\+: you get it to free. Make sure to always use cascading lookup. Additionally, an X\+DG conforming application should not write system/ keys.

\subsection*{Open\+I\+CC Compatibility}

Based on that, Elektra now also implements the draft for \href{http://www.openicc.info/wiki/index.php?title=OpenICC_Configuration_0.1}{\tt the Open\+I\+CC specification}.

The mount command looks like quite complicated, but it consists of simple parts\+: \begin{DoxyVerb}kdb mount --resolver=resolver_fm_xhp_x \
  color/settings/openicc-devices.json /org/freedesktop/openicc \
  yajl rename cut=org/freedesktop/openicc
\end{DoxyVerb}


We already know the first two lines\+: we use the X\+DG resolver already introduced above. Only the file name and the path where it should be mounted differs.

The plugin yajl is a storage plugin that reads/writes json. The plugin rename was the missing link to support Open\+I\+CC (thanks to Felix Berlakovich for closing this gap). It is needed, because every Open\+I\+CC file starts like this\+: \begin{DoxyVerb}{ "org": { "freedesktop": { "openicc": {
\end{DoxyVerb}


Because the backend is mounted at /org/freedesktop/openicc, it would lead to keys below /org/freedesktop/openicc/org/freedesktop/openicc which we obviously do not want. So we simply get rid of the common prefix by cutting it out using the rename plugin.

Of course this renaming functionality can be used in every situation and is not limited to Open\+I\+CC.

\subsection*{Tools}

A large number of old and new tools were added, mostly for convenience e.\+g.\+: \begin{DoxyVerb}kdb mount-openicc
\end{DoxyVerb}


saves you from writing the long mount command we had in the previous section.

To get a list of all tools that are installed, now the command (which is also an external tool and as such currently not displayed in kdb --help)\+: \begin{DoxyVerb}kdb list-tools
\end{DoxyVerb}


is available. Do not be surprised\+: on typical installations this will be a large list. You can run each of these tools by using kdb $<$command$>$. Most of the tools, however, are part of the test suite, which you can run using\+: \begin{DoxyVerb}kdb run_all
\end{DoxyVerb}


Other tools are \char`\"{}old friends\char`\"{}, e.\+g. convert-\/fstab written in 2006 by Avi Alkalay still works\+: \begin{DoxyVerb}kdb convert-fstab | kdb import system/filesystems xmltool
\end{DoxyVerb}


It will parse your /etc/fstab and generate a X\+ML. This X\+ML then can be imported. Other convert tools directly produce kdb commands, though.

kdb now uses K\+DB itself for many commands\+:


\begin{DoxyItemize}
\item /sw/kdb/current/resolver .. You always want a different default resolver than that was compiled in as default when mounting backends?
\item /sw/kdb/current/format .. If you are annoyed by the default format dump format for import/export.
\item /sw/kdb/current/plugins .. If you always forget to add some plugins when mounting something.
\end{DoxyItemize}

By default the plugin \char`\"{}sync\char`\"{} is added automatically (it makes sure that fsync is executed on config files, the directory is already done by the resolver), you should not remove it from /sw/kdb/current/plugins otherwise the next mount command will not add it. To preserve it use a space separated list, e.\+g.\+: \begin{DoxyVerb}kdb set user/sw/kdb/current/plugins "sync syslog"
\end{DoxyVerb}


Last, but not least, kdb get now supports cascading get\+: \begin{DoxyVerb}kdb get /sw/kdb/current/plugins
\end{DoxyVerb}


This feature allows you to see the configuration exactly as seen by the application.

Other options\+:


\begin{DoxyItemize}
\item -\/123 options for hiding nth column in {\ttfamily kdb mount}
\item hide warnings during script usage of {\ttfamily kdb mount}
\item -\/0 option accepted in some tools (null termination)
\item Mount got a new -\/c option for backend configuration. For example -\/c cut=org/freedesktop/openicc would be the parameter cut for all plugins. Have a look at \#146 if you want to use it.
\end{DoxyItemize}

\subsection*{Compatibility}

The core A\+PI (kdb.\+h), as always, stayed A\+P\+I/\+A\+BI compatible. The only changes in kdb.\+h is the addition of K\+E\+Y\+\_\+\+C\+A\+S\+C\+A\+D\+I\+N\+G\+\_\+\+N\+A\+ME and K\+E\+Y\+\_\+\+M\+E\+T\+A\+\_\+\+N\+A\+ME. So applications compiled against 0.\+8.\+10 and using these constants, will not work with Elektra 0.\+8.\+9.

The constants allow us to create following kinds of keys\+:


\begin{DoxyItemize}
\item empty names\+: this was always possible, because invalid names (including empty names) did not cause key\+New to abort
\item metanames\+: this is a new feature that allows us to compare key names with metakeys
\item cascading names\+: names starting with / have the special meaning that they do not specify which namespace they have. When such names are used for
\begin{DoxyItemize}
\item \hyperlink{group__kdb_ga28e385fd9cb7ccfe0b2f1ed2f62453a1}{kdb\+Get()} and \hyperlink{group__kdb_ga11436b058408f83d303ca5e996832bcf}{kdb\+Set()} keys are retrieved from all namespaces
\item \hyperlink{group__keyset_gaa34fc43a081e6b01e4120daa6c112004}{ks\+Lookup()} keys are searched in all namespaces
\item \hyperlink{group__keyset_gad2e30fb6d4739d917c5abb2ac2f9c1a1}{ks\+Lookup\+By\+Name()} is now just a wrapper for \hyperlink{group__keyset_gaa34fc43a081e6b01e4120daa6c112004}{ks\+Lookup()}. The method does not do much except creating a key and passing them to \hyperlink{group__keyset_gaa34fc43a081e6b01e4120daa6c112004}{ks\+Lookup()}.
\end{DoxyItemize}
\end{DoxyItemize}

Usage in C is\+: \begin{DoxyVerb}Key *c = keyNew("/org/freedesktop", KEY_CASCADING_NAME, KEY_END);
Key *m = keyNew("comment/#0", KEY_META_NAME, KEY_END);
\end{DoxyVerb}


The same functionality exists, of course, in available in all bindings, too.

Changes in non-\/core A\+PI are\+:


\begin{DoxyItemize}
\item xmltool now does not output default (unchanged) uid,gid and mode
\item ks\+Lookup\+By\+Spec from \hyperlink{kdbproposal_8h}{kdbproposal.\+h} was removed, is now integrated into ks\+Lookup
\item extension key\+Name\+Get\+Namespace was removed
\item the hosts comment format has changed
\item the default resolver has changed (uses passwd)
\item \hyperlink{classkdb_1_1tools_1_1Backend_a1650b149ebf313ee8cd3472247212263}{kdb\+::tools\+::\+Backend\+::\+Backend} constructor, try\+Plugin and add\+Plugin have changed\+:
\begin{DoxyItemize}
\item mountname is now automatically calculated
\item add\+Plugin allows us to add a Key\+Set to validate plugins with different contracts correctly
\end{DoxyItemize}
\item C++ binding now throws std\+::bad\+\_\+alloc on allocation problems (and not Invalid\+Name)
\end{DoxyItemize}

\subsection*{C\+Make}

It is now possible to remove a plugin/binding/tools by prefixing a name with \char`\"{}-\/\char`\"{}. The new \char`\"{}-\/element\char`\"{} syntax is accepted by T\+O\+O\+LS, B\+I\+N\+D\+I\+N\+GS and P\+L\+U\+G\+I\+NS. It is very handy in combination with A\+LL, e.\+g.\+: \begin{DoxyVerb}-DPLUGINS="ALL;-xmltool"
\end{DoxyVerb}


will include all plugins except xmltool.

\subsection*{Improved comments}

Comment preserving was improved a lot. Especially, the hosts plugin was rewritten completely. Now multiple different comment styles can be intermixed without losing information. E.\+g. some I\+NI formats support both ; and \# for comments. With the new comments it is possible to preserve that information and even better\+: applications can iterate over that information (metadata).

To mount the new hosts plugin use (if you already have mounted it, you have nothing to do)\+: \begin{DoxyVerb}kdb mount /etc/hosts system/hosts hosts
\end{DoxyVerb}


The hosts plugin now seperates from ipv4 and ipv6 which makes the host names canonical again, e.\+g.\+: \begin{DoxyVerb}kdb get system/hosts/ipv4/localhost
kdb get system/hosts/ipv6/localhost
\end{DoxyVerb}


To access the inline-\/comment, use\+: \begin{DoxyVerb}kdb getmeta system/hosts/ipv4/localhost "comment/#0"
\end{DoxyVerb}


For other meta information, see\+: \begin{DoxyVerb}kdb lsmeta system/hosts/ipv4/localhost
\end{DoxyVerb}


Additionally, a small A\+PI for specific metadata operations emerges. These operations will be moved to a separate library and will not stay in Elektra’s core library.

\subsection*{Proposal}


\begin{DoxyItemize}
\item lookup options\+:
\begin{DoxyItemize}
\item K\+D\+B\+\_\+\+O\+\_\+\+S\+P\+EC uses the lookup key as specification
\item K\+D\+B\+\_\+\+O\+\_\+\+C\+R\+E\+A\+TE creates a key if it could not be found
\end{DoxyItemize}
\item elektra\+Key\+Get\+Meta\+Key\+Set creates a Key\+Set from metadata
\item elektra\+Ks\+Filter allows us to filter a Key\+Set arbitrarily (not only key\+Is\+Below in case of ks\+Cut). It reintroduces more functional programming.
\item key\+Get\+Namespace was reintroduced. In one of the next versions of Elektra we will introduce new namespaces. key\+Get\+Namespace allows the compiler to output a warning when some namespaces are not handled in your C/\+C++ code.
\end{DoxyItemize}

\subsection*{Java binding}

Elektra now fully supports applications written in Java and also Plugins written in the same language.

The \href{https://github.com/ElektraInitiative/libelektra/tree/master/src/bindings/jna}{\tt new binding was developed using jna.} For the \href{https://github.com/ElektraInitiative/libelektra/tree/master/src/plugins/jni}{\tt plugin interface J\+NI} was used. We developed already \href{https://master.libelektra.org/src/bindings/jna/libelektra4j/src/main/java/org/libelektra/plugin}{\tt some plugins}.

\subsection*{Qt-\/\+Gui}

Raffael Pancheri released the version 0.\+0.\+2 of the Qt-\/\+Gui\+:


\begin{DoxyItemize}
\item added Backend Wizard for mounting
\item user can hover over Tree\+View items and quickly see key name, key value and metakeys
\item it is now easily possible to create and edit arrays
\item added header to Meta\+Area for better clarity
\item many small layout and view update fixes
\end{DoxyItemize}

\subsection*{Further stuff and small fixes}


\begin{DoxyItemize}
\item Two new error/warnings information\+: mountpoint and configfile. It is added automatically and all tools will print it.
\item C++ I/O for key(s) now allows null terminator next to new-\/line terminator
\item fix error plugin\+: now use on\+\_\+open/trigger\+\_\+warnings to be consistent
\item fix metaset\+: now correctly append new key
\item arrays are also available when compiled with mingw (but tests still have to be excluded for successful compilation)
\item fix \#136
\item fix long help text in {\ttfamily kdb check}
\item signed release tags are now used
\end{DoxyItemize}

\subsection*{Get It!}

You can download the release from \href{http://www.markus-raab.org/ftp/elektra/releases/elektra-0.8.10.tar.gz}{\tt here}


\begin{DoxyItemize}
\item size\+: 1915277
\item md5sum\+: 2b16a4b555bc187562a0b38919d822a1
\item sha1\+: 08b1d0139fc5eb0d03c52408478e68b91b1825dc
\item sha256\+: 526e2ed61e87d89966eb36ddad78d8139b976e01ce18aab340d8a1df47132355
\end{DoxyItemize}

already built A\+P\+I-\/\+Docu can be found \href{https://doc.libelektra.org/api/0.8.10/html/}{\tt here}

\subsection*{Stay tuned!}

Subscribe to the \href{https://doc.libelektra.org/news/feed.rss}{\tt new R\+SS feed} to always get the release notifications.

\href{https://doc.libelektra.org/news/6ce57ecf-420a-4a31-821e-1c5fe5532eb4.html}{\tt Permalink to this N\+E\+WS entry}

For more information, see \href{https://www.libelektra.org}{\tt https\+://www.\+libelektra.\+org}

Best regards, Markus 