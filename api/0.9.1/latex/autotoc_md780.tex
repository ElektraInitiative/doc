
\begin{DoxyItemize}
\item infos = Information about the yamlcpp plugin is in keys below
\item infos/author = René Schwaiger \href{mailto:sanssecours@me.com}{\texttt{ sanssecours@me.\+com}}
\item infos/licence = B\+SD
\item infos/needs = base64 directoryvalue
\item infos/provides = storage/yaml
\item infos/recommends =
\item infos/placements = getstorage setstorage
\item infos/status = maintained unittest preview unfinished concept discouraged
\item infos/metadata =
\item infos/description = This storage plugin reads and writes data in the Y\+A\+ML format
\end{DoxyItemize}\hypertarget{autotoc_md780_src_plugins_yamlcpp_README_md}{}\section{Y\+A\+M\+L C\+PP}\label{autotoc_md780_src_plugins_yamlcpp_README_md}
\hypertarget{autotoc_md780_autotoc_md781}{}\subsection{Introduction}\label{autotoc_md780_autotoc_md781}
The Y\+A\+ML C\+PP plugin reads and writes configuration data via the \href{https://github.com/jbeder/yaml-cpp}{\texttt{ yaml-\/cpp}} library.\hypertarget{autotoc_md780_autotoc_md782}{}\subsection{Usage}\label{autotoc_md780_autotoc_md782}
You can mount this plugin via {\ttfamily kdb mount}\+:


\begin{DoxyCode}{0}
\DoxyCodeLine{sudo kdb mount config.yaml /tests/yamlcpp yamlcpp}
\end{DoxyCode}


. To unmount the plugin use {\ttfamily kdb umount}\+:


\begin{DoxyCode}{0}
\DoxyCodeLine{sudo kdb umount /tests/yamlcpp}
\end{DoxyCode}


. The following examples show how you can store and retrieve data via {\ttfamily yamlcpp}.

\`{}\`{}{\ttfamily  @section autotoc\+\_\+md783 Mount yamlcpp plugin to cascading namespace}/tests/yamlcpp\`{} sudo kdb mount config.\+yaml /tests/yamlcpp yamlcpp\hypertarget{autotoc_md780_autotoc_md784}{}\section{Manually add a mapping to the database}\label{autotoc_md780_autotoc_md784}
echo \char`\"{}🔑 \+: 🐳\char`\"{} $>$ {\ttfamily kdb file /tests/yamlcpp} \hypertarget{autotoc_md780_autotoc_md785}{}\section{Retrieve the value of the manually added key}\label{autotoc_md780_autotoc_md785}
kdb get /tests/yamlcpp/🔑 \#$>$ 🐳\hypertarget{autotoc_md780_autotoc_md786}{}\section{Manually add syntactically incorrect data}\label{autotoc_md780_autotoc_md786}
echo \char`\"{}some key\+: @some  value\char`\"{} $>$$>$ {\ttfamily kdb file /tests/yamlcpp} kdb get \char`\"{}/tests/yamlcpp/some key\char`\"{} \hypertarget{autotoc_md780_autotoc_md787}{}\section{S\+T\+D\+E\+R\+R\+: .$\ast$yaml-\/cpp\+: error at line 2, column 11\+: unknown token.$\ast$}\label{autotoc_md780_autotoc_md787}
\hypertarget{autotoc_md780_autotoc_md788}{}\section{E\+R\+R\+O\+R\+: C03100}\label{autotoc_md780_autotoc_md788}
\hypertarget{autotoc_md780_autotoc_md789}{}\section{R\+E\+T\+: 5}\label{autotoc_md780_autotoc_md789}
\hypertarget{autotoc_md780_autotoc_md790}{}\section{Overwrite incorrect data}\label{autotoc_md780_autotoc_md790}
echo \char`\"{}🔑\+: value\char`\"{} $>$ {\ttfamily kdb file /tests/yamlcpp}\hypertarget{autotoc_md780_autotoc_md791}{}\section{Add some values via $<$tt$>$kdb set$<$/tt$>$}\label{autotoc_md780_autotoc_md791}
kdb set /tests/yamlcpp 🎵 kdb set /tests/yamlcpp/fleetwood mac kdb set /tests/yamlcpp/the chain\hypertarget{autotoc_md780_autotoc_md792}{}\section{Retrieve the new values}\label{autotoc_md780_autotoc_md792}
kdb get /tests/yamlcpp \#$>$ 🎵 kdb get /tests/yamlcpp/the \#$>$ chain kdb get /tests/yamlcpp/fleetwood \#$>$ mac\hypertarget{autotoc_md780_autotoc_md793}{}\section{Undo modifications}\label{autotoc_md780_autotoc_md793}
kdb rm -\/r /tests/yamlcpp sudo kdb umount /tests/yamlcpp 
\begin{DoxyCode}{0}
\DoxyCodeLine{\#\# Arrays}
\DoxyCodeLine{}
\DoxyCodeLine{YAML CPP provides support for Elektra’s array data type.}
\end{DoxyCode}
 \hypertarget{autotoc_md780_autotoc_md794}{}\section{Mount yamlcpp plugin to $<$tt$>$user/tests/yamlcpp$<$/tt$>$}\label{autotoc_md780_autotoc_md794}
sudo kdb mount config.\+yaml user/tests/yamlcpp yamlcpp\hypertarget{autotoc_md780_autotoc_md795}{}\section{Manually add an array to the database}\label{autotoc_md780_autotoc_md795}
echo \textquotesingle{}sunny\+:\textquotesingle{} $>$ {\ttfamily kdb file user/tests/yamlcpp} echo \textquotesingle{} -\/ Charlie\textquotesingle{} $>$$>$ {\ttfamily kdb file user/tests/yamlcpp} echo \textquotesingle{} -\/ Dee\textquotesingle{} $>$$>$ {\ttfamily kdb file user/tests/yamlcpp}\hypertarget{autotoc_md780_autotoc_md796}{}\section{List the array entries}\label{autotoc_md780_autotoc_md796}
kdb ls user/tests/yamlcpp \#$>$ user/tests/yamlcpp/sunny \#$>$ user/tests/yamlcpp/sunny/\#0 \#$>$ user/tests/yamlcpp/sunny/\#1\hypertarget{autotoc_md780_autotoc_md797}{}\section{Read an array entry}\label{autotoc_md780_autotoc_md797}
kdb get user/tests/yamlcpp/sunny/\#1 \#$>$ Dee\hypertarget{autotoc_md780_autotoc_md798}{}\section{You can retrieve the last index of an array by reading the metakey $<$tt$>$array$<$/tt$>$}\label{autotoc_md780_autotoc_md798}
kdb meta-\/get user/tests/yamlcpp/sunny array \hypertarget{autotoc_md780_autotoc_md799}{}\section{1}\label{autotoc_md780_autotoc_md799}
\hypertarget{autotoc_md780_autotoc_md800}{}\section{Extend the array}\label{autotoc_md780_autotoc_md800}
kdb set user/tests/yamlcpp/sunny/\#2 Dennis kdb set user/tests/yamlcpp/sunny/\#3 Frank kdb set user/tests/yamlcpp/sunny/\#4 Mac\hypertarget{autotoc_md780_autotoc_md801}{}\section{The plugin supports empty array fields}\label{autotoc_md780_autotoc_md801}
kdb set user/tests/yamlcpp/sunny/\#\+\_\+10 \textquotesingle{}The Waitress\textquotesingle{} kdb meta-\/get user/tests/yamlcpp/sunny array \#$>$ \#\+\_\+10 kdb get user/tests/yamlcpp/sunny/\#\+\_\+9 \hypertarget{autotoc_md780_autotoc_md802}{}\section{R\+E\+T\+: 11}\label{autotoc_md780_autotoc_md802}
\hypertarget{autotoc_md780_autotoc_md803}{}\section{Retrieve the last array entry}\label{autotoc_md780_autotoc_md803}
kdb get user/tests/yamlcpp/sunny/\$(kdb meta-\/get user/tests/yamlcpp/sunny array) \#$>$ The Waitress\hypertarget{autotoc_md780_autotoc_md804}{}\section{The plugin also supports empty arrays (arrays without any elements)}\label{autotoc_md780_autotoc_md804}
kdb meta-\/set user/tests/yamlcpp/empty array \textquotesingle{}\textquotesingle{} kdb export user/tests/yamlcpp/empty yamlcpp \#$>$ \mbox{[}\mbox{]}\hypertarget{autotoc_md780_autotoc_md805}{}\section{For arrays with at least one value we do not need to set the type $<$tt$>$array$<$/tt$>$}\label{autotoc_md780_autotoc_md805}
kdb set user/tests/yamlcpp/movies kdb set user/tests/yamlcpp/movies/\#0 \textquotesingle{}A Silent Voice\textquotesingle{} kdb meta-\/get user/tests/yamlcpp/movies array \#$>$ \#0 kdb export user/tests/yamlcpp/movies yamlcpp \#$>$ -\/ A Silent Voice\hypertarget{autotoc_md780_autotoc_md806}{}\section{Undo modifications to the key database}\label{autotoc_md780_autotoc_md806}
kdb rm -\/r user/tests/yamlcpp sudo kdb umount user/tests/yamlcpp 
\begin{DoxyCode}{0}
\DoxyCodeLine{\#\#\# Nested Arrays}
\DoxyCodeLine{}
\DoxyCodeLine{The plugin also supports nested arrays.}
\end{DoxyCode}
 \hypertarget{autotoc_md780_autotoc_md807}{}\section{Mount yamlcpp plugin to $<$tt$>$user/tests/yamlcpp$<$/tt$>$}\label{autotoc_md780_autotoc_md807}
sudo kdb mount config.\+yaml user/tests/yamlcpp yamlcpp\hypertarget{autotoc_md780_autotoc_md808}{}\section{Add some key value pairs}\label{autotoc_md780_autotoc_md808}
kdb set user/tests/yamlcpp/key value kdb set user/tests/yamlcpp/array/\#0 scalar kdb set user/tests/yamlcpp/array/\#1/key value kdb set user/tests/yamlcpp/array/\#1/🔑 🙈

kdb ls user/tests/yamlcpp \#$>$ user/tests/yamlcpp/array \#$>$ user/tests/yamlcpp/array/\#0 \#$>$ user/tests/yamlcpp/array/\#1/key \#$>$ user/tests/yamlcpp/array/\#1/🔑 \#$>$ user/tests/yamlcpp/key\hypertarget{autotoc_md780_autotoc_md809}{}\section{Retrieve part of an array value}\label{autotoc_md780_autotoc_md809}
kdb get user/tests/yamlcpp/array/\#1/key \#$>$ value\hypertarget{autotoc_md780_autotoc_md810}{}\section{Since an array saves a list of values, an array parent}\label{autotoc_md780_autotoc_md810}
\hypertarget{autotoc_md780_autotoc_md811}{}\section{-\/ which represent the array -\/ does not store a value!}\label{autotoc_md780_autotoc_md811}
echo \char`\"{}user/tests/yamlcpp/array\+: “\`{}kdb get user/tests/yamlcpp/array\`{}”\char`\"{} \#$>$ user/tests/yamlcpp/array\+: “”\hypertarget{autotoc_md780_autotoc_md812}{}\section{Remove part of an array value}\label{autotoc_md780_autotoc_md812}
kdb rm user/tests/yamlcpp/array/\#1/key

kdb ls user/tests/yamlcpp \#$>$ user/tests/yamlcpp/array \#$>$ user/tests/yamlcpp/array/\#0 \#$>$ user/tests/yamlcpp/array/\#1/🔑 \#$>$ user/tests/yamlcpp/key\hypertarget{autotoc_md780_autotoc_md813}{}\section{The plugin stores array keys using Y\+A\+M\+L sequences.}\label{autotoc_md780_autotoc_md813}
\hypertarget{autotoc_md780_autotoc_md814}{}\section{Since yaml-\/cpp stores keys in arbitrary order -\/}\label{autotoc_md780_autotoc_md814}
\hypertarget{autotoc_md780_autotoc_md815}{}\section{either $<$tt$>$key$<$/tt$>$ or $<$tt$>$array$<$/tt$>$ could be the “first” key -\/}\label{autotoc_md780_autotoc_md815}
\hypertarget{autotoc_md780_autotoc_md816}{}\section{we remove $<$tt$>$key$<$/tt$>$ before we retrieve the data. This way}\label{autotoc_md780_autotoc_md816}
\hypertarget{autotoc_md780_autotoc_md817}{}\section{we make sure that the output below will always look}\label{autotoc_md780_autotoc_md817}
\hypertarget{autotoc_md780_autotoc_md818}{}\section{the same.}\label{autotoc_md780_autotoc_md818}
kdb rm user/tests/yamlcpp/key kdb file user/tests/yamlcpp $\vert$ xargs cat \#$>$ array\+: \#$>$ -\/ scalar \#$>$ -\/ 🔑\+: 🙈\hypertarget{autotoc_md780_autotoc_md819}{}\section{Undo modifications to the key database}\label{autotoc_md780_autotoc_md819}
kdb rm -\/r user/tests/yamlcpp sudo kdb umount user/tests/yamlcpp 
\begin{DoxyCode}{0}
\DoxyCodeLine{\#\#\# Sparse Arrays}
\DoxyCodeLine{}
\DoxyCodeLine{Since Elektra allows [“holes”](@ref doc\_decisions\_holes\_md) in a key set, YAML CPP has to support small key sets that describe relatively complex data.}
\end{DoxyCode}
 \hypertarget{autotoc_md780_autotoc_md820}{}\section{Mount yamlcpp plugin}\label{autotoc_md780_autotoc_md820}
sudo kdb mount config.\+yaml user/tests/yamlcpp yamlcpp

kdb set user/tests/yamlcpp/\#0/map/\#1/\#0 value kdb file user/tests/yamlcpp $\vert$ xargs cat \#$>$ -\/ map\+: \#$>$ -\/ $\sim$ \#$>$ -\/ \#$>$ -\/ value\hypertarget{autotoc_md780_autotoc_md821}{}\section{The plugin adds the missing array parents to the key set}\label{autotoc_md780_autotoc_md821}
kdb ls user/tests/yamlcpp \#$>$ user/tests/yamlcpp \#$>$ user/tests/yamlcpp/\#0/map \#$>$ user/tests/yamlcpp/\#0/map/\#0 \#$>$ user/tests/yamlcpp/\#0/map/\#1 \#$>$ user/tests/yamlcpp/\#0/map/\#1/\#0\hypertarget{autotoc_md780_autotoc_md822}{}\section{Undo modifications to the key database}\label{autotoc_md780_autotoc_md822}
kdb rm -\/r user/tests/yamlcpp sudo kdb umount user/tests/yamlcpp 
\begin{DoxyCode}{0}
\DoxyCodeLine{\#\# Metadata}
\DoxyCodeLine{}
\DoxyCodeLine{The plugin supports metadata. The example below shows how a basic `Key` including some metadata, looks inside the YAML configuration file:}
\DoxyCodeLine{}
\DoxyCodeLine{```yaml}
\DoxyCodeLine{key without metadata: value}
\DoxyCodeLine{key with metadata: !elektra/meta}
\DoxyCodeLine{  - value2}
\DoxyCodeLine{  - metakey: metavalue}
\DoxyCodeLine{    empty metakey:}
\DoxyCodeLine{    another metakey: another metavalue}
\end{DoxyCode}


. As we can see above the value containing metadata is marked by the tag handle {\ttfamily !elektra/meta}. The data type contains a list with two elements. The first element of this list specifies the value of the key, while the second element contains a map saving the metadata for the key. The data above represents the following key set in Elektra if we mount the file directly to the namespace {\ttfamily user}\+:

\tabulinesep=1mm
\begin{longtabu}spread 0pt [c]{*{4}{|X[-1]}|}
\hline
\PBS\centering \cellcolor{\tableheadbgcolor}\textbf{ Name  }&\PBS\centering \cellcolor{\tableheadbgcolor}\textbf{ Value  }&\PBS\centering \cellcolor{\tableheadbgcolor}\textbf{ Metaname  }&\PBS\centering \cellcolor{\tableheadbgcolor}\textbf{ Metavalue   }\\\cline{1-4}
\endfirsthead
\hline
\endfoot
\hline
\PBS\centering \cellcolor{\tableheadbgcolor}\textbf{ Name  }&\PBS\centering \cellcolor{\tableheadbgcolor}\textbf{ Value  }&\PBS\centering \cellcolor{\tableheadbgcolor}\textbf{ Metaname  }&\PBS\centering \cellcolor{\tableheadbgcolor}\textbf{ Metavalue   }\\\cline{1-4}
\endhead
\PBS\centering user/key without metadata  &\PBS\centering value1  &\PBS\centering —  &\PBS\centering —   \\\cline{1-4}
\PBS\centering user/key with metadata  &\PBS\centering value2  &\PBS\centering metakey  &\PBS\centering metavalue   \\\cline{1-4}
\PBS\centering &\PBS\centering &\PBS\centering empty metakey  &\PBS\centering —   \\\cline{1-4}
\PBS\centering &\PBS\centering &\PBS\centering another metakey  &\PBS\centering another metavalue   \\\cline{1-4}
\end{longtabu}


. The example below shows how we can read and write metadata using the {\ttfamily yamlcpp} plugin via {\ttfamily kdb}.

\`{}\`{}{\ttfamily  @section autotoc\+\_\+md823 Mount yamlcpp plugin to}user/tests/yamlcpp\`{} sudo kdb mount config.\+yaml user/tests/yamlcpp yamlcpp\hypertarget{autotoc_md780_autotoc_md824}{}\section{Manually add a key including metadata to the database}\label{autotoc_md780_autotoc_md824}
echo \char`\"{}🔑\+: !elektra/meta \mbox{[}🦄, \{comment\+: Unicorn\}\mbox{]}\char`\"{} $>$ {\ttfamily kdb file user/tests/yamlcpp} kdb meta-\/ls user/tests/yamlcpp/🔑 \#$>$ comment kdb meta-\/get user/tests/yamlcpp/🔑 comment \#$>$ Unicorn\hypertarget{autotoc_md780_autotoc_md825}{}\section{Add a new key and add some metadata to the new key}\label{autotoc_md780_autotoc_md825}
kdb set user/tests/yamlcpp/brand new kdb meta-\/set user/tests/yamlcpp/brand comment \char`\"{}\+The Devil And God Are Raging Inside Me\char`\"{} kdb meta-\/set user/tests/yamlcpp/brand rationale \char`\"{}\+Because I Love It\char`\"{}\hypertarget{autotoc_md780_autotoc_md826}{}\section{Retrieve metadata}\label{autotoc_md780_autotoc_md826}
kdb meta-\/ls user/tests/yamlcpp/brand \#$>$ comment \#$>$ rationale kdb meta-\/get user/tests/yamlcpp/brand rationale \#$>$ Because I Love It\hypertarget{autotoc_md780_autotoc_md827}{}\section{Undo modifications to the key database}\label{autotoc_md780_autotoc_md827}
kdb rm -\/r user/tests/yamlcpp sudo kdb umount user/tests/yamlcpp 
\begin{DoxyCode}{0}
\DoxyCodeLine{We can also invoke additional plugins that use metadata like `type`.}
\end{DoxyCode}
 sudo kdb mount config.\+yaml user/tests/yamlcpp yamlcpp type kdb set user/tests/yamlcpp/typetest/number 21 kdb meta-\/set user/tests/yamlcpp/typetest/number check/type short

kdb set user/tests/yamlcpp/typetest/number \char`\"{}\+One\char`\"{} \hypertarget{autotoc_md780_autotoc_md828}{}\section{R\+E\+T\+: 5}\label{autotoc_md780_autotoc_md828}
\hypertarget{autotoc_md780_autotoc_md829}{}\section{S\+T\+D\+E\+R\+R\+: .$\ast$\+Validation Semantic.$\ast$}\label{autotoc_md780_autotoc_md829}
\hypertarget{autotoc_md780_autotoc_md830}{}\section{E\+R\+R\+O\+R\+: C03200}\label{autotoc_md780_autotoc_md830}
kdb get user/tests/yamlcpp/typetest/number \#$>$ 21\hypertarget{autotoc_md780_autotoc_md831}{}\section{Undo modifications to the key database}\label{autotoc_md780_autotoc_md831}
kdb rm -\/r user/tests/yamlcpp sudo kdb umount user/tests/yamlcpp 
\begin{DoxyCode}{0}
\DoxyCodeLine{\#\# Binary Data}
\DoxyCodeLine{}
\DoxyCodeLine{YAML CPP also supports [base64](https://tools.ietf.org/html/rfc4648) encoded data via the [Base64](@ref src\_plugins\_base64\_README\_md) plugin.}
\end{DoxyCode}
 \hypertarget{autotoc_md780_autotoc_md832}{}\section{Mount Y\+A\+M\+L C\+P\+P plugin at $<$tt$>$user/tests/binary$<$/tt$>$}\label{autotoc_md780_autotoc_md832}
sudo kdb mount test.\+yaml user/tests/binary yamlcpp \hypertarget{autotoc_md780_autotoc_md833}{}\section{Manually add binary data}\label{autotoc_md780_autotoc_md833}
echo \textquotesingle{}bin\+: !!binary a\+Gk=\textquotesingle{} $>$ {\ttfamily kdb file user/tests/binary}\hypertarget{autotoc_md780_autotoc_md834}{}\section{Base 64 decodes the data $<$tt$>$a\+Gk=$<$/tt$>$ to $<$tt$>$hi$<$/tt$>$ and stores the value in binary form.}\label{autotoc_md780_autotoc_md834}
\hypertarget{autotoc_md780_autotoc_md835}{}\section{The command $<$tt$>$kdb get$<$/tt$>$ prints the data as hexadecimal byte values.}\label{autotoc_md780_autotoc_md835}
kdb get user/tests/binary/bin \#$>$ \textbackslash{}x68\textbackslash{}x69\hypertarget{autotoc_md780_autotoc_md836}{}\section{Add a string value to the database}\label{autotoc_md780_autotoc_md836}
kdb set user/tests/binary/text mate \hypertarget{autotoc_md780_autotoc_md837}{}\section{Base 64 does not modify textual values}\label{autotoc_md780_autotoc_md837}
kdb get user/tests/binary/text \#$>$ mate\hypertarget{autotoc_md780_autotoc_md838}{}\section{The Base 64 plugin re-\/encodes binary data before Y\+A\+M\+L C\+P\+P stores the key set. Hence the}\label{autotoc_md780_autotoc_md838}
\hypertarget{autotoc_md780_autotoc_md839}{}\section{configuration file contains the value $<$tt$>$a\+Gk=$<$/tt$>$ even after Y\+A\+M\+L C\+P\+P wrote a new configuration.}\label{autotoc_md780_autotoc_md839}
grep -\/q \textquotesingle{}bin\+: !.$\ast$ a\+Gk=\textquotesingle{} {\ttfamily kdb file user/tests/binary} \hypertarget{autotoc_md780_autotoc_md840}{}\section{R\+E\+T\+: 0}\label{autotoc_md780_autotoc_md840}
\hypertarget{autotoc_md780_autotoc_md841}{}\section{Undo modifications to the database}\label{autotoc_md780_autotoc_md841}
kdb rm -\/r user/tests/binary sudo kdb umount user/tests/binary 
\begin{DoxyCode}{0}
\DoxyCodeLine{\#\# Null \& Empty}
\DoxyCodeLine{}
\DoxyCodeLine{Sometimes you only want to save a key without a value (null key) or a key with an empty value. The commands below show that YAML CPP supports this scenario properly.}
\end{DoxyCode}
 \hypertarget{autotoc_md780_autotoc_md842}{}\section{Mount Y\+A\+M\+L C\+P\+P plugin at $<$tt$>$user/tests/yamlcpp$<$/tt$>$}\label{autotoc_md780_autotoc_md842}
sudo kdb mount test.\+yaml user/tests/yamlcpp yamlcpp\hypertarget{autotoc_md780_autotoc_md843}{}\section{Check if the plugin saves null keys correctly}\label{autotoc_md780_autotoc_md843}
kdb set user/tests/yamlcpp/null kdb set user/tests/yamlcpp/null/level1/level2 kdb meta-\/set user/tests/yamlcpp/null/level1/level2 comment \textquotesingle{}Null key\textquotesingle{}

kdb ls user/tests/yamlcpp/null \#$>$ user/tests/yamlcpp/null \#$>$ user/tests/yamlcpp/null/level1/level2 kdb get -\/v user/tests/yamlcpp/null $\vert$ grep -\/q \textquotesingle{}The key is null.\textquotesingle{} kdb get -\/v user/tests/yamlcpp/null/level1/level2 $\vert$ grep -\/q \textquotesingle{}The key is null.\textquotesingle{}\hypertarget{autotoc_md780_autotoc_md844}{}\section{Check if the plugin saves empty keys correctly}\label{autotoc_md780_autotoc_md844}
kdb set user/tests/yamlcpp/empty \char`\"{}\char`\"{} kdb set user/tests/yamlcpp/empty/level1/level2

kdb ls user/tests/yamlcpp/empty \#$>$ user/tests/yamlcpp/empty \#$>$ user/tests/yamlcpp/empty/level1/level2 kdb get -\/v user/tests/yamlcpp/empty $\vert$ grep -\/vq \textquotesingle{}The key is null.\textquotesingle{}\hypertarget{autotoc_md780_autotoc_md845}{}\section{Undo modifications to the database}\label{autotoc_md780_autotoc_md845}
kdb rm -\/r user/tests/yamlcpp sudo kdb umount user/tests/yamlcpp 
\begin{DoxyCode}{0}
\DoxyCodeLine{\#\# Binary Values}
\DoxyCodeLine{}
\DoxyCodeLine{Elektra [saves binary data as either `0` or `1`](@ref doc\_decisions\_boolean\_md). The YAML CPP plugin supports this design decision by converting between YAML’s and Elektra’s boolean type.}
\end{DoxyCode}
 \hypertarget{autotoc_md780_autotoc_md846}{}\section{Mount Y\+A\+M\+L C\+P\+P plugin at $<$tt$>$user/tests/yamlcpp$<$/tt$>$}\label{autotoc_md780_autotoc_md846}
sudo kdb mount config.\+yaml user/tests/yamlcpp yamlcpp \hypertarget{autotoc_md780_autotoc_md847}{}\section{Manually add boolean key}\label{autotoc_md780_autotoc_md847}
echo \textquotesingle{}truth\+: true\textquotesingle{} $>$ {\ttfamily kdb file user/tests/yamlcpp}

kdb get user/tests/yamlcpp/truth \#$>$ 1\hypertarget{autotoc_md780_autotoc_md848}{}\section{Add another boolean value}\label{autotoc_md780_autotoc_md848}
kdb set user/tests/yamlcpp/success 0 kdb get user/tests/yamlcpp/success \#$>$ 0\hypertarget{autotoc_md780_autotoc_md849}{}\section{Undo modifications to the database}\label{autotoc_md780_autotoc_md849}
kdb rm -\/r user/tests/yamlcpp sudo kdb umount user/tests/yamlcpp 
\begin{DoxyCode}{0}
\DoxyCodeLine{\#\# Dependencies}
\DoxyCodeLine{}
\DoxyCodeLine{This plugin requires [yaml-cpp][]. On a Debian based OS the package for the library is called [`libyaml-cpp-dev`](https://packages.debian.org/libyaml-cpp-dev). On macOS you can install the package [`yaml-cpp`](https://repology.org/project/yaml-cpp) via [HomeBrew](https://brew.sh).}
\DoxyCodeLine{}
\DoxyCodeLine{\#\# Limitations}
\DoxyCodeLine{}
\DoxyCodeLine{\#\#\# Leaf Values}
\DoxyCodeLine{}
\DoxyCodeLine{One of the limitations of this plugin is, that it only supports values inside [leaf nodes](https://github.com/ElektraInitiative/libelektra/issues/106). Let us look at an example to show what that means. The YAML file below:}
\DoxyCodeLine{}
\DoxyCodeLine{```yaml}
\DoxyCodeLine{root:}
\DoxyCodeLine{  subtree: 🍂}
\DoxyCodeLine{  below root: leaf}
\DoxyCodeLine{level 1:}
\DoxyCodeLine{  level 2:}
\DoxyCodeLine{    level 3: 🍁}
\end{DoxyCode}


stores all of the values ({\ttfamily 🍂}, {\ttfamily leaf} and {\ttfamily 🍁}) in the leaves of the mapping. The drawing below makes this situation a little bit clearer.



The key set that this plugin creates using the data above looks like this (assuming we mount the plugin to {\ttfamily user/tests/yamlcpp})\+:

\tabulinesep=1mm
\begin{longtabu}spread 0pt [c]{*{2}{|X[-1]}|}
\hline
\PBS\centering \cellcolor{\tableheadbgcolor}\textbf{ Name  }&\PBS\centering \cellcolor{\tableheadbgcolor}\textbf{ Value   }\\\cline{1-2}
\endfirsthead
\hline
\endfoot
\hline
\PBS\centering \cellcolor{\tableheadbgcolor}\textbf{ Name  }&\PBS\centering \cellcolor{\tableheadbgcolor}\textbf{ Value   }\\\cline{1-2}
\endhead
user/tests/yamlcpp/level  &\\\cline{1-2}
user/tests/yamlcpp/level 1/level 2  &\\\cline{1-2}
user/tests/yamlcpp/level 1/level 2/level 3  &🍁   \\\cline{1-2}
user/tests/yamlcpp/root  &\\\cline{1-2}
user/tests/yamlcpp/root/below root  &leaf   \\\cline{1-2}
user/tests/yamlcpp/root/subtree  &🍂   \\\cline{1-2}
\end{longtabu}


. Now why is this plugin unable to store values outside leaf nodes? For example, why can we not store a value inside {\ttfamily user/tests/yamlcpp/level 1/level 2}? To answer this question we need to look at the Y\+A\+ML representation\+:


\begin{DoxyCode}{0}
\DoxyCodeLine{level 1:}
\DoxyCodeLine{  level 2:}
\DoxyCodeLine{    level 3: 🍁}
\end{DoxyCode}


. In a naive approach we might just try to add a value e.\+g. {\ttfamily 🙈} right next to level 2\+:


\begin{DoxyCode}{0}
\DoxyCodeLine{level 1:}
\DoxyCodeLine{  level 2: 🙈}
\DoxyCodeLine{    level 3:  🍁}
\end{DoxyCode}


. This however would be not correct, since then the Y\+A\+ML node {\ttfamily level 2} would contain both a scalar value ({\ttfamily 🙈}) and a mapping ({\ttfamily \{ level 3\+: 🍁 \}}). We could solve this dilemma using a list\+:


\begin{DoxyCode}{0}
\DoxyCodeLine{level 1:}
\DoxyCodeLine{  level 2:}
\DoxyCodeLine{    - 🙈}
\DoxyCodeLine{    - level 3: 🍁}
\end{DoxyCode}


. However, if we use this approach we are not able to support Elektra’s array type properly.