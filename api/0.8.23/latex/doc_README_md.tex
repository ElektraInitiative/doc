This folder contains documentation for “\+Elektra – the configuration framework for everyone”. If you do not know what Elektra is, then we recommend that you check out our \href{https://www.libelektra.org/home}{\tt homepage} first. This Read\+Me deals with the content of the documentation folder and should give you a hint where to look for specific information.

\subsection*{Introductory}


\begin{DoxyItemize}
\item \hyperlink{md_doc_help_elektra-glossary_doc_help_elektra-glossary_md}{Glossary}\+: The glossary explains common terminology used in the documentation.
\item \hyperlink{doc_WHY_md}{Why}\+: This document describes why you should use Elektra.
\item \hyperlink{doc_BIGPICTURE_md}{Big Picture}\+: This document provides an birds eye view of Elektra and the key database (K\+DB).
\item \hyperlink{doc_GOALS_md}{Goals}\+: We specify the goals and target audiences for Elektra in this document.
\item \hyperlink{doc_SECURITY_md}{Security}\+: This guideline shows how Elektra handles security concerns.
\item \hyperlink{md_doc_tutorials_README_doc_tutorials_README_md}{Tutorials}\+: The tutorials folder provides various {\bfseries user related tutorials}. If you are interested in {\bfseries developer related tutorials} instead, then please take a look at the folder \hyperlink{md_doc_dev_README_doc_dev_README_md}{dev}.
\item News\+: The news folder contains release notes and other recent information about Elektra.
\item \hyperlink{doc_paper_README_md}{Paper}\+: This directory contains a research paper about Elektra, also available in \href{http://joss.theoj.org/papers/10.21105/joss.00044}{\tt P\+DF} format.
\end{DoxyItemize}

\subsection*{Using Elektra}


\begin{DoxyItemize}
\item \hyperlink{doc_INSTALL_md}{Installation}\+: These instructions tell you how you can install Elektra in your favorite operating system.
\item \hyperlink{doc_COMPILE_md}{Compile}\+: If you want to compile Elektra from source please take a look at this document.
\item Help\+: This folder contains our man pages in Markdown format. The folder man contains these man pages in roff format, which you can read using the Unix utility \href{https://en.wikipedia.org/wiki/Man_page}{\tt {\ttfamily man}} if you already installed Elektra.
\end{DoxyItemize}

\subsubsection*{A\+PI}


\begin{DoxyItemize}
\item A\+PI\+: This overview of the application programming interface tells you how you can develop an application that uses Elektra.
\item \hyperlink{doc_DESIGN_md}{Design}\+: This document describes the design of Elektra’s C A\+PI.
\end{DoxyItemize}

\subsection*{Advanced Information}


\begin{DoxyItemize}
\item \href{/home/markus/Projekte/Elektra/current/doc/METADATA.ini}{\tt Metadata}\+: This document specifies data about the K\+DB (meta information), like supported data types and configuration options.
\item \href{/home/markus/Projekte/Elektra/current/doc/CONTRACT.ini}{\tt Contract}\+: The plugin contract specifies keys and values that an \hyperlink{md_src_plugins_README_src_plugins_README_md}{Elektra plugin} provides.
\end{DoxyItemize}

\subsection*{Contributing}


\begin{DoxyItemize}
\item \hyperlink{doc_CODING_md}{Coding}\+: The coding guidelines describe the basic rules you should keep in mind when you want to contribute code to Elektra.
\item \hyperlink{doc_GIT_md}{Git}\+: This document describes how we use the version control system \href{https://git-scm.com}{\tt git} to develop Elektra.
\item \hyperlink{doc_IDEAS_md}{Ideas}\+: If you want to contribute to Elektra and do not know what, you can either take a look here or at our \href{http://libelektra.org/issues}{\tt issue tracker}.
\item To\+Do\+: This folder contains various To\+Do items for future releases of Elektra.
\item \hyperlink{doc_AUTHORS_md}{Authors}\+: This file lists information about Elektra’s authors.
\end{DoxyItemize}

\subsection*{Other}


\begin{DoxyItemize}
\item \hyperlink{doc_images_README_md}{Images}\+: The images folder contains logos and other promotional material.
\item \hyperlink{doc_decisions_README_md}{Decisions}\+: If you are interested in why Elektra uses a certain technology or strategy, then please check out the documents in this folder.
\item \hyperlink{doc_markdownlinkconverter_README_md}{Markdown Link Converter}\+: This tool converts links in Markdown files to make them usable in our \href{https://doc.libelektra.org/api/current/html}{\tt Doxygen documentation}.
\item Usecases\+: This folder contains use cases for our \href{https://www.libelektra.org/auth/login}{\tt snippet sharing service} and the upcoming web user interface for the K\+DB. 
\end{DoxyItemize}