
\begin{DoxyItemize}
\item infos = Information about the blockresolver plugin is in keys below
\item infos/author = Thomas Waser \href{mailto:thomas.waser@libelektra.org}{\tt thomas.\+waser@libelektra.\+org}
\item infos/licence = B\+SD
\item infos/needs =
\item infos/provides = resolver
\item infos/recommends =
\item infos/placements = rollback getresolver setresolver commit
\item infos/status = maintained conformant compatible coverage specific unittest tested nodep libc configurable preview experimental difficult unfinished concept
\item infos/metadata =
\item infos/description = resolver for parts in a file configuration file
\end{DoxyItemize}

The {\ttfamily blockresolver} can be used to only resolve a tagged block inside a configuration file.

\subsubsection*{Implementation details}

{\ttfamily blockresolver} extracts the requested block from the configurations file and writes it into a temporary file. Afterwards Elektra will only work on the temporary file until kdb\+Set is called. On kdb\+Set the contents of the temporary file will be merged with parts outside of the requested block from the original file.

\subsection*{Usage}

\begin{DoxyVerb}`kdb mount -R blockresolver /path/to/my/file /mountpoint -c identifier="identifier-tag"`
\end{DoxyVerb}


where {\ttfamily identifier} specifies the tag {\ttfamily blockresolver} will search for in the configuration file.

A block consists of 2 parts\+:
\begin{DoxyItemize}
\item beginning\+: the identifier suffixed with {\ttfamily start}
\item end\+: the identifier suffixed with {\ttfamily stop}
\end{DoxyItemize}

\subsection*{Limitations}

Currently the identifier must be unique.

\subsection*{Example}


\begin{DoxyCode}
# Backup-and-Restore:system/examples/blockresolver
sudo kdb mount -R blockresolver /tmp/test.block system/examples/blockresolver -c identifier=">>> block
       config" ini

# create testfile
echo 'text'                   >  /tmp/test.block
echo 'more text'              >> /tmp/test.block
echo 'some more text'         >> /tmp/test.block
echo '>>> block config start' >> /tmp/test.block
echo '[section1]'             >> /tmp/test.block
echo 'key1 = val1'            >> /tmp/test.block
echo '[section2]'             >> /tmp/test.block
echo 'key2 = val2'            >> /tmp/test.block
echo '>>> block config stop'  >> /tmp/test.block
echo 'text again'             >> /tmp/test.block
echo 'and more text'          >> /tmp/test.block
echo 'text'                   >> /tmp/test.block

# check testfile
cat /tmp/test.block
#> text
#> more text
#> some more text
#> >>> block config start
#> [section1]
#> key1 = val1
#> [section2]
#> key2 = val2
#> >>> block config stop
#> text again
#> and more text
#> text

# only the block between the tags is read!
kdb export system/examples/blockresolver ini
#> [section1]
#> key1 = val1
#> [section2]
#> key2 = val2

# add a new key to the resolved block
kdb set system/examples/blockresolver/section1/key12 val12

cat /tmp/test.block
#> text
#> more text
#> some more text
#> >>> block config start
#> [section1]
#> key1 = val1
#> key12 = val12
#> [section2]
#> key2 = val2
#> >>> block config stop
#> text again
#> and more text
#> text

# cleanup
kdb rm -r system/examples/blockresolver
sudo kdb umount system/examples/blockresolver
\end{DoxyCode}
 