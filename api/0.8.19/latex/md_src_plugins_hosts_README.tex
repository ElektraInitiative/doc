
\begin{DoxyItemize}
\item infos = Information about H\+O\+S\+T\+S plugin is in keys below
\item infos/author = Markus Raab \href{mailto:elektra@libelektra.org}{\tt elektra@libelektra.\+org}
\item infos/licence = B\+S\+D
\item infos/provides = storage/hosts
\item infos/needs =
\item infos/recommends = glob error network
\item infos/placements = getstorage setstorage
\item infos/status = maintained unittest nodep libc limited
\item infos/description = This plugin reads and writes /etc/hosts files.
\end{DoxyItemize}

The {\ttfamily /etc/hosts} file is a simple text file that associates I\+P addresses with hostnames, one line per I\+P address. The format is described in {\ttfamily hosts(5)}.

\subsection*{Special values}

\subsubsection*{Hostnames}

Canonical hostnames are stored as key names with the I\+P address as key value.

\subsubsection*{Aliases}

Aliases are stored as sub keys with a read only duplicate of the associated ip address as value.

\subsubsection*{Comments}

Comments are stored according to the comment metadata specification (see \href{/home/markus/Projekte/Elektra/current/doc/METADATA.ini}{\tt /doc/\+M\+E\+T\+A\+D\+A\+T\+A.ini} for more information)

\subsubsection*{Multi-\/\+Line Comments}

Since line breaks are preserved, you can identify multi line comments by their trailing line break.

\subsection*{Restrictions}

The ordering of the hosts is stored in metakeys of type \char`\"{}order\char`\"{}. The value is an ascending number. Ordering of aliases is N\+O\+T preserved.

\subsection*{Examples}

Mount the plugin\+: \begin{DoxyVerb}$ kdb mount --with-recommends /etc/hosts system/hosts hosts
\end{DoxyVerb}


Print out all known hosts and their aliases\+: \begin{DoxyVerb}$ kdb ls system/hosts
\end{DoxyVerb}


Get I\+P address of ipv4 host \char`\"{}localhost\char`\"{}\+: \begin{DoxyVerb}$ kdb get system/hosts/ipv4/localhost
\end{DoxyVerb}


Check if a comment is belonging to host \char`\"{}localhost\char`\"{}\+: \begin{DoxyVerb}$ kdb lsmeta system/hosts/ipv4/localhost
\end{DoxyVerb}


Try to change the host \char`\"{}localhost\char`\"{}, should fail because it is not an ipv4 adress\+: \begin{DoxyVerb}$ kdb set system/hosts/ipv4/localhost ::1
\end{DoxyVerb}


``` \section*{Backup-\/and-\/\+Restore\+:/examples/hosts}

sudo kdb mount --with-\/recommends hosts /examples/hosts hosts \# \section*{Create hosts file for testing}

\# \$ echo \char`\"{}127.\+0.\+0.\+1\textbackslash{}\textbackslash{}tlocalhost\char`\"{} $>$ {\ttfamily kdb file /examples/hosts} \$ echo \char`\"{}\+::1\textbackslash{}\textbackslash{}tlocalhost\char`\"{} $>$$>$ {\ttfamily kdb file /examples/hosts} \# \section*{Check the file}

\# \$ cat {\ttfamily kdb file /examples/hosts} 127.\+0.\+0.\+1 localhost \+:\+:1 localhost \# \section*{Check if the values are read correctly}

\# kdb get /examples/hosts/ipv4/localhost 127.\+0.\+0.\+1 kdb get /examples/hosts/ipv6/localhost \+:\+:1 \# \section*{Should both fail with error 51 and return 5}

\# kdb set /examples/hosts/ipv4/localhost \+:\+:1 \section*{R\+E\+T\+:5}

\section*{E\+R\+R\+O\+R\+S\+:51}

kdb set /examples/hosts/ipv6/localhost 127.\+0.\+0.\+1 \section*{R\+E\+T\+:5}

\section*{E\+R\+R\+O\+R\+S\+:51}

\# \section*{cleanup}

\# kdb rm -\/r /examples/hosts sudo kdb umount /examples/hosts ``` 