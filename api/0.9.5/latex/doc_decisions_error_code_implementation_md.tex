In the previous error concept it was very useful to generate macros as we often added new errors. The code generation itself, however, was not ideal for cross compilation.


\begin{DoxyItemize}
\item Error codes do not change very often
\end{DoxyItemize}


\begin{DoxyItemize}
\item Migrate to a more modern and easier way to generate code with our mustache system, and let this generate the (mapping) code for all compiled languages (C, C++, Java, Rust, Go). All we get out of this is the removal of {\ttfamily std\+::cout $<$$<$ ...} code from C++ but not much more.
\item Migrate to C\+Make code that generates such macros/classes. Mustache templates have to be supplied with the input data somehow. Either we have to use a custom executable that is compiled at build time. In that case we would just get rid of the {\ttfamily std\+::cout $<$$<$ ...} in the C++ code, but not much else would change. The other option is to use the default mustache executable, which is a Ruby script and therefore requires Ruby to be installed. Also kdb gen cannot be reused, since that would require compiling kdb first, which needs \hyperlink{kdberrors_8h}{kdberrors.\+h}.
\end{DoxyItemize}

Write down the few macros manually, and also manually write down exceptions for the language bindings (and also the mappings from Elektra\textquotesingle{}s internal errors to nice errors specific for the languages)

Since error codes and crucial parts Elektra\textquotesingle{}s core implementation will not change often this is the best approach with minimal effort.

The existing code will be refactored so that error macros directly call macros.

When adding a new error code, language bindings have to be adapted accordingly such as the Rust or Java Binding.


\begin{DoxyItemize}
\item Updates to error codes will imply syncing every binding. This though should happen in very rare circumstances.
\end{DoxyItemize}


\begin{DoxyItemize}
\item The current bindings stay untouched
\item Tutorial on how to write a binding should be written
\end{DoxyItemize}


\begin{DoxyItemize}
\item See also \href{https://issues.libelektra.org/2814}{\tt Issue \#2814}
\item See also \href{https://issues.libelektra.org/2871}{\tt Issue \#2871} 
\end{DoxyItemize}