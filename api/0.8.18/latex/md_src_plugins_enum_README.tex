
\begin{DoxyItemize}
\item infos = Information about the enum plugin is in keys below
\item infos/author = Thomas Waser \href{mailto:thomas.waser@libelektra.org}{\tt thomas.\+waser@libelektra.\+org}
\item infos/licence = B\+S\+D
\item infos/provides = check
\item infos/needs =
\item infos/recommends =
\item infos/placements = presetstorage
\item infos/status = productive maintained tested nodep libc nodoc
\item infos/metadata = check/enum
\item infos/description =
\end{DoxyItemize}

The Enum plugin checks string values of Keys by comparing it against a list of valid values.

\subsection*{Usage}

The plugin checks every Key in the Keyset for the Metakey {\ttfamily check/enum} containing a list with the syntax `'string1', 'string2', 'string3', ..., 'string\+N'` and compares each value with the string value of the Key. If no match is found an error is returned.

\subsection*{Example}

\begin{DoxyVerb}    kdb mount enum.ecf /example/enum enum
    kdb set user/example/enum/value middle # init to something valid
    kdb setmeta user/example/enum/value check/enum "'low', 'middle', 'high'"
    kdb set user/example/enum/value low # success
    kdb set user/example/enum/value no  # fail\end{DoxyVerb}
 