In Elektra string-\/keys are preferred. Nevertheless, binary keys, i.\+e. strings with null-\/characters embedded are possible.

Sometimes a key does not exist at all. When using \mbox{\hyperlink{group__keyset_gaa34fc43a081e6b01e4120daa6c112004}{ks\+Lookup()}} you will get a null pointer. In the C++ Binding the null pointer is wrapped in a key. Thus this is also sometimes called {\bfseries{N\+U\+LL K\+E\+YS}}.

Null values are binary values without content. They are the only keys that have the size 0.

Null values are always binary values.

Empty values point to a string that only contains a null byte.

Empty values are possible for both string and binary values. 