{\ttfamily kdb check \mbox{[}$<$plugin$>$\mbox{]}}

\subsection*{D\+E\+S\+C\+R\+I\+P\+T\+I\+O\+N}

This command is used to perform checks on the key database or an Elektra plugin.

Where the option argument, {\ttfamily plugin} is the plugin that a user wants to check. Use {\ttfamily -\/c} to pass options to that plugin. If no {\ttfamily plugin} argument is provided a check will be performed on the key database itself. Special values are returned upon exit to represent the outcome of a check.

\subsection*{O\+P\+T\+I\+O\+N\+S}


\begin{DoxyItemize}
\item {\ttfamily -\/\+H}, {\ttfamily -\/-\/help}\+: Show the man page.
\item {\ttfamily -\/\+V}, {\ttfamily -\/-\/version}\+: Print version info.
\item {\ttfamily -\/p}, {\ttfamily -\/-\/profile}=$<$profile$>$\+: Use a different kdb profile.
\item {\ttfamily -\/f}, {\ttfamily -\/-\/force}\+: The user can also use this tool to perform write tests. Please note that this can result in configuration files being changed!
\item {\ttfamily -\/v}, {\ttfamily -\/-\/verbose}\+: Explain what is happening.
\item {\ttfamily -\/c}, {\ttfamily -\/-\/plugins-\/config}=$<$pluginconfig$>$\+: Add a plugin configuration in addition to {\ttfamily /module}.
\end{DoxyItemize}

\subsection*{R\+E\+T\+U\+R\+N V\+A\+L\+U\+E\+S}

There are two different types of checks, a check on a plugin (by specifying the name of a plugin as an argument) or a check on the key database itself.

The outcome of a check on the key database is returned as an exit status. This integer represents an 8-\/bit pattern. Each bit represents a specific outcome as described below\+:


\begin{DoxyItemize}
\item 0\+: No errors (no output)
\item Bit 1\+: Warning on opening the key database.
\item Bit 2\+: Error on opening the key database.
\item Bit 3\+: Warning on getting the value of a key.
\item Bit 4\+: Error on getting the value of a key.
\item Bit 5\+: Warning on setting the value of a key. (only checked when {\ttfamily -\/f} is used)
\item Bit 6\+: Error on setting the value of a key (only checked when {\ttfamily -\/f} is used)
\item Bit 7\+: Warning on closing the key database.
\item Bit 8\+: Error on closing the key database.
\end{DoxyItemize}

So if the following number was returned {\ttfamily 9} the user could figure out more detail by considering the bits\+: {\ttfamily 00001001} The user would know that there was a warning on open and an error on get.

If a plugin name is given, checks will only be done on the given plugin. The returned values for a check on a plugin are returned as much simpler numbers.

Return values on plugin checking\+:


\begin{DoxyItemize}
\item 0\+: Everything ok. (no output)
\item 1\+: No such plugin found or plugin could not be opened.
\item 2\+: Plugin did not pass checks.
\item 3\+: Plugin has warnings.
\end{DoxyItemize}

Please report any output caused by official plugins to \href{http://git.libelektra.org/issues}{\tt http\+://git.\+libelektra.\+org/issues}.

Since the error code is a return value, it is not automatically displayed to the shell. If the user wants to have the value printed, they must do so manually (by running a command such as {\ttfamily echo \$?}).

\subsection*{E\+X\+A\+M\+P\+L\+E\+S}

To check the Key Database\+: {\ttfamily kdb check}

To check the Key Database and then print the result\+: {\ttfamily kdb check} followed by\+: {\ttfamily echo \$?}

To check the Key Database including write checks\+: {\ttfamily kdb check -\/f} Note that this type of check may change configuration files.

To check the {\ttfamily line} plugin\+: {\ttfamily kdb check line}

\subsection*{S\+E\+E A\+L\+S\+O}


\begin{DoxyItemize}
\item For an introductions into plugins, read \hyperlink{md_doc_help_elektra-plugins-framework_doc_help_elektra-plugins-framework_md}{elektra-\/plugins-\/framemwork(7)}.
\item To list all plugins use \hyperlink{md_doc_help_kdb-list_doc_help_kdb-list_md}{kdb-\/list(1)}.
\item For information on a plugin use \hyperlink{md_doc_help_kdb-info_doc_help_kdb-info_md}{kdb-\/info(1)}. 
\end{DoxyItemize}