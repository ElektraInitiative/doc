{\ttfamily kdb meta-\/show $<$key name$>$}~\newline


Where {\ttfamily key name} is the name of the key the user would like to access.

This command is used to print all metakeys and its values for a given key. A metakey is information stored in a key which describes that key.

The handling of cascading {\ttfamily key name} does not differ to {\ttfamily kdb get}. Make sure to use the namespace {\ttfamily spec}, if you want metadata from there.

This command will return the following values as an exit status\+:~\newline



\begin{DoxyItemize}
\item 0\+: No errors.
\item 1\+: Key not found. (Invalid {\ttfamily path})
\end{DoxyItemize}


\begin{DoxyItemize}
\item {\ttfamily -\/H}, {\ttfamily -\/-\/help}\+: Show the man page.
\item {\ttfamily -\/V}, {\ttfamily -\/-\/version}\+: Print version info.
\item {\ttfamily -\/p}, {\ttfamily -\/-\/profile $<$profile$>$}\+: Use a different kdb profile.
\item {\ttfamily -\/C}, {\ttfamily -\/-\/color $<$when$>$}\+: Print never/auto(default)/always colored output.
\item {\ttfamily -\/v}, {\ttfamily -\/-\/verbose}\+: Explain what is happening. Prints additional information in case of errors/warnings.
\item {\ttfamily -\/d}, {\ttfamily -\/-\/debug}\+: Give debug information. Prints additional debug information in case of errors/warnings.
\end{DoxyItemize}


\begin{DoxyCode}
# Backup-and-Restore: user:/tests/examples

# We use `dump` as storage format here
sudo kdb mount ls.ecf user:/tests/examples dump

# Create the keys we use for the examples
kdb set user:/tests/examples/kdb-meta-show test
kdb meta-set user:/tests/examples/kdb-meta-show meta1 val1
kdb meta-set user:/tests/examples/kdb-meta-show meta2 val2
kdb meta-set user:/tests/examples/kdb-meta-show meta3 val3
kdb meta-set user:/tests/examples/kdb-meta-show meta4 val4

# list all meta keys for /tests/examples/kdb-meta-show
kdb meta-show /tests/examples/kdb-meta-show
#> meta1: val1
#> meta2: val2
#> meta3: val3
#> meta4: val4

kdb rm -r user:/tests/examples
sudo kdb umount user:/tests/examples
\end{DoxyCode}



\begin{DoxyItemize}
\item How to set metadata\+: \hyperlink{doc_help_kdb-meta-set_md}{kdb-\/meta-\/set(1)}
\item For more about cascading keys see \hyperlink{doc_help_elektra-cascading_md}{elektra-\/cascading(7)}
\item \hyperlink{doc_help_elektra-metadata_md}{elektra-\/metadata(7)} for an explanation of the metadata concepts.
\item \hyperlink{doc_help_elektra-key-names_md}{elektra-\/key-\/names(7)} for an explanation of key names. 
\end{DoxyItemize}