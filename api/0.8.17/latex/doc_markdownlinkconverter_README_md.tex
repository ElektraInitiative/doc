The Markdown link Converter, which filters markdown pages before the processing of doxygen, converts the links in markdown pages. It is set up as input filter in doxygen, if a markdown file is desired to be in the A\+P\+I documentation it only must have the extension {\ttfamily .md} and be in the {\ttfamily I\+N\+P\+U\+T} path.

The Markdown link Converter gives each markdown file a header {\ttfamily \{ \#header \}} which is attached to a title and converts the links to refer to this headers. This conversion happens in 2 passes, which is needed because there can be files with no title.

\subsection*{Usage for manual invocation}

\begin{DoxyVerb}    markdownlinkconverter [<cmake-cache-file>] <input-file>
\end{DoxyVerb}


{\bfseries The $<$input-\/file$>$ parameter must be an absolute path!}

\subsection*{Conventions}


\begin{DoxyItemize}
\item Links starting with {\ttfamily @ref}, {\ttfamily \#} for anchors and {\ttfamily http} for extern links wont be touched.
\item All other links to markdown or arbitrary source files will be converted.
\item To refer to a folder use {\ttfamily /} at the end of a link. This feature was introduced to be compatible with github, where you can show the content of a folder in combination with the R\+E\+A\+D\+M\+E.\+md of the containing folder. Links ending with {\ttfamily /} will be changed to {\ttfamily /\+R\+E\+A\+D\+M\+E.md}.
\item Anchors wont work in imported markdown pages.
\end{DoxyItemize}

\subsection*{Link Validation}

The link validation works with a simple try to {\ttfamily fopen} the file, which the link refers to.

\subsection*{Further improvements (which will be introduced in a later version)\+:}


\begin{DoxyItemize}
\item optimize pdf output (also U\+T\+F-\/8 encoding)
\item if title contains --, this should be$\ast$ add http link print 
\end{DoxyItemize}