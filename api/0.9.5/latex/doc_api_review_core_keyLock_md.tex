
\begin{DoxyItemize}
\item start = 2021-\/02-\/21 21\+:45
\item end = 2021-\/02-\/21 22\+:05
\item reviewer = Stefan Hanreich \href{mailto:stefanhani@gmail.com}{\tt stefanhani@gmail.\+com}
\end{DoxyItemize}

{\ttfamily int key\+Lock (Key $\ast$ key, elektra\+Lock\+Flags what)}

(bullet points are in order of appearance)


\begin{DoxyItemize}
\item \mbox{[}x\mbox{]} First line explains briefly what the function does
\item \mbox{[} \mbox{]} Simple example or snippet how to use the function -\/ \mbox{[} \mbox{]} add
\item \mbox{[}x\mbox{]} Longer description of function containing common use cases
\item \mbox{[}x\mbox{]} Description of functions reads nicely
\item \mbox{[} \mbox{]} {\ttfamily @pre}
\begin{DoxyItemize}
\item \mbox{[} \mbox{]} key is a valid key
\item \mbox{[} \mbox{]} what contains valid elektra\+Lock\+Flags
\end{DoxyItemize}
\item \mbox{[} \mbox{]} {\ttfamily @post}
\begin{DoxyItemize}
\item \mbox{[} \mbox{]} parts of the keys, as stated in the what-\/parameter, are locked
\end{DoxyItemize}
\item \mbox{[} \mbox{]} {\ttfamily @invariant}
\begin{DoxyItemize}
\item \mbox{[} \mbox{]} key stays a valid key
\end{DoxyItemize}
\item \mbox{[} \mbox{]} {\ttfamily @param} for every parameter
\begin{DoxyItemize}
\item \mbox{[} \mbox{]} key
\item \mbox{[} \mbox{]} what
\end{DoxyItemize}
\item \mbox{[} \mbox{]} {\ttfamily @return} / {\ttfamily @retval}
\begin{DoxyItemize}
\item \mbox{[} \mbox{]} move above see also
\item \mbox{[} \mbox{]} move {\ttfamily $>$0} to default case ({\ttfamily @return})
\end{DoxyItemize}
\item \mbox{[} \mbox{]} {\ttfamily @since}
\begin{DoxyItemize}
\item \mbox{[} \mbox{]} add
\end{DoxyItemize}
\item \mbox{[} \mbox{]} ``
\begin{DoxyItemize}
\item \mbox{[} \mbox{]} add
\end{DoxyItemize}
\item \mbox{[} \mbox{]} {\ttfamily @see}
\begin{DoxyItemize}
\item \mbox{[} \mbox{]} add {\ttfamily \hyperlink{group__key_ga769882e86e34a95cefcf8f260ef97e06}{key\+Is\+Locked()}}
\end{DoxyItemize}
\end{DoxyItemize}


\begin{DoxyItemize}
\item Abbreviations used in function names must be defined in the \hyperlink{doc_help_elektra-glossary_md}{Glossary}
\item \mbox{[}x\mbox{]} Function names should neither be too long, nor too short
\item \mbox{[}x\mbox{]} Function name should be clear and unambiguous
\item \mbox{[}x\mbox{]} Abbreviations used in parameter names must be defined in the \hyperlink{doc_help_elektra-glossary_md}{Glossary}
\item \mbox{[}x\mbox{]} Parameter names should neither be too long, nor too short
\item \mbox{[} \mbox{]} Parameter names should be clear and unambiguous
\begin{DoxyItemize}
\item \mbox{[} \mbox{]} rename what to flags?
\end{DoxyItemize}
\end{DoxyItemize}

(only in P\+Rs)


\begin{DoxyItemize}
\item \hyperlink{doc_dev_symbol-versioning_md}{Symbol versioning} is correct for breaking changes
\item A\+B\+I/\+A\+PI changes are forward-\/compatible (breaking backwards-\/compatibility to add additional symbols is fine)
\end{DoxyItemize}


\begin{DoxyItemize}
\item \mbox{[}x\mbox{]} Function parameters should use enum types instead of boolean types wherever sensible
\item \mbox{[}x\mbox{]} Wherever possible, function parameters should be {\ttfamily const}
\item \mbox{[}x\mbox{]} Wherever possible, return types should be {\ttfamily const}
\item \mbox{[}x\mbox{]} Functions should have the least amount of parameters feasible
\end{DoxyItemize}


\begin{DoxyItemize}
\item \mbox{[}x\mbox{]} Functions should do exactly one thing
\item \mbox{[}x\mbox{]} Function name has the appropriate prefix
\item \mbox{[} \mbox{]} Order of signatures in kdb.\+h.\+in is the same as Doxygen
\begin{DoxyItemize}
\item \mbox{[} \mbox{]} move below {\ttfamily \hyperlink{group__key_ga4aabc4272506dd63161db2bbb42de8ae}{key\+Get\+Ref()}}-\/family of functions
\end{DoxyItemize}
\item \mbox{[}x\mbox{]} No functions with similar purpose exist
\end{DoxyItemize}


\begin{DoxyItemize}
\item \mbox{[}x\mbox{]} Memory Management should be handled by the function wherever possible
\end{DoxyItemize}


\begin{DoxyItemize}
\item \mbox{[}x\mbox{]} Function is easily extensible, e.\+g., with flags
\item \mbox{[} \mbox{]} Documentation does not impose limits, that would hinder further extensions
\begin{DoxyItemize}
\item \mbox{[} \mbox{]} keys cannot be unlocked
\end{DoxyItemize}
\end{DoxyItemize}


\begin{DoxyItemize}
\item \mbox{[}x\mbox{]} Function code is fully covered by tests
\item \mbox{[} \mbox{]} All possible error states are covered by tests
\begin{DoxyItemize}
\item \mbox{[} \mbox{]} test null pointer
\end{DoxyItemize}
\item \mbox{[}x\mbox{]} All possible enum values are covered by tests
\item \mbox{[}x\mbox{]} No inconsistencies between tests and documentation
\end{DoxyItemize}

Also check return values of function in tests