You might want to read \hyperlink{doc_dev_data-structures_md}{about data structures first}.

\subsection*{Terminology}

It is sometimes unavoidable that errors occur that ultimately have an impact for the user. One example for such an error is that hard disc space is exhausted. For a library it is necessary to pass information about the facts and circumstances to the user, because the user wants to be informed why a requested action failed. So Elektra gathers all information in these situations. We call this resulting information {\bfseries error information} or {\bfseries warning information} depending on the severity.

If the error is critical and ultimately causes a situation in which the post conditions cannot be fulfilled we say that Elektra is in a {\bfseries faulty state}. Such a faulty state will change the control flow inside Elektra completely. Elektra is unable to resolve the problem without assistance. After cleaning up resources, a faulty state leads to immediate return from the function with an {\bfseries error code}. As a user expects from a library, Elektra never calls {\ttfamily exit()} or something similar, regardless of how fatal the error is. In this situation error information should be set.

On the other hand, for many problems the work can go on with reasonable defaults. Nevertheless, the user will be warned that a problem occurred using the {\bfseries warning information}. These situations do not influence the control flow. But applications can choose to react differently in the presence of warning information. They may not be interested in any warning information at all. It is no problem if warning information is ignored because they are stored and remain accessible in a circular buffer. The implementation prevents an overflow of the buffer. Instead the oldest warnings are overwritten.

When error or warning information is presented to the user, it is called {\itshape error message} or {\itshape warning message}. The user may reply to this message in which way to continue.

\subsubsection*{Error vs. Warning Information}

When an error in a faulty state occurs, the error information must still hold the original error information. So even if a new problem would cause a faulty state otherwise, the error information must be omitted or transformed to warning information. In some places only the addition of warning information is possible\+:


\begin{DoxyItemize}
\item The main purpose of {\ttfamily \hyperlink{group__kdb_gadb54dc9fda17ee07deb9444df745c96f}{kdb\+Close()}} is to free the handle. This operation is always successful and is carried out even if some of the resources cannot be freed. Therefore, in {\ttfamily \hyperlink{group__kdb_gadb54dc9fda17ee07deb9444df745c96f}{kdb\+Close()}}, setting error information is prohibited. Warning information is, however, very useful to tell the user the circumstance that some actions during cleanup failed.
\item Also in {\ttfamily \hyperlink{group__kdb_ga6808defe5870f328dd17910aacbdc6ca}{kdb\+Open()}}, only adding warning information is allowed. If {\ttfamily \hyperlink{group__kdb_ga6808defe5870f328dd17910aacbdc6ca}{kdb\+Open()}} is not able to open a plugin, the affected backend will be dropped. The user is certainly interested why that happened. But we decided not to make this a faulty state, because the application might not even access the faulty part of the key hierarchy. An exception to this rule is if {\ttfamily \hyperlink{group__kdb_ga6808defe5870f328dd17910aacbdc6ca}{kdb\+Open()}} fails to open the default backend. This situation will induce a faulty state.
\item In {\ttfamily \hyperlink{group__kdb_ga11436b058408f83d303ca5e996832bcf}{kdb\+Set()}}, the clean-\/up of resources involves calling plugins. But during this process Elektra is in a faulty state, so only adding of warning information is allowed. This ensures that the original error information is passed unchanged to the user.
\end{DoxyItemize}

On the other hand, any access to the key database can produce warning information. Plugins are allowed to yield warning information at any place and for any reason. For example, the storage plugin reading a configuration file with an old syntax, is free to report the circumstance as warning information. Warning information is also useful when more than one fact needs to be reported.

\subsection*{Error Information}

\subsubsection*{Reporting Errors}

Reporting errors is a critical task. Users expect different aspects\+:


\begin{DoxyItemize}
\item The {\bfseries user of the application} does not want to see any error message at all. If it is inevitable, he or she wants little, but very concrete information, about what he or she needs to do. The message should be short and concise. Some error information may already be captured by the application, but others like “no more free disk space” have to be displayed. Conflicts should also be presented to the user. It is a good idea to ask how to proceed if a diversity of possible reactions exists. In case of conflicts, the user may have additional knowledge about the other program which has caused the problem. The user is more likely to decide correctly by which strategy the configuration shall be restored.
\item The {\bfseries user of the library} wants more detailed information. Categories of how severe the error is can help to decide how to proceed. Even more important is the information if it makes sense to try the same action again. If, for example, an unreliable network connection or file system is used, the same action can work in a second try.
\item A {\bfseries developer of the library} (Library refers to both Elektra’s core and plugins.) wants full information about anything needed to be able to reproduce and locate potential bugs. Ideally the error information should even mention the file and line where the error occurred. This can help developers to decide if there is a bug inside Elektra or if the problem lies somewhere else.
\item Vast information is needed to support correct error handling in {\itshape other programming languages}. In languages supporting exceptions, class name, inheritance or interface information may be necessary. Language specific extensions are, however, not limited to exceptions. Other ways of handling errors are continuations or {\ttfamily longjmp} in C. A plugin is free to add, for example, {\ttfamily jmp\+\_\+buf} information to the error information.
\end{DoxyItemize}

It is certainly not a good idea to put all this previously mentioned information into a single string. Elektra chooses another way described in the next chapter.

\subsubsection*{Metadata}

As stated above, a library always informs the user about what has happened, but does not print or log anything itself. One way to return error information is to add a parameter containing the error information. In the case of Elektra, all {\ttfamily K\+DB} methods have a key as parameter. This key is additionally passed to every plugin. The idea is to add the error and warning information as metadata to this key. This approach does not limit flexibility, because a key can hold a potentially unlimited number of metakeys.

The error information is categorised in metadata as follows\+:


\begin{DoxyItemize}
\item \mbox{[}error\mbox{]} indicates that a faulty state is present. The value of the metakey contains the name of all the subkeys that are used for error indication. Metakeys do not guarantee any particular order on iteration. Instead the user can find out the information by looking at this metavalue.
\end{DoxyItemize}

Additional metakeys yield all the details.


\begin{DoxyItemize}
\item \mbox{[}error/number\mbox{]} yields a unique number for every error.
\item \mbox{[}error/description\mbox{]} is a description for the error to be displayed to the user. For example, the metavalue can hold the text “could not write to file”.
\item \mbox{[}error/reason\mbox{]} specifies the reason of the error. The human readable message is in the metavalue of {\ttfamily error/reason}. It states why the error occurred. One example for it is \textquotesingle{}\textquotesingle{}no disc space available\textquotesingle{}\textquotesingle{}.
\item \mbox{[}error/module\mbox{]} indicates the name of the specific module or plugin.
\item \mbox{[}error/file\mbox{]} yields the source file from where the error information comes.
\item \mbox{[}error/line\mbox{]} represents the exact line of that source file.
\end{DoxyItemize}

As we see, the system is powerful because any other text or information can be added in a flexible manner by using additional metakeys.

\subsubsection*{Error Specification}

The error specification in Elektra is written in simple colon-\/separated text files. Each entry has a unique identifier and all the information we already discussed above. No part of Elektra ever reads this file directly. Instead it is used to generate source code which contains everything needed to add a particular error or warning information. With that file we achieved a central place for error-\/related information. All other locations are automatically generated instead of having error-\/prone duplicated code. This principle is called “\+Don\textquotesingle{}t repeat yourself”.

In Elektra’s core and plugins, C macros are responsible for setting the error information and adding warning information. In C only a macro is able to add the file name and line number of the source code. In language bindings other code may be generated.

\subsubsection*{Sources of Errors}

{\ttfamily Key} and {\ttfamily Key\+Set} functions cannot expose more error information than the error code they return. But, of course, errors are also possible in these functions. Typically, errors concern invalid key names or null pointers. These problems are mostly programming errors with only local effects.

The most interesting error situations, however, all occur in {\ttfamily K\+DB}. The error system described here is dedicated to the four main {\ttfamily K\+DB} functions\+: {\ttfamily \hyperlink{group__kdb_ga6808defe5870f328dd17910aacbdc6ca}{kdb\+Open()}}, {\ttfamily \hyperlink{group__kdb_ga28e385fd9cb7ccfe0b2f1ed2f62453a1}{kdb\+Get()}}, {\ttfamily \hyperlink{group__kdb_ga11436b058408f83d303ca5e996832bcf}{kdb\+Set()}} and {\ttfamily \hyperlink{group__kdb_gadb54dc9fda17ee07deb9444df745c96f}{kdb\+Close()}}. The place where the configuration is checked and made persistent is the source of most error information. At this specific place a large variety of errors can happen ranging from opening, locking up and saving the file. Sometimes in plugins, nearly every line needs to deal with an error situation.

\subsection*{Exceptions}

Exceptions are a mechanism of the language and not just an implementation detail. Exceptions are not intended to force the user to do something, but to enrich the possibilities. In this section, we discuss two issues related to exceptions. On the one hand, we will see how Elektra supports programming languages that provide exceptions. On the other hand, we will see how the research in exceptions helps Elektra to provide more robustness.

\subsubsection*{Language Bindings}

C does not know anything about exceptions. But Elektra was designed so that both plugins and applications can be written in languages that provide exceptions. One design goal of Elektra’s error system is to transport exception-\/related information in a language neutral way from the plugins to the applications. To do so, a language binding of the plugin needs to catch every exception and transform it into error information and return an error code.

Elektra recognizes the error code, stops the processing of plugins, switches to a faulty state and gives all the plugins a chance to do the necessary cleanups. The error information is passed to the application as usual. If the application is written in C or does not want to deal with exceptions for another reason, we are finished because the application gets the error information inside metadata as expected. But, if the application is written in another language, the binding translates the error code to an exception and throws it. It is worth noting that the source and target language do not need to be the same.

Such a system needs a central specification of error information. We already introduced such a specification file in error specification. The exception classes and converters can be generated from it. An exception converter is either a long sequence of try-\/catch statements that transforms every known exception into an appropriate metakeys. Each exception thrown by the plugin has to be caught. Alternatively, a converter can be a long switch statement for every error number. In every case the appropriate exception is thrown.

The motivation for using exceptions is that in C every return value has to be checked. On the other hand, the C++ exception mechanism allows the programmer to throw an exception in any place and catch it where it is needed. So in C++ the code is not cluttered with parts responsible for error handling. Instead, in a single place all exceptions of a plugin can be transformed to Elektra’s error or warning information. The code for this place can be generated automatically using an exception converter.

Applications not written in C can also benefit from an exception converter. Instead of using the metadata mechanism, the error information can be converted to the exception mechanism already used for that application. We see that Elektra is minimally invasive in this regard.

\subsubsection*{Exception Safety}

We can learn from the way languages define the semantics for {\bfseries exception safety}. Exception safety is a concept which ensures that all resources are freed regardless of the chosen return path. {\bfseries Basic guarantees} make sure that some invariants remain on an object even in exceptional cases. On the other hand, {\bfseries strong guarantees} assure that the investigated operation is successful or has no effect at all. Methods are said to be exception safe if the object remains in a valid state. The idea of exception safety is to ensure that no resource leaking is possible. {\ttfamily \hyperlink{group__kdb_ga11436b058408f83d303ca5e996832bcf}{kdb\+Set()}} written in C++ would look like\+:


\begin{DoxyCode}
\{c++\}
try \{
        plugin[1].set(); // may throw
        plugin[2].set(); // may throw
        plugin[3].set(); // may throw
        ...

        plugin[PLUGIN\_COMMIT].set(); // now all changes are committed
\} catch (...) \{
        // error situation, roll back the changes
        plugin[1].error(); // does not throw
        plugin[2].error(); // does not throw
        plugin[3].error(); // does not throw
        ...

        // now all changes are rolled back
        return -1;
\} // now do all actions on success after commit
plugin[POSTCOMMIT].set(); // does not throw
...
return 1; // commit successful
\end{DoxyCode}


This pseudo code is much clearer than the corresponding C code. Let us explain the guarantee Elektra provides using this example. One by one plugin gets its chance to process the configuration. If any plugin fails, the semantics resemble that of a thrown exception. All other plugins will not be executed. Instead, the plugins get a chance to recover from the faulty state. In this catch part, the plugins are not allowed to yield any error information, but they are allowed to add warnings.

Continue reading \hyperlink{doc_dev_algorithm_md}{with the algorithm}. 