
\begin{DoxyItemize}
\item infos = Information about Y\+A\+JL plugin is in keys below
\item infos/author = Markus Raab \href{mailto:elektra@libelektra.org}{\tt elektra@libelektra.\+org}
\item infos/licence = B\+SD
\item infos/provides = storage/json
\item infos/needs =
\item infos/recommends = rebase directoryvalue comment type
\item infos/placements = getstorage setstorage
\item infos/status = maintained coverage unittest
\item infos/description = J\+S\+ON using Y\+A\+JL
\end{DoxyItemize}

This is a plugin reading and writing J\+S\+ON files using the library \href{http://lloyd.github.com/yajl/}{\tt yail}

The plugin was tested with yajl version 1.\+0.\+8-\/1 from Debian 6 and yajl version 2.\+0.\+4-\/2 from Debian 7.

Examples of files which are used for testing can be found below the folder in \char`\"{}src/plugins/yajl/yajl\char`\"{}.

The J\+S\+ON grammar can be found \href{http://www.ietf.org/rfc/rfc4627.txt}{\tt here}.

A validator can be found \href{http://jsonlint.com/}{\tt here}.

Supports every Key\+Set except when arrays are intermixed with other keys. Has only limited support for metadata.

\subsection*{Dependencies}


\begin{DoxyItemize}
\item {\ttfamily libyajl-\/dev} (version 1 and 2 should work)
\end{DoxyItemize}

\subsection*{Types}

The type of the data is available via the metadata {\ttfamily type}\+:


\begin{DoxyItemize}
\item {\ttfamily string}\+: The J\+S\+ON string type.
\item {\ttfamily boolean}\+: The J\+S\+ON boolean type (true or false)
\item {\ttfamily double}\+: For J\+S\+ON numbers.
\end{DoxyItemize}

If no metadata {\ttfamily type} is given, the type is either\+:


\begin{DoxyItemize}
\item {\ttfamily null} on binary null-\/key
\item {\ttfamily string} otherwise
\end{DoxyItemize}

Any other type/value will still be treated as string, but the warning {\ttfamily \#78} will be added because of the potential data loss.

\subsection*{Special values}

In J\+S\+ON it is possible to have empty arrays and objects. In Elektra this is mapped using the special names \begin{DoxyVerb}###empty_array
\end{DoxyVerb}


and \begin{DoxyVerb}___empty_map
\end{DoxyVerb}


Arrays are mapped to Elektra’s array convention \#0, \#1,..

\subsection*{Restrictions}


\begin{DoxyItemize}
\item Only U\+T\+F-\/8 is supported. Use the {\ttfamily iconv} plugin if your locale are not U\+T\+F-\/8. When using non-\/\+U\+T\+F-\/8 the plugin will be able to write the file, but cannot parse it back again. You will error \#77, invalid bytes in U\+T\+F8 string.
\item Everything is string if not tagged by metakey \char`\"{}type\char`\"{} Only valid J\+S\+ON types can be used in type, otherwise there are some fall backs to string but warnings are produced.
\item Values in non-\/leaves are discarded.
\item Arrays will be normalized (to \#0, \#1, ..)
\item Comments of various J\+S\+O\+N-\/dialects are discarded.
\end{DoxyItemize}

Because of these potential problems a type checker, comments filter and directory value filter are highly recommended.

\subsection*{Usage}

The following example shows you how you can read and write data using this plugin.

``{\ttfamily  $<$h1$>$Mount the plugin to the cascading namespace}/examples/yajl` sudo kdb mount config.\+json /examples/yajl yajl

\section*{Manually add a key-\/value pair to the database}

printf \textquotesingle{}\{ \char`\"{}number\char`\"{}\+: 1337 \}\textquotesingle{} $>$ {\ttfamily kdb file /examples/yajl}

\section*{Retrieve the new value}

kdb get /examples/yajl/number \#$>$ 1337

\section*{Determine the data type of the value}

kdb getmeta /examples/yajl/number type \#$>$ double

\section*{Add another key-\/value pair}

kdb set /examples/yajl/key value \#$>$ Using name user/examples/yajl/key \#$>$ Create a new key user/examples/yajl/key with string \char`\"{}value\char`\"{}

\section*{Retrieve the new value}

kdb get /examples/yajl/key \#$>$ value

\section*{Check the format of the configuration file}

kdb file user/examples/yajl/ $\vert$ xargs cat \#$>$ \{ \#$>$ \char`\"{}key\char`\"{}\+: \char`\"{}value\char`\"{}, \#$>$ \char`\"{}number\char`\"{}\+: 1337 \#$>$ \}

\section*{Undo modifications to the database}

kdb rm -\/r /examples/yajl sudo kdb umount /examples/yajl ```

\subsection*{Open\+I\+CC Device Config}

This plugin was specifically designed and tested for the {\ttfamily Open\+I\+C\+C\+\_\+device\+\_\+config\+\_\+\+DB} although it is of course not limited to it.

Mount the plugin\+: \begin{DoxyVerb}kdb mount --resolver=resolver_fm_xhp_x color/settings/openicc-devices.json /org/freedesktop/openicc yajl rename cut=org/freedesktop/openicc
\end{DoxyVerb}


or\+: \begin{DoxyVerb}kdb mount-openicc
\end{DoxyVerb}


Then you can copy the Open\+I\+C\+C\+\_\+device\+\_\+config\+\_\+\+D\+B.\+json to systemwide or user config, e.\+g. \begin{DoxyVerb}cp src/plugins/yajl/examples/OpenICC_device_config_DB.json /etc/xdg
cp src/plugins/yajl/examples/OpenICC_device_config_DB.json ~/.config

kdb ls system/org/freedesktop/openicc
\end{DoxyVerb}


prints out then all device entries available in the config \begin{DoxyVerb}kdb get system/org/freedesktop/openicc/device/camera/0/EXIF_manufacturer
\end{DoxyVerb}


prints out \char`\"{}\+Glasshuette\char`\"{} with the example config in source

You can export the whole system openicc config to ini with\+: \begin{DoxyVerb}kdb export system/org/freedesktop/openicc simpleini > dump.ini
\end{DoxyVerb}


or import it\+: \begin{DoxyVerb}kdb import system/org/freedesktop/openicc ini < dump.ini\end{DoxyVerb}
 