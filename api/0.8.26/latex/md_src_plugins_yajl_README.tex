
\begin{DoxyItemize}
\item infos = Information about Y\+A\+JL plugin is in keys below
\item infos/author = Markus Raab \href{mailto:elektra@libelektra.org}{\tt elektra@libelektra.\+org}
\item infos/licence = B\+SD
\item infos/provides = storage/json
\item infos/needs = directoryvalue
\item infos/recommends = rebase comment type
\item infos/placements = getstorage setstorage
\item infos/status = maintained coverage unittest
\item infos/description = J\+S\+ON using Y\+A\+JL
\end{DoxyItemize}

This is a plugin reading and writing J\+S\+ON files using the library \href{http://lloyd.github.com/yajl/}{\tt yail}

The plugin was tested with yajl version 1.\+0.\+8-\/1 from Debian 6 and yajl version 2.\+0.\+4-\/2 from Debian 7.

Examples of files which are used for testing can be found below the folder in \char`\"{}src/plugins/yajl/yajl\char`\"{}.

The J\+S\+ON grammar can be found \href{http://www.ietf.org/rfc/rfc4627.txt}{\tt here}.

A validator can be found \href{http://jsonlint.com/}{\tt here}.

Supports every Key\+Set except when arrays are intermixed with other keys. Has only limited support for metadata.

\subsection*{Dependencies}


\begin{DoxyItemize}
\item {\ttfamily libyajl-\/dev} (version 1 and 2 should work)
\end{DoxyItemize}

\subsection*{Types}

The type of the data is available via the metadata {\ttfamily type}\+:


\begin{DoxyItemize}
\item {\ttfamily string}\+: The J\+S\+ON string type.
\item {\ttfamily boolean}\+: The J\+S\+ON boolean type (true or false)
\item {\ttfamily double}\+: For J\+S\+ON numbers.
\end{DoxyItemize}

If no metadata {\ttfamily type} is given, the type is either\+:


\begin{DoxyItemize}
\item {\ttfamily null} on binary null-\/key
\item {\ttfamily string} otherwise
\end{DoxyItemize}

Any other type/value will still be treated as string, but the warning {\ttfamily \#78} will be added because of the potential data loss.

\subsection*{Special values}

In J\+S\+ON it is possible to have empty arrays and objects. In Elektra this is mapped using the special names \begin{DoxyVerb}###empty_array
\end{DoxyVerb}


and \begin{DoxyVerb}___empty_map
\end{DoxyVerb}


Arrays are mapped to Elektra’s array convention \#0, \#1,..

\subsection*{Restrictions}


\begin{DoxyItemize}
\item Only U\+T\+F-\/8 is supported. Use the {\ttfamily iconv} plugin if your locale are not U\+T\+F-\/8. When using non-\/\+U\+T\+F-\/8 the plugin will be able to write the file, but cannot parse it back again. You will error \#77, invalid bytes in U\+T\+F8 string.
\item Everything is string if not tagged by metakey \char`\"{}type\char`\"{} Only valid J\+S\+ON types can be used in type, otherwise there are some fall backs to string but warnings are produced.
\item Arrays will be normalized (to \#0, \#1, ..)
\item Comments of various J\+S\+O\+N-\/dialects are discarded.
\item Mixing of arrays and maps is not detected and leads to corrupted J\+S\+ON files. Please specify arrays to avoid such situations.
\item The plugin creates adds an empty root key to the database, even if you did not add this key (see \href{http://issues.libelektra.org/2132}{\tt http\+://issues.\+libelektra.\+org/2132}).
\end{DoxyItemize}

Because of these potential problems a type checker and comments filter are highly recommended.

\subsection*{Usage}

The following example shows you how you can read and write data using this plugin.

``{\ttfamily  $<$h1$>$Mount the plugin to the cascading namespace}/tests/yajl` sudo kdb mount config.\+json /tests/yajl yajl

\section*{Manually add a key-\/value pair to the database}

printf \textquotesingle{}\{ \char`\"{}number\char`\"{}\+: 1337 \}\textquotesingle{} $>$ {\ttfamily kdb file /tests/yajl}

\section*{Retrieve the new value}

kdb get /tests/yajl/number \#$>$ 1337

\section*{Determine the data type of the value}

kdb getmeta /tests/yajl/number type \#$>$ double

\section*{Add another key-\/value pair}

kdb set /tests/yajl/key value \section*{S\+T\+D\+O\+U\+T-\/\+R\+E\+G\+EX\+: .$\ast$\+Create a new key (user$\vert$system)/tests/yajl/key with string \char`\"{}value\char`\"{}}

\section*{Retrieve the new value}

kdb get /tests/yajl/key \#$>$ value

\section*{Check the format of the configuration file}

\section*{The Directory Value plugin creates the entry}

\section*{\char`\"{}\+\_\+\+\_\+\+\_\+dirdata\char`\"{}\+: \char`\"{}\char`\"{},}

\section*{, since the plugin added an empty root key}

\section*{({\ttfamily user/tests/yajl/}).}

\section*{See also\+: \href{http://issues.libelektra.org/2132}{\tt http\+://issues.\+libelektra.\+org/2132}}

kdb file /tests/yajl/ $\vert$ xargs cat \#$>$ \{ \#$>$ \char`\"{}\+\_\+\+\_\+\+\_\+dirdata\char`\"{}\+: \char`\"{}\char`\"{}, \#$>$ \char`\"{}key\char`\"{}\+: \char`\"{}value\char`\"{}, \#$>$ \char`\"{}number\char`\"{}\+: 1337 \#$>$ \}

\section*{Add an array}

kdb set /tests/yajl/piggy/\#0 straw kdb set /tests/yajl/piggy/\#1 sticks kdb set /tests/yajl/piggy/\#2 bricks

\section*{Retrieve an array key}

kdb get /tests/yajl/piggy/\#2 \#$>$ bricks

\section*{Check the format of the configuration file}

kdb file /tests/yajl $\vert$ xargs cat \#$>$ \{ \#$>$ \char`\"{}\+\_\+\+\_\+\+\_\+dirdata\char`\"{}\+: \char`\"{}\char`\"{}, \#$>$ \char`\"{}key\char`\"{}\+: \char`\"{}value\char`\"{}, \#$>$ \char`\"{}number\char`\"{}\+: 1337, \#$>$ \char`\"{}piggy\char`\"{}\+: \mbox{[} \#$>$ \char`\"{}\+\_\+\+\_\+\+\_\+dirdata\+: \char`\"{}, \#$>$ \char`\"{}straw\char`\"{}, \#$>$ \char`\"{}sticks\char`\"{}, \#$>$ \char`\"{}bricks\char`\"{} \#$>$ \mbox{]} \#$>$ \}

\section*{Undo modifications to the database}

kdb rm -\/r /tests/yajl sudo kdb umount /tests/yajl 
\begin{DoxyCode}
### Directory Values

The YAJL plugin support values in directory keys via the [Directory Value](@ref
       src\_plugins\_directoryvalue\_README\_md) plugin.
\end{DoxyCode}
 \section*{Mount the plugin to {\ttfamily user/tests/yajl}}

sudo kdb mount config.\+json user/tests/yajl yajl

\section*{Add two directory keys and one leaf key}

kdb set user/tests/yajl/roots \textquotesingle{}Things Fall Apart\textquotesingle{} kdb set user/tests/yajl/roots/bloody \textquotesingle{}Radical Face\textquotesingle{} kdb set user/tests/yajl/roots/bloody/roots \textquotesingle{}No Roots\textquotesingle{}

\section*{Add an array containing two elements}

kdb set user/tests/yajl/now \textquotesingle{}, Now\textquotesingle{} kdb set user/tests/yajl/now/\#0 \textquotesingle{}Neighbors\textquotesingle{} kdb set user/tests/yajl/now/\#1 \textquotesingle{}Threads\textquotesingle{}

kdb ls user/tests/yajl \#$>$ user/tests/yajl \#$>$ user/tests/yajl/now \#$>$ user/tests/yajl/now/\#0 \#$>$ user/tests/yajl/now/\#1 \#$>$ user/tests/yajl/roots \#$>$ user/tests/yajl/roots/bloody \#$>$ user/tests/yajl/roots/bloody/roots

\section*{Retrieve directory values}

kdb get user/tests/yajl/roots \#$>$ Things Fall Apart kdb get user/tests/yajl/roots/bloody \#$>$ Radical Face

\section*{Retrieve leaf value}

kdb get user/tests/yajl/roots/bloody/roots \#$>$ No Roots

\section*{Check array}

kdb get user/tests/yajl/now \#$>$ , Now kdb getmeta user/tests/yajl/now array \#$>$ \#1 kdb get user/tests/yajl/now/\#0 \#$>$ Neighbors kdb get user/tests/yajl/now/\#1 \#$>$ Threads

\section*{Undo modifications to the database}

kdb rm -\/r user/tests/yajl sudo kdb umount user/tests/yajl 
\begin{DoxyCode}
## OpenICC Device Config

This plugin was specifically designed and tested for the
`OpenICC\_device\_config\_DB` although it is of course not limited
to it.

Mount the plugin:

```bash
kdb mount --resolver=resolver\_fm\_xhp\_x color/settings/openicc-devices.json \(\backslash\)
  /org/freedesktop/openicc yajl rename cut=org/freedesktop/openicc
\end{DoxyCode}


or\+:


\begin{DoxyCode}
kdb mount-openicc
\end{DoxyCode}


Then you can copy the {\ttfamily Open\+I\+C\+C\+\_\+device\+\_\+config\+\_\+\+D\+B.\+json} to systemwide or user config, e.\+g.


\begin{DoxyCode}
cp src/plugins/yajl/examples/OpenICC\_device\_config\_DB.json /etc/xdg
cp src/plugins/yajl/examples/OpenICC\_device\_config\_DB.json ~/.config

kdb ls system/org/freedesktop/openicc
\end{DoxyCode}


prints out then all device entries available in the config


\begin{DoxyCode}
kdb get system/org/freedesktop/openicc/device/camera/0/EXIF\_manufacturer
\end{DoxyCode}


prints out \char`\"{}\+Glasshuette\char`\"{} with the example config in source

You can export the whole system openicc config to ini with\+:


\begin{DoxyCode}
kdb export system/org/freedesktop/openicc simpleini > dump.ini
\end{DoxyCode}


or import it\+:


\begin{DoxyCode}
kdb import system/org/freedesktop/openicc ini < dump.ini
\end{DoxyCode}
 