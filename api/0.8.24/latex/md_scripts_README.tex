{\ttfamily kdb find-\/tools} \mbox{[}-\/h\mbox{]} \mbox{[}--warnings\mbox{]} \mbox{[}--good\mbox{]} \mbox{[}--alltags\mbox{]} \mbox{[}-\/n N\+A\+ME\mbox{]} \mbox{[}-\/a A\+U\+T\+H\+OR\mbox{]} \mbox{[}-\/d D\+A\+TE\mbox{]} \mbox{[}-\/t T\+A\+GS \mbox{[}T\+A\+GS ...\mbox{]}\mbox{]} \mbox{[}-\/b B\+R\+I\+EF\mbox{]} \mbox{[}-\/e E\+X\+E\+C\+U\+TE\mbox{]}

\subsection*{D\+E\+S\+C\+R\+I\+P\+T\+I\+ON}

If you are looking for a tool, then you have found the right tool to find tools! {\ttfamily kdb find-\/tools} provides search and list functionality for tools.

Just enter {\ttfamily kdb find-\/tools} to get a list of names, type and short description of all available tools.

If you are looking for something special, then there are two ways\+:


\begin{DoxyEnumerate}
\item Tag Search\+: Type {\ttfamily kdb find-\/tools -\/-\/alltags} to get a list of all Tags in use. Then you can search with {\ttfamily kdb -\/t \mbox{[}T\+A\+GS \mbox{[}T\+A\+GS ...\mbox{]}\mbox{]}}
\item Full Text Search\+:
\begin{DoxyItemize}
\item {\ttfamily kdb find-\/tools -\/n N\+A\+ME} to search for a script name.
\item {\ttfamily kdb find-\/tools -\/b B\+R\+I\+EF} to search for a short text.
\item {\ttfamily kdb find-\/tools -\/a A\+U\+T\+H\+OR} to search for a author.
\item {\ttfamily kdb find-\/tools -\/d D\+A\+TE} to search for a creation date.
\item {\ttfamily kdb find-\/tools -\/e E\+X\+E\+C\+U\+TE} to search for a type.
\end{DoxyItemize}
\end{DoxyEnumerate}

All methods can be combined. For example if you search all bash scripts which do some configuration work. You can type {\ttfamily kdb find-\/tools -\/t configuration -\/e bash}.

\subsection*{The right Way to add your script to the find tools}

Meta Tags as comments in the beginning of a script are parsed. Mate Tags start with an {\ttfamily @}, here is a list of all Meta Tags\+:

\tabulinesep=1mm
\begin{longtabu} spread 0pt [c]{*{2}{|X[-1]}|}
\hline
\rowcolor{\tableheadbgcolor}\textbf{ Meta\+Tag }&\textbf{ Meaning  }\\\cline{1-2}
\endfirsthead
\hline
\endfoot
\hline
\rowcolor{\tableheadbgcolor}\textbf{ Meta\+Tag }&\textbf{ Meaning  }\\\cline{1-2}
\endhead
@author &Names and Emails (in $<$$>$) of the Authors as comma separated list \\\cline{1-2}
@brief &A Short Description (One Line!) \\\cline{1-2}
@tags &Comma Separated List of Tags, there is a list of common tags below \\\cline{1-2}
@date &Date when the script was created, use D\+D.\+M\+M.\+Y\+Y\+YY as format \\\cline{1-2}
\end{longtabu}
Do not mind the \textquotesingle{}\textbackslash{}\textquotesingle{} at the beginning it is a doxygen escaping.

Beware, that these metatags should be applied at the beginning of the file (in the first 10 rows)!

\subsection*{Tags}

List of Common Tags\+:

\tabulinesep=1mm
\begin{longtabu} spread 0pt [c]{*{2}{|X[-1]}|}
\hline
\rowcolor{\tableheadbgcolor}\textbf{ @tags }&\textbf{ Description  }\\\cline{1-2}
\endfirsthead
\hline
\endfoot
\hline
\rowcolor{\tableheadbgcolor}\textbf{ @tags }&\textbf{ Description  }\\\cline{1-2}
\endhead
configure &This script is used for the build configuration \\\cline{1-2}
convert &This script is used convert things \\\cline{1-2}
generator &This script is a generator \\\cline{1-2}
creator &This script creates things \\\cline{1-2}
env &This script does some env stuff \\\cline{1-2}
mount &This script mounts things \\\cline{1-2}
reformat &This script reformats things \\\cline{1-2}
debian &Special script for debian system \\\cline{1-2}
\end{longtabu}


If you choose to add a tag to the {\ttfamily @tags} then do not forget to add it in the tags map of the {\ttfamily find-\/tools} script and in the table here.

\subsection*{Example}

\begin{DoxyVerb}        <Start of File>
        #!/usr/bin/bash
        #
        # @author Kurt Micheli <kurt.micheli@libelektra.org>
        # @brief This is a example
        # @date 01.06.2016
        # @tags configure, creator, arch\end{DoxyVerb}


\subsection*{Notes}

The Metatag System of Epydoc is used (\href{http://epydoc.sourceforge.net/manual-fields.html#module-metadata-variables}{\tt http\+://epydoc.\+sourceforge.\+net/manual-\/fields.\+html\#module-\/metadata-\/variables}) and extended with special tags. 