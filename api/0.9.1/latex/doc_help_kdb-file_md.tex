{\ttfamily kdb file $<$key name$>$}~\newline


Where {\ttfamily key name} is the name of the key to check.~\newline


This command prints which file a given key is stored in.~\newline
 While many keys are stored in a default key database file, many others are stored in any number of configuration files located all over the system.~\newline
 This tool is made to allow users to find out the file that a key is actually stored in.~\newline
 This command makes use of Elektra’s {\ttfamily resolver} plugin which the uer can learn more about by running the command {\ttfamily kdb plugin-\/info resolver}.


\begin{DoxyItemize}
\item {\ttfamily -\/H}, {\ttfamily -\/-\/help}\+: Show the man page.
\item {\ttfamily -\/V}, {\ttfamily -\/-\/version}\+: Print version info.
\item {\ttfamily -\/p}, {\ttfamily -\/-\/profile $<$profile$>$}\+: Use a different kdb profile.
\item {\ttfamily -\/C}, {\ttfamily -\/-\/color $<$when$>$}\+: Print never/auto(default)/always colored output.
\item {\ttfamily -\/n}, {\ttfamily -\/-\/no-\/newline}\+: Suppress the newline at the end of the output.
\item {\ttfamily -\/N}, {\ttfamily -\/-\/namespace $<$namespace$>$}\+: Specify the namespace to use when writing cascading keys.
\item {\ttfamily -\/v}, {\ttfamily -\/-\/verbose}\+: Explain what is happening. Prints additional information in case of errors/warnings.
\item {\ttfamily -\/d}, {\ttfamily -\/-\/debug}\+: Give debug information. Prints additional debug information in case of errors/warnings.
\end{DoxyItemize}


\begin{DoxyItemize}
\item {\ttfamily /sw/elektra/kdb/\#0/current/namespace}\+: Specifies which default namespace should be used when setting a cascading name. By default it is {\ttfamily user}, except if you are root, then it is {\ttfamily system}.
\end{DoxyItemize}

To find which file a key is stored in\+:~\newline
 {\ttfamily kdb file user/example/key}~\newline



\begin{DoxyItemize}
\item \mbox{\hyperlink{doc_help_elektra-mounting_md}{elektra-\/mounting(7)}}
\item \mbox{\hyperlink{doc_help_elektra-namespaces_md}{elektra-\/namespaces(7)}}
\item \mbox{\hyperlink{doc_help_elektra-key-names_md}{elektra-\/key-\/names(7)}} for an explanation of key names. 
\end{DoxyItemize}