
\begin{DoxyItemize}
\item infos = Information about Y\+A\+I\+L plugin is in keys below
\item infos/author = Markus Raab \href{mailto:elektra@libelektra.org}{\tt elektra@libelektra.\+org}
\item infos/licence = B\+S\+D
\item infos/needs =
\item infos/provides = storage
\item infos/placements = getstorage setstorage
\item infos/recommends = rebase directoryvalue comment type
\item infos/description = J\+S\+O\+N using Y\+A\+I\+L
\end{DoxyItemize}

This is a plugin reading and writing json files using the library \href{http://lloyd.github.com/yajl/}{\tt yail}

The plugin was tested with yajl version 1.\+0.\+8-\/1 from Debian 6 and yajl version 2.\+0.\+4-\/2 from Debian 7.

Examples of files which are used for testing can be found below the folder in \char`\"{}src/plugins/yajl/examples\char`\"{}.

The json grammar can be found \href{http://www.ietf.org/rfc/rfc4627.txt}{\tt here}.

A validator can be found \href{http://jsonlint.com/}{\tt here}.

Supports every Key\+Set except when arrays are intermixed with other keys. Has only limited support for metadata.

\subsection*{Dependencies}


\begin{DoxyItemize}
\item {\ttfamily libyajl-\/dev} (version 1 and 2 should work)
\end{DoxyItemize}

\subsection*{Special values}

In json it is possible to have empty arrays and objects. In Elektra this is mapped using the special names \begin{DoxyVerb}    ###empty_array
\end{DoxyVerb}


and \begin{DoxyVerb}    ___empty_map
\end{DoxyVerb}


Arrays are mapped to Elektra's array convention \#0, \#1,..

\subsection*{Restrictions}


\begin{DoxyItemize}
\item Everything is string if not tagged by meta key \char`\"{}type\char`\"{} Only valid json types can be used in type, otherwise there are some fall backs to string but warnings are produced.
\item Values in non-\/leaves are discarded.
\item Arrays will be normalized (to \#0, \#1, ..)
\item Comments of various json-\/dialects are discarded.
\end{DoxyItemize}

Because of these potential problems a type checker, comments filter and directory value filter are highly recommended.

\subsection*{Open\+I\+C\+C Device Config}

This plugin was specifically designed and tested for the {\ttfamily Open\+I\+C\+C\+\_\+device\+\_\+config\+\_\+\+D\+B} although it is of course not limited to it.

Mount the plugin\+: \begin{DoxyVerb}    kdb mount --resolver=resolver_fm_xhp_x color/settings/openicc-devices.json /org/freedesktop/openicc yajl rename cut=org/freedesktop/openicc
\end{DoxyVerb}


or\+: \begin{DoxyVerb}    kdb mount-openicc
\end{DoxyVerb}


Then you can copy the Open\+I\+C\+C\+\_\+device\+\_\+config\+\_\+\+D\+B.\+json to systemwide or user config, e.\+g. \begin{DoxyVerb}    cp src/plugins/yajl/examples/OpenICC_device_config_DB.json /etc/xdg
    cp src/plugins/yajl/examples/OpenICC_device_config_DB.json ~/.config

    kdb ls system/org/freedesktop/openicc
\end{DoxyVerb}


prints out then all device entries available in the config \begin{DoxyVerb}    kdb get system/org/freedesktop/openicc/device/camera/0/EXIF_manufacturer
\end{DoxyVerb}


prints out \char`\"{}\+Glasshuette\char`\"{} with the example config in souce

You can export the whole system openicc config to ini with\+: \begin{DoxyVerb}    kdb export system/org/freedesktop/openicc simpleini > dump.ini
\end{DoxyVerb}


or import it\+: \begin{DoxyVerb}    kdb import system/org/freedesktop/openicc ini < dump.ini\end{DoxyVerb}
 