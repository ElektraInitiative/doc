
\begin{DoxyItemize}
\item infos = Information about the lua plugin is in keys below
\item infos/author = Manuel Mausz \href{mailto:manuel-elektra@mausz.at}{\tt manuel-\/elektra@mausz.\+at}
\item infos/licence = B\+SD
\item infos/provides =
\item infos/needs =
\item infos/placements = getstorage setstorage
\item infos/status = unittest configurable global memleak
\item infos/description = proxy that calls other plugins (scripts) written in lua
\end{DoxyItemize}

The plugin uses Lua to do magic things. It basically allows to call plugins written in Lua.

What a Lua script can do is not really limited by design, so any kind of plugin may be implemented. The lua plugin is especially useful to write filter and logging scripts.

The lua plugin accepts only the {\bfseries script} configuration parameter holding the path to a Lua script. The mount command would look like


\begin{DoxyCode}
kdb mount file.ini /lua ini lua script=/path/to/filter\_script.lua
\end{DoxyCode}


if the {\bfseries ini} plugin should be used for storage and the lua plugin only serves to invoke the filter script.

For a Lua script that serves as (J\+S\+ON) storage plugin itself, one could also use


\begin{DoxyCode}
kdb mount file.json /lua lua script=/path/to/json\_plugin.lua
\end{DoxyCode}


Lua scripts can implement the following functions


\begin{DoxyItemize}
\item elektra\+Open(config, error\+Key)
\item elektra\+Get(returned, parent\+Key)
\item elektra\+Set(returned, parent\+Key)
\item elektra\+Error(returned, parent\+Key)
\item elektra\+Close(error\+Key)
\end{DoxyItemize}

where {\itshape config} \& {\itshape returned} are Key\+Sets and {\itshape error\+Key} \& {\itshape parent\+Key} are Keys. For the return codes of the functions, the same rules as for normal plugins apply.

If a function is not available, it simply is not called. A script does not have to implement all functions therefore.

Access to {\bfseries kdb} can be retrieved using the Lua import


\begin{DoxyCode}
require("kdb")
\end{DoxyCode}


An example script that prints some information for each method call would be\+:


\begin{DoxyCode}
function elektraOpen(config, errorKey)
  print("Lua script method 'elektraOpen' called")
  return 0
end

function elektraGet(returned, parentKey)
  print("Lua script method 'elektraGet' called")
  return 1
end

function elektraSet(returned, parentKey)
  print("Lua script method 'elektraSet' called")
  return 1
end

function elektraError(returned, parentKey)
  print("Lua script method 'elektraError' called")
  return 1
end

function elektraClose(errorKey)
  print("Lua script method 'elektraClose' called")
  return 0
end
\end{DoxyCode}


Further examples can be found in the lua directory.

Note, this is a technical preview. It might have severe bugs and the A\+PI might change in the future.

Be also aware that a Lua script will never be as performant as a native C/\+C++ plugin. Spinning up the interpreter takes additional time and resources. 