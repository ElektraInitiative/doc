
\begin{DoxyItemize}
\item start = 2021-\/02-\/14 02\+:12
\item end = 2021-\/02-\/14 02\+:44
\item reviewer = Stefan Hanreich \href{mailto:stefanhani@gmail.com}{\tt stefanhani@gmail.\+com}
\end{DoxyItemize}

{\ttfamily const char $\ast$key\+Name(const Key $\ast$key)}

(bullet points are in order of appearance)


\begin{DoxyItemize}
\item \mbox{[}x\mbox{]} First line explains briefly what the function does
\item \mbox{[}x\mbox{]} Simple example or snippet how to use the function
\item \mbox{[} \mbox{]} Longer description of function containing common use cases
\item \mbox{[} \mbox{]} Description of functions reads nicely
\item \mbox{[} \mbox{]} {\ttfamily @pre}
\begin{DoxyItemize}
\item \mbox{[} \mbox{]} \begin{DoxyPrecond}{Precondition}
key must be a valid key
\end{DoxyPrecond}

\end{DoxyItemize}
\item \mbox{[} \mbox{]} {\ttfamily @post}
\begin{DoxyItemize}
\item \mbox{[} \mbox{]} \begin{DoxyPostcond}{Postcondition}
Pointer to key that can change over time
\end{DoxyPostcond}

\end{DoxyItemize}
\item \mbox{[} \mbox{]} {\ttfamily @invariant}
\begin{DoxyItemize}
\item \mbox{[} \mbox{]} key name stays valid
\end{DoxyItemize}
\item \mbox{[}x\mbox{]} {\ttfamily @param} for every parameter
\item \mbox{[} \mbox{]} {\ttfamily @return} / {\ttfamily @retval}
\begin{DoxyItemize}
\item \mbox{[} \mbox{]} add default return value to {\ttfamily @retval}
\end{DoxyItemize}
\item \mbox{[} \mbox{]} {\ttfamily @since}
\begin{DoxyItemize}
\item \mbox{[} \mbox{]} add
\end{DoxyItemize}
\item \mbox{[}x\mbox{]} `{\ttfamily }
\item {\ttfamily \mbox{[}x\mbox{]}}\begin{DoxySeeAlso}{See also}
`
\end{DoxySeeAlso}

\end{DoxyItemize}


\begin{DoxyItemize}
\item \mbox{[}x\mbox{]} Abbreviations used in function names must be defined in the \hyperlink{doc_help_elektra-glossary_md}{Glossary}
\item \mbox{[}x\mbox{]} Function names should neither be too long, nor too short
\item \mbox{[} \mbox{]} Function name should be clear and unambiguous
\begin{DoxyItemize}
\item \mbox{[} \mbox{]} Function name might lead to confusions with {\ttfamily \hyperlink{group__keyname_gab29a850168d9b31c9529e90cf9ab68be}{key\+Get\+Name()}}, make their intentions more clear in their names
\end{DoxyItemize}
\item \mbox{[}x\mbox{]} Abbreviations used in parameter names must be defined in the \hyperlink{doc_help_elektra-glossary_md}{Glossary}
\item \mbox{[}x\mbox{]} Parameter names should neither be too long, nor too short
\item \mbox{[}x\mbox{]} Parameter names should be clear and unambiguous
\end{DoxyItemize}

(only in P\+Rs)


\begin{DoxyItemize}
\item \hyperlink{doc_dev_symbol-versioning_md}{Symbol versioning} is correct for breaking changes
\item A\+B\+I/\+A\+PI changes are forward-\/compatible (breaking backwards-\/compatibility to add additional symbols is fine)
\end{DoxyItemize}


\begin{DoxyItemize}
\item Function parameters should use enum types instead of boolean types wherever sensible
\item \mbox{[}x\mbox{]} Wherever possible, function parameters should be {\ttfamily const}
\item \mbox{[}x\mbox{]} Wherever possible, return types should be {\ttfamily const}
\item \mbox{[}x\mbox{]} Functions should have the least amount of parameters feasible
\end{DoxyItemize}


\begin{DoxyItemize}
\item \mbox{[}x\mbox{]} Functions should do exactly one thing
\item \mbox{[}x\mbox{]} Function name has the appropriate prefix
\item \mbox{[}x\mbox{]} Order of signatures in kdb.\+h.\+in is the same as Doxygen
\item \mbox{[}x\mbox{]} No functions with similar purpose exist
\begin{DoxyItemize}
\item \mbox{[} \mbox{]} {\ttfamily key\+Unescaped\+Name} might be considered too similar
\end{DoxyItemize}
\end{DoxyItemize}


\begin{DoxyItemize}
\item \mbox{[}x\mbox{]} Memory Management should be handled by the function wherever possible
\end{DoxyItemize}


\begin{DoxyItemize}
\item \mbox{[}x\mbox{]} Function is easily extensible, e.\+g., with flags
\item \mbox{[}x\mbox{]} Documentation does not impose limits, that would hinder further extensions
\end{DoxyItemize}


\begin{DoxyItemize}
\item \mbox{[}x\mbox{]} Function code is fully covered by tests
\item \mbox{[} \mbox{]} All possible error states are covered by tests
\begin{DoxyItemize}
\item \mbox{[} \mbox{]} empty key\+Name is not covered in tests
\end{DoxyItemize}
\item All possible enum values are covered by tests
\item \mbox{[}x\mbox{]} No inconsistencies between tests and documentation
\end{DoxyItemize}