
\begin{DoxyItemize}
\item start = 2021-\/02-\/14 02\+:55
\item end = 2021-\/02-\/14 03\+:22
\item reviewer = Stefan Hanreich \href{mailto:stefanhani@gmail.com}{\tt stefanhani@gmail.\+com}
\end{DoxyItemize}

{\ttfamily Key $\ast$key\+New(const char $\ast$keyname, ...)}

(bullet points are in order of appearance)


\begin{DoxyItemize}
\item \mbox{[}x\mbox{]} First line explains briefly what the function does
\item \mbox{[}x\mbox{]} Simple example or snippet how to use the function
\item \mbox{[} \mbox{]} Longer description of function containing common use cases
\begin{DoxyItemize}
\item \mbox{[} \mbox{]} is 0 a valid name? returns N\+U\+LL if 0 is param
\end{DoxyItemize}
\item \mbox{[}x\mbox{]} Description of functions reads nicely
\item \mbox{[} \mbox{]} {\ttfamily @pre}
\begin{DoxyItemize}
\item \mbox{[} \mbox{]} \begin{DoxyPrecond}{Precondition}
name must be a valid key name
\end{DoxyPrecond}

\end{DoxyItemize}
\item \mbox{[} \mbox{]} {\ttfamily @post}
\begin{DoxyItemize}
\item \mbox{[} \mbox{]} \begin{DoxyPostcond}{Postcondition}
returns a valid key object
\end{DoxyPostcond}

\end{DoxyItemize}
\item \mbox{[} \mbox{]} {\ttfamily @invariant}
\begin{DoxyItemize}
\item \mbox{[} \mbox{]} add
\end{DoxyItemize}
\item \mbox{[} \mbox{]} {\ttfamily @param} for every parameter
\begin{DoxyItemize}
\item \mbox{[} \mbox{]} add {\ttfamily @param} for {\ttfamily ...}
\end{DoxyItemize}
\item \mbox{[}x\mbox{]} {\ttfamily @return} / {\ttfamily @retval}
\item \mbox{[} \mbox{]} {\ttfamily @since}
\begin{DoxyItemize}
\item \mbox{[} \mbox{]} add
\end{DoxyItemize}
\item \mbox{[}x\mbox{]} `{\ttfamily }
\item {\ttfamily \mbox{[}x\mbox{]}}\begin{DoxySeeAlso}{See also}
`
\end{DoxySeeAlso}

\end{DoxyItemize}


\begin{DoxyItemize}
\item \mbox{[}x\mbox{]} Abbreviations used in function names must be defined in the \hyperlink{doc_help_elektra-glossary_md}{Glossary}
\item \mbox{[}x\mbox{]} Function names should neither be too long, nor too short
\item \mbox{[}x\mbox{]} Function name should be clear and unambiguous
\item \mbox{[}x\mbox{]} Abbreviations used in parameter names must be defined in the \hyperlink{doc_help_elektra-glossary_md}{Glossary}
\item \mbox{[}x\mbox{]} Parameter names should neither be too long, nor too short
\item \mbox{[}x\mbox{]} Parameter names should be clear and unambiguous
\end{DoxyItemize}

(only in P\+Rs)


\begin{DoxyItemize}
\item \hyperlink{doc_dev_symbol-versioning_md}{Symbol versioning} is correct for breaking changes
\item A\+B\+I/\+A\+PI changes are forward-\/compatible (breaking backwards-\/compatibility to add additional symbols is fine)
\end{DoxyItemize}


\begin{DoxyItemize}
\item \mbox{[}x\mbox{]} Function parameters should use enum types instead of boolean types wherever sensible
\item \mbox{[}x\mbox{]} Wherever possible, function parameters should be {\ttfamily const}
\item \mbox{[}x\mbox{]} Wherever possible, return types should be {\ttfamily const}
\item \mbox{[}x\mbox{]} Functions should have the least amount of parameters feasible
\end{DoxyItemize}


\begin{DoxyItemize}
\item \mbox{[}x\mbox{]} Functions should do exactly one thing
\begin{DoxyItemize}
\item it doesn\textquotesingle{}t technically -\/ but its a convenience function
\end{DoxyItemize}
\item \mbox{[}x\mbox{]} Function name has the appropriate prefix
\item \mbox{[}x\mbox{]} Order of signatures in kdb.\+h.\+in is the same as Doxygen
\item \mbox{[}x\mbox{]} No functions with similar purpose exist
\end{DoxyItemize}


\begin{DoxyItemize}
\item \mbox{[}x\mbox{]} Memory Management should be handled by the function wherever possible
\end{DoxyItemize}


\begin{DoxyItemize}
\item \mbox{[}x\mbox{]} Function is easily extensible, e.\+g., with flags
\item \mbox{[} \mbox{]} Documentation does not impose limits, that would hinder further extensions
\begin{DoxyItemize}
\item \mbox{[} \mbox{]} Docs say {\ttfamily Key $\ast$k = key\+New(0);} has same effect as {\ttfamily Key $\ast$k =key\+New(\char`\"{}\char`\"{}, K\+E\+Y\+\_\+\+E\+ND);}
\end{DoxyItemize}
\end{DoxyItemize}


\begin{DoxyItemize}
\item \mbox{[}x\mbox{]} Function code is fully covered by tests
\item \mbox{[}x\mbox{]} All possible error states are covered by tests
\item \mbox{[} \mbox{]} All possible enum values are covered by tests
\begin{DoxyItemize}
\item \mbox{[} \mbox{]} K\+E\+Y\+\_\+\+M\+E\+TA
\item \mbox{[} \mbox{]} K\+E\+Y\+\_\+\+F\+L\+A\+GS
\end{DoxyItemize}
\item \mbox{[} \mbox{]} No inconsistencies between tests and documentation
\begin{DoxyItemize}
\item \mbox{[} \mbox{]} Documentation says i can work with {\ttfamily Key $\ast$k =key\+New(\char`\"{}\char`\"{}, K\+E\+Y\+\_\+\+E\+ND);} Tests say {\ttfamily k = key\+New (\char`\"{}\char`\"{}, K\+E\+Y\+\_\+\+E\+ND); succeed\+\_\+if (k == N\+U\+LL, \char`\"{}should be invalid\char`\"{}); key\+Del (k);}
\item \mbox{[} \mbox{]} same as above with {\ttfamily key\+New(0)}
\end{DoxyItemize}
\end{DoxyItemize}