
\begin{DoxyItemize}
\item guid\+: 3c00a5f1-\/c017-\/4555-\/92b5-\/a2cf6e0803e3
\item author\+: Markus Raab
\item pub\+Date\+: Thu, 17 Sep 2015 17\+:32\+:16 +0200
\item short\+Desc\+: adds elektrify-\/getenv \& other improvements
\end{DoxyItemize}

Again we managed to release with many new features, many fixes and also other improvements.

\subsection*{Elektrify-\/getenv}

getenv(3) is one of the most popular ways to retrieve configuration, even though it has many known problems\+:


\begin{DoxyItemize}
\item no standard way to modify it
\item relogin (or restart of shell) necessary
\item names are flat (no hierarchical structure)
\item cannot be set for individual applications
\item different in at, cron and similar scripts
\end{DoxyItemize}

With elektrify-\/getenv we wrote a solution which solves most of the problems. We use the L\+D\+\_\+\+P\+R\+E\+L\+O\+AD technique to {\itshape additionally} retrieve values from Elektra, and not only the environment.

You simply can do\+:


\begin{DoxyCode}
kdb set user/env/override/HTTP\_PROXY "http://my.proxy:8080"
\end{DoxyCode}


This will set the {\ttfamily H\+T\+T\+P\+\_\+\+P\+R\+O\+XY} environment variable to {\ttfamily \href{http://my.proxy:8080}{\tt http\+://my.\+proxy\+:8080}}. Configuration can be retrieved with {\ttfamily kdb get}\+:


\begin{DoxyCode}
kdb get user/env/override/HTTP\_PROXY
lynx   # or start another www-browser with the newly set HTTP\_PROXY
\end{DoxyCode}


Or using the man pages\+: \begin{DoxyVerb}kdb elektrify-getenv man man --elektra:MANWIDTH=40
\end{DoxyVerb}


Will use M\+A\+N\+W\+I\+D\+TH 40 for this invocation of man man. This feature is handy, if an option is only available by environment, but not by command-\/line arguments, because sometimes environment variables are not trivial to set (e.\+g. in Makefiles).

Some more examples\+: \begin{DoxyVerb}kdb set user/env/override/MANOPT -- "--regex -LC"
kdb elektrify-getenv getenv MANOPT   # to check if it is set as expected
kdb getenv MANOPT   # if /etc/ld.so.preload is active
\end{DoxyVerb}


So is this the final solution for configuration and manual elektrification of applications is not needed anymore?

The answer is\+: no and yes.

It is quite satisfactory for configuration that is inherently sharable (not different from one application to another) {\itshape and} needs the environment semantics, i.\+e. some subprocesses should have different configuration than others, e.\+g. in a specific terminal.

But it might not be a good solution for your own application, because libgetenv(3) implies many architectural decision, that other elektrified applications would decide differently, e.\+g.\+:


\begin{DoxyItemize}
\item it uses global variables (getenv(3) has no handle)
\item it uses mutex for multi-\/threading safety
\item the A\+PI getenv(3) only returns {\ttfamily char$\ast$} and has no support for other data types
\end{DoxyItemize}

For more information see https\+://git.libelektra.\+org/blob/master/src/libs/getenv/\+R\+E\+A\+D\+ME.md \char`\"{}src/libgetenv/\+R\+E\+A\+D\+M\+E.\+md\char`\"{}

\subsection*{Compatibility}

As always, the A\+PI and A\+PI is fully forward-\/compatible, i.\+e. programs compiled against an older 0.\+8 versions of Elektra will continue to work.

Because {\ttfamily key\+Unescaped\+Name} and {\ttfamily key\+Get\+Unescaped\+Name\+Size} is added in this release, it is not backward-\/compatible, i.\+e. programs compiled against 0.\+8.\+13, might {\itshape not} work with older 0.\+8 libraries.

The function {\ttfamily key\+Unescaped\+Name} provides access to an unescaped name, i.\+e. one where {\ttfamily /} and {\ttfamily \textbackslash{}\textbackslash{}} are literal symbols and do not have any special meaning. {\ttfamily N\+U\+LL} characters are used as path separators. This function makes it trivial and efficient to iterate over all path names, as already exploited in all bindings\+:


\begin{DoxyItemize}
\item \href{https://git.libelektra.org/blob/master/src/bindings/jna/HelloElektra.java}{\tt jna (java)}
\item \href{https://git.libelektra.org/blob/master/src/bindings/swig/lua/tests/test_key.lua}{\tt lua}
\item \href{https://git.libelektra.org/blob/master/src/bindings/swig/python2/tests/testpy2_key.py}{\tt python2}
\item \href{https://git.libelektra.org/blob/master/src/bindings/swig/python/tests/test_key.py}{\tt python3}
\end{DoxyItemize}

Other small changes/additions in bindings\+:


\begin{DoxyItemize}
\item fix key constructor, thanks to Manuel Mausz
\item add copy and deepcopy in python (+examples,+testcases), thanks to Manuel Mausz
\item dup() in python3 returned wrong type (S\+W\+IG wrapper), thanks to Toscano Pino for reporting, thanks to Manuel Mausz for fixing it
\end{DoxyItemize}

Doxygen 1.\+8.\+8 is preferred and the configfile was updated to this version.

The symbols of nickel (for the ni plugin) do not longer leak from the Elektra library. As such, old versions of testmod\+\_\+ni won\textquotesingle{}t work with Elektra 0.\+8.\+13. A version-\/script is now in use to only export following symbols\+:


\begin{DoxyItemize}
\item kdb$\ast$
\item key$\ast$
\item ks$\ast$
\item libelektra$\ast$ for module loading system
\item elektra$\ast$ for proposed and other functions (no A\+B\+I/\+A\+PI compatibility here!)
\end{DoxyItemize}

In this release, E\+N\+A\+B\+L\+E\+\_\+\+C\+X\+X11 was changed to {\ttfamily ON} by default.

Note that in the next release 0.\+8.\+14 there will be two changes\+:


\begin{DoxyItemize}
\item According to \href{https://git.libelektra.org/issues/262}{\tt issue \#262}, we plan to remove the option E\+N\+A\+B\+L\+E\+\_\+\+C\+X\+X11 and require the compiler to be C++11 compatible. If you have any system you are not able to build Elektra with -\/\+D\+E\+N\+A\+B\+L\+E\+\_\+\+C\+X\+X11=ON (which is the default for 0.\+8.\+13) please report that immediately.
\item the python3 bindings will be renamed to python
\end{DoxyItemize}

By not having to care for pre-\/\+C++11 compilers, we hope to attract more developers. The core part is still in C99 so that Elektra can be used on systems where libc++ is not available. Many new plugins are still written in C99, also with the purpose of not depending on C++.

\subsection*{Python Plugins}

A technical preview of \href{https://git.libelektra.org/blob/master/src/plugins/python}{\tt python3} and \href{https://git.libelektra.org/blob/master/src/plugins/python2}{\tt python2} plugins has been added.

With them its possible to write any plugin with python scripts.

Note, they are a technical preview. They might have severe bugs and the A\+PI might change in the future. Nevertheless, it is already possible to, e.\+g. develop storage plugins with it.

They are not included in {\ttfamily A\+LL} plugins. To use it, you have to specify it\+: \begin{DoxyVerb}-PLUGINS="ALL;python;python2"
\end{DoxyVerb}


Thanks to Manuel Mausz for to this work on the plugins and the patience in all the last minute fixes!

\subsection*{Qt-\/gui 0.\+0.\+8}

The G\+UI was improved and the most annoying bugs are fixed\+:


\begin{DoxyItemize}
\item only reload and write config files if something has changed
\item use merging in a way that only a conflict free merge will be written, thanks to Felix Berlakovich
\item made sure keys can only be renamed if the new name/value/metadata is different from the existing ones
\item fixed 1) and 2) of \#233
\item fixed \#235
\item fixed qml warning when deleting key
\item fixed qml typerror when accepting an edit
\end{DoxyItemize}

A big thanks to Raffael Pancheri!

\subsection*{K\+DB Tool}

The commandline tool {\ttfamily kdb} also got some improvements. Most noteworthy is that {\ttfamily kdb get -\/v} now gives a complete trace for every key that was tried. This is very handy if you have a complex specification with many fallback and override links.

It also shows default values and warnings in the case of context-\/oriented features.

Furthermore\+:


\begin{DoxyItemize}
\item Add {\ttfamily -\/v} for setmeta
\item Copy will warn when it won\textquotesingle{}t overwrite another key (behaviour did not change)
\item improve help text, thanks to Ian Donnelly
\end{DoxyItemize}

\subsection*{Documentation Initiative}

As Michael Haberler from \href{http://www.machinekit.io/}{\tt machinekit} pointed out it was certainly not easy for someone to get started with Elektra. With the documentation initiative we are going to change that.


\begin{DoxyItemize}
\item The discussion in \href{https://git.libelektra.org/issues}{\tt github issues} should clarify many things
\item We start writing man pages in ronn-\/format(7), thanks to Ian Donnelly for current work
\item Kurt Micheli is woring on improved doxygen docu + pdf generation
\item Daniel Bugl already restructed the main page
\item Daniel Bugl also improved formatting
\item doc\+: use 
\begin{DoxyRetVals}{Return values}
{\em more,thanks} & to Pino Toscano\\
\hline
\end{DoxyRetVals}

\item doxygen\+: fix template to use {\ttfamily @} and not {\ttfamily \textbackslash{}\textbackslash{}}.
\item S\+VG logo is preferred, thanks to Daniel Bugl
\item doc\+: use 
\begin{DoxyRetVals}{Return values}
{\em more,thanks} & to Pino Toscano\\
\hline
\end{DoxyRetVals}

\item many typo fixes, thanks to Pino Toscano
\item fix broken links, thanks to Manuel Mausz, Daniel Bugl and Michael Haberler
\end{DoxyItemize}

Any further help is very welcome! This call is especially addressed to beginners in Elektra because they obviously know best which documentation is lacking and what they would need.

\subsection*{Portability}

{\ttfamily kdb-\/full} and {\ttfamily kdb-\/static} work fine now for Windows 64bit, thanks to Manuel Mausz. The wresolver is now more relaxed with unset environment.

All issues for mac\+OS were resolved. With the exception of elektrify-\/getenv everything should work now, thanks to Mihael Pranjic\+:


\begin{DoxyItemize}
\item fix mktemp
\item testscripts
\item recursive mutex simplification
\item clearenv ifdef
\end{DoxyItemize}

and thanks to Daniel Bugl\+:


\begin{DoxyItemize}
\item R\+P\+A\+TH fixed, so that {\ttfamily kdb} works
\end{DoxyItemize}

furthermore\+:


\begin{DoxyItemize}
\item fix {\ttfamily \+\_\+\+\_\+\+F\+U\+N\+C\+T\+I\+O\+N\+\_\+\+\_\+} to {\ttfamily \+\_\+\+\_\+func\+\_\+\+\_\+} (C99), thanks to Pino Toscano
\item avoid compilation error when J\+N\+I\+\_\+\+V\+E\+R\+S\+I\+O\+N\+\_\+1\+\_\+8 is missing
\item fix (twice, because of an accidental revert) the T\+A\+R\+G\+E\+T\+\_\+\+C\+M\+A\+K\+E\+\_\+\+F\+O\+L\+D\+ER, thanks to Pino Toscano
\end{DoxyItemize}

Thanks to Manuel Mausz for to testing and improving portability!

\subsection*{Packaging and Build System}


\begin{DoxyItemize}
\item \href{https://packages.qa.debian.org/e/elektra/news/20150726T155000Z.html}{\tt 0.\+8.\+12 packaged+migrated to testing}, thanks to Pino Toscano
\item fix build with external gtest, thanks to Pino Toscano
\item switch from Find\+Elektra.\+cmake to Elektra\+Config.\+cmake, thanks to Pino Toscano
\item use {\ttfamily cmake\+\_\+parse\+\_\+arguments} instead of {\ttfamily parse\+\_\+arguments}, thanks to Manuel Mausz
\end{DoxyItemize}

\subsection*{Further Fixes}


\begin{DoxyItemize}
\item Key\+::release() will also work when Key holds a null-\/pointer
\item Key\+::get\+Name() avoids std\+::string exception
\item support for copy module was introduced, thanks to Manuel Mausz
\item be more P\+O\+S\+IX compatible in shell scripts ({\ttfamily type} to {\ttfamily command -\/v} and avoid {\ttfamily echo -\/e}) thanks to Pino Toscano
\item fix vararg type for K\+E\+Y\+\_\+\+F\+L\+A\+GS, thanks to Pino Toscano
\item fix crash of example, thanks to Pino Toscano
\item add proper licence file for Modules (C\+O\+P\+Y\+I\+N\+G-\/\+C\+M\+A\+K\+E-\/\+S\+C\+R\+I\+P\+TS), thanks to Pino Toscano
\item fix X\+DG resolver issue when no given path in X\+D\+G\+\_\+\+C\+O\+N\+F\+I\+G\+\_\+\+D\+I\+RS is valid
\item make dbus example work again
\item fix compiler warnings for gcc and clang
\item fix Valgrind suppressions
\item Installation of GI binding is fixed, thanks to Dāvis
\item make uninstall is fixed and docu improved
\end{DoxyItemize}

\subsection*{Notes}

There are some misconceptions about Elektra and semi structured data (like X\+ML, J\+S\+ON). Elektra is a key-\/value storage, that internally represents everything with key and values. Even though, Elektra can use X\+ML and J\+S\+ON files elegantly, there are limitations what X\+ML and J\+S\+ON can represent. X\+ML, e.\+g., cannot have holes within its structure, while this is obviously easily possible with key-\/value. And J\+S\+ON, e.\+g., cannot have non-\/array entries within an array. This is a more general issue of that configuration files in general are constrained in what they are able to express. The solution to this problem is validation, i.\+e. keys that does not fit in the underlying format are rejected. Note there is no issue the other way round\+: special characteristics of configuration files can always be captured in Elektra’s metadata.

\subsection*{Get It!}

You can download the release from \href{https://www.libelektra.org/ftp/elektra/releases/elektra-0.8.13.tar.gz}{\tt here} and now also \href{https://github.com/ElektraInitiative/ftp/tree/master/releases/elektra-0.8.13.tar.gz}{\tt here on github}


\begin{DoxyItemize}
\item name\+: elektra-\/0.\+8.\+13.\+tar.\+gz
\item size\+: 2141758
\item md5sum\+: 6e7640338f440e67aba91bd64b64f613
\item sha1\+: ca58524d78e5d39a540a4db83ad527354524db5e
\item sha256\+: f5c672ef9f7826023a577ca8643d0dcf20c3ad85720f36e39f98fe61ffe74637
\end{DoxyItemize}

This release tarball now is also available \href{https://www.libelektra.org/ftp/elektra/releases/elektra-0.8.13.tar.gz.gpg}{\tt signed by me using gpg}

already built A\+P\+I-\/\+Docu can be found \href{https://doc.libelektra.org/api/0.8.13/html/}{\tt here}

\subsection*{Stay tuned!}

Subscribe to the \href{https://doc.libelektra.org/news/feed.rss}{\tt R\+SS feed} to always get the release notifications.

For any questions and comments, please contact the \href{https://lists.sourceforge.net/lists/listinfo/registry-list}{\tt Mailing List} the issue tracker \href{https://git.libelektra.org/issues}{\tt on github} or by mail \href{mailto:elektra@markus-raab.org}{\tt elektra@markus-\/raab.\+org}.

\href{https://doc.libelektra.org/news/3c00a5f1-c017-4555-92b5-a2cf6e0803e3.html}{\tt Permalink to this N\+E\+WS entry}

For more information, see \href{https://libelektra.org}{\tt https\+://libelektra.\+org}

Best regards, Markus 