
\begin{DoxyItemize}
\item infos = Information about the toml plugin is in keys below
\item infos/author = Jakob Fischer \href{mailto:jakobfischer93@gmail.com}{\tt jakobfischer93@gmail.\+com}
\item infos/licence = B\+SD
\item infos/provides = storage/toml
\item infos/needs = base64
\item infos/recommends = type
\item infos/placements = getstorage setstorage
\item infos/status = experimental unfinished
\item infos/metadata = order comment/\# comment/\#/start comment/\#/space type tomltype origvalue
\item infos/description = This storage plugin reads and writes T\+O\+ML files using Flex and Bison.
\end{DoxyItemize}\hypertarget{autotoc_md693_src_plugins_toml_README_md}{}\section{Introduction}\label{autotoc_md693_src_plugins_toml_README_md}
This plugin is a storage plugin for reading and writing T\+O\+ML files. The plugin retains most of the file structure of a read T\+O\+ML file such as comments, empty lines and T\+O\+ML tables. It supports all kinds of T\+O\+ML specific types and tables, including nested inline tables and multiline strings.

For parsing T\+O\+ML files, the plugins uses Flex and Bison.\hypertarget{autotoc_md693_autotoc_md694}{}\section{Requirements}\label{autotoc_md693_autotoc_md694}
The plugin needs Flex ($>$=2.\+6.\+2) and Bison ($>$=3).\hypertarget{autotoc_md693_autotoc_md695}{}\section{Types}\label{autotoc_md693_autotoc_md695}
\hypertarget{autotoc_md693_autotoc_md696}{}\subsection{Reading}\label{autotoc_md693_autotoc_md696}
On reading, the plugin will set the {\ttfamily type} metakey for strings, integers, floats and boolean values, if applicable. For decimal integers, the metakey is set to {\ttfamily long\+\_\+long}. For binary/octal/hexadecimal integers, the metakey is set to {\ttfamily unsigned\+\_\+long\+\_\+long}. For floats, the metakey will be set to {\ttfamily double}. These types are chosen to conform with the T\+O\+ML format as stated on the offical \href{https://toml.io/en/v1.0.0-rc.1}{\tt T\+O\+ML} page.

On reading, non-\/decimal integers will get converted to decimal. The non-\/decimal representation will be stored in the {\ttfamily origvalue} metakey of the key. When writing a key, the value of that metakey will be written instead, in order to retain the original format. Note that the {\ttfamily origvalue} metakey gets removed if the value of the key changes.

``` sudo kdb mount test.\+toml user/tests/storage/types toml type\hypertarget{autotoc_md693_autotoc_md697}{}\section{Create a T\+O\+M\+L file with 4 keys}\label{autotoc_md693_autotoc_md697}
echo \textquotesingle{}plain\+\_\+decimal = 1000\textquotesingle{} $>$$>$ {\ttfamily kdb file user/tests/storage/types} echo \textquotesingle{}file\+\_\+permissions = 0o777\textquotesingle{} $>$$>$ {\ttfamily kdb file user/tests/storage/types} echo \textquotesingle{}pi = 3.\+1415\textquotesingle{} $>$$>$ {\ttfamily kdb file user/tests/storage/types} echo \textquotesingle{}division\+\_\+gone\+\_\+wrong = -\/inf\textquotesingle{} $>$$>$ {\ttfamily kdb file user/tests/storage/types}\hypertarget{autotoc_md693_autotoc_md698}{}\section{Print the content of the toml file}\label{autotoc_md693_autotoc_md698}
cat {\ttfamily kdb file user/tests/storage/types} \hypertarget{autotoc_md693_autotoc_md699}{}\section{$>$ plain\+\_\+decimal = 1000}\label{autotoc_md693_autotoc_md699}
\hypertarget{autotoc_md693_autotoc_md700}{}\section{$>$ file\+\_\+permissions = 0o777}\label{autotoc_md693_autotoc_md700}
\hypertarget{autotoc_md693_autotoc_md701}{}\section{$>$ pi = 3.\+1415}\label{autotoc_md693_autotoc_md701}
\hypertarget{autotoc_md693_autotoc_md702}{}\section{$>$ division\+\_\+gone\+\_\+wrong = -\/inf}\label{autotoc_md693_autotoc_md702}
\hypertarget{autotoc_md693_autotoc_md703}{}\section{Print types and values of the keys with `kdb`}\label{autotoc_md693_autotoc_md703}
kdb meta-\/get \textquotesingle{}user/tests/storage/types/plain\+\_\+decimal\textquotesingle{} \textquotesingle{}type\textquotesingle{} \hypertarget{autotoc_md693_autotoc_md704}{}\section{$>$ long\+\_\+long}\label{autotoc_md693_autotoc_md704}
kdb get \textquotesingle{}user/tests/storage/types/plain\+\_\+decimal\textquotesingle{} \hypertarget{autotoc_md693_autotoc_md705}{}\section{$>$ 1000}\label{autotoc_md693_autotoc_md705}
kdb meta-\/get \textquotesingle{}user/tests/storage/types/file\+\_\+permissions\textquotesingle{} \textquotesingle{}type\textquotesingle{} \hypertarget{autotoc_md693_autotoc_md706}{}\section{$>$ unsigned\+\_\+long\+\_\+long}\label{autotoc_md693_autotoc_md706}
\hypertarget{autotoc_md693_autotoc_md707}{}\section{The octal value will be converted to decimal}\label{autotoc_md693_autotoc_md707}
kdb get \textquotesingle{}user/tests/storage/types/file\+\_\+permissions\textquotesingle{} \hypertarget{autotoc_md693_autotoc_md708}{}\section{$>$ 511}\label{autotoc_md693_autotoc_md708}
kdb meta-\/get \textquotesingle{}user/tests/storage/types/pi\textquotesingle{} \textquotesingle{}type\textquotesingle{} \hypertarget{autotoc_md693_autotoc_md709}{}\section{$>$ double}\label{autotoc_md693_autotoc_md709}
kdb get \textquotesingle{}user/tests/storage/types/pi\textquotesingle{} \hypertarget{autotoc_md693_autotoc_md710}{}\section{$>$ 3.\+1415}\label{autotoc_md693_autotoc_md710}
kdb meta-\/get \textquotesingle{}user/tests/storage/types/division\+\_\+gone\+\_\+wrong\textquotesingle{} \textquotesingle{}type\textquotesingle{} \hypertarget{autotoc_md693_autotoc_md711}{}\section{$>$ double}\label{autotoc_md693_autotoc_md711}
kdb get \textquotesingle{}user/tests/storage/types/division\+\_\+gone\+\_\+wrong\textquotesingle{} \hypertarget{autotoc_md693_autotoc_md712}{}\section{$>$ -\/inf}\label{autotoc_md693_autotoc_md712}
\hypertarget{autotoc_md693_autotoc_md713}{}\section{Cleanup}\label{autotoc_md693_autotoc_md713}
kdb rm -\/r user/tests/storage/types sudo kdb umount user/tests/storage/types 
\begin{DoxyCode}
## Writing

On writing, for most values, the plugin will infer the appropriate type and will write them accordingly.
This means, values, that match a TOML float, integer or date will be written without any quotes around
       them.
If a value does not match any of these types, it will be written as a string.

The plugin uses an existing `type` metakey only to check if it should write a value as a `string` or a
       `boolean`.

With this functionality, numbers can be written as a string to the file, if wanted.

To write a boolean value, the `type` metakey must be set to `boolean` for the key.
Otherwise, no conversion to the TOML boolean values will take place.
Per default, Elektra uses 0/1 to represent boolean values.
\end{DoxyCode}
 sudo kdb mount test.\+toml user/tests/storage/types toml type\hypertarget{autotoc_md693_autotoc_md714}{}\section{Create a key, which may be a integer, boolean or string}\label{autotoc_md693_autotoc_md714}
kdb set \textquotesingle{}user/tests/storage/types/value\textquotesingle{} \textquotesingle{}1\textquotesingle{}\hypertarget{autotoc_md693_autotoc_md715}{}\section{The plugin infers `long\+\_\+long` for this value}\label{autotoc_md693_autotoc_md715}
kdb meta-\/get \textquotesingle{}user/tests/storage/types/value\textquotesingle{} \textquotesingle{}type\textquotesingle{} \hypertarget{autotoc_md693_autotoc_md716}{}\section{$>$ long\+\_\+long}\label{autotoc_md693_autotoc_md716}
\hypertarget{autotoc_md693_autotoc_md717}{}\section{The value is written as an integer}\label{autotoc_md693_autotoc_md717}
cat {\ttfamily kdb file user/tests/storage/types} \hypertarget{autotoc_md693_autotoc_md718}{}\section{$>$ value = 1}\label{autotoc_md693_autotoc_md718}
\hypertarget{autotoc_md693_autotoc_md719}{}\section{Manually set the `type` metakey to boolean}\label{autotoc_md693_autotoc_md719}
kdb meta-\/set \textquotesingle{}user/tests/storage/types/value\textquotesingle{} \textquotesingle{}type\textquotesingle{} \textquotesingle{}boolean\textquotesingle{}\hypertarget{autotoc_md693_autotoc_md720}{}\section{The value is written as a boolean}\label{autotoc_md693_autotoc_md720}
cat {\ttfamily kdb file user/tests/storage/types} \hypertarget{autotoc_md693_autotoc_md721}{}\section{$>$ value = true}\label{autotoc_md693_autotoc_md721}
\hypertarget{autotoc_md693_autotoc_md722}{}\section{Manually set the `type` metakey to string}\label{autotoc_md693_autotoc_md722}
kdb meta-\/set \textquotesingle{}user/tests/storage/types/value\textquotesingle{} \textquotesingle{}type\textquotesingle{} \textquotesingle{}string\textquotesingle{}\hypertarget{autotoc_md693_autotoc_md723}{}\section{The value is written as a string}\label{autotoc_md693_autotoc_md723}
cat {\ttfamily kdb file user/tests/storage/types} \hypertarget{autotoc_md693_autotoc_md724}{}\section{$>$ value = \char`\"{}1\char`\"{}}\label{autotoc_md693_autotoc_md724}
\hypertarget{autotoc_md693_autotoc_md725}{}\section{Cleanup}\label{autotoc_md693_autotoc_md725}
kdb rm -\/r user/tests/storage/types sudo kdb umount user/tests/storage/types 
\begin{DoxyCode}
# Numbers

The plugin supports reading and writing of any kind of number supported by the TOML format, such as
       floating point numbers and binary/octal/decimal/hexadecimal integers.
To write a non-decimal integer, add the corresponding prefix to the number (`0b` for binary, `0o` for
       octal, `0x` for hexadecimal).
The value will be written in the given base to the file, but converted to decimal within Elektra (see
       [Reading](##reading)).
Note that the plugin doesn't warn about invalid prefix/digit combinations. If the combination is not valid,
       it will be written as a string instead.

If the `type` plugin is enabled and you want to change the value of an existing number key which needs
       conversion (all keys which have a `origvalue`),
you have to change the value of `origvalue` instead of the key value. Otherwise the `type` plugin will give
       an error.
\end{DoxyCode}
 \hypertarget{autotoc_md693_autotoc_md726}{}\section{Mount a new T\+O\+M\+L file}\label{autotoc_md693_autotoc_md726}
sudo kdb mount test.\+toml user/tests/storage/numbers toml type\hypertarget{autotoc_md693_autotoc_md727}{}\section{Write an octal value}\label{autotoc_md693_autotoc_md727}
kdb set \textquotesingle{}user/tests/storage/numbers/a\textquotesingle{} \textquotesingle{}0o777\textquotesingle{}\hypertarget{autotoc_md693_autotoc_md728}{}\section{Get the converted decimal value}\label{autotoc_md693_autotoc_md728}
kdb get \textquotesingle{}user/tests/storage/numbers/a\textquotesingle{} \hypertarget{autotoc_md693_autotoc_md729}{}\section{$>$ 511}\label{autotoc_md693_autotoc_md729}
\hypertarget{autotoc_md693_autotoc_md730}{}\section{Get the original octal value}\label{autotoc_md693_autotoc_md730}
kdb meta-\/get \textquotesingle{}user/tests/storage/numbers/a\textquotesingle{} \textquotesingle{}origvalue\textquotesingle{} \hypertarget{autotoc_md693_autotoc_md731}{}\section{$>$ 0o777}\label{autotoc_md693_autotoc_md731}
\hypertarget{autotoc_md693_autotoc_md732}{}\section{Get the type of the number; since it\textquotesingle{}s originally octal, we get `unsigned\+\_\+long\+\_\+long`}\label{autotoc_md693_autotoc_md732}
kdb meta-\/get \textquotesingle{}user/tests/storage/numbers/a\textquotesingle{} \textquotesingle{}type\textquotesingle{} \hypertarget{autotoc_md693_autotoc_md733}{}\section{$>$ unsigned\+\_\+long\+\_\+long}\label{autotoc_md693_autotoc_md733}
\hypertarget{autotoc_md693_autotoc_md734}{}\section{Change the value by changing the `origvalue` metakey}\label{autotoc_md693_autotoc_md734}
kdb meta-\/set \textquotesingle{}user/tests/storage/numbers/a\textquotesingle{} \textquotesingle{}origvalue\textquotesingle{} \textquotesingle{}0o666\textquotesingle{}\hypertarget{autotoc_md693_autotoc_md735}{}\section{Get the new value as decimal}\label{autotoc_md693_autotoc_md735}
kdb get \textquotesingle{}user/tests/storage/numbers/a\textquotesingle{} \hypertarget{autotoc_md693_autotoc_md736}{}\section{$>$ 438}\label{autotoc_md693_autotoc_md736}
\hypertarget{autotoc_md693_autotoc_md737}{}\section{Change the value to an invalid octal value.}\label{autotoc_md693_autotoc_md737}
kdb meta-\/set \textquotesingle{}user/tests/storage/numbers/a\textquotesingle{} \textquotesingle{}origvalue\textquotesingle{} \textquotesingle{}0o888\textquotesingle{}\hypertarget{autotoc_md693_autotoc_md738}{}\section{The key value is no longer considered a number}\label{autotoc_md693_autotoc_md738}
kdb meta-\/get \textquotesingle{}user/tests/storage/numbers/a\textquotesingle{} \textquotesingle{}type\textquotesingle{} \hypertarget{autotoc_md693_autotoc_md739}{}\section{$>$ string}\label{autotoc_md693_autotoc_md739}
\hypertarget{autotoc_md693_autotoc_md740}{}\section{Cleanup}\label{autotoc_md693_autotoc_md740}
kdb rm -\/r user/tests/storage/numbers sudo kdb umount user/tests/storage/numbers 
\begin{DoxyCode}
# Strings

The plugin can read any kind of TOML string: bare, basic, literal, basic multiline and literal multiline.
However, it will write back all non-bare strings as basic strings or it's multiline version.
Therefore, any string set with `kdb set` must be treated as a basic string and possible escape sequences
       and special meanings of quotation characters must be taken care of.
\end{DoxyCode}
 \hypertarget{autotoc_md693_autotoc_md741}{}\section{Mount T\+O\+M\+L file}\label{autotoc_md693_autotoc_md741}
sudo kdb mount test\+\_\+strings.\+toml user/tests/storage toml type\hypertarget{autotoc_md693_autotoc_md742}{}\section{setting a string containing a newline escape sequence}\label{autotoc_md693_autotoc_md742}
kdb set \textquotesingle{}user/tests/storage/string\textquotesingle{} \textquotesingle{}I am a basic string a literal one.\textquotesingle{}

kdb get \textquotesingle{}user/tests/storage/string\textquotesingle{} \hypertarget{autotoc_md693_autotoc_md743}{}\section{$>$ I am a basic string}\label{autotoc_md693_autotoc_md743}
\hypertarget{autotoc_md693_autotoc_md744}{}\section{$>$ ot a literal one}\label{autotoc_md693_autotoc_md744}
\hypertarget{autotoc_md693_autotoc_md745}{}\section{setting the string again, but escape the backslash with another backslash}\label{autotoc_md693_autotoc_md745}
kdb set \textquotesingle{}user/tests/storage/string\textquotesingle{} \textquotesingle{}I am a basic string\textbackslash{}not a literal one.\textquotesingle{}

kdb get \textquotesingle{}user/tests/storage/string\textquotesingle{} \hypertarget{autotoc_md693_autotoc_md746}{}\section{$>$ I am a basic string\textbackslash{}not a literal one}\label{autotoc_md693_autotoc_md746}
\hypertarget{autotoc_md693_autotoc_md747}{}\section{Cleanup}\label{autotoc_md693_autotoc_md747}
kdb rm -\/r user/tests/storage sudo kdb umount user/tests/storage 
\begin{DoxyCode}
The plugin supports all kinds of escape sequences used by TOML in basic and basic multiline strings, like
       `\(\backslash\)n`, `\(\backslash\)r`, `\(\backslash\)t` and
even `\(\backslash\)u`/`\(\backslash\)U` for Unicode escape sequences. `\(\backslash\)t` is interpreted to be 4 spaces.

# Binary/NULL values

The plugin handles binary values by using the [base64](@ref src\_plugins\_base64\_README\_md) plugin.
As a result, binary values get written as base64 encoded strings, which start with the special prefix
       `@BASE64`.
`NULL` key values are written as special strings of value `@NULL`.
\end{DoxyCode}
 \hypertarget{autotoc_md693_autotoc_md748}{}\section{Mount T\+O\+M\+L file}\label{autotoc_md693_autotoc_md748}
sudo kdb mount test\+\_\+binary.\+toml user/tests/storage toml type\hypertarget{autotoc_md693_autotoc_md749}{}\section{Creating a key with a N\+U\+L\+L value}\label{autotoc_md693_autotoc_md749}
kdb set \textquotesingle{}user/tests/storage/nullkey\textquotesingle{} \hypertarget{autotoc_md693_autotoc_md750}{}\section{$>$ Create a new key user/test/nullkey with null value}\label{autotoc_md693_autotoc_md750}
\hypertarget{autotoc_md693_autotoc_md751}{}\section{Print file content}\label{autotoc_md693_autotoc_md751}
cat {\ttfamily kdb file user/tests/storage} \hypertarget{autotoc_md693_autotoc_md752}{}\section{$>$ nullkey = \textquotesingle{}@\+N\+U\+L\+L\textquotesingle{}}\label{autotoc_md693_autotoc_md752}
\hypertarget{autotoc_md693_autotoc_md753}{}\section{Write base64 encoded data to the file}\label{autotoc_md693_autotoc_md753}
echo \char`\"{}base64 = \textquotesingle{}@\+B\+A\+S\+E64\+S\+S\+Bhb\+S\+Bi\+Y\+X\+Nl\+I\+D\+Y0\+I\+G\+Vu\+Y29k\+Z\+W\+Qg\+Zm9y\+I\+G5v\+I\+H\+Jl\+Y\+X\+Nvbi4=\textquotesingle{}\char`\"{} $>$ {\ttfamily kdb file user/test}\hypertarget{autotoc_md693_autotoc_md754}{}\section{Print the value of the key, which is a binary value}\label{autotoc_md693_autotoc_md754}
kdb get \textquotesingle{}user/test/base64\textquotesingle{} \#$>$ \hypertarget{autotoc_md693_autotoc_md755}{}\section{Print the value again, but apply the escape codes}\label{autotoc_md693_autotoc_md755}
echo -\/e {\ttfamily kdb get \textquotesingle{}user/test/base64\textquotesingle{}} \#$>$ I am base 64 encoded for no reason.\hypertarget{autotoc_md693_autotoc_md756}{}\section{Cleanup}\label{autotoc_md693_autotoc_md756}
kdb rm -\/r user/tests/storage sudo kdb umount user/tests/storage 
\begin{DoxyCode}
# TOML specific structures

TOML specific structures are represented by the metakey `tomltype` on a certain key.
It will be set when the TOML plugin reads a TOML structure from a file. Additionally, this metakey can be
       set by the user, if they want a certain TOML structure to be written.
No automatic inference of this metakey is done on writing.

## Simple Tables

TOML's simple tables are represented by setting the `tomltype` metakey to `simpletable`.
\end{DoxyCode}
 \hypertarget{autotoc_md693_autotoc_md757}{}\section{Mount T\+O\+M\+L file}\label{autotoc_md693_autotoc_md757}
sudo kdb mount test\+\_\+table.\+toml user/tests/storage toml type\hypertarget{autotoc_md693_autotoc_md758}{}\section{Create three keys, which are all a subkey of \textquotesingle{}common\textquotesingle{},}\label{autotoc_md693_autotoc_md758}
\hypertarget{autotoc_md693_autotoc_md759}{}\section{but we have no \textquotesingle{}common\textquotesingle{} simple table key yet}\label{autotoc_md693_autotoc_md759}
kdb set \textquotesingle{}user/tests/storage/common/a\textquotesingle{} \textquotesingle{}0\textquotesingle{} kdb set \textquotesingle{}user/tests/storage/common/b\textquotesingle{} \textquotesingle{}1\textquotesingle{} kdb set \textquotesingle{}user/tests/storage/common/c\textquotesingle{} \textquotesingle{}2\textquotesingle{}\hypertarget{autotoc_md693_autotoc_md760}{}\section{Print the content of the resulting T\+O\+M\+L file}\label{autotoc_md693_autotoc_md760}
cat {\ttfamily kdb file user/tests/storage} \#$>$ common.\+a = 0 \#$>$ common.\+b = 1 \#$>$ common.\+c = 2\hypertarget{autotoc_md693_autotoc_md761}{}\section{Create a simple table key}\label{autotoc_md693_autotoc_md761}
kdb meta-\/set \textquotesingle{}user/tests/storage/common\textquotesingle{} \textquotesingle{}tomltype\textquotesingle{} \textquotesingle{}simpletable\textquotesingle{}\hypertarget{autotoc_md693_autotoc_md762}{}\section{Print the content of the resulting T\+O\+M\+L file}\label{autotoc_md693_autotoc_md762}
cat {\ttfamily kdb file user/tests/storage} \#$>$ \mbox{[}common\mbox{]} \#$>$ a = 0 \#$>$ b = 1 \#$>$ c = 2\hypertarget{autotoc_md693_autotoc_md763}{}\section{Cleanup}\label{autotoc_md693_autotoc_md763}
kdb rm -\/r user/tests/storage sudo kdb umount user/tests/storage 
\begin{DoxyCode}
## Table Arrays

Table arrays are represented by setting the `tomltype` metakey to `tablearray`. It is not required to also
       set the array metakey, since the plugin will set the metakey, if it is missing.
\end{DoxyCode}
 \hypertarget{autotoc_md693_autotoc_md764}{}\section{Mount T\+O\+M\+L file}\label{autotoc_md693_autotoc_md764}
sudo kdb mount test\+\_\+table\+\_\+array.\+toml user/tests/storage toml type\hypertarget{autotoc_md693_autotoc_md765}{}\section{Create a table array containing two entries, each with a key \textquotesingle{}a\textquotesingle{} and \textquotesingle{}b\textquotesingle{}}\label{autotoc_md693_autotoc_md765}
kdb meta-\/set \textquotesingle{}user/tests/storage/tablearray\textquotesingle{} \textquotesingle{}tomltype\textquotesingle{} \textquotesingle{}tablearray\textquotesingle{} kdb set \textquotesingle{}user/tests/storage/tablearray/\#0/a\textquotesingle{} \textquotesingle{}1\textquotesingle{} kdb set \textquotesingle{}user/tests/storage/tablearray/\#0/b\textquotesingle{} \textquotesingle{}2\textquotesingle{}

kdb set \textquotesingle{}user/tests/storage/tablearray/\#1/a\textquotesingle{} \textquotesingle{}3\textquotesingle{} kdb set \textquotesingle{}user/tests/storage/tablearray/\#1/b\textquotesingle{} \textquotesingle{}4\textquotesingle{}\hypertarget{autotoc_md693_autotoc_md766}{}\section{Print the highest index of the table array}\label{autotoc_md693_autotoc_md766}
kdb meta-\/get \textquotesingle{}user/tests/storage/tablearray\textquotesingle{} \textquotesingle{}array\textquotesingle{} \#$>$ \#1\hypertarget{autotoc_md693_autotoc_md767}{}\section{Print the content of the resulting T\+O\+M\+L file}\label{autotoc_md693_autotoc_md767}
cat {\ttfamily kdb file user/tests/storage} \#$>$ \mbox{[}\mbox{[}tablearray\mbox{]}\mbox{]} \#$>$ a = 1 \#$>$ b = 2 \#$>$ \mbox{[}\mbox{[}tablearray\mbox{]}\mbox{]} \#$>$ a = 3 \#$>$ b = 4\hypertarget{autotoc_md693_autotoc_md768}{}\section{Cleanup}\label{autotoc_md693_autotoc_md768}
kdb rm -\/r user/tests/storage sudo kdb umount user/tests/storage 
\begin{DoxyCode}
## Inline Tables

Inline tables are represented by setting the `tomltype` metakey to `inlinetable`. The plugin also supports
       reading/writing nested inline tables.
\end{DoxyCode}
 \hypertarget{autotoc_md693_autotoc_md769}{}\section{Mount T\+O\+M\+L file}\label{autotoc_md693_autotoc_md769}
sudo kdb mount test\+\_\+inline\+\_\+table.\+toml user/tests/storage toml type\hypertarget{autotoc_md693_autotoc_md770}{}\section{Create a table array containing two entries, each with a key \textquotesingle{}a\textquotesingle{} and \textquotesingle{}b\textquotesingle{}}\label{autotoc_md693_autotoc_md770}
kdb meta-\/set \textquotesingle{}user/tests/storage/inlinetable\textquotesingle{} \textquotesingle{}tomltype\textquotesingle{} \textquotesingle{}inlinetable\textquotesingle{} kdb set \textquotesingle{}user/tests/storage/inlinetable/a\textquotesingle{} \textquotesingle{}1\textquotesingle{} kdb set \textquotesingle{}user/tests/storage/inlinetable/b\textquotesingle{} \textquotesingle{}2\textquotesingle{} kdb meta-\/set \textquotesingle{}user/tests/storage/inlinetable/nested\textquotesingle{} \textquotesingle{}tomltype\textquotesingle{} \textquotesingle{}inlinetable\textquotesingle{} kdb set \textquotesingle{}user/tests/storage/inlinetable/nested/x\textquotesingle{} \textquotesingle{}3\textquotesingle{} kdb set \textquotesingle{}user/tests/storage/inlinetable/nested/y\textquotesingle{} \textquotesingle{}4\textquotesingle{}\hypertarget{autotoc_md693_autotoc_md771}{}\section{Print the content of the resulting T\+O\+M\+L file}\label{autotoc_md693_autotoc_md771}
cat {\ttfamily kdb file user/tests/storage} \#$>$ inlinetable = \{ a = 1, b = 2, nested = \{ x = 3, y = 4 \} \}\hypertarget{autotoc_md693_autotoc_md772}{}\section{Cleanup}\label{autotoc_md693_autotoc_md772}
kdb rm -\/r user/tests/storage sudo kdb umount user/tests/storage 
\begin{DoxyCode}
## Arrays

Arrays are recognized by the `array` metakey. On writing, the plugin will detect arrays automatically and
       set the appropriate metakey if it is missing.
\end{DoxyCode}
 \hypertarget{autotoc_md693_autotoc_md773}{}\section{Mount T\+O\+M\+L file}\label{autotoc_md693_autotoc_md773}
sudo kdb mount test\+\_\+array.\+toml user/tests/storage toml type\hypertarget{autotoc_md693_autotoc_md774}{}\section{Create array elements}\label{autotoc_md693_autotoc_md774}
kdb set \textquotesingle{}user/tests/storage/array/\#0\textquotesingle{} \textquotesingle{}1\textquotesingle{} kdb set \textquotesingle{}user/tests/storage/array/\#1\textquotesingle{} \textquotesingle{}2\textquotesingle{} kdb set \textquotesingle{}user/tests/storage/array/\#2\textquotesingle{} \textquotesingle{}3\textquotesingle{}\hypertarget{autotoc_md693_autotoc_md775}{}\section{Print the highest index of the array}\label{autotoc_md693_autotoc_md775}
kdb meta-\/get \textquotesingle{}user/tests/storage/array\textquotesingle{} \textquotesingle{}array\textquotesingle{} \#$>$ \#2\hypertarget{autotoc_md693_autotoc_md776}{}\section{Print the content of the resulting T\+O\+M\+L file}\label{autotoc_md693_autotoc_md776}
cat {\ttfamily kdb file user/tests/storage} \#$>$ array = \mbox{[}1, 2, 3\mbox{]}\hypertarget{autotoc_md693_autotoc_md777}{}\section{Cleanup}\label{autotoc_md693_autotoc_md777}
kdb rm -\/r user/tests/storage sudo kdb umount user/tests/storage 
\begin{DoxyCode}
# Comments and Empty Lines

The plugin preserves all comments with only one limitation for arrays. The amount of whitespace in front of
       a comment is also saved.
For this purpose, each tab will get translated to 4 spaces.

Comments can also be created by assigning meta keys to a key.
The meta keys must be of the form `comment/#n`, where `n` is a positive number, indicating the position of
       the comment relative to the key.
An index of 0 is always the inline comment of the key.
Indices greater than zero are for comments preceding the given key, where 1 is the top-most comment and the
       highest index comment is right above the key.

Spaces can be added to a comment by creating a `comment/#n/space` metakey with the amount of spaces to the
       key.

File ending comments must be assigned to the file root key.

Empty lines in front of a key can be created by adding an empty `comment/#n/start` entry to it. In this
       case, no `comment/#n` key is needed.
\end{DoxyCode}
 \hypertarget{autotoc_md693_autotoc_md778}{}\section{Mount T\+O\+M\+L file}\label{autotoc_md693_autotoc_md778}
sudo kdb mount test\+\_\+comments.\+toml user/tests/storage toml type\hypertarget{autotoc_md693_autotoc_md779}{}\section{create a key-\/value pair, ready for comment decoration}\label{autotoc_md693_autotoc_md779}
kdb set \textquotesingle{}user/tests/storage/key\textquotesingle{} \textquotesingle{}1\textquotesingle{}\hypertarget{autotoc_md693_autotoc_md780}{}\section{add an inline comment with 4 leading spaces}\label{autotoc_md693_autotoc_md780}
kdb meta-\/set \textquotesingle{}user/tests/storage/key\textquotesingle{} \textquotesingle{}comment/\#0\textquotesingle{} \textquotesingle{} This value is very interesting\textquotesingle{} kdb meta-\/set \textquotesingle{}user/tests/storage/key\textquotesingle{} \textquotesingle{}comment/\#0/space\textquotesingle{} \textquotesingle{}4\textquotesingle{}\hypertarget{autotoc_md693_autotoc_md781}{}\section{add some comments preceding the key}\label{autotoc_md693_autotoc_md781}
kdb meta-\/set \textquotesingle{}user/tests/storage/key\textquotesingle{} \textquotesingle{}comment/\#1\textquotesingle{} \textquotesingle{} I am the top-\/most comment relative to my key.\textquotesingle{} kdb meta-\/set \textquotesingle{}user/tests/storage/key\textquotesingle{} \textquotesingle{}comment/\#2\textquotesingle{} \textquotesingle{} I am in the middle. Just boring.\textquotesingle{} kdb meta-\/set \textquotesingle{}user/tests/storage/key\textquotesingle{} \textquotesingle{}comment/\#3\textquotesingle{} \textquotesingle{} I am in the line right above my key.\textquotesingle{}\hypertarget{autotoc_md693_autotoc_md782}{}\section{add file ending comments and empty lines}\label{autotoc_md693_autotoc_md782}
kdb meta-\/set \textquotesingle{}user/tests/storage\textquotesingle{} \textquotesingle{}comment/\#1\textquotesingle{} \textquotesingle{} First file-\/ending comment\textquotesingle{} kdb meta-\/set \textquotesingle{}user/tests/storage\textquotesingle{} \textquotesingle{}comment/\#2/start\textquotesingle{} \textquotesingle{}\textquotesingle{} kdb meta-\/set \textquotesingle{}user/tests/storage\textquotesingle{} \textquotesingle{}comment/\#3\textquotesingle{} \textquotesingle{} Second file-\/ending comment. I am the last line of the file.\textquotesingle{}\hypertarget{autotoc_md693_autotoc_md783}{}\section{Print the content of the resulting T\+O\+M\+L file}\label{autotoc_md693_autotoc_md783}
cat {\ttfamily kdb file user/tests/storage} \#$>$ \# I am the top-\/most comment relative to my key. \#$>$ \# I am in the middle. Just boring. \#$>$ \# I am in the line right above my key. \#$>$ key = 1 \# This value is very interesting \#$>$ \# First file-\/ending comment \#$>$ \#$>$ \# Second file-\/ending comment. I am the last line of the file.\hypertarget{autotoc_md693_autotoc_md784}{}\section{Cleanup}\label{autotoc_md693_autotoc_md784}
kdb rm -\/r user/tests/storage sudo kdb umount user/tests/storage 
\begin{DoxyCode}
## Comments in Arrays

Any amount of comments can be placed between array elements or between the first element and the opening
       brackets.

However, only one comment - an inline comment - can be placed after the last element and the closing
       brackets.
On reading, the plugin discards any non-inline comments between the last element and the closing brackets.
\end{DoxyCode}
 \hypertarget{autotoc_md693_autotoc_md785}{}\section{Mount T\+O\+M\+L file}\label{autotoc_md693_autotoc_md785}
sudo kdb mount test\+\_\+array\+\_\+comments.\+toml user/tests/storage toml type\hypertarget{autotoc_md693_autotoc_md786}{}\section{Create array elements}\label{autotoc_md693_autotoc_md786}
kdb set \textquotesingle{}user/tests/storage/array/\#0\textquotesingle{} \textquotesingle{}1\textquotesingle{} kdb set \textquotesingle{}user/tests/storage/array/\#1\textquotesingle{} \textquotesingle{}2\textquotesingle{} kdb set \textquotesingle{}user/tests/storage/array/\#2\textquotesingle{} \textquotesingle{}3\textquotesingle{}\hypertarget{autotoc_md693_autotoc_md787}{}\section{Add inline comment after the array}\label{autotoc_md693_autotoc_md787}
kdb meta-\/set \textquotesingle{}user/tests/storage/array\textquotesingle{} \textquotesingle{}comment/\#0\textquotesingle{} \textquotesingle{} Inline comment after the array\textquotesingle{} kdb meta-\/set \textquotesingle{}user/tests/storage/array\textquotesingle{} \textquotesingle{}comment/\#0/start\textquotesingle{} \textquotesingle{}\#\textquotesingle{} kdb meta-\/set \textquotesingle{}user/tests/storage/array\textquotesingle{} \textquotesingle{}comment/\#0/space\textquotesingle{} \textquotesingle{}5\textquotesingle{}\hypertarget{autotoc_md693_autotoc_md788}{}\section{Add comments for array elements}\label{autotoc_md693_autotoc_md788}
kdb meta-\/set \textquotesingle{}user/tests/storage/array/\#0\textquotesingle{} \textquotesingle{}comment/\#0\textquotesingle{} \textquotesingle{} Inline comment of first element\textquotesingle{} kdb meta-\/set \textquotesingle{}user/tests/storage/array/\#0\textquotesingle{} \textquotesingle{}comment/\#0/start\textquotesingle{} \textquotesingle{}\#\textquotesingle{} kdb meta-\/set \textquotesingle{}user/tests/storage/array/\#0\textquotesingle{} \textquotesingle{}comment/\#0/space\textquotesingle{} \textquotesingle{}4\textquotesingle{}

kdb meta-\/set \textquotesingle{}user/tests/storage/array/\#0\textquotesingle{} \textquotesingle{}comment/\#1\textquotesingle{} \textquotesingle{} Comment preceding the first element\textquotesingle{} kdb meta-\/set \textquotesingle{}user/tests/storage/array/\#0\textquotesingle{} \textquotesingle{}comment/\#1/start\textquotesingle{} \textquotesingle{}\#\textquotesingle{} kdb meta-\/set \textquotesingle{}user/tests/storage/array/\#0\textquotesingle{} \textquotesingle{}comment/\#1/space\textquotesingle{} \textquotesingle{}4\textquotesingle{}

kdb meta-\/set \textquotesingle{}user/tests/storage/array/\#0\textquotesingle{} \textquotesingle{}comment/\#2\textquotesingle{} \textquotesingle{} Another comment preceding the first element\textquotesingle{} kdb meta-\/set \textquotesingle{}user/tests/storage/array/\#0\textquotesingle{} \textquotesingle{}comment/\#2/start\textquotesingle{} \textquotesingle{}\#\textquotesingle{} kdb meta-\/set \textquotesingle{}user/tests/storage/array/\#0\textquotesingle{} \textquotesingle{}comment/\#2/space\textquotesingle{} \textquotesingle{}6\textquotesingle{}

kdb meta-\/set \textquotesingle{}user/tests/storage/array/\#1\textquotesingle{} \textquotesingle{}comment/\#0\textquotesingle{} \textquotesingle{} Inline comment of second element\textquotesingle{} kdb meta-\/set \textquotesingle{}user/tests/storage/array/\#1\textquotesingle{} \textquotesingle{}comment/\#0/start\textquotesingle{} \textquotesingle{}\#\textquotesingle{} kdb meta-\/set \textquotesingle{}user/tests/storage/array/\#1\textquotesingle{} \textquotesingle{}comment/\#0/space\textquotesingle{} \textquotesingle{}4\textquotesingle{}

kdb meta-\/set \textquotesingle{}user/tests/storage/array/\#1\textquotesingle{} \textquotesingle{}comment/\#1\textquotesingle{} \textquotesingle{} Comment preceding the second element\textquotesingle{} kdb meta-\/set \textquotesingle{}user/tests/storage/array/\#1\textquotesingle{} \textquotesingle{}comment/\#1/start\textquotesingle{} \textquotesingle{}\#\textquotesingle{} kdb meta-\/set \textquotesingle{}user/tests/storage/array/\#1\textquotesingle{} \textquotesingle{}comment/\#1/space\textquotesingle{} \textquotesingle{}6\textquotesingle{}

kdb meta-\/set \textquotesingle{}user/tests/storage/array/\#2\textquotesingle{} \textquotesingle{}comment/\#0\textquotesingle{} \textquotesingle{} Inline comment of the last element\textquotesingle{} kdb meta-\/set \textquotesingle{}user/tests/storage/array/\#2\textquotesingle{} \textquotesingle{}comment/\#0/start\textquotesingle{} \textquotesingle{}\#\textquotesingle{} kdb meta-\/set \textquotesingle{}user/tests/storage/array/\#2\textquotesingle{} \textquotesingle{}comment/\#0/space\textquotesingle{} \textquotesingle{}5\textquotesingle{}\hypertarget{autotoc_md693_autotoc_md789}{}\section{Print the content of the resulting T\+O\+M\+L file}\label{autotoc_md693_autotoc_md789}
cat {\ttfamily kdb file user/tests/storage} \#$>$ array = \mbox{[} \# Comment preceding the first element \#$>$ \# Another comment preceding the first element \#$>$ 1, \# Inline comment of first element \#$>$ \# Comment preceding the second element \#$>$ 2, \# Inline comment of second element \#$>$ 3 \# Inline comment of the last element \#$>$ \mbox{]} \# Inline comment after the array\hypertarget{autotoc_md693_autotoc_md790}{}\section{Cleanup}\label{autotoc_md693_autotoc_md790}
kdb rm -\/r user/tests/storage sudo kdb umount user/tests/storage 
\begin{DoxyCode}
# Order

The plugin preserves the file order by the usage of the metakey `order`. When reading a file, the order
       metakey will be set according to the order as read in the file.
If new keys are added, eg. via `kdb set`, the order of the set key will be set to the next-to-highest order
       value present in the existing key set.

However, the order is only relevant between elements with the same TOML-parent. For example keys of a
       simple table are only sorted with respect to each other, not with any keys outside that table. If that table has
       it's order changed and moves to another position in the file, so will it's subkeys.

When sorting elements under the same TOML-parent, tables (simple and array) will always be sorted after
       non-table elements, regardless of their order.
With this limitation, we prevent that a newly set key, that is not part of a certain table array/simple
       table, would be placed after the table declaration, making it a member of that table on a subsequent read.
\end{DoxyCode}
 \hypertarget{autotoc_md693_autotoc_md791}{}\section{Mount T\+O\+M\+L file}\label{autotoc_md693_autotoc_md791}
sudo kdb mount test\+\_\+order.\+toml user/tests/storage toml type\hypertarget{autotoc_md693_autotoc_md792}{}\section{Create three keys in reverse alphabetical order under the subkey common}\label{autotoc_md693_autotoc_md792}
\hypertarget{autotoc_md693_autotoc_md793}{}\section{Additionally, create one key not in the common subkey space}\label{autotoc_md693_autotoc_md793}
kdb set \textquotesingle{}user/tests/storage/common/c\textquotesingle{} \textquotesingle{}0\textquotesingle{} kdb set \textquotesingle{}user/tests/storage/common/b\textquotesingle{} \textquotesingle{}1\textquotesingle{} kdb set \textquotesingle{}user/tests/storage/common/a\textquotesingle{} \textquotesingle{}2\textquotesingle{} kdb set \textquotesingle{}user/tests/storage/d\textquotesingle{} \textquotesingle{}3\textquotesingle{}\hypertarget{autotoc_md693_autotoc_md794}{}\section{Print the content of the resulting T\+O\+M\+L file}\label{autotoc_md693_autotoc_md794}
\hypertarget{autotoc_md693_autotoc_md795}{}\section{The keys are ordered as they were set}\label{autotoc_md693_autotoc_md795}
cat {\ttfamily kdb file user/tests/storage} \#$>$ common.\+c = 0 \#$>$ common.\+b = 1 \#$>$ common.\+a = 2 \#$>$ d = 3\hypertarget{autotoc_md693_autotoc_md796}{}\section{Create a simple table for the three keys under `common`}\label{autotoc_md693_autotoc_md796}
kdb meta-\/set \textquotesingle{}user/tests/storage/common\textquotesingle{} \textquotesingle{}tomltype\textquotesingle{} \textquotesingle{}simpletable\textquotesingle{}\hypertarget{autotoc_md693_autotoc_md797}{}\section{Print the content of the resulting T\+O\+M\+L file}\label{autotoc_md693_autotoc_md797}
cat {\ttfamily kdb file user/tests/storage} \#$>$ d = 3 \#$>$ \mbox{[}common\mbox{]} \#$>$ c = 0 \#$>$ b = 1 \#$>$ a = 2\hypertarget{autotoc_md693_autotoc_md798}{}\section{Cleanup}\label{autotoc_md693_autotoc_md798}
kdb rm -\/r user/tests/storage sudo kdb umount user/tests/storage ```

In this example, {\ttfamily d} and {\ttfamily common} have the same parent, the file root. This means, they need to be sorted with each other. {\ttfamily d} would be placed before {\ttfamily common} by it\textquotesingle{}s order, since it was set before, and thus, has lesser order. However, their order never gets compared, since {\ttfamily common} is a simple table and {\ttfamily d} is not, so {\ttfamily d} will get sorted before the table regardless of order.\hypertarget{autotoc_md693_autotoc_md799}{}\section{Limitations}\label{autotoc_md693_autotoc_md799}
While the plugin has good capabilities in handling the T\+O\+ML file format, it currently lacks some features possible with Elektra\+:


\begin{DoxyItemize}
\item Sparse arrays are not preserved on writing, they get a continuous array without index holes.
\item Values on non-\/leaf keys are currently not supported, they get discarded.
\item Custom metakeys cannot be written by the plugin, so they get discarded.
\end{DoxyItemize}

Additionally, there are some minor limitations related to the T\+O\+ML file format, mostly related to the preservation of the original file structure\+:


\begin{DoxyItemize}
\item On writing, each string is written as a T\+O\+ML basic string (single or multiline). See \href{#Strings}{\tt Strings}
\item Comments and newlines between the last array element and closing brackets are discarded.
\item Trailing commas in arrays and inline tables are discarded
\item Only spaces in front of comments are preserved. 
\end{DoxyItemize}