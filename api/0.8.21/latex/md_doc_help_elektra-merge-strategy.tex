In elektra-\/tools a three way merging was implemented. It can also use be used for two way merging, e.\+g. for importing.

Note\+: For a two-\/way merge, the {\ttfamily ours} version of the keys is used in place of {\ttfamily base}


\begin{DoxyItemize}
\item {\ttfamily base}\+: The {\ttfamily base} Key\+Set is the original version of the Key\+Set.
\item {\ttfamily ours}\+: The {\ttfamily ours} Key\+Set represents the user\textquotesingle{}s current version of the Key\+Set. This Key\+Set differs from {\ttfamily base} for every key you changed.
\item {\ttfamily theirs}\+: The {\ttfamily theirs} Key\+Set usually represents the default version of a Key\+Set (usually the package maintainer\textquotesingle{}s version). This Key\+Set differs from {\ttfamily base} for every key someone has changed.
\end{DoxyItemize}

The three-\/way merge works by comparing the {\ttfamily ours} Key\+Set and the {\ttfamily theirs} Key\+Set to the {\ttfamily base} Key\+Set. By looking for differences in these Key\+Sets, a new Key\+Set called {\ttfamily result} is created that represents a merge of these Key\+Sets.

\subsection*{S\+T\+R\+A\+T\+E\+G\+I\+ES}

Currently the following strategies exist\+:


\begin{DoxyItemize}
\item preserve\+: Automerge only those keys where just one side deviates from base (default).
\item ours\+: Whenever a conflict exists, use our version.
\item theirs\+: Whenever a conflict exists, use their version.
\item cut\+: Removes existing keys below the resultpath and replaces them with the merged keyset.
\item unchanged\+: (E\+X\+P\+E\+R\+I\+M\+E\+N\+T\+AL, only for kdb-\/mount) Do not fail if the operation does not change anything.
\item import\+: (D\+E\+P\+R\+E\+C\+A\+T\+ED, avoid using it) Preserves existing keys in the resultpath if they do not exist in the merged keyset. If the key does exist in the merged keyset, it will be overwritten. 
\end{DoxyItemize}