
\begin{DoxyItemize}
\item author\+: Markus Raab
\item pub\+Date\+: Mon, 28 Jul 2014 12\+:00\+:00 +0100
\item short\+Desc\+: adds python2 bindings, 3-\/way merge \& improvements
\end{DoxyItemize}

Again, we managed to have a great feature release with dozens of corrections!

\subsection*{New Features}

Thanks to Manuel Mausz for further improving lua, python3 bindings and the new python2 bindings.

The G\+SoC efforts have their first large contribution to Elektra, the 3 way merge finally arrives with this release. It is still a long way to go, however, because augeas plugins can only be mounted with a workaround and the package integration of the 3-\/way merge is still in its infancy. More information about G\+SoC and its progress can be found \href{http://community.libelektra.org/wp}{\tt here}.

A special thanks to Felix Berlakovich for his contributions to the 3-\/way merge, including meta merging, conflict resolving strategies and extensive testing.

Additionally, he added the plugins keytometa, ini and greatly improved the glob plugin. These plugins are technical previews and will receive some improvements in the next release, too.

Now a script for tab completion is available here, again thanks to Felix Berlakovich.

The contextual values now got a \href{https://github.com/ElektraInitiative/libelektra/tree/master/src/tools/gen}{\tt tutorial} and small fixes.

\subsection*{Corrections}

Thanks to Pino Toscano for fixing a lot of spelling errors, simplify R\+P\+A\+TH setting, respect \$\+H\+O\+ME and \$\+T\+M\+P\+D\+IR, improvements of test cases, and his debian-\/packaging efforts.

In the kdb tool not allowed subfolders are now checked properly and the output of warnings comes before output of the error. This fixes the problem that in the case of a longer list of warnings one did not see the error anymore.

Fix compilation warnings on clang and gcc 4.\+9. Also improve test coverage on kdb tool and some plugins.

Fix kdb import/export for some plugins (Should now work with any storage plugin again).

kdb run\+\_\+all should run flawlessly with this release. Remaining problems with not installed test data were fixed. kdb run\+\_\+all also checks if the test cases do not modify any existing key and keeps a backup if this happened.

Some remaining mem leaks in rare circumstances were fixed. Valgrind should now never report any leaks, if it does, please report the issue.

\subsection*{A\+PI Changes}

Added del\+Meta() for C++, because set\+Meta() with N\+U\+LL will set the number 0 and not remove the meta.

Arguments of is\+Below, is\+Direct\+Below, is\+Below\+Or\+Same are swapped for better readability. k.\+is\+Below(root) now means the obvious thing. The change only effects the C++ binding, key\+Is\+Below is unaffected by the change.

\subsection*{Documentation}

\href{/home/markus/Projekte/Elektra/current/doc/METADATA.ini}{\tt Specification of metadata} and \href{/home/markus/Projekte/Elektra/current/doc/CONTRACT.ini}{\tt contracts} written/greatly improved.

\hyperlink{doc_decisions_README_md}{Decisions} are introduced again.

Most often the Key\+Set is ideal, e.\+g. when doing full iteration or when performing set operations. In some cases, however, a hierarchical data structure fits better. This is especially true for G\+U\+Is. Luckily, Keys can be in multiple data structures because of their reference counting.

\subsection*{Other Stuff}

We now fully embrace github\+:
\begin{DoxyItemize}
\item We use its issue tracker (all issues from local text files were moved there)
\item We have rewritten many R\+E\+A\+D\+M\+Es to use githubs markdown
\item On pull requests the build server checks if the merge would break the build.
\item All previous gitorious users are now at github. (Most had an account anyway)
\end{DoxyItemize}

Raffael Pancheri also made progress with its qt-\/gui. It now features a model that implements great parts of Elektra’s features. Unfortunately the model cannot be serialised and thus changes cannot made persistent. Also undo and other important use-\/cases are still not there. The G\+UI looks very clean and was evaluated in a S\+US study on 23.\+07.\+2014. The current implementation can be found \href{https://github.com/0003088/qt-gui}{\tt here}.

Many distributions already have Elektra packages
\begin{DoxyItemize}
\item \href{https://admin.fedoraproject.org/pkgdb/package/elektra/}{\tt Fedora}
\item \href{http://packages.gentoo.org/package/app-admin/elektra}{\tt Gentoo}
\item \href{https://aur.archlinux.org/packages/elektra/}{\tt Arch Linux}
\end{DoxyItemize}

In some distributions Elektra packages are available, but are not up-\/to-\/date. Pino Toscano is working on get them (actually Debian, but others are derived from it) up-\/to-\/date.
\begin{DoxyItemize}
\item \href{https://launchpad.net/ubuntu/+source/elektra}{\tt Ubuntu}
\item \href{https://packages.debian.org/de/wheezy/libelektra3}{\tt Debian}
\item \href{http://community.linuxmint.com/software/view/elektra}{\tt Linux Mint}
\end{DoxyItemize}

A special thanks to Kai-\/\+Uwe Behrmann for providing packages for \href{http://software.opensuse.org/download.html?project=home%3Abekun&package=elektra}{\tt Cent\+OS, Fedora, Open\+S\+U\+SE, R\+H\+EL and S\+LE}.

\subsection*{Get It!}

You can download the release from\+:

\href{http://www.markus-raab.org/ftp/elektra/releases/elektra-0.8.7.tar.gz}{\tt http\+://www.\+markus-\/raab.\+org/ftp/elektra/releases/elektra-\/0.\+8.\+7.\+tar.\+gz} size\+: 1566800 md5sum\+: 4996df62942791373b192c793d912b4c sha1\+: 00887cc8edb3dea1bc110f69ea64f6b700c29402 sha256\+: 698ebd41d540eb0c6427c17c13a6a0f03eef94655fbd40655c9b42d612ea1c9b

already build A\+P\+I-\/\+Docu can be found here\+:

\href{https://doc.libelektra.org/api/0.8.7/html/}{\tt https\+://doc.\+libelektra.\+org/api/0.\+8.\+7/html/}

Best regards, Markus 