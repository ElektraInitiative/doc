The graph below shows an (incomplete) list of available packages for Elektra.

\href{https://repology.org/metapackage/elektra/versions}{\texttt{ }}

For the following Linux distributions and package managers 0.\+8 packages are available\+:


\begin{DoxyItemize}
\item \href{https://aur.archlinux.org/packages/elektra/}{\texttt{ Arch Linux}}
\item \href{https://github.com/openwrt/packages/tree/master/libs/elektra}{\texttt{ Openwrt}}
\item \href{https://software.opensuse.org/package/elektra}{\texttt{ Open\+Suse}}
\item \href{https://packages.debian.org/de/jessie/libelektra4}{\texttt{ Debian}}
\item \href{https://launchpad.net/ubuntu/+source/elektra}{\texttt{ Ubuntu}}
\item \href{http://packages.gentoo.org/package/app-admin/elektra}{\texttt{ Gentoo}}
\item \href{https://community.linuxmint.com/software/view/elektra-bin}{\texttt{ Linux Mint}}
\item \href{https://github.com/Linuxbrew/homebrew-core/blob/master/Formula/elektra.rb}{\texttt{ Linux\+Brew}}
\item \href{https://pkgs.alpinelinux.org/package/edge/testing/x86_64/elektra}{\texttt{ Alpine Linux}}
\end{DoxyItemize}

For \href{https://build.opensuse.org/package/show/home:bekun:devel/elektra}{\texttt{ Open\+S\+U\+SE, Cent\+OS, Fedora, R\+H\+EL and S\+LE}} Kai-\/\+Uwe Behrmann kindly provides packages \href{http://software.opensuse.org/download.html?project=home%3Abekun%3Adevel&package=libelektra4}{\texttt{ for download}}.

To use the debian repository of the latest builds from master put following lines in {\ttfamily /etc/apt/sources.list}\+:

For Stretch\+:


\begin{DoxyCode}{0}
\DoxyCodeLine{deb     [trusted=yes] https://debian-stretch-repo.libelektra.org/ stretch main}
\DoxyCodeLine{deb-src [trusted=yes] https://debian-stretch-repo.libelektra.org/ stretch main}
\end{DoxyCode}


Which can also be done using\+:


\begin{DoxyCode}{0}
\DoxyCodeLine{sudo apt-get install apt-transport-https}
\DoxyCodeLine{echo "deb     [trusted=yes] https://debian-stretch-repo.libelektra.org/ stretch main" | sudo tee /etc/apt/sources.list.d/elektra.list}
\end{DoxyCode}


Or alternatively, you can use (if you do not mind many dependences just to add one line to a config file)\+:


\begin{DoxyCode}{0}
\DoxyCodeLine{sudo apt-get install software-properties-common apt-transport-https}
\DoxyCodeLine{sudo add-apt-repository "deb     [trusted=yes] https://debian-stretch-repo.libelektra.org/ stretch main"}
\end{DoxyCode}


For Jessie (not updated anymore, contains 0.\+8.\+24 packages which were created shortly before 0.\+8.\+25 release)


\begin{DoxyCode}{0}
\DoxyCodeLine{deb     [trusted=yes] https://debian-stable.libelektra.org/elektra-stable/ jessie main}
\DoxyCodeLine{deb-src [trusted=yes] https://debian-stable.libelektra.org/elektra-stable/ jessie main}
\end{DoxyCode}


For Wheezy (not updated anymore, contains 0.\+8.\+19-\/8121 packages)\+:


\begin{DoxyCode}{0}
\DoxyCodeLine{deb     [trusted=yes] https://build.libelektra.org/debian/ wheezy main}
\DoxyCodeLine{deb-src [trusted=yes] https://build.libelektra.org/debian/ wheezy main}
\end{DoxyCode}


To get all packaged plugins, bindings and tools install\+:


\begin{DoxyCode}{0}
\DoxyCodeLine{apt-get install libelektra4-all}
\end{DoxyCode}


For a small installation with command-\/line tools available use\+:


\begin{DoxyCode}{0}
\DoxyCodeLine{apt-get install elektra-bin}
\end{DoxyCode}


If you want to rebuild Elektra from Debian unstable or our repositories, add a {\ttfamily deb-\/src} entry to {\ttfamily /etc/apt/sources.list} and then run\+:


\begin{DoxyCode}{0}
\DoxyCodeLine{apt-get source -b elektra}
\end{DoxyCode}


To build Debian Packages from the source you might want to use\+:


\begin{DoxyCode}{0}
\DoxyCodeLine{dpkg-buildpackage -us -uc -sa}
\end{DoxyCode}


(You need to be in the Debian branch, see \mbox{\hyperlink{doc_GIT_md}{G\+IT}})

You can install Elektra using \href{http://brew.sh}{\texttt{ Homebrew}} via the shell command\+:


\begin{DoxyCode}{0}
\DoxyCodeLine{brew install elektra}
\end{DoxyCode}


. We also provide a tap containing a more elaborate formula \href{http://github.com/ElektraInitiative/homebrew-elektra}{\texttt{ here}}.

Please refer to the section OS Independent below.

First follow the steps in \mbox{\hyperlink{doc_COMPILE_md}{C\+O\+M\+P\+I\+LE}}.

After you completed building Elektra on your own, there are multiple options how to install it. For example, with make or c\+Pack tools.


\begin{DoxyCode}{0}
\DoxyCodeLine{sudo make install}
\DoxyCodeLine{sudo ldconfig  \# See troubleshooting below}
\end{DoxyCode}


To uninstall Elektra use (will not be very clean, e.\+g. it will not remove directories and {\ttfamily $\ast$.pyc} files)\+:


\begin{DoxyCode}{0}
\DoxyCodeLine{sudo make uninstall}
\DoxyCodeLine{sudo ldconfig}
\end{DoxyCode}


or in the build directory (will not honor {\ttfamily D\+E\+S\+T\+D\+IR}!)\+:


\begin{DoxyCode}{0}
\DoxyCodeLine{xargs rm < install\_manifest.txt}
\end{DoxyCode}


First follow the steps in \mbox{\hyperlink{doc_COMPILE_md}{C\+O\+M\+P\+I\+LE}}.

Then use\+:


\begin{DoxyCode}{0}
\DoxyCodeLine{cpack}
\end{DoxyCode}


which should create a package for distributions where a Generator is implemented. See \href{/home/mpranj/workspace/libelektra/scripts/cmake/ElektraPackaging.cmake}{\texttt{ this cmake file}} for available Generators and send a merge request for your system.

If you encounter the problem that the library can not be found (output like this)


\begin{DoxyCode}{0}
\DoxyCodeLine{kdb: error while loading shared libraries:}
\DoxyCodeLine{     libelektra-core.so.4: cannot open shared object file: No such file or directory}
\end{DoxyCode}


or\+:


\begin{DoxyCode}{0}
\DoxyCodeLine{kdb: error while loading shared libraries:}
\DoxyCodeLine{     libelektratools.so.2: cannot open shared object file: No such file or directory}
\end{DoxyCode}


you need to place a configuration file at {\ttfamily /etc/ld.so.\+conf.\+d/} (e.\+g. {\ttfamily /etc/ld.so.\+conf.\+d/elektra.conf}). Note that under Alpine Linux this file is called {\ttfamily /etc/ld-\/musl-\/x86\+\_\+64.path} or similar, depending on your architecture.

Add the path where the library has been installed (on Alpine Linux this had to be {\ttfamily usr/lib/elektra} for it to work)


\begin{DoxyCode}{0}
\DoxyCodeLine{/usr/lib/local/}
\end{DoxyCode}


and run {\ttfamily ldconfig} as root.

For some of the plugins and tools that ship with Elektra, additional installation manuals have been written. You can find them in the \mbox{\hyperlink{md_doc_tutorials_README_doc_tutorials_README_md}{tutorial overview}}.


\begin{DoxyItemize}
\item \mbox{\hyperlink{doc_COMPILE_md}{C\+O\+M\+P\+I\+LE}}.
\item \mbox{\hyperlink{doc_TESTING_md}{T\+E\+S\+T\+I\+NG}}. 
\end{DoxyItemize}