\subsection*{Structure of the Key Database}

The {\itshape key database} of Elektra is {\itshape hierarchically structured}. This means that keys are organized similar to directories in a file system.

Let us add some keys to the database. To add a key we can use {\ttfamily kdb}, the {\itshape key database access tool}\+:


\begin{DoxyCode}
kdb set <key> <value>
\end{DoxyCode}


Now add the the key $\ast$$\ast$/a$\ast$$\ast$ with the Value {\bfseries Value 1} and the key $\ast$$\ast$/b/c$\ast$$\ast$ with the Value {\bfseries Value 2}\+:


\begin{DoxyCode}
kdb set /a 'Value 1'
#> Using name user/a
#> Create a new key user/a with string "Value 1"
kdb set /b/c 'Value 2'
#> Using name user/b/c
#> Create a new key user/b/c with string "Value 2"
\end{DoxyCode}




Here you see the internal structure of the database after adding the keys $\ast$$\ast$/a$\ast$$\ast$ and $\ast$$\ast$/b/c$\ast$$\ast$. For instance the key $\ast$$\ast$/b/c$\ast$$\ast$ has the path $\ast$$\ast$/$\ast$$\ast$ -\/$>$ {\bfseries b} -\/$>$ {\bfseries c}.

Note how the name of the key determines the path to its value.

You can use the file system analogy as a mnemonic to remember these commands (like the file system commands in your favorite operating system)\+:


\begin{DoxyItemize}
\item {\ttfamily kdb ls $<$path$>$} lists keys below {\itshape path}
\item {\ttfamily kdb rm $<$key$>$} removes a {\itshape key}
\item {\ttfamily kdb cp $<$source$>$ $<$dest$>$} copies a key to another path
\item {\ttfamily kdb get $<$key$>$} gets the value of {\itshape key}
\end{DoxyItemize}

For example {\ttfamily kdb get /b/c} should return {\ttfamily Value 2} now, if you set the values before.

\subsection*{Namespaces}

Now we abandon the file system analogy and introduce the concept of {\itshape namespaces}.

Every key in Elektra belongs to one of these namespaces\+:


\begin{DoxyItemize}
\item {\bfseries spec} for specification of other keys
\item {\bfseries proc} for in-\/memory keys (e.\+g. command-\/line)
\item {\bfseries dir} for dir keys in current working directory
\item {\bfseries user} for user keys in home directory
\item {\bfseries system} for system keys in {\ttfamily /etc} or {\ttfamily /}
\end{DoxyItemize}

All namespaces save their keys in a {\itshape separate hierarchical structure} from the other namespaces.

But when we set the keys $\ast$$\ast$/a$\ast$$\ast$ and $\ast$$\ast$/b/c$\ast$$\ast$ before we didn\textquotesingle{}t provide a namespace. So I hear you asking, if every key has to belong to a namespace, where are the keys? They are in the {\itshape user} namespace, as you can verify with\+:


\begin{DoxyCode}
kdb ls user | grep -E '(/a|/b/c)'
#> user/a
#> user/b/c
\end{DoxyCode}


When we don\textquotesingle{}t provide a namespace Elektra assumes a default namespace, which should be {\bfseries user} for non-\/root users. So if you are a normal user the command {\ttfamily kdb set /b/c \textquotesingle{}Value 2\textquotesingle{}} was synonymous to {\ttfamily kdb set user/b/c \textquotesingle{}Value 2\textquotesingle{}}.

At this point the key database should have this structure\+: 

\subsubsection*{Cascading Keys}

Another question you may ask yourself now is, what happens if we lookup a key without providing a namespace. So let us retrieve the key $\ast$$\ast$/b/c$\ast$$\ast$ with the -\/v flag in order to make {\itshape kdb} more talkative.


\begin{DoxyCode}
kdb get -v /b/c
# STDOUT-REGEX: got \(\backslash\)d+ keys
#>  searching spec/b/c, found: <nothing>, options: KDB\_O\_CALLBACK
#>  searching proc/b/c, found: <nothing>, options:
#>  searching dir/b/c, found: <nothing>, options:
#>  searching user/b/c, found: user/b/c, options:
#> The resulting key name is user/b/c
#> Value 2
\end{DoxyCode}


Here you see how Elektra searches all namespaces for matching keys in this order\+: {\bfseries spec}, {\bfseries proc}, {\bfseries dir}, {\bfseries user} and finally {\bfseries system}

If a key is found in a namespace, it masks the key in all subsequent namespaces, which is the reason why the system namespace isn\textquotesingle{}t searched. Finally the virtual key $\ast$$\ast$/b/c$\ast$$\ast$ gets resolved to the real key {\bfseries user/b/c}. Because of the way a key without a namespace is retrieved, we call keys, that start with \textquotesingle{}$\ast$$\ast$/$\ast$$\ast$\textquotesingle{} {\bfseries cascading keys}. You can find out more about cascading lookups in the \hyperlink{doc_tutorials_cascading_md}{cascading tutorial}.

Having namespaces enables both admins and users to set specific parts of the application\textquotesingle{}s configuration, as you will see in the following example.

\subsection*{How it Works on the Command Line (kdb)}

We will provide an example of how you can configure \hyperlink{md_doc_help_elektra-glossary_doc_help_elektra-glossary_md}{elektrified} applications.

Our exemplary application will be the key database access tool {\ttfamily kdb} as this should already be installed on your system.

{\ttfamily kdb} can be configured by the following configuration data\+:


\begin{DoxyItemize}
\item \+\_\+/sw/elektra/kdb/\#$\ast$$\ast$\+X$\ast$$\ast$/$\ast$$\ast$\+P\+R\+O\+F\+I\+L\+E$\ast$$\ast$/verbose\+\_\+ -\/ sets the verbosity of kdb
\item \+\_\+/sw/elektra/kdb/\#$\ast$$\ast$\+X$\ast$$\ast$/$\ast$$\ast$\+P\+R\+O\+F\+I\+L\+E$\ast$$\ast$/quiet\+\_\+ -\/ if kdb should suppress non-\/error messages
\item \+\_\+/sw/elektra/kdb/\#$\ast$$\ast$\+X$\ast$$\ast$/$\ast$$\ast$\+P\+R\+O\+F\+I\+L\+E$\ast$$\ast$/namespace\+\_\+ -\/ specifies the default namespace used, when setting a cascading name
\end{DoxyItemize}

{\bfseries X} is a placeholder for the {\itshape major version number} and {\bfseries P\+R\+O\+F\+I\+LE} stands for the name of a {\itshape profile} to which this configuration applies. If we want to set configuration for the default profile we can set {\bfseries P\+R\+O\+F\+I\+LE} to \%. The name of the key follows the convention described \hyperlink{md_doc_help_elektra-key-names_doc_help_elektra-key-names_md}{here}.

Say we want to set {\ttfamily kdb} to be more verbose when it is used in the current directory. In this case we have to set {\itshape verbose} to 1 in the {\itshape dir} namespace of the current directory. 
\begin{DoxyCode}
kdb set "dir/sw/elektra/kdb/#0/%/verbose" 1
#> Create a new key dir/sw/elektra/kdb/#0/%/verbose with string "1"
\end{DoxyCode}
 \begin{quote}
The configuration for a directory is actually stored in this directory. By default the configuration is contained in a folder named {\ttfamily .dir}, as you can verify with {\ttfamily kdb file dir} ({\itshape kdb file} tells you the file where a key is stored in).

For the purpose of demonstration we chose to only manipulate the verbosity of kdb. Note that setting {\ttfamily dir/sw/elektra/kdb/\#0/\%/namespace} to {\ttfamily dir} can be handy if you want to work with configuration of an application in a certain directory. \end{quote}


If we now search for some key, {\ttfamily kdb} will behave just as if we have called it with the {\ttfamily -\/v} option. 
\begin{DoxyCode}
kdb get /some/key
# STDOUT-REGEX: got \(\backslash\)d+ keys
#> searching spec/some/key, found: <nothing>, options: KDB\_O\_CALLBACK
#>     searching proc/some/key, found: <nothing>, options:
#>     searching dir/some/key, found: <nothing>, options:
#>     searching user/some/key, found: <nothing>, options:
#>     searching system/some/key, found: <nothing>, options:
#>     searching default of spec/some/key, found: <nothing>, options: KDB\_O\_NOCASCADING
#> Did not find key
\end{DoxyCode}


Verbosity is not always useful because it distracts from the essential. So we may decide that we want {\ttfamily kdb} to be only verbose if we are debugging it. So let us move the default configuration to another profile\+: 
\begin{DoxyCode}
kdb mv -r "dir/sw/elektra/kdb/#0/%" "dir/sw/elektra/kdb/#0/debug"
#> using common basename: dir/sw/elektra/kdb/#0
#> key: dir/sw/elektra/kdb/#0/%/verbose will be renamed to: dir/sw/elektra/kdb/#0/debug/verbose
#> Will write out:
#> dir/sw/elektra/kdb/#0/debug/verbose
\end{DoxyCode}


If we now call {\ttfamily kdb get /some/key} it will behave non-\/verbose, but if we call it with the {\itshape debug} profile {\ttfamily kdb get -\/p debug /some/key} the configuration under $\ast$$\ast$/sw/elektra/kdb/\#0/debug$\ast$$\ast$ applies.

We configured kdb only for the current directory. If we like this configuration we could move it to the system namespace, so that every user can enjoy a preconfigured {\itshape debug} profile. 
\begin{DoxyCode}
sudo kdb mv -r "dir/sw/elektra/kdb" "system/sw/elektra/kdb"
#> using common basename: /sw/elektra/kdb
#> key: dir/sw/elektra/kdb/#0/%/verbose will be renamed to: system/sw/elektra/kdb/#0/%/verbose
#> Will write out:
#> system/sw/elektra/kdb/#0/%/verbose
\end{DoxyCode}


Now every user can use the {\itshape debug} profile with kdb.

{\itshape Cascading keys} are keys that start with $\ast$$\ast$/$\ast$$\ast$ and are a way of making key lookups much easier. Let\textquotesingle{}s say you want to see the configuration from the example above. You do not need to search every namespace by yourself. Just make a lookup for $\ast$$\ast$/sw/elektra/kdb/\#0/debug/verbose$\ast$$\ast$, like this\+:


\begin{DoxyCode}
kdb get "/sw/elektra/kdb/#0/debug/verbose"
#> 1
\end{DoxyCode}


When using cascading key the best key will be searched at run-\/time. If you are only interested in the system key, you would use\+:


\begin{DoxyCode}
kdb get "system/sw/elektra/kdb/#0/debug/verbose"
#> 1
\end{DoxyCode}


Because of {\itshape cascading keys} a user can override the behavior of the {\itshape debug} profile by setting the corresponding keys in his {\itshape user} namespace (as we discussed \href{#cascading-keys}{\tt before}). If a user sets {\itshape verbose} in his user namespace to 0 she overrides the default behavior from the {\itshape system} namespace.


\begin{DoxyCode}
kdb set "user/sw/elektra/kdb/#0/debug/verbose" 0
#> Create a new key user/sw/elektra/kdb/#0/debug/verbose with string "0"
kdb get "/sw/elektra/kdb/#0/debug/verbose"
#> 0
\end{DoxyCode}


Now {\ttfamily kdb get -\/p debug /some/key} is not verbose anymore for this user. 