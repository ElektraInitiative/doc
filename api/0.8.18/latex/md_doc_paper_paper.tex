

 title\+: 'Elektra\+: universal framework to access configuration parameters' tags\+:
\begin{DoxyItemize}
\item configuration
\item context awareness
\item configuration files
\item interception
\item integration authors\+:
\end{DoxyItemize}

name\+: Markus Raab orcid\+: 0000-\/0002-\/1493-\/9065 affiliation\+: T\+U Wien (T\+U\+W) date\+: 23 July 2016 \subsection*{bibliography\+: paper.\+bib }\hypertarget{md_doc_paper_paper_doc_paper_paper_md}{}\section{Summary}\label{md_doc_paper_paper_doc_paper_paper_md}
Applications today typically directly work with custom configuration files. Up to now, it was difficult to integrate and specify the configuration of such applications. Elektra \mbox{[}\mbox{]} provides a framework to bridge this gap \mbox{[}\mbox{]}.

The research value of Elektra is two-\/fold\+: First Elektra allows us to intercept unmodified applications \mbox{[}\mbox{]} and extend their configuration access with the features Elektra has \mbox{[}\mbox{]}. Second Elektra provides a plugin framework \mbox{[}\mbox{]} where researchers can experiment with new configuration features. Elektra allows us to directly apply novel ideas on many existing applications for case studies. Furthermore a code generator \mbox{[}\mbox{]} simplifies writing new applications \mbox{[}\mbox{]}.

\section*{References}