This Vagrantfile provisions a machine based on debian/jessie64 with all tools to build Elektra.

If \href{https://www.vagrantup.com/}{\tt Vagrant} is installed on your machine change to the directory containing the file you are currently reading and build a box for vagrant with 
\begin{DoxyCode}
$ vagrant up && vagrant package && vagrant box add buildelektra package.box && vagrant destroy -f
\end{DoxyCode}
 This will take some time, but when its done you have a new vagrantbox as you can verify with {\ttfamily vagrant box list}. Amongst your boxes you should see the box {\ttfamily buildelektra}.

You can now set up a new VM from this box easily\+: Enter a directory where you want to set up the VM 
\begin{DoxyCode}
# enter a directory where you want to set up the VM
$ mkdir ~/vagrant/buildelektra && cd $\_
# now you create a Vagrantfile ...
$ vagrant init buildelektra
# ... and start the VM
$ vagrant up
\end{DoxyCode}


When the machine is running, access it with 
\begin{DoxyCode}
$ vagrant ssh
\end{DoxyCode}
 In this S\+SH session you can interact with the machine.

By default Vagrant synchronizes the folder on the host machine containing the vagrantfile with the folder {\ttfamily /vagrant} in the VM. Therefore we will build a .deb package of Elektra in this folder.


\begin{DoxyCode}
# in the VM change to the synched folder
$ cd /vagrant
# build the commit with the tag "0.8.19"
$ sudo buildelektra -b elektra 0.8.19
\end{DoxyCode}


When you are done leave the VM with {\ttfamily C\+T\+R\+L-\/D}. The folder should now contain the created .deb file.

You can either shut the VM down with {\ttfamily vagrant halt} or delete it with {\ttfamily vagrant destroy}. 