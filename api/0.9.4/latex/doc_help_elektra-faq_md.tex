If this F\+AQ does not contain your question, \href{https://git.libelektra.org/issues}{\tt please open an issue}. You can simply remove all template text and it is enough if the issue only contains your question. If you want you can \href{https://git.libelektra.org/issues/labels/question}{\tt label it as question}, but we can also categorize it for you.

Please do not waste too much time to find something out yourself. Information where people get stuck is valuable to improve Elektra and its documentation. Even if you find out directly after you posted the question\+: the pointer can be helpful for other people having the same problem.

No, it is not. And even if it were, the story does not end with implementing a configuration file parser but, at least, you also need\+:


\begin{DoxyItemize}
\item operating-\/system-\/specific code to locate configuration files
\item tools to change the configuration files
\item validation to make such changes user friendly
\end{DoxyItemize}

Every successful project has implemented many features Elektra has. But Elektra has the distinctive advantage that you can pick the features as you need them. Not used plugins do not cause any overhead or dependency. If you need new plugins or bindings, there is a community which can help you. Furthermore, Elektra has a defined A\+PI and you can swap implementations as needed.

So it pays off to use Elektra -- in the short and in the long term.

Short answer\+: Try \href{https://puppet.libelektra.org}{\tt puppet-\/libelektra} to see how useful it can be.

Longer answer\+:

Elektra abstracts \hyperlink{doc_help_elektra-glossary_md}{configuration settings}, something desperately needed within configuration management. Instead of rewriting complete configuration files, which might create security problems, Elektra operates precisely on the configuration setting you want to change\+: leaving others as chosen by the application or distribution. Furthermore, Elektra also allows us to {\itshape specify} configuration settings, which again brings benefits for configuration management tools.

Elektra is a radical step needed towards better configuration management\+: Let us fix how applications access configuration settings, so that we can properly access them, for example, from configuration management tools.

As an intermediate step, we can \hyperlink{doc_help_elektra-mounting_md}{mount} existing configuration files and operate on them.

Absolutely, you can think of libelektra as a small library in C that reads a configuration file and returns a data structure if you do not use any of its advanced features.

In fact, from the view of system-\/calls, a properly written configuration parser within your application would do exactly the same as Elektra does\+:


\begin{DoxyCode}
stat("/etc/kdb/elektra.ecf", \{st\_mode=S\_IFREG|0644, st\_size=1996, ...\}) = 0
open("/etc/kdb/elektra.ecf", O\_RDONLY)  = 3
read(3, "..., 8191) = 1996
close(3)                                = 0
\end{DoxyCode}


Writing configuration files is much more tricky, as Elektra avoids data loss in the case of concurrent writes, even if the other application does not use Elektra (this only works on modern file systems with high-\/resolution timestamps).

Elektra does not interfere with restarting. It is a passive library. It provides some techniques for reloading but they are optional. We recommend, however, that you keep the in-\/memory and persistent configuration always in sync via immediate writes on changes and immediate reloading after notifications.

In case of doubt \href{https://git.libelektra.org/issues}{\tt please open an issue}. If the question was already answered or is already in the documentation, we will simply point it out to you.

So do not worry too much, do not hesitate to ask any question. We welcome feedback, only then we can improve the documentation such as this F\+A\+Q!

Please check the \href{https://git.libelektra.org/issues}{\tt issue tracker} if it has already been reported. If it has not, please \href{https://git.libelektra.org/issues/new}{\tt fill out the template}. If you are in doubt, please report it.

Due to the modular architecture we can accept many contributions as plugins. Please only make sure that the R\+E\+A\+D\+M\+E.\+md clearly states the purpose and quality of the plugin.

Please start by reading here.

New B\+SD license which allows us to have plugins link against G\+PL and G\+P\+L-\/incompatible libraries. If you compile Elektra, e.\+g., with G\+PL plugins, the result is G\+PL.

If you already use 0.\+8, you might want to continue using it until 1.\+0 is released. As announced \hyperlink{doc_news_2019-08-06_0_9_0_md}{here}, the 0.\+9 series introduces incompatible changes as needed, in particular cleanup for the 1.\+0 release is done.

For details of versioning principles, see \hyperlink{doc_VERSION_md}{here}.

\hyperlink{doc_AUTHORS_md}{List of authors}.


\begin{DoxyItemize}
\item If questions about configuration come up, point users to \href{https://www.libelektra.org}{\tt https\+://www.\+libelektra.\+org}
\item Display the S\+VG logos found at \href{https://master.libelektra.org/doc/images/logo}{\tt https\+://master.\+libelektra.\+org/doc/images/logo}
\item and already rastered logos at \href{https://github.com/ElektraInitiative/blobs/tree/master/images/logos}{\tt https\+://github.\+com/\+Elektra\+Initiative/blobs/tree/master/images/logos}
\item Distribute the flyer found at \href{https://github.com/ElektraInitiative/blobs/raw/master/flyers/flyer.odt}{\tt https\+://github.\+com/\+Elektra\+Initiative/blobs/raw/master/flyers/flyer.\+odt}
\item And of course\+: talk about it! 
\end{DoxyItemize}