This document describes how and when to write a language binding for the high-\/level A\+PI.

Writing bindings for high-\/level A\+PI is different from writing bindings for other parts of Elektra. This is mainly because the high-\/level A\+PI has different goals.

The goals of any high-\/level A\+PI (or binding to the C high-\/level A\+PI) for Elektra should be\+:


\begin{DoxyItemize}
\item {\bfseries{Type Safety\+:}} The A\+PI should use Elektra\textquotesingle{}s type system as dictated by the {\ttfamily type} plugin. The various types shall be mapped to native types of the target language. All A\+PI calls interacting with keys should reflect the type of the key. Type mismatches should produce errors.
\item {\bfseries{Easy to Use\+:}} The A\+PI should be easy to use and abstract as much of the low-\/level A\+PI as possible. The main part of the A\+PI should not consists of more than an initialization method, typed {\ttfamily get} and {\ttfamily set} calls and if required by the language a method freeing the acquired resources.
\item {\bfseries{No Errors in Getters\+:}} It is a stated goal of our high-\/level A\+PI to make {\ttfamily get} calls not able to fail. This means {\ttfamily get} calls {\itshape cannot} return an error under normal conditions. This can be ensure via checks during initialization or via code-\/generation. Both are based on the specification. Without a specification it is not possible to prevent errors because of missing keys, since we have no way of knowing which keys should exist. ~\newline
~\newline
 There is an exception to this rule. Some target languages have a standard error concept, which not only forces the user to handle arising errors, but also does so in a simple and concise way. An example of this is Rust. Its {\ttfamily Result} type forces the user to handle errors and the {\ttfamily ?} operator allows to do this in a concise way. ~\newline
 However, we still recommend to avoid errors as far as possible.
\item {\bfseries{Idiomatic Code\+:}} If the target language has a concept of \char`\"{}idiomatic code\char`\"{}, the A\+PI should fall into that category.
\end{DoxyItemize}

Based on the goals above, you should decide, whether it is possible to write a binding for the C high-\/level A\+PI while preserving the goals.

If you decide to write a binding, proceed with this tutorial.

If not, designing an A\+PI to meet our goals is up to you. While designing the A\+PI you should keep in mind that you can always use code-\/generator ({\ttfamily kdb gen}) templates like the C A\+PI does.

Since the C high-\/level A\+PI consists of a shared library A\+PI and a code-\/generated part, your binding will also have these two parts.

Writing the binding for the {\ttfamily highlevel} library works the same way as writing a language binding for any part of Elektra. The only additional challenge is that the binding should still meet the same goals and (as far as possible) fulfill the same guarantees as the C A\+PI.

Some languages like to split bindings into two parts. One version that maps the C A\+PI one-\/to-\/one and another part that builds on the first one. The second part then uses the languages additional features. Since our A\+PI isn\textquotesingle{}t intended to be used directly, but through generated code, it might not be necessary to write the second part. It may be sufficient to write (or generate) a one-\/to-\/one mirror of the C A\+PI and use that in the code-\/generator template.

The generated code should still be understandable. So if writing a more idiomatic A\+PI on top of the direct mapping, significantly simplifies the generated code (and maybe also the template), you should write such an A\+PI.

Your programming language of choice must provide a way to call into C code (like \href{https://golang.org/cmd/cgo/}{\texttt{ cgo}}).

In general we prefer (in this order)\+:


\begin{DoxyEnumerate}
\item Automatically and statically generated bindings (like \href{http://www.swig.org/}{\texttt{ swig}}).
\item Manually written bindings that directly call to library without any statically build part (like J\+NI and GI). They are a bit slower but can directly access libelektra.\+so.
\item Manually written bindings that are built statically.
\end{DoxyEnumerate}

If you want to manually write a binding make sure you have a good understanding of the possible limitations the interop layer can have (e.\+g. variadic functions, freeing of resources, ...).

What you will also need is to set up the compiler + linker flags. For this we recommend \href{https://www.freedesktop.org/wiki/Software/pkg-config/}{\texttt{ pkg-\/config}}, because Elektra already provides {\ttfamily .pc} (and {\ttfamily cmake}) files.

For garbage collected (GC) languages freeing memory by hand is not something you usually do and since the GC has no knowledge of memory allocated in C we have two options\+:


\begin{DoxyItemize}
\item Forcing the user to call the appropriate functions like {\ttfamily \mbox{\hyperlink{group__key_ga3df95bbc2494e3e6703ece5639be5bb1}{key\+Del()}}} themselves, which is very developer unfriendly and error prone or
\item Using language features like Java / C++ Destructors or Go\textquotesingle{}s {\ttfamily runtime.\+Set\+Finalizer()} function to automatically and reliably release the memory as soon as the native objects are garbage collected.
\end{DoxyItemize}

If you decide on mapping the functionality of kdb.h 1\+:1 it is pretty straightforward -\/ whereas if you want to adapt or enhance some A\+PI\textquotesingle{}s to leverage language features like iterators or operator overloading feel free to do so. The less “alien” the binding feels to its users the better.

Remember that Elektra has internal iterators (for metadata+keysets) but in general we prefer external iterators by either copying the Key\+Set per iterator or using {\ttfamily ks\+At\+Cursor}.

Some languages for example cannot call variadic functions because in C the amount of parameters has to be known at compile-\/time. In Go for example this is not the case since it supports variable length arguments at runtime with the {\ttfamily ...} operator.

This is unfortunate because the low-\/level bindings rely heavily on variadic functions. It is possible to work around this problem by either


\begin{DoxyEnumerate}
\item Writing helper functions in C that call these variadic functions with a fixed amount of parameters or
\item Imitating the behavior of a function e.\+g. {\ttfamily \mbox{\hyperlink{group__key_gad23c65b44bf48d773759e1f9a4d43b89}{key\+New()}}} by calling multiple functions\+: {\ttfamily \mbox{\hyperlink{group__key_gad23c65b44bf48d773759e1f9a4d43b89}{key\+New()}}} and {\ttfamily \mbox{\hyperlink{group__keymeta_gae1f15546b234ffb6007d8a31178652b9}{key\+Set\+Meta()}}} for every metakey/value that was passed.
\end{DoxyEnumerate}

How to create a template for {\ttfamily kdb gen} is detailed in \mbox{\hyperlink{doc_tutorials_code-generator_md}{this tutorial}}.

The created template should support the same input keysets (and parent keys) as the {\ttfamily highlevel} template. If and how exactly you implement the advanced features (structs, unions, ...) is up to you. Note\+: enums should always be supported (if your language has them) as they are one of the {\ttfamily type} plugins types.

For example, in C++ the generated code could consist of nested structs with overloaded operators. In that case the structs features doesn\textquotesingle{}t really make sense, since everything is already structs. Of course you could reuse some of the {\ttfamily gen/$\ast$} metadata to allow some cross-\/compatibility. 