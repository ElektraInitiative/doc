Each plugin in a backend can cause run-\/time errors. Additionally, the chaining of the plugins can introduce further run-\/time errors. For example, a plugin can modify keys so that the next plugin cannot process these keys anymore. Or a plugin can omit changes to the keys that are required by the next plugin. To deal with such situations in a controlled way, each plugin exports a contract that describes the interaction with other parts of the backend. A {\ttfamily Key\+Set} contains the description.

Mounting of backends actually takes place at run-\/time, we will refer to it as mount-\/time. The time when applications access the key database, however, will be called run-\/time.

The contract checker revises contracts of plugins during the mount-\/time. Afterwards, at run-\/time, no such type errors can occur. {\ttfamily kdb mount} implements the contract checker. It can refuse to add a plugin to the backend because of a conflict or constraint. As long as not all contracts are satisfied {\ttfamily kdb mount} waits for more plugins to be attached.

The content of the contract is well-\/specified in \href{/home/markus/Projekte/Elektra/current/doc/CONTRACT.ini}{\tt C\+O\+N\+T\+R\+A\+C\+T.\+ini}.

Plugins are the only place to export contracts, but they are not the only party. It is also possible that the administrator extends the requirements to every backend, for example, if notification is required. This implies that every backend has to provide a specific additional service that will be checked using contracts. Plugins are also involved in contracts with Elektra’s core. This topic will be discussed in \hyperlink{doc_dev_algorithm_md}{elektra-\/algorithm} in which the algorithm used by Elektra’s core is explained.

\subsection*{Assertions}

Contracts as introduced by Meyer define preconditions and postconditions on routines. Assertions handle these conditions by checking them before entering or after leaving a routine. Because most of this can happen only at run time, it often leaves the user alone with an exception.

Elektra goes a step further. Instead of exiting the normal control flow when a precondition is not met, it is the responsibility of a special plugin to handle the situation and make sure that the precondition is met afterwards. Sometimes this is not possible. In these cases, the plugins check the necessary conditions and return with error code when they are not met. As we will see, these situations do not occur very often.

Because of the modular approach, we can have several checkers, and correcting and checking can be combined. Plugins can work together to reach a certain goal.

{\ttfamily Key\+Set} and {\ttfamily Key} already handle most parts of checking pre-\/ and postconditions imposed on data structures. Elektra’s core provides preconditions and weak postconditions for the plugins.

\subsection*{S\+EE A\+L\+SO}


\begin{DoxyItemize}
\item \href{/home/markus/Projekte/Elektra/current/doc/CONTRACT.ini}{\tt C\+O\+N\+T\+R\+A\+C\+T.\+ini}
\item \hyperlink{doc_dev_plugins-ordering_md}{plugins-\/ordering} 
\end{DoxyItemize}