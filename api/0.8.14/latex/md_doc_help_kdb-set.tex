{\ttfamily kdb set $<$key-\/name$>$ \mbox{[}$<$value$>$\mbox{]}}

Where {\ttfamily key-\/name} is the path to the key you wish to set the value of (or create) and {\ttfamily value} is the value you would like to set the key to. If the {\ttfamily value} argument is not passed, the key will be set to a value of {\ttfamily null}.

\subsection*{D\+E\+S\+C\+R\+I\+P\+T\+I\+O\+N}

This command allows the user to set the value of an individual key.

\subsection*{E\+M\+P\+T\+Y V\+A\+L\+U\+E\+S}

To set a key to an empty value, {\ttfamily \char`\"{}\char`\"{}} should be passed for the {\ttfamily value} argument.

\subsection*{O\+P\+T\+I\+O\+N\+S}


\begin{DoxyItemize}
\item {\ttfamily -\/\+H}, {\ttfamily -\/-\/help}\+: Show the man page.
\item {\ttfamily -\/\+V}, {\ttfamily -\/-\/version}\+: Print version info.
\item {\ttfamily -\/v}, {\ttfamily -\/-\/verbose}\+: Explain what is happening.
\item {\ttfamily -\/\+N}, {\ttfamily -\/-\/namespace ns}\+: Specify the namespace to use when writing cascading keys Default\+: value of {\ttfamily /sw/kdb/current/namespace} or user.
\end{DoxyItemize}

\subsection*{K\+D\+B}


\begin{DoxyItemize}
\item {\ttfamily /sw/kdb/current/namespace}\+: Specifies which default namespace should be used when setting a cascading name. Note, that as root you can set {\ttfamily user/sw/kdb/current/namespace} to {\ttfamily system} to get the expected default. (by default the namespace is user)
\end{DoxyItemize}

\subsection*{E\+X\+A\+M\+P\+L\+E\+S}

To set a Key to the value {\ttfamily Hello World!}\+: {\ttfamily kdb set user/example/key \char`\"{}\+Hello World!\char`\"{}}

To create a new key with a null value\+: {\ttfamily kdb set user/example/key}

To set a key to an empty value\+: {\ttfamily kdb set user/example/key \char`\"{}\char`\"{}}

\subsection*{S\+E\+E A\+L\+S\+O}


\begin{DoxyItemize}
\item For difference between empty and null values, see \hyperlink{md_doc_help_elektra-values_doc_help_elektra-values_md}{elektra-\/values(7)} 
\end{DoxyItemize}