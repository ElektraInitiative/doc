
\begin{DoxyItemize}
\item infos = Information about the uname plugin is in keys below
\item infos/author = Markus Raab \href{mailto:elektra@libelektra.org}{\tt elektra@libelektra.\+org}
\item infos/licence = B\+SD
\item infos/provides = storage/info
\item infos/needs =
\item infos/placements = getstorage setstorage
\item infos/status = maintained unittest shelltest nodep readonly limited concept
\item infos/description = Includes uname information into the key database.
\end{DoxyItemize}

This plugin is a storage plugin that will use the syscall {\ttfamily uname (2)}. No resolver is needed for that plugin to work.

This plugin defines following keynames below its mount point\+:


\begin{DoxyItemize}
\item sysname
\item nodename
\item release
\item version
\item machine
\end{DoxyItemize}

This plugin is read-\/only.


\begin{DoxyCode}
# To mount uname information using this plugin:
kdb mount -R noresolver none user:/tests/uname uname

# List available data
kdb ls user:/tests/uname/
#> user:/tests/uname/machine
#> user:/tests/uname/nodename
#> user:/tests/uname/release
#> user:/tests/uname/sysname
#> user:/tests/uname/version

# Read the OS name
kdb get user:/tests/uname/sysname
# STDOUT-REGEX: CYGWIN\_NT.*|Darwin|DragonFly|FreeBSD|Linux|OpenBSD

# Read the OS version number
kdb get user:/tests/uname/release
# STDOUT-REGEX: [0-9]+(\(\backslash\).[0-9]+)*[[:alnum:][:punct:]]*

# Unmount the plugin
kdb umount user:/tests/uname
\end{DoxyCode}
 