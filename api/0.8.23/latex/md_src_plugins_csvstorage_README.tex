
\begin{DoxyItemize}
\item infos = Information about the csvstorage plugin is in keys below
\item infos/author = Thomas Waser \href{mailto:thomas.waser@libelektra.org}{\tt thomas.\+waser@libelektra.\+org}
\item infos/licence = B\+SD
\item infos/provides = storage/csv
\item infos/needs =
\item infos/recommends =
\item infos/placements = getstorage setstorage
\item infos/status = maintained unittest nodep libc configurable limited nodoc
\item infos/description = parses csv files
\end{DoxyItemize}

This plugin allows you to read and write C\+SV files within Elektra. It aims to be compatible with R\+FC 4180. Rows and columns are written in Elektra arrays ({\ttfamily \#0}, {\ttfamily \#1},..). Using configuration you can give columns a name.

\subsection*{Configuration}

{\ttfamily delimiter} Tells the plugin what delimiter is used in the file. The default delimiter is {\ttfamily ,} and will be used if {\ttfamily delimiter} is not set.

{\ttfamily header} Tells the plugin to use the first line as a header if it\textquotesingle{}s set to \char`\"{}colname\char`\"{}. The columns will get the corresponding names. Skip the first line if it\textquotesingle{}s set to \char`\"{}skip\char`\"{} or treat the first line as a record if it\textquotesingle{}s set to \char`\"{}record\char`\"{}. If {\ttfamily header} is not set, or set to \char`\"{}record\char`\"{}, the columns get named \#0,\#1,... (array key naming)

{\ttfamily columns} If this key is set the plugin will yield an error for every file that doesn\textquotesingle{}t have exactly the amount of columns as specified in {\ttfamily columns}.

{\ttfamily columns/names} Sets the column names. Only usable in combination with the {\ttfamily columns} key. The number of subkeys must match the number of columns. Conflicts with usage of {\ttfamily header}.

{\ttfamily columns/index} specify which column should be used to index records instead of the record number.

{\ttfamily /export/$<$column name$>$} only export column {\ttfamily column name}. Configuration can be give multiple times for different columns to export.

\subsection*{Examples}

First line should determine the headers\+: \begin{DoxyVerb}kdb mount test.csv /csv csvstorage "delimiter=;,header=colname,columns=2,columns/names,columns/names/#0=col0Name,columns/names/#1=col1Name"
\end{DoxyVerb}


\subsubsection*{Usage}

The example below shows how you can use this plugin to read and write C\+SV files.

``{\ttfamily  $<$h1$>$Mount plugin to cascading namespace}/examples/csv` \section*{We use the column names from the first line of the}

\section*{config file as key names}

sudo kdb mount config.\+csv /examples/csv csvstorage \char`\"{}header=colname,columns/names/\#0=col0\+Name,columns/names/\#1=col1\+Name\char`\"{}

\section*{Add some data}

printf \textquotesingle{}band,album~\newline
\textquotesingle{} $>$$>$ {\ttfamily kdb file /examples/csv} printf \textquotesingle{}Converge,All We Love We Leave Behind~\newline
\textquotesingle{} $>$$>$ {\ttfamily kdb file /examples/csv} printf \textquotesingle{}mewithout\+You,Pale Horses~\newline
\textquotesingle{} $>$$>$ {\ttfamily kdb file /examples/csv} printf \textquotesingle{}Kate Tempest,Everybody Down~\newline
\textquotesingle{} $>$$>$ {\ttfamily kdb file /examples/csv}

kdb ls /examples/csv \#$>$ user/examples/csv/\#0 \#$>$ user/examples/csv/\#0/album \#$>$ user/examples/csv/\#0/band \#$>$ user/examples/csv/\#1 \#$>$ user/examples/csv/\#1/album \#$>$ user/examples/csv/\#1/band \#$>$ user/examples/csv/\#2 \#$>$ user/examples/csv/\#2/album \#$>$ user/examples/csv/\#2/band \#$>$ user/examples/csv/\#3 \#$>$ user/examples/csv/\#3/album \#$>$ user/examples/csv/\#3/band

\section*{The first array element contains the column names}

kdb get /examples/csv/\#0/band \#$>$ band kdb get /examples/csv/\#0/album \#$>$ album

\section*{Retrieve data from the last entry}

kdb get /examples/csv/\#3/album \#$>$ Everybody Down kdb get /examples/csv/\#3/band \#$>$ Kate Tempest

\section*{Change an existing item}

kdb set /examples/csv/\#1/album \textquotesingle{}You Fail Me\textquotesingle{} \section*{Retrieve the new item}

kdb get /examples/csv/\#1/album \#$>$ You Fail Me

\section*{The plugin stores the index of the last column}

\section*{in all of the parent keys.}

kdb get user/examples/csv/\#0 \#$>$ \#1 kdb get user/examples/csv/\#1 \#$>$ \#1 kdb get user/examples/csv/\#2 \#$>$ \#1 kdb get user/examples/csv/\#3 \#$>$ \#1

\section*{The configuration file reflects the changes}

kdb file /examples/csv $\vert$ xargs cat \#$>$ album,band \#$>$ You Fail Me,Converge \#$>$ Pale Horses,mewithout\+You \#$>$ Everybody Down,Kate Tempest

\section*{Undo changes to the key database}

kdb rm -\/r /examples/csv sudo kdb umount /examples/csv 
\begin{DoxyCode}
# Directory Values

By default the `csvstorage` plugin saves the name of the last column in each parent key. If you want to
       store a different value, you can do
so using the [Directory Value](@ref src\_plugins\_directoryvalue\_README\_md) plugin.
\end{DoxyCode}
 \section*{Mount plugin together with {\ttfamily directoryvalue} to cascading namespace {\ttfamily /examples/csv}}

kdb mount config.\+csv /examples/csv csvstorage directoryvalue

\section*{Add some data}

printf \textquotesingle{}Schindler’s List,1993,8.\+9~\newline
\textquotesingle{} $>$$>$ {\ttfamily kdb file /examples/csv} printf \textquotesingle{}Léon\+: The Professional,1994,8.\+5~\newline
\textquotesingle{} $>$$>$ {\ttfamily kdb file /examples/csv}

\section*{Retrieve data}

kdb ls /examples/csv \#$>$ user/examples/csv/\#0 \#$>$ user/examples/csv/\#0/\#0 \#$>$ user/examples/csv/\#0/\#1 \#$>$ user/examples/csv/\#0/\#2 \#$>$ user/examples/csv/\#1 \#$>$ user/examples/csv/\#1/\#0 \#$>$ user/examples/csv/\#1/\#1 \#$>$ user/examples/csv/\#1/\#2

kdb get user/examples/csv/\#0/\#0 \#$>$ Schindler’s List kdb get user/examples/csv/\#1/\#2 \#$>$ 8.\+5

\section*{The plugin stores the index of the last column in the parent keys}

kdb get user/examples/csv/\#0 \#$>$ \#2 kdb get user/examples/csv/\#1 \#$>$ \#2

\section*{Since we use the Directory Value plugin we can also change the data in a parent key}

kdb set user/examples/csv/\#0 \textquotesingle{}Movie – Year – Rating\textquotesingle{} kdb set user/examples/csv/\#1 \textquotesingle{}It’s a Me.\textquotesingle{}

\section*{Retrieve data stored in parent keys}

kdb get user/examples/csv/\#0 \#$>$ Movie – Year – Rating kdb get user/examples/csv/\#1 \#$>$ It’s a Me.

\section*{Undo changes to the key database}

kdb rm -\/r /examples/csv sudo kdb umount /examples/csv 
\begin{DoxyCode}
# Column as index
\end{DoxyCode}
 kdb mount config.\+csv /examples/csv csvstorage \char`\"{}delimiter=;,header=colname,columns/index=\+I\+M\+D\+B\char`\"{}

printf \textquotesingle{}I\+M\+DB;Title;Year~\newline
\textquotesingle{} $>$$>$ {\ttfamily kdb file /examples/csv} printf \textquotesingle{}tt0108052;Schindler´s List;1993~\newline
\textquotesingle{} $>$$>$ {\ttfamily kdb file /examples/csv} printf \textquotesingle{}tt0110413;Léon\+: The Professional;1994~\newline
\textquotesingle{} $>$$>$ {\ttfamily kdb file /examples/csv}

kdb ls /examples/csv \#$>$ user/examples/csv/tt0108052 \#$>$ user/examples/csv/tt0108052/\+I\+M\+DB \#$>$ user/examples/csv/tt0108052/\+Title \#$>$ user/examples/csv/tt0108052/\+Year \#$>$ user/examples/csv/tt0110413 \#$>$ user/examples/csv/tt0110413/\+I\+M\+DB \#$>$ user/examples/csv/tt0110413/\+Title \#$>$ user/examples/csv/tt0110413/\+Year

kdb get /examples/csv/tt0108052/\+Title \#$>$ Schindler´s List

kdb rm -\/r /examples/csv sudo kdb umount /examples/csv


\begin{DoxyCode}
# Export filter
\end{DoxyCode}
 kdb mount config.\+csv /examples/csv csvstorage \char`\"{}delimiter=;,header=colname,columns/index=\+I\+M\+D\+B\char`\"{}

printf \textquotesingle{}I\+M\+DB;Title;Year~\newline
\textquotesingle{} $>$$>$ {\ttfamily kdb file /examples/csv} printf \textquotesingle{}tt0108052;Schindler´s List;1993~\newline
\textquotesingle{} $>$$>$ {\ttfamily kdb file /examples/csv} printf \textquotesingle{}tt0110413;Léon\+: The Professional;1994~\newline
\textquotesingle{} $>$$>$ {\ttfamily kdb file /examples/csv}

kdb export /examples/csv csvstorage -\/c \char`\"{}delimiter=;,header=colname,columns/index=\+I\+M\+D\+B,export=,export/\+I\+M\+D\+B=,export/\+Title=\char`\"{} \#$>$ I\+M\+DB;Title \#$>$ tt0108052;Schindler´s List \#$>$ tt0110413;Léon\+: The Professional

kdb export /examples/csv csvstorage -\/c \char`\"{}delimiter=;,header=colname,columns/index=\+I\+M\+D\+B,export=,export/\+I\+M\+D\+B=,export/\+Year=\char`\"{} \#$>$ I\+M\+DB;Year \#$>$ tt0108052;1993 \#$>$ tt0110413;1994

kdb rm -\/r /examples/csv sudo kdb umount /examples/csv

```

\subsection*{Limitations}


\begin{DoxyItemize}
\item Does not work on file streams (e.\+g. {\ttfamily kdb import} without file) 
\end{DoxyItemize}