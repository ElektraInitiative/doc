The use of arbitrary metadata has extensive effects in Elektra's semantics. They become simpler and more suited to carry key value pairs. The semantics now gives us independence of the underlying file system. So none of the file system's restrictions apply anymore. No constraints on the length of a key name disturbs the user any more. Additionally, key names can be arbitrarily deep nested. Depth is the number of unescaped {\ttfamily /} in the key name.

The directory concept is enforced by default. Keys can be created everywhere. Keys always can have a value. The only constraint is that key names are unique and occur in one of the \hyperlink{md_doc_help_elektra-namespaces_doc_help_elektra-namespaces_md}{namespaces}. Every Key has an absolute name. There is no concept of relative names in Elektra's Keys except for metakeys belonging to a key. Every other Key is independent of each other. We just do not care if there is another key below or above the accessed one in the storage or not.

Some applications need specific structure in the keys. Plugins can introduce and enforce relationships between keys. They can implement a type system, check if holes are present and check the structure and interrelations. They may propagate the metadata and introduce inheritance. We see that plugins are able to add more semantics to Elektra.

There are no symbolic links, hard links, device files or anything else different from key value pairs. Again, most of these behaviours can be mimicked using metadata. Especially, links are available using the metadata {\ttfamily override} and {\ttfamily fallback}.

Hidden keys are not useful for Elektra. Instead comments or other metadata contain information about keys that is not considered to belong to the configuration. If hidden keys are desired, we can still write a plugin to filter specific keys out.

This section explains why using file system semantics for configuration is not a good idea.

\subsubsection*{filesys}

{\ttfamily filesys} was the first backend. It implemented the principle that every key is represented by a single file. The key name was actually mapped to a file name and the value and the comment was written to that file.

If the backend {\ttfamily filesys} was the ideal solution, Elektra's A\+P\+I (application programming interface) would be of limited use. E.\+g.\+cascading, type checking and optional cross-\/cutting features would be missing. The storage problem itself and the location of a key in a key database would be solved. because well-\/established A\+P\+Is for accessing files are available in every applicable programming language.

Elektra 0.\+7 already supported more than one backend, but {\ttfamily filesys} was the only backend implementing the full semantics.

\subsubsection*{Limitations of File Systems}

Here we will discuss, why the file system's semantics are not well suited for configuration at all.

One file per key turned out to be inefficient because of the file system's practical limitations. In most file systems, a file needs about four kilobytes (Depends on the block size, four kilobytes is a common value often used as default.) of space, no matter how little content is in it. Thus the file system wastes 99.\+9\% of the space if keys have a payload of four bytes. Additionally, every file allocates a file node, which might be limited, too. We can argue, however, that we can use a file system which does not have these problem.

Many additional restrictions occur for portable access. The file name length in P\+O\+S\+I\+X is limited to fourteen characters. Additionally, issues with case sensitivity are likely. The common denominator for all file systems is a surprisingly small one. If, for example, the traditional F\+A\+T file system should be supported, file names are limited to eight characters and case insensitivity.

On the one hand, there are many file system features that are not needed for configuration. File systems have a strict hierarchy. It is not possible to create a file in a non-\/existing directory. We will refer to such a missing object as {\bfseries hole}. File systems do not support such holes.

A single {\bfseries root directory} is not a useful concept for configuration. Instead, the system configuration and each user configuration has its own root. These root keys themselves are typically not needed.

There is additional metadata of files which is typically not needed for configuration\+: atime, mtime, ctime, uid, gid and mode just to name a few. Additional file types, for example, device files, links, fifos and sockets, are not needed either. Features like sparse files are ridiculous for the small strings, that key values typically are.

On the other hand, there are many {\itshape features missing} in file systems that we need in a serious key database. Creating a whole hierarchy of files at once atomically is not possible. Ways to achieve this are currently academic and not portable. Directories cannot have any content next to the files below. Swap semantics are missing\+: it is not possible to rename a file without removing the target first.

To sum up, file systems are not suitable to hold configuration with one entry per file. Instead, they are perfectly suitable to hold larger pieces of information like configuration files. 