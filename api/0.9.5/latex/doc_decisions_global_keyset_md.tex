Some plugins need to communicate more data than is possible to do with metadata. This can limit the functionality of plugins, which need to exchange binary or very complex data.

To make the communication between plugins easier, plugins will additionally get a handle to a global keyset via {\ttfamily \hyperlink{group__plugin_ga436cda13ed70c0face08661a90620bf6}{elektra\+Plugin\+Get\+Global\+Key\+Set()}}. The global keyset is tied to a K\+DB handle, initialized on {\ttfamily \hyperlink{group__kdb_ga844e1299a84c3fbf1d3a905c5c893ba5}{kdb\+Open()}} and deleted on {\ttfamily \hyperlink{group__kdb_gadb54dc9fda17ee07deb9444df745c96f}{kdb\+Close()}}.

The global keyset handle is initialized and accessible for all plugins except manually created plugins (by calling e.\+g. {\ttfamily \hyperlink{elektra_2plugin_8c_a32a70a7876542c51d153164ac5108a57}{elektra\+Plugin\+Open()}}).

This decision removes the need to exchange information between plugins via the parent\+Key.

The need for a global keyset arose when developing a global cache plugin. A global cache plugin needs to store internal and binary information for the K\+DB, which is simply not possible with metadata.

Plugins are responsible for cleaning up their part of the global keyset.