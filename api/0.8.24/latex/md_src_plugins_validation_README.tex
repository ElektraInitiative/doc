
\begin{DoxyItemize}
\item infos = Information about validation plugin is in keys below
\item infos/author = Markus Raab \href{mailto:elektra@libelektra.org}{\tt elektra@libelektra.\+org}
\item infos/licence = B\+SD
\item infos/provides = check
\item infos/needs =
\item infos/placements = presetstorage
\item infos/status = maintained unittest nodep libc
\item infos/metadata = check/validation check/validation/message check/validation/ignorecase check/validation/match check/validation/invert check/validation/type
\item infos/description = Validates key values using regular expressions
\end{DoxyItemize}

This plugin is a check plugin which checks string values of Keys using regular expressions.

\subsection*{Usage}

The validation plugin looks for two metakeys. {\ttfamily check/validation} gives a regular expression to check against. If it is present, {\ttfamily check/validation/message} may contain an optional humanly readable message that will be passed with the error information.

Important\+: To validate against the whole string, you have to start the regular expression with {\ttfamily $^\wedge$} and end it with {\ttfamily \$}. Otherwise expressions that e.\+g. match the empty string, always return true. Alternatively, you can use {\ttfamily check/validation/match=L\+I\+NE}.

\subsection*{Configuration}

Metadata can be supplied to configure the validation\+:


\begin{DoxyItemize}
\item {\ttfamily check/validation/match}\+: You can check against {\ttfamily L\+I\+NE}, {\ttfamily W\+O\+RD} or {\ttfamily A\+NY}
\item {\ttfamily check/validation/ignorecase}\+: If you want to ignore case.
\item {\ttfamily check/validation/invert}\+: If you want to invert match.
\end{DoxyItemize}

\subsection*{Implementation}

The implementation consists of a loop checking for every key if it has the mentioned metakey. The check itself is done by the P\+O\+S\+IX regular expression library with the interface {\ttfamily regcomp}, {\ttfamily regexec}, {\ttfamily regerror} and {\ttfamily regfree}. The flag {\ttfamily R\+E\+G\+\_\+\+E\+X\+T\+E\+N\+D\+ED} is passed so that the regular expression will be compiled as an extended regular expression. {\ttfamily R\+E\+G\+\_\+\+N\+O\+S\+UB} gives a better performance and subexpressions cannot be used in this setup anyway.

\subsection*{Exported Methods}

The plugin also exports the function {\ttfamily ks\+Lookup\+R\+E()} that does a lookup in a Key\+Set using a regular expression. It starts from the current cursor of the Key\+Set and stops when the first value matches. Finally, this key is returned. 