{\ttfamily kdb shell}

This command is used to start an instance of the kdb shell.~\newline
 The kdb shell allows for a user to interactively view, edit, or otherwise work with the key database.~\newline


The kdb shell offers a number of commands to interact with the key database.


\begin{DoxyItemize}
\item {\ttfamily kdb\+Get $<$name$>$}\+: Get the value of a key.
\item {\ttfamily kdb\+Set $<$name$>$}\+: Set the value of a key.
\item {\ttfamily key\+Set\+Name $<$name$>$}\+: Set the name of the current key.
\item {\ttfamily key\+Set\+Meta $<$name$>$ $<$string$>$}\+: Set a metakey associated with the current key.
\item {\ttfamily key\+Set\+String $<$string$>$}\+: Set a string value for the current key.
\item {\ttfamily ks\+Append\+Key}\+: Append the current key to the current keyset.
\item {\ttfamily ks\+Cut $<$name$>$}\+: Cut the current keyset.
\item {\ttfamily ks\+Output}\+: Prints the keys in the current keyset.
\end{DoxyItemize}


\begin{DoxyItemize}
\item {\ttfamily -\/H}, {\ttfamily -\/-\/help}\+: Show the man page.
\item {\ttfamily -\/V}, {\ttfamily -\/-\/version}\+: Print version info.
\item {\ttfamily -\/p}, {\ttfamily -\/-\/profile $<$profile$>$}\+: Use a different kdb profile.
\item {\ttfamily -\/C}, {\ttfamily -\/-\/color $<$when$>$}\+: Print never/auto(default)/always colored output.
\end{DoxyItemize}

To execute commands from a textfile, you can use\+:~\newline
 {\ttfamily cat commands.\+txt $\vert$ kdb shell}

To have readline functionality (line editing, history, ...), you can use\+:~\newline
 {\ttfamily rlwrap kdb shell}

To learn more about these commands and how they work, refer to the \href{https://doc.libelektra.org/api/latest/html}{\tt Elektra A\+PI Documentation}. 