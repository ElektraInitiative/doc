
\begin{DoxyItemize}
\item infos = Information about base64 plugin is in keys below
\item infos/author = Peter Nirschl \href{mailto:peter.nirschl@gmail.com}{\tt peter.\+nirschl@gmail.\+com}
\item infos/licence = B\+SD
\item infos/provides = binary
\item infos/needs =
\item infos/recommends =
\item infos/placements = postgetstorage presetstorage
\item infos/status = maintained unittest nodep libc final configurable
\item infos/metadata =
\item infos/description = Base64 Encoding
\end{DoxyItemize}\hypertarget{md_src_plugins_base64_README_src_plugins_base64_README_md}{}\section{Base64}\label{md_src_plugins_base64_README_src_plugins_base64_README_md}
\subsection*{Introduction}

The Base64 Encoding (specified in \href{https://www.ietf.org/rfc/rfc4648.txt}{\tt R\+F\+C4648}) is used to encode arbitrary binary data to A\+S\+C\+II strings.

This is useful for configuration files that must contain A\+S\+C\+II strings only.

The {\ttfamily base64} plugin encodes binary values before {\ttfamily kdb set} writes the configuration to the file. The values are decoded back to their original value after {\ttfamily kdb get} has read from the configuration file.

\subsection*{Usage}

To mount a simple backend that uses the Base64 encoding, you can use\+:


\begin{DoxyCode}
sudo kdb mount test.ecf /tests/base64/test base64
\end{DoxyCode}


. To unmount the plugin use the following command\+:


\begin{DoxyCode}
sudo kdb umount /tests/base64/test
\end{DoxyCode}


. All encoded binary values will look something like this\+: \begin{DoxyVerb}@BASE64SGVsbG8gV29ybGQhCg==
\end{DoxyVerb}


\subsection*{Modes}

The plugin supports two different modes\+:


\begin{DoxyEnumerate}
\item Escaping Mode
\item Meta Mode
\end{DoxyEnumerate}

. By default the plugin uses escaping mode which has the advantage that a storage plugin does not have to change its behavior at all to work in conjunction with Base64.

\subsubsection*{Escaping Mode}

In order to identify the base64 encoded content, the values are marked with the prefix {\ttfamily @B\+A\+S\+E64}. To distinguish between the {\ttfamily @} as character and {\ttfamily @} as Base64 marker, all strings starting with {\ttfamily @} will be modified so that they begin with {\ttfamily @@}.

See the documentation of the \hyperlink{md_src_plugins_null_README_src_plugins_null_README_md}{null plugin}, as it uses the same pattern for masking {\ttfamily @}.

\paragraph*{Example}

The following example shows how you can use this plugin together with the I\+NI plugin to store binary data.


\begin{DoxyCode}
# Mount the INI and Base64 plugin
kdb mount config.ini user/tests/base64 ini base64

# Copy binary data
kdb cp system/elektra/modules/base64/exports/get user/tests/base64/binary

# Print binary data
kdb get user/tests/base64/binary
# STDOUT-REGEX: ^(\(\backslash\)\(\backslash\)x[0-9a-f]\{1,2\})+$

# The value inside the configuration file is encoded by the Base64 plugin
kdb file user/tests/base64 | xargs cat
# STDOUT-REGEX: binary = "@BASE64[a-zA-Z0-9+/]+=\{0,2\}"

# Undo modifications
kdb rm -r user/tests/base64
kdb umount user/tests/base64
\end{DoxyCode}


\subsubsection*{Meta Mode}

Some file formats such as \href{http://yaml.org}{\tt Y\+A\+ML} already support Base64 encoded data. In Y\+A\+ML \href{http://yaml.org/type/binary.html}{\tt binary data} starts with the tag {\ttfamily !!binary} followed by a Base64 encoded scalar\+:


\begin{DoxyCode}
!!binary SGVsbG8sIFlBTUwh # “Hello, YAML!”
\end{DoxyCode}


. For Y\+A\+ML it would not make sense to use the format of the escaping mode\+:


\begin{DoxyCode}
# Another YAML implementation will not be able to decode this data
# because of the prefix `@BASE64`!
!!binary "@BASE64SGVsbG8sIFlBTUwh"
\end{DoxyCode}


. Base64 supports another mode called “meta mode”. In this mode the Base64 plugin encodes the value, but does not add a prefix. To use the escaping mode a plugin must add the configuration key {\ttfamily /binary/meta}. Afterwards the Base64 plugin encodes and decodes all data that contains the metakey {\ttfamily type} with the value {\ttfamily binary}.

The diagram below shows how the Base64 conversion process works in conjunction with the Y\+A\+ML C\+PP plugin.



\paragraph*{Example}

The following example shows you how you can use the I\+NI plugin together with Base64’s meta mode.

``{\ttfamily  $<$h1$>$Mount I\+NI and Base64 plugin (provides}binary{\ttfamily ) with the configuration key}binary/meta` kdb mount config.\+ini user/tests/base64 ini base64 binary/meta=

\section*{Save base64 encoded data {\ttfamily \char`\"{}value\char`\"{}} ({\ttfamily 0x76616c7565})}

kdb set user/tests/base64/encoded dm\+Fsd\+W\+UA kdb file user/tests/base64/encoded $\vert$ xargs cat $\vert$ grep encoded \#$>$ encoded = dm\+Fsd\+W\+UA

\section*{Tell Base64 plugin to decode and encode key value}

kdb setmeta user/tests/base64/encoded type binary

\section*{Receive key data (the {\ttfamily \textbackslash{}x0} at the end marks the end of the string)}

kdb get user/tests/base64/encoded \#$>$ 

\section*{Undo modifications}

kdb rm -\/r user/tests/base64 kdb umount user/tests/base64 ```

For another usage example, please take a look at the Read\+Me of the \hyperlink{md_src_plugins_yamlcpp_README_src_plugins_yamlcpp_README_md}{Y\+A\+ML C\+PP plugin}. 