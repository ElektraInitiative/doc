\subsection*{Status}

The graph below shows an (incomplete) list of available packages for Elektra.

\href{https://repology.org/metapackage/elektra/versions}{\tt }

\subsection*{Linux}

For the following Linux distributions and package managers 0.\+8 packages are available\+:


\begin{DoxyItemize}
\item \href{https://aur.archlinux.org/packages/elektra/}{\tt Arch Linux}
\item \href{https://github.com/openwrt/packages/tree/master/libs/elektra}{\tt Openwrt}
\item \href{https://software.opensuse.org/package/elektra}{\tt Open\+Suse}
\item \href{https://packages.debian.org/de/jessie/libelektra4}{\tt Debian}
\item \href{https://launchpad.net/ubuntu/+source/elektra}{\tt Ubuntu}
\item \href{http://packages.gentoo.org/package/app-admin/elektra}{\tt Gentoo}
\item \href{https://community.linuxmint.com/software/view/elektra-bin}{\tt Linux Mint}
\item \href{https://github.com/Linuxbrew/homebrew-core/blob/master/Formula/elektra.rb}{\tt Linux\+Brew}
\item \href{https://pkgs.alpinelinux.org/package/edge/testing/x86_64/elektra}{\tt Alpine Linux}
\end{DoxyItemize}

For \href{https://build.opensuse.org/package/show/home:bekun:devel/elektra}{\tt Open\+S\+U\+SE, Cent\+OS, Fedora, R\+H\+EL and S\+LE} Kai-\/\+Uwe Behrmann kindly provides packages \href{http://software.opensuse.org/download.html?project=home%3Abekun%3Adevel&package=libelektra4}{\tt for download}.

\subsubsection*{Debian}

To use the debian repository of the latest builds from master put following lines in {\ttfamily /etc/apt/sources.list}\+:

For Stretch\+:


\begin{DoxyCode}
deb     [trusted=yes] https://debian-stretch-repo.libelektra.org/ stretch main
deb-src [trusted=yes] https://debian-stretch-repo.libelektra.org/ stretch main
\end{DoxyCode}


Which can also be done using\+:


\begin{DoxyCode}
sudo apt-get install apt-transport-https
echo "deb     [trusted=yes] https://debian-stretch-repo.libelektra.org/ stretch main" | sudo tee
       /etc/apt/sources.list.d/elektra.list
\end{DoxyCode}


Or alternatively, you can use (if you do not mind many dependences just to add one line to a config file)\+:


\begin{DoxyCode}
sudo apt-get install software-properties-common apt-transport-https
sudo add-apt-repository "deb     [trusted=yes] https://debian-stretch-repo.libelektra.org/ stretch main"
\end{DoxyCode}


For Jessie (not updated anymore, contains 0.\+8.\+24 packages which were created shortly before 0.\+8.\+25 release)


\begin{DoxyCode}
deb     [trusted=yes] https://debian-stable.libelektra.org/elektra-stable/ jessie main
deb-src [trusted=yes] https://debian-stable.libelektra.org/elektra-stable/ jessie main
\end{DoxyCode}


For Wheezy (not updated anymore, contains 0.\+8.\+19-\/8121 packages)\+:


\begin{DoxyCode}
deb     [trusted=yes] https://build.libelektra.org/debian/ wheezy main
deb-src [trusted=yes] https://build.libelektra.org/debian/ wheezy main
\end{DoxyCode}


To get all packaged plugins, bindings and tools install\+:


\begin{DoxyCode}
apt-get install libelektra4-all
\end{DoxyCode}


For a small installation with command-\/line tools available use\+:


\begin{DoxyCode}
apt-get install elektra-bin
\end{DoxyCode}


If you want to rebuild Elektra from Debian unstable or our repositories, add a {\ttfamily deb-\/src} entry to {\ttfamily /etc/apt/sources.list} and then run\+:


\begin{DoxyCode}
apt-get source -b elektra
\end{DoxyCode}


To build Debian Packages from the source you might want to use\+:


\begin{DoxyCode}
dpkg-buildpackage -us -uc -sa
\end{DoxyCode}


(You need to be in the Debian branch, see \hyperlink{doc_GIT_md}{G\+IT})

\subsection*{mac\+OS}

You can install Elektra using \href{http://brew.sh}{\tt Homebrew} via the shell command\+:


\begin{DoxyCode}
brew install elektra
\end{DoxyCode}


. We also provide a tap containing a more elaborate formula \href{http://github.com/ElektraInitiative/homebrew-elektra}{\tt here}.

\subsection*{Windows}

Please refer to the section OS Independent below.

\subsection*{OS Independent}

First follow the steps in \hyperlink{doc_COMPILE_md}{C\+O\+M\+P\+I\+LE}.

After you completed building Elektra on your own, there are multiple options how to install it. For example, with make or c\+Pack tools.

\subsubsection*{make}


\begin{DoxyCode}
sudo make install
sudo ldconfig  # See troubleshooting below
\end{DoxyCode}


To uninstall Elektra use (will not be very clean, e.\+g. it will not remove directories and {\ttfamily $\ast$.pyc} files)\+:


\begin{DoxyCode}
sudo make uninstall
sudo ldconfig
\end{DoxyCode}


or in the build directory (will not honor {\ttfamily D\+E\+S\+T\+D\+IR}!)\+:


\begin{DoxyCode}
xargs rm < install\_manifest.txt
\end{DoxyCode}


\subsubsection*{C\+Pack}

First follow the steps in \hyperlink{doc_COMPILE_md}{C\+O\+M\+P\+I\+LE}.

Then use\+:


\begin{DoxyCode}
cpack
\end{DoxyCode}


which should create a package for distributions where a Generator is implemented. See \href{/home/markus/Projekte/Elektra/current/cmake/ElektraPackaging.cmake}{\tt this cmake file} for available Generators and send a merge request for your system.

\subsection*{Troubleshooting}

\subsubsection*{Error Loading Libraries}

If you encounter the problem that the library can not be found (output like this)


\begin{DoxyCode}
kdb: error while loading shared libraries:
     libelektra-core.so.4: cannot open shared object file: No such file or directory
\end{DoxyCode}


or\+:


\begin{DoxyCode}
kdb: error while loading shared libraries:
     libelektratools.so.2: cannot open shared object file: No such file or directory
\end{DoxyCode}


you need to place a configuration file at {\ttfamily /etc/ld.so.\+conf.\+d/} (e.\+g. {\ttfamily /etc/ld.so.\+conf.\+d/elektra.conf}). Note that under Alpine Linux this file is called {\ttfamily /etc/ld-\/musl-\/x86\+\_\+64.path} or similar, depending on your architecture.

Add the path where the library has been installed (on Alpine Linux this had to be {\ttfamily usr/lib/elektra} for it to work)


\begin{DoxyCode}
/usr/lib/local/
\end{DoxyCode}


and run {\ttfamily ldconfig} as root.

\subsection*{Installation Manuals}

For some of the plugins and tools that ship with Elektra, additional installation manuals have been written. You can find them in the \hyperlink{md_doc_tutorials_README_doc_tutorials_README_md}{tutorial overview}.

\subsection*{See Also}


\begin{DoxyItemize}
\item \hyperlink{doc_COMPILE_md}{C\+O\+M\+P\+I\+LE}.
\item \hyperlink{doc_TESTING_md}{T\+E\+S\+T\+I\+NG}. 
\end{DoxyItemize}