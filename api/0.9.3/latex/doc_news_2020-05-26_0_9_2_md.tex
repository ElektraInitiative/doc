
\begin{DoxyItemize}
\item guid\+: ea79f59e-\/f471-\/4658-\/a11b-\/1371802814c2
\item author\+: Mihael Pranjic
\item pub\+Date\+: Tue, 26 May 2020 19\+:33\+:30 +0200
\item short\+Desc\+: K\+DE and G\+N\+O\+ME Integration, {\ttfamily elektrad} in Go
\end{DoxyItemize}

We are proud to release Elektra 0.\+9.\+2.

With the 0.\+9.\+x series of releases we shift our focus to bugfixes and stability improvements as needed for the K\+DE and G\+N\+O\+ME integration. We do not guarantee any compatibility in this series.

Elektra serves as a universal and secure framework to access configuration settings in a global, hierarchical key database. For more information, visit \href{https://libelektra.org}{\tt https\+://libelektra.\+org}.

You can also read the news \href{https://www.libelektra.org/news/0.9.2-release}{\tt on our website}


\begin{DoxyItemize}
\item K\+DE integration
\item G\+N\+O\+ME Integration
\item {\ttfamily elektrad} rewritten in Go
\end{DoxyItemize}

We created a \href{https://github.com/ElektraInitiative/kconfig}{\tt fork} of \href{https://kde.org/}{\tt K\+DE\textquotesingle{}s} {\ttfamily K\+Config} configuration system and patched it to use libelektra. We have done some initial testing and replaced the {\ttfamily K\+Config} library for \href{https://kate-editor.org/}{\tt Kate} and \href{https://www.kdevelop.org/}{\tt K\+Develop} successfully.

Additionally, we added a new Elektra plugin called {\ttfamily kconfig}, which can read and write kconfig\textquotesingle{}s I\+NI files. The plugin enables smooth migration of existing K\+DE configurations. \+\_\+(\+Dardan Haxhimustafa)\+\_\+ and \+\_\+(\+Felix Resch)\+\_\+

We continued work on Elektra\textquotesingle{}s bindings for \href{https://developer.gnome.org/gio/stable/GSettings.html}{\tt G\+N\+O\+ME G\+Settings}. Our implementation should be able to replace the widely used \href{https://wiki.gnome.org/Projects/dconf}{\tt dconf} backend. Elektra\textquotesingle{}s {\ttfamily gsettings} bindings are not yet ready for production use, but they are already able to replace {\ttfamily dconf} for a complete G\+N\+O\+ME session without problems. We are still lacking proper dbus integration for change notifications. \+\_\+(Mihael Pranjić)\+\_\+

\href{https://www.libelektra.org/tools/elektrad}{\tt elektrad} provides an H\+T\+TP A\+PI to access Elektra remotely. {\ttfamily elektrad} is now completely rewritten in Go, which drastically improves the performance by leveraging the new \href{https://github.com/ElektraInitiative/go-elektra/}{\tt go-\/elektra} bindings instead of calling the {\ttfamily kdb} command-\/line tool on every request. The new {\ttfamily elektrad} creates a session per user to reuse the same \href{https://doc.libelektra.org/api/current/html/group__kdb.html}{\tt K\+DB handle} for correct conflict handling and better performance. \+\_\+(\+Raphael Gruber)\+\_\+

You can try out the latest Elektra release using our docker image \href{https://hub.docker.com/r/elektra/elektra}{\tt elektra/elektra}. This is the quickest way to get started with Elektra without compiling and other obstacles.

Get started with Elektra by running {\ttfamily docker run -\/it elektra/elektra}.

We removed the {\ttfamily maintained} status of the following \href{https://www.libelektra.org/plugins/readme}{\tt plugins}\+:


\begin{DoxyItemize}
\item blockresolver
\item csvstorage
\item gitresolver
\item list
\item multifile
\item spec
\end{DoxyItemize}

New maintainers are very much welcomed!


\begin{DoxyItemize}
\item Improved error message for Augeas to show lens\+Path. \+\_\+(\+Michael Zronek)\+\_\+
\end{DoxyItemize}


\begin{DoxyItemize}
\item The \href{https://master.libelektra.org/tests/shell/shell_recorder/tutorial_wrapper}{\tt Markdown Shell Recorder} test of the plugin does not require Bash any more. \+\_\+(René Schwaiger)\+\_\+
\end{DoxyItemize}


\begin{DoxyItemize}
\item The crypto plugin no longer supports Botan and Open\+S\+SL as provider of cryptographic functions. The support has been removed to improve the maintainability of the code. \+\_\+(\+Peter Nirschl)\+\_\+
\item The unit test of the crypto plugin attempts to kill the gpg-\/agent if a regular shutdown via {\ttfamily connect-\/gpg-\/agent} failed. \+\_\+(\+Peter Nirschl)\+\_\+
\end{DoxyItemize}


\begin{DoxyItemize}
\item The plugin now only interprets a \href{https://doc.libelektra.org/api/current/html/group__keyset.html}{\tt Key\+Set} as \hyperlink{doc_tutorials_arrays_md}{array} if the parent contains the meta key {\ttfamily array}. \+\_\+(René Schwaiger)\+\_\+
\end{DoxyItemize}


\begin{DoxyItemize}
\item Improve handling of temporary files after encryption and decryption by trying to perform a manual copy if the call of {\ttfamily rename} fails. This problem might occur if another file system is mounted at {\ttfamily /tmp}. \+\_\+(\+Peter Nirschl)\+\_\+
\end{DoxyItemize}


\begin{DoxyItemize}
\item Write support for the K\+Config I\+NI format was added. \+\_\+(\+Dardan Haxhimustafa)\+\_\+
\end{DoxyItemize}


\begin{DoxyItemize}
\item Configure line (-\/\+D\+B\+I\+N\+D\+I\+N\+GS=\char`\"{}..\char`\"{}) for S\+W\+IG based bindings have been changed from {\ttfamily swig\+\_\+foo} to {\ttfamily foo}. \+\_\+(\+Manuel Mausz)\+\_\+
\item Exclude S\+W\+IG bindings if S\+W\+IG Version is 4.\+0.\+1 and Python is $>$= 3.\+8 or Ruby is $>$= 2.\+7 due to incompatibility (\#3378, \#3379). \+\_\+(Mihael Pranjić)\+\_\+
\end{DoxyItemize}


\begin{DoxyItemize}
\item Added bindings for libelektratools. \+\_\+(\+Manuel Mausz)\+\_\+
\item Add test for kdb\+Ensure. \+\_\+(Mihael Pranjić)\+\_\+
\end{DoxyItemize}


\begin{DoxyItemize}
\item Removed python2 binding, as python2 support ended. \+\_\+(\+Manuel Mausz)\+\_\+
\end{DoxyItemize}


\begin{DoxyItemize}
\item The \href{https://master.libelektra.org/tests/shell/shell_recorder/tutorial_wrapper}{\tt Markdown Shell Recorder} test of the plugin now correctly requires the \hyperlink{autotoc_md831_src_plugins_xmltool_README_md}{`xmltool` plugin}. \+\_\+(René Schwaiger)\+\_\+
\end{DoxyItemize}


\begin{DoxyItemize}
\item We removed the plugin in favor of \hyperlink{autotoc_md955_src_plugins_yanlr_README_md}{Yan LR}. \+\_\+(René Schwaiger)\+\_\+
\end{DoxyItemize}


\begin{DoxyItemize}
\item The plugin now always prints a newline at the end of the Y\+A\+ML output. \+\_\+(René Schwaiger)\+\_\+
\item The plugin does not interpret a key set such as
\end{DoxyItemize}


\begin{DoxyCode}
user/example
user/example/#0
user/example/#1
user/example/#2
\end{DoxyCode}


as array unless the parent key {\ttfamily user/example} contains the meta key {\ttfamily array}. \+\_\+(René Schwaiger)\+\_\+


\begin{DoxyItemize}
\item Y\+A\+ML C\+PP now always sets and requires the metakey {\ttfamily type} with the value {\ttfamily boolean} for boolean data. \+\_\+(René Schwaiger)\+\_\+
\item We limited the scope of a logging function of the module. This makes it possible to build Elektra again, if
\begin{DoxyItemize}
\item you enabled the logger ({\ttfamily E\+N\+A\+B\+L\+E\+\_\+\+L\+O\+G\+G\+ER=ON}),
\item build the “full” ({\ttfamily B\+U\+I\+L\+D\+\_\+\+F\+U\+LL=ON}) version of Elektra, and
\item include both the Directory Value and Y\+A\+ML C\+PP plugin in your build configuration. \+\_\+(René Schwaiger)\+\_\+
\end{DoxyItemize}
\end{DoxyItemize}


\begin{DoxyItemize}
\item The C\+Make code of the plugin does not print error messages produced by the tool {\ttfamily ldd} any more. \+\_\+(René Schwaiger)\+\_\+
\item The plugin now also supports A\+N\+T\+LR 4.\+8. \+\_\+(René Schwaiger)\+\_\+
\item We limited the scope of the logging code of the module. For more information, please take a look at the last news entry of the Y\+A\+ML C\+PP plugin. \+\_\+(René Schwaiger)\+\_\+
\end{DoxyItemize}


\begin{DoxyItemize}
\item The plugin now supports an offset into {\ttfamily argv} given by the {\ttfamily /offset} config key. If {\ttfamily /offset} is set, {\ttfamily gopts} will ignore a number of arguments at the start of {\ttfamily argv}. This can be used in e.\+g. python scripts to ignore the interpreter arguments. \+\_\+(Klemens Böswirth)\+\_\+
\item {\ttfamily gopts} now also writes help message into the key {\ttfamily proc/elektra/gopts/help/message} in addition to setting {\ttfamily proc/elektra/gopts/help = 1}. This is especially useful in non-\/\+C/\+C++ environments. \+\_\+(Klemens Böswirth)\+\_\+
\item {\ttfamily gopts} is also affected by the changes and improvements to the {\ttfamily opts} library outlined below.
\end{DoxyItemize}


\begin{DoxyItemize}
\item Respect {\ttfamily X\+D\+G\+\_\+\+C\+A\+C\+H\+E\+\_\+\+H\+O\+ME} when resolving the mmap cache directory. \+\_\+(Mihael Pranjić)\+\_\+
\end{DoxyItemize}

The text below summarizes updates to the \href{https://www.libelektra.org/libraries/readme}{\tt C (and C++)-\/based libraries} of Elektra.


\begin{DoxyItemize}
\item We clarified compatibility requirements for Elektra and its plugins and bindings. Furthermore, we renamed {\ttfamily system/elektra/version/constants/\+K\+D\+B\+\_\+\+V\+E\+R\+S\+I\+O\+N\+\_\+\+M\+I\+C\+RO} to {\ttfamily system/elektra/version/constants/\+K\+D\+B\+\_\+\+V\+E\+R\+S\+I\+O\+N\+\_\+\+P\+A\+T\+CH} to be compatible with \href{https://semver.org/}{\tt Semantic Versioning 2.\+0.\+0}. \+\_\+(\+Markus Raab)\+\_\+
\end{DoxyItemize}


\begin{DoxyItemize}
\item The library function {\ttfamily elektra\+Get\+Opts} now supports sub-\/commands. Sub-\/commands are best explained by looking at an application that uses them, like {\ttfamily git}. For example {\ttfamily add} is a sub-\/command in {\ttfamily git add}, and interprets {\ttfamily -\/p} differently from {\ttfamily git}\+: {\ttfamily git -\/p add} is {\ttfamily git -\/-\/paginate add}, but {\ttfamily git add -\/p} is {\ttfamily git add -\/-\/patch}. {\ttfamily elektra\+Get\+Opts} now implements this notion of sub-\/commands. For more information take a look at the \hyperlink{doc_tutorials_command-line-options_md}{tutorial for command-\/line-\/options}. By extension this functionality is also available via the {\ttfamily gopts} plugin. \+\_\+(Klemens Böswirth)\+\_\+
\item The generated help message was improved. It now also gives details about parameter arguments, sub-\/commands and environment variables in addition to the existing support for option arguments. This also means that it is no longer possible to have multiple keys with the {\ttfamily args=remaining} metadata (because their {\ttfamily opt/help} may not be the same). \+\_\+(Klemens Böswirth)\+\_\+
\end{DoxyItemize}

Bindings allow you to utilize Elektra using \href{https://www.libelektra.org/bindings/readme}{\tt various programming languages}. This section keeps you up to date with the multi-\/language support provided by Elektra.


\begin{DoxyItemize}
\item Removed python2 plugin, as python2 support ended. \+\_\+(\+Manuel Mausz)\+\_\+
\end{DoxyItemize}


\begin{DoxyItemize}
\item Published {\ttfamily elektra} and {\ttfamily elektra-\/sys} to crates.\+io. \+\_\+(\+Philipp Gackstatter)\+\_\+
\end{DoxyItemize}


\begin{DoxyItemize}
\item Update {\ttfamily kdb cache} tool synopsis to reflect man page. \+\_\+(Mihael Pranjić)\+\_\+
\item Pull elektrad, webui and webd out of shared web folder to allow fine grained selection of tools. \+\_\+(\+Raphael Gruber)\+\_\+
\item webd has updated dependencies. \+\_\+(Mihael Pranjić)\+\_\+
\end{DoxyItemize}


\begin{DoxyItemize}
\item The fish completion script now recognizes the new names of subcommands (e.\+g. {\ttfamily meta-\/set} instead of {\ttfamily setmeta} ) introduced with Elektra {\ttfamily 0.\+9.\+1}. \+\_\+(René Schwaiger)\+\_\+
\item The script reformat-\/cmake now reformats the code with {\ttfamily cmake-\/format} 0.\+6.\+3. \+\_\+(René Schwaiger)\+\_\+
\item The scripts
\begin{DoxyItemize}
\item reformat-\/c, and
\item reformat-\/java
\end{DoxyItemize}

now uses {\ttfamily clang-\/format} 9 to reformat the code base. \+\_\+(René Schwaiger)\+\_\+
\item The script reformat-\/shell now makes sure that you do not use {\ttfamily shfmt} 3, which formats parts of the code base slightly differently. \+\_\+(René Schwaiger)\+\_\+
\end{DoxyItemize}


\begin{DoxyItemize}
\item Improved formatting of the \hyperlink{doc_tutorials_validation_md}{`validation tutorial`} \+\_\+(Anton Hößl)\+\_\+
\item We fixed some minor spelling mistakes. \+\_\+(René Schwaiger)\+\_\+
\item We updated the man pages of the \hyperlink{doc_tutorials_install-webui_md}{`web`} tool. \+\_\+(René Schwaiger)\+\_\+
\item Updated documentation for Ubuntu-\/\+Bionic Packages. \+\_\+(\+Djordje Bulatovic)\+\_\+
\item Fixed an old path of the reformatting script in the \hyperlink{doc_tutorials_run_reformatting_script_with_docker_md}{`docker reformatting tutorial`} \+\_\+(\+Jakob Fischer)\+\_\+
\end{DoxyItemize}


\begin{DoxyItemize}
\item We now use \href{https://github.com/google/googletest}{\tt Google Test} version {\ttfamily 1.\+10} to test Elektra. \+\_\+(René Schwaiger)\+\_\+
\item The C++ test code does not produce warnings about a missing macro argument for {\ttfamily ...} any more. \+\_\+(René Schwaiger)\+\_\+
\item Whitelisted many broken links. \+\_\+(Mihael Pranjić)\+\_\+
\item Enabled regex in link checker. \+\_\+(Mihael Pranjić)\+\_\+
\item The formatting check now also works correctly, if it is invoked multiple times. \+\_\+(René Schwaiger)\+\_\+
\item {\ttfamily K\+D\+B\+\_\+\+E\+X\+E\+C\+\_\+\+P\+A\+TH} is not being set globally to contain the build directory any longer. \+\_\+(\+Peter Nirschl)\+\_\+
\item Rewrite gpg-\/agent shutdown logic to use {\ttfamily fork} and {\ttfamily execv} instead of {\ttfamily system}. \+\_\+(\+Peter Nirschl)\+\_\+
\item Removed a broken link from the link checker. \+\_\+(\+Djordje Bulatovic)\+\_\+
\end{DoxyItemize}


\begin{DoxyItemize}
\item We do not use implicit typing in the code of the
\begin{DoxyItemize}
\item {\ttfamily augeas},
\item {\ttfamily base64}, and
\item {\ttfamily blockresolver}
\end{DoxyItemize}

plugin any more. After this update, the code compiles without any warnings, even though we now use the compiler switch {\ttfamily -\/\+Wconversion}. \+\_\+(René Schwaiger)\+\_\+
\end{DoxyItemize}


\begin{DoxyItemize}
\item Debian 9 “stretch” (oldstable) is now the oldest supported platform. \+\_\+(René Schwaiger)\+\_\+
\end{DoxyItemize}


\begin{DoxyItemize}
\item Generating the build system now requires C\+Make {\ttfamily 3.\+4} (released in November 2015). \+\_\+(René Schwaiger)\+\_\+
\item We fixed warnings about C\+Make policy \href{https://cmake.org/cmake/help/latest/policy/CMP0078.html}{\tt C\+M\+P0078} and \href{https://cmake.org/cmake/help/latest/policy/CMP0086.html}{\tt C\+M\+P0086}. \+\_\+(René Schwaiger)\+\_\+
\item The C\+Make functions {\ttfamily add\+\_\+msr\+\_\+test} and {\ttfamily add\+\_\+msr\+\_\+test\+\_\+plugin} do not export the list of required plugins as environment variable any more. \+\_\+(René Schwaiger)\+\_\+
\item The C\+Make code of the code generation does not print warnings about unknown regex operators any more. \+\_\+(René Schwaiger)\+\_\+
\end{DoxyItemize}


\begin{DoxyItemize}
\item We updated some of the software in the Dockerfile for Debian sid. \+\_\+(René Schwaiger)\+\_\+
\item Building the documentation Dockerfile for Debian Stretch works again. \+\_\+(René Schwaiger)\+\_\+
\item Use Python 3, S\+W\+IG 4.\+0 and Ruby 2.\+5 in the Dockerfile for Debian sid. \+\_\+(Mihael Pranjić)\+\_\+
\item Disable python binding on {\ttfamily debian-\/unstable-\/full-\/clang} due to upstream \href{https://github.com/ElektraInitiative/libelektra/issues/3379}{\tt issue}. \+\_\+(Mihael Pranjić)\+\_\+
\item Use current ruby-\/dev on Debian sid image as Ruby 2.\+5 has been dropped. \+\_\+(Mihael Pranjić)\+\_\+
\end{DoxyItemize}


\begin{DoxyItemize}
\item We fixed a minor problem with the package install procedure on mac\+OS build jobs. \+\_\+(René Schwaiger)\+\_\+
\item We updated the startup command for D-\/\+Bus on mac\+OS. \+\_\+(René Schwaiger)\+\_\+
\item We removed python2 (E\+OL and removed from homebrew). \+\_\+(Mihael Pranjić)\+\_\+
\item Use latest mac\+OS Catalina Xcode stable. \+\_\+(Mihael Pranjić)\+\_\+
\item Use newer Free\+B\+SD images and use image family instead of concrete image names. \+\_\+(Mihael Pranjić)\+\_\+
\item Disable tcl plugin on Free\+B\+SD images because of test failures (see \#3353). \+\_\+(Mihael Pranjić)\+\_\+
\item Disable curlget plugin for mac\+OS jobs (see \#3382). \+\_\+(Mihael Pranjić)\+\_\+
\item Add more dependencies to Fedora image to cover many tests. \+\_\+(Mihael Pranjić)\+\_\+
\item Installed Ruby 2.\+6 to test the ruby bindings and plugins. \+\_\+(Mihael Pranjić)\+\_\+
\item Upgraded Fedora image to current stable (version 32). \+\_\+(Mihael Pranjić)\+\_\+
\end{DoxyItemize}


\begin{DoxyItemize}
\item Fixed \href{https://coveralls.io/github/ElektraInitiative/libelektra}{\tt coveralls} coverage report. \+\_\+(Mihael Pranjić)\+\_\+
\item The build jobs {\ttfamily debian-\/unstable-\/clang-\/asan} and {\ttfamily debian-\/unstable-\/full-\/clang} now use Clang 9 to compile Elektra. \+\_\+(René Schwaiger)\+\_\+
\item Added the {\ttfamily Jenkins.\+monthly} in the Jenkins\textquotesingle{} scripts file. \+\_\+(\+Djordje Bulatovic)\+\_\+
\item Enabled building packages for Bionic. \+\_\+(\+Djordje Bulatovic)\+\_\+
\item Improve gpgme unit test stability. \+\_\+(\+Peter Nirschl)\+\_\+
\item Publishing packages for Bionic to community. \+\_\+(\+Djordje Bulatovic)\+\_\+
\item Added Fedora 32 image to main build stage, moved Fedora 31 to full build stage. \+\_\+(Mihael Pranjić)\+\_\+
\item Fixed path for publishing in Jenkinsfile. \+\_\+(\+Djordje Bulatovic)\+\_\+
\item Reliably build the rust bindings based on the same version, by adding back the {\ttfamily Cargo.\+lock} file. \+\_\+(\+Philipp Gackstatter)\+\_\+
\end{DoxyItemize}


\begin{DoxyItemize}
\item Restyled now also checks the formatting of C, C++ and Java code in the repository. \+\_\+(René Schwaiger)\+\_\+
\end{DoxyItemize}


\begin{DoxyItemize}
\item Use newer Xcode 11.\+4 and ruby 2.\+6.\+5 on mac\+OS builds and use mac\+OS 10.\+15. \+\_\+(Mihael Pranjić)\+\_\+
\item Disable curlget plugin for mac\+OS jobs (see \#3382). \+\_\+(Mihael Pranjić)\+\_\+
\end{DoxyItemize}


\begin{DoxyItemize}
\item We now automatically close issues after one year of inactivity. \+\_\+(Mihael Pranjić)\+\_\+
\end{DoxyItemize}

The website is generated from the repository, so all information about plugins, bindings and tools are always up to date. Furthermore, we changed\+:


\begin{DoxyItemize}
\item Fix and re-\/enable website auto-\/deployment. \+\_\+(Mihael Pranjić)\+\_\+
\item Update docker images for website frontend and backend to debian buster. Update dependencies to newer versions. \+\_\+(Mihael Pranjić)\+\_\+
\item Remove obsolete parts from the website. \+\_\+(Mihael Pranjić)\+\_\+
\end{DoxyItemize}

We are currently working on following topics\+:


\begin{DoxyItemize}
\item Elektrify K\+DE \+\_\+(\+Dardan Haxhimustafa)\+\_\+, \+\_\+(\+Felix Resch)\+\_\+ and \+\_\+(Mihael Pranjić)\+\_\+
\item Elektrify G\+N\+O\+ME \+\_\+(Mihael Pranjić)\+\_\+
\item Elektrify L\+C\+Dproc \+\_\+(Klemens Böswirth)\+\_\+ and \+\_\+(\+Jakob Fischer)\+\_\+
\item Packaging for popular Linux distributions \+\_\+(\+Djordje Bulatovic)\+\_\+
\item Improve 3-\/way merge. \+\_\+(Dominic Jäger)\+\_\+
\item Go bindings and improved Web-\/\+UI \+\_\+(\+Raphael Gruber)\+\_\+
\item T\+O\+ML plugin as new default storage \+\_\+(\+Jakob Fischer)\+\_\+
\item Shell completion \+\_\+(Ulrike Schäfer)\+\_\+
\item Improve Elektra developer experience \+\_\+(\+Hani Torabi)\+\_\+
\item Ansible bindings \+\_\+(\+Thomas Waser)\+\_\+
\item Plugin interface improvements \+\_\+(\+Vid Leskovar)\+\_\+
\end{DoxyItemize}

We closed \href{https://github.com/ElektraInitiative/libelektra/milestone/22?closed=1}{\tt 40 issues} for this release.

About 23 authors changed 653 files with 15221 insertions(+) and 18890 deletions(-\/) in 815 commits.

Thanks to all authors for making this release possible!

\href{https://github.com/sanssecours}{\tt René Schwaiger} finished \href{https://github.com/sanssecours/Configuration-File-Parsing/releases}{\tt his thesis} about parsing techniques and parsing tools for configuration files.

We welcome new contributors! Read \href{https://www.libelektra.org/devgettingstarted/ideas}{\tt here} about how to get started.

As first step, you could give us feedback about these release notes. Contact us via our \href{https://issues.libelektra.org}{\tt issue tracker}.

You can download the release from \href{https://www.libelektra.org/ftp/elektra/releases/elektra-0.9.2.tar.gz}{\tt here} or \href{https://github.com/ElektraInitiative/ftp/blob/master/releases/elektra-0.9.2.tar.gz?raw=true}{\tt Git\+Hub}

The \href{https://github.com/ElektraInitiative/ftp/blob/master/releases/elektra-0.9.2.tar.gz.hashsum?raw=true}{\tt hashsums are\+:}


\begin{DoxyItemize}
\item author\+: mpranj
\item file\+: elektra-\/0.\+9.\+2.\+tar.\+gz
\item size\+: 7416188
\item md5sum\+: 6e92ebcbef31cdeab91d228b61456947
\item sha1\+: 8f874de3e7a47baa55d7c5106efbbae635fff499
\item sha256\+: 6f2fcf8aaed8863e1cc323265ca2617751ca50dac974b43a0811bcfd4a511f2e
\end{DoxyItemize}

The release tarball is also available signed by Mihael Pranjic using Gnu\+PG from \href{https://www.libelektra.org/ftp/elektra/releases/elektra-0.9.2.tar.gz.gpg}{\tt here} or on \href{https://github.com/ElektraInitiative/ftp/blob/master/releases/elektra-0.9.2.tar.gz.gpg?raw=true}{\tt Git\+Hub}.

The following G\+PG Key was used to sign this release\+: 9\+C18145\+C22\+F9\+E746\+D743\+D\+E\+C59\+E\+C\+C0\+F4\+C\+F0359\+C7B

Already built A\+P\+I-\/\+Docu can be found \href{https://doc.libelektra.org/api/0.9.2/html/}{\tt here} or on \href{https://github.com/ElektraInitiative/doc/tree/master/api/0.9.2}{\tt Git\+Hub}.

Subscribe to the \href{https://www.libelektra.org/news/feed.rss}{\tt R\+SS feed} to always get the release notifications.

If you also want to participate, or for any questions and comments please contact us via our issue tracker \href{http://issues.libelektra.org}{\tt on Git\+Hub}.

\href{https://www.libelektra.org/news/0.9.2-release}{\tt Permalink to this N\+E\+WS entry}

For more information, see \href{https://libelektra.org}{\tt https\+://libelektra.\+org}

Best regards, \href{https://www.libelektra.org/developers/authors}{\tt Elektra Initiative} 