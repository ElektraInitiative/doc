
\begin{DoxyItemize}
\item infos = Information about the shell plugin is in keys below
\item infos/author = Name \href{mailto:name@libelektra.org}{\tt name@libelektra.\+org}
\item infos/licence = B\+S\+D
\item infos/needs =
\item infos/provides =
\item infos/placements = postgetstorage postcommit postrollback
\item infos/status = preview unfinished nodoc
\item infos/description =
\end{DoxyItemize}

The shell plugin executes shell commandos after set, get or error.

The configuration keys


\begin{DoxyItemize}
\item {\ttfamily execute/set}
\item {\ttfamily execute/get}
\item {\ttfamily execute/error}
\end{DoxyItemize}

are used to store the shell commands.

The configuration keys


\begin{DoxyItemize}
\item {\ttfamily execute/set/return}
\item {\ttfamily execute/get/return}
\item {\ttfamily execute/error/return}
\end{DoxyItemize}

can be compared against the return values of the shell commandos.

\subsection*{Example}

``` \% cat /tmp/log cat\+: /tmp/log\+: No such file or directory

\% kdb mount /tmp/test.ini system/shelltest ini array= shell 'execute/set=echo set $>$$>$ /tmp/log,execute/get=echo get $>$$>$ /tmp/log,execute/get/return=0'

\% kdb set system/shelltest Create a new key system/shelltest with null value

\% cat /tmp/log get set ``` 