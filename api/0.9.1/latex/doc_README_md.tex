This folder contains documentation for “\+Elektra – the configuration framework for everyone”. If you do not know what Elektra is, then we recommend that you check out our \href{https://www.libelektra.org/home}{\texttt{ homepage}} first. This Read\+Me deals with the content of the documentation folder and should give you a hint where to look for specific information.


\begin{DoxyItemize}
\item \mbox{\hyperlink{doc_GOALS_md}{Goals}}\+: We specify the goals and target audiences for Elektra in this document.
\item \mbox{\hyperlink{doc_WHY_md}{Why}}\+: This document describes why you should use Elektra.
\item \mbox{\hyperlink{doc_VISION_md}{Vision}}\+: This document describes the vision behind Elektra.
\item \mbox{\hyperlink{doc_BIGPICTURE_md}{Big Picture}}\+: This document provides an birds eye view of Elektra and the key database (K\+DB).
\item \mbox{\hyperlink{doc_SECURITY_md}{Security}}\+: This guideline shows how Elektra handles security concerns.
\item \mbox{\hyperlink{md_doc_tutorials_README_doc_tutorials_README_md}{Tutorials}}\+: The tutorials folder provides various {\bfseries{user related tutorials}}. If you are interested in {\bfseries{developer related tutorials}} instead, then please take a look at the folder \mbox{\hyperlink{md_doc_dev_README_doc_dev_README_md}{dev}}.
\item News\+: The news folder contains release notes and other recent information about Elektra.
\item \mbox{\hyperlink{doc_paper_README_md}{Paper}}\+: This directory contains a research paper about Elektra, also available in \href{http://joss.theoj.org/papers/10.21105/joss.00044}{\texttt{ P\+DF}} format.
\end{DoxyItemize}


\begin{DoxyItemize}
\item \mbox{\hyperlink{doc_INSTALL_md}{Installation}}\+: These instructions tell you how you can install Elektra in your favorite operating system.
\item \mbox{\hyperlink{doc_COMPILE_md}{Compile}}\+: If you want to compile Elektra from source please take a look at this document.
\item Help\+: This folder contains our man pages in Markdown format. The folder man contains these man pages in roff format, which you can read using the Unix utility \href{https://en.wikipedia.org/wiki/Man_page}{\texttt{ {\ttfamily man}}} if you already installed Elektra.
\end{DoxyItemize}


\begin{DoxyItemize}
\item A\+PI\+: This overview of the application programming interface tells you how you can develop an application that uses Elektra.
\item \mbox{\hyperlink{doc_DESIGN_md}{Design}}\+: This document describes the design of Elektra’s C A\+PI.
\end{DoxyItemize}


\begin{DoxyItemize}
\item \href{/home/mpranj/workspace/libelektra/doc/METADATA.ini}{\texttt{ Metadata}}\+: This document specifies data about the K\+DB (meta information), like supported data types and configuration options.
\item \href{/home/mpranj/workspace/libelektra/doc/CONTRACT.ini}{\texttt{ Contract}}\+: The plugin contract specifies keys and values that an \mbox{\hyperlink{src_plugins_README_md}{Elektra plugin}} provides.
\end{DoxyItemize}


\begin{DoxyItemize}
\item \mbox{\hyperlink{doc_CODING_md}{Coding}}\+: The coding guidelines describe the basic rules you should keep in mind when you want to contribute code to Elektra.
\item \mbox{\hyperlink{doc_GIT_md}{Git}}\+: This document describes how we use the version control system \href{https://git-scm.com}{\texttt{ git}} to develop Elektra.
\item \mbox{\hyperlink{doc_IDEAS_md}{Ideas}}\+: If you want to contribute to Elektra and do not know what, you can either take a look here or at our \href{http://libelektra.org/issues}{\texttt{ issue tracker}}.
\item To\+Do\+: This folder contains various To\+Do items for future releases of Elektra.
\item \mbox{\hyperlink{doc_AUTHORS_md}{Authors}}\+: This file lists information about Elektra’s authors.
\end{DoxyItemize}


\begin{DoxyItemize}
\item \mbox{\hyperlink{doc_images_README_md}{Images}}\+: The images folder contains logos and other promotional material.
\item \mbox{\hyperlink{doc_decisions_README_md}{Decisions}}\+: If you are interested in why Elektra uses a certain technology or strategy, then please check out the documents in this folder.
\item \mbox{\hyperlink{doc_markdownlinkconverter_README_md}{Markdown Link Converter}}\+: This tool converts links in Markdown files to make them usable in our \href{https://doc.libelektra.org/api/current/html}{\texttt{ Doxygen documentation}}.
\item Usecases\+: This folder contains use cases for our \href{https://www.libelektra.org/auth/login}{\texttt{ snippet sharing service}} and the upcoming web user interface for the K\+DB.
\item \mbox{\hyperlink{doc_help_elektra-glossary_md}{Glossary}}\+: The glossary explains common terminology used in the documentation. 
\end{DoxyItemize}