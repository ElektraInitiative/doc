
\begin{DoxyItemize}
\item guid\+: 38640673-\/3e07-\/4cff-\/9647-\/f6bdd89b1b08
\item author\+: Markus Raab
\item pub\+Date\+: Tue, 04 Nov 2014 10\+:48\+:14 +0100
\item short\+Desc\+: adds qt-\/gui, several optimizations \& fixes
\end{DoxyItemize}

Again we managed to do an amazing feature release in just two month. In 416 commits we modified 393 files with 23462 insertions(+) and 9046 deletions(-\/).

\subsection*{Most awaited}

The most awaited feature in this release is certainly the {\itshape qt-\/gui} developed by Raffael Pancheri. It includes a rich feature set including searching, unmounting, importing and exporting. A lot of functionality is quite stable now, even though its version is 0.\+0.\+1 alpha. If you find any bugs or want to give general feedback, feel free to use the issue tracker of the Elektra project. A screenshot can be found \href{https://github.com/ElektraInitiative/libelektra/blob/master/doc/images/screenshot-qt-gui.png}{\tt here} To compile it (together with Elektra), see the R\+E\+A\+D\+ME \href{https://github.com/ElektraInitiative/libelektra/tree/master/src/tools/qt-gui}{\tt here}

Manuel Mausz also has been very active and developed glib+gi bindings. These bindings make Elektra more friendly to the glib/gtk/gnome world. Using the gobject introspection python3 and lua bindings were developed. Additionally he got rid of all clang warnings.

Felix Berlakovich also made progress\+: \href{https://github.com/ElektraInitiative/libelektra/tree/master/src/plugins/ini}{\tt the ini plugin} now supports multiline and which can be dynamically turned on and off, i.\+e. during mounting (thanks to Felix)

Last, but not least, Kai-\/\+Uwe ported Elektra to Windows7. Min\+GW is now one more supported compiler (tested on build-\/server, see later). Astonishingly, it was only little effort necessary\+: Basically we only needed a new implementation of the resolver, which is now called {\itshape wresolver}. Different from the {\itshape resolver} it lacks the sophisticated multi-\/process and multi-\/thread atomicity properties. On the plus side we now have a resolver that is very easy to study and understand and still works as file resolver ({\itshape noresolver} does not).

\subsection*{Interfaces}

A\+B\+I/\+A\+PI of the C-\/\+A\+PI is still completely stable even though under the hood a lot was changed. All testcases compiled against the previous version still run against Elektra 0.\+8.\+9.

This is, however, not the case for libtools. For Min\+GW porting it was necessary to rename an enum related to merging because it conflicted with an already defined M\+A\+C\+RO.

For maintainers also some changes are necessary. For Min\+GW and to actually use the flexibility of the new resolver variants two new C\+Make Variables are introduced\+: K\+D\+B\+\_\+\+D\+E\+F\+A\+U\+L\+T\+\_\+\+R\+E\+S\+O\+L\+V\+ER and K\+D\+B\+\_\+\+D\+E\+F\+A\+U\+L\+T\+\_\+\+S\+T\+O\+R\+A\+GE.

More importantly for maintainers the C\+Make variables regarding S\+W\+IG bindings are now abandoned in favour to the new variable B\+I\+N\+D\+I\+N\+GS that works like P\+L\+U\+G\+I\+NS and T\+O\+O\+LS. Just start with \begin{DoxyVerb}-DBINDINGS=ALL
\end{DoxyVerb}


and C\+Make should remove the bindings that have missing dependencies on your system. Remember that glib and gi (i.\+e. {\itshape gi\+\_\+python3} and {\itshape gi\+\_\+lua}) bindings were introduced, too. Additionally, the {\itshape cpp} binding can now be deactivated if not added to B\+I\+N\+D\+I\+N\+GS.

Finally, the {\itshape gen} tool added a Python package called {\ttfamily support}.

\subsection*{Other Bits}

A proof of concept storage plugin {\ttfamily regexstore} was added. It allows one to capture individual configuration options within an otherwise not understood configuration file (e.\+g. for vimrc or emacs where the configuration file may contain programming constructs).

Most tests now also work with the B\+U\+I\+L\+D\+\_\+\+S\+H\+A\+R\+ED variant (from our knowledge all would work now, but some are still excluded if B\+U\+I\+L\+D\+\_\+\+F\+U\+LL and B\+U\+I\+L\+D\+\_\+\+S\+T\+A\+T\+IC is disabled. Please report issues if you want to use uncommon C\+Make combinations).

A small but very important step towards specifying configuration files is the new proposed A\+PI method ks\+Lookup\+By\+Spec (and ks\+Lookup implementing cascading search). It introduces a {\ttfamily logical view} of configuration that in difference to the {\ttfamily physical view} of configuration does not have namespaces, but everything is below the root \char`\"{}/\char`\"{}. Additionally, contextual values now allow one to be compile-\/time configured using C++-\/\+Policies. These are small puzzle pieces that will fit into a greater picture at a later time.

A (data) race detection tool was implemented. Using it a configurable number of processes and threads it tries to \hyperlink{group__kdb_ga11436b058408f83d303ca5e996832bcf}{kdb\+Set()} a different configuration at (nearly) the same time.

With this tool the resolver could be greatly be improved (again). It now uses stat with nanosecond precision that will be updated for every successful \hyperlink{group__kdb_ga11436b058408f83d303ca5e996832bcf}{kdb\+Set()}. Even if the configuration file was modified manually (not using Elektra) the next \hyperlink{group__kdb_ga11436b058408f83d303ca5e996832bcf}{kdb\+Set()} then is much more likely to fail. Additionally a recursive mutex now protects the file locking mechanism.

The build server now additionally has following build jobs\+:


\begin{DoxyItemize}
\item \href{https://build.libelektra.org/job/elektra-gcc-i386/}{\tt i386 build\+:}\+: We had an i386 regression, because none of the developers seems to use i386 anymore.
\item \href{https://build.libelektra.org/job/elektra-gcc-configure-debian/}{\tt Configure Debian Script} Calls the scripts/configure-\/debian(-\/wheezy).
\item \href{https://build.libelektra.org/job/elektra-local-installation/}{\tt Local Installation\+:} We had an regression that local installation was not possible because of a bash completion file installed to /etc. This build tests if it is possible to install Elektra in your home directory (and calls kdb run\+\_\+all afterwards)
\item \href{https://build.libelektra.org/job/elektra-test-bindings/}{\tt Test bindings\+:} Compiles and tests A\+LL bindings.
\item \href{https://build.libelektra.org/job/elektra-gcc-configure-mingw-w64/}{\tt Mingw\+:} Compiles Elektra using mingw.
\end{DoxyItemize}

Many more examples were written and are used within doxygen. Most snippets now can also be found in compilable files\+:


\begin{DoxyItemize}
\item \href{https://github.com/ElektraInitiative/libelektra/tree/master/examples/keyNew.c}{\tt key\+New examples}
\item \href{https://github.com/ElektraInitiative/libelektra/tree/master/examples/keyCopy.c}{\tt key\+Copy examples}
\item \href{https://github.com/ElektraInitiative/libelektra/tree/master/src/bindings/cpp/examples/cpp_example_dup.cpp}{\tt C++ deep dup}
\item \href{https://github.com/ElektraInitiative/libelektra/tree/master/src/bindings/cpp/examples/cpp_example_ordering.cpp}{\tt How to put Key in different data structures}
\item \href{https://github.com/ElektraInitiative/libelektra/tree/master/scripts/mount-augeas}{\tt Mount some config files using augeas}
\item \href{https://github.com/ElektraInitiative/libelektra/tree/master/scripts/mount-info}{\tt Mount system information}
\end{DoxyItemize}

Most plugins now internally use the same C\+Make function {\ttfamily add\+\_\+plugin} which makes plugin handling more consistent.

Felix converted the M\+E\+T\+A\+D\+A\+TA spec to ini files and added a proposal how comments can be improved.

\subsubsection*{Refactoring\+:}


\begin{DoxyItemize}
\item reuse of utilities in gen code generator
\item the gen support library is now in its own package ({\ttfamily support})
\item refactor array handling
\item internal comparision functions (key\+Compare\+By\+Name)
\end{DoxyItemize}

\subsubsection*{Optimization\+:}


\begin{DoxyItemize}
\item lookup\+By\+Name does not need to allocate two keys
\item lookups in generated code
\item prefer to use allocation on stack
\end{DoxyItemize}

\subsubsection*{Fixes\+:}


\begin{DoxyItemize}
\item disable cast that segfaults on i386 (only testing code was affected)
\item fix key\+Add\+Base\+Name in xmltool and testing code
\item support non-\/system installation (e.\+g. in home directory)
\item rewrote test cases to use succeed\+\_\+if\+\_\+same to avoid crashes on null pointers
\item allow one to use python 2.\+6 for kdb gen
\item improve exception messages
\item use memcasecmp (fix lookup ignoring case)
\item fix memory leaks (ini)
\item text messages for some warnings/errors
\item fix many issues regarding C\+Make, more variants of setting C\+Make options are now allowed.
\item cmake policies fixes allow us to use cmake version $>$ 3
\end{DoxyItemize}

\subsection*{Get It!}

You can download the release from \href{http://www.markus-raab.org/ftp/elektra/releases/elektra-0.8.9.tar.gz}{\tt here}


\begin{DoxyItemize}
\item size\+: 1936524
\item md5sum\+: 001c4ec67229046509a0cb9eda223dc6
\item sha1\+: 79ea9b83c08ed4c347ed0100b5e0e2d3309b9d04
\item sha256\+: e0895bba28a27fb37f36f59ef77c95235f3a9c54fb71aa6f648566774d276568
\end{DoxyItemize}

already built A\+P\+I-\/\+Docu can be found \href{https://doc.libelektra.org/api/0.8.9/html/}{\tt here}

For more information, see \href{https://www.libelektra.org}{\tt https\+://www.\+libelektra.\+org}

Best regards, Markus 