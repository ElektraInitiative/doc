
\begin{DoxyItemize}
\item infos = Information about the dump plugin is in keys below
\item infos/author = Markus Raab \href{mailto:elektra@libelektra.org}{\tt elektra@libelektra.\+org}
\item infos/licence = B\+SD
\item infos/needs =
\item infos/provides = storage/dump
\item infos/recommends =
\item infos/placements = getstorage setstorage
\item infos/status = productive maintained conformant unittest tested nodep -\/1000
\item infos/metadata =
\item infos/description = Dumps into a format tailored for complete Key\+Set semantics
\end{DoxyItemize}

This plugin is a storage plugin that supports full Elektra semantics. Combined with a resolver plugin it already assembles a fully featured backend. No other plugins are needed.

\subsection*{Format}

The file format edf (Elektra dump format) consists of a simple command language with arguments. When an argument is binary or string data the length needs to be passed first. Because the size is known in advance, any binary dump is accepted. Terminating characters present no problem. The commands are assembled similar to the ones present in Elektra’s A\+PI.

The file starts with the magic word {\ttfamily kdb\+Open} followed by a version number. Processing can be stopped immediately when it is not in Elektra’s dump format at all. A wrong version number most likely indicates that the version of the plugin is too old to recognise all commands in the file. The basic idea of the dump plugin is to write out the way that the Key\+Set needs to be constructed. The dump plugin interprets such a file. The file also looks similar to C code that would create the Key\+Set. Keys can contain any binary values and arbitrary metadata and are still stored and parsed correctly. The dump plugin can even reconstruct pointers to metadata to save memory. When a pointer to the same region of memory is found, a special command {\ttfamily key\+Copy\+Meta} is written out that is able to reconstruct the data structure the way it was before. The commands were designed to make parsing of the file an easy task.

\subsubsection*{Format Examples}

The serialised configuration can look like (0 bytes at end of strings are omitted)\+: \begin{DoxyVerb}kdbOpen 1
ksNew 207
keyNew 27 1
system/elektra/mountpoints
keyMeta 8 27
commentBelow are the mountpoints.
keyEnd
keyNew 32 19
system/elektra/mountpoints/dbusserialised Backend
keyEnd keyNew 39 1
system/elektra/mountpoints/dbus/config
keyMeta 8 72
commentThis is a configuration for a backend, see subkeys for more information
keyEnd
keyNew 53 1
system/elektra/mountpoints/fstab/config/struct/FStab
keyCopyMeta 59 11
system/elektra/mountpoints/file
systems/config/struct/FStabcheck/type
keyEnd
ksEnd
\end{DoxyVerb}


\subsection*{Limitations}

(status -\/1000)


\begin{DoxyItemize}
\item It is quite slow
\item Files cannot easily edited by hand
\end{DoxyItemize}

\subsection*{Examples}

Export a Key\+Set using {\ttfamily dump}\+: \begin{DoxyVerb}kdb export system/example dump > example.ecf
\end{DoxyVerb}


Import a Key\+Set using {\ttfamily dump}\+: \begin{DoxyVerb}cat example.ecf | kdb import system/example dump
\end{DoxyVerb}


Using grep/diff or other Unix tools on the dump file. Make sure that you treat it as text file, e.\+g.\+: \begin{DoxyVerb}grep --text mountpoints example.ecf\end{DoxyVerb}
 