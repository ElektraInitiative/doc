{\ttfamily kdb set $<$key name$>$ \mbox{[}$<$value$>$\mbox{]}}

Where {\ttfamily key name} is the name of the key you wish to set the value of (or create) and {\ttfamily value} is the value you would like to set the key to. If the {\ttfamily value} argument is not passed, the key will be set to a value of {\ttfamily null}.

\subsection*{D\+E\+S\+C\+R\+I\+P\+T\+I\+ON}

This command allows the user to set the value of an individual key.

\subsection*{E\+M\+P\+TY V\+A\+L\+U\+ES}

To set a key to an empty value, {\ttfamily \char`\"{}\char`\"{}} should be passed for the {\ttfamily value} argument.

\subsection*{O\+P\+T\+I\+O\+NS}


\begin{DoxyItemize}
\item {\ttfamily -\/H}, {\ttfamily -\/-\/help}\+: Show the man page.
\item {\ttfamily -\/V}, {\ttfamily -\/-\/version}\+: Print version info.
\item {\ttfamily -\/p}, {\ttfamily -\/-\/profile $<$profile$>$}\+: Use a different kdb profile.
\item {\ttfamily -\/C}, {\ttfamily -\/-\/color $<$when$>$}\+: Print never/auto(default)/always colored output.
\item {\ttfamily -\/v}, {\ttfamily -\/-\/verbose}\+: Explain what is happening.
\item {\ttfamily -\/q}, {\ttfamily -\/-\/quiet}\+: Suppress non-\/error messages.
\item {\ttfamily -\/N}, {\ttfamily -\/-\/namespace=NS}\+: Specify the namespace to use when writing cascading keys. See \href{#KDB}{\tt below in K\+DB}.
\end{DoxyItemize}

\subsection*{K\+DB}


\begin{DoxyItemize}
\item {\ttfamily /sw/elektra/kdb/\#0/current/verbose}\+: Same as {\ttfamily -\/v}\+: Explain what is happening.
\item {\ttfamily /sw/elektra/kdb/\#0/current/quiet}\+: Same as {\ttfamily -\/q}\+: Suppress default messages.
\item {\ttfamily /sw/elektra/kdb/\#0/current/namespace}\+: Specifies which default namespace should be used when setting a cascading name. By default the namespace is user, except {\ttfamily kdb} is used as root, then {\ttfamily system} is the default.
\end{DoxyItemize}

\subsection*{E\+X\+A\+M\+P\+L\+ES}

To set a Key to the value {\ttfamily Hello World!}\+: {\ttfamily kdb set user/example/key \char`\"{}\+Hello World!\char`\"{}}

To create a new key with a null value\+: {\ttfamily kdb set user/example/key}

To set a key to an empty value\+: {\ttfamily kdb set user/example/key \char`\"{}\char`\"{}}

To create bookmarks\+: {\ttfamily kdb set user/sw/elektra/kdb/\#0/current/bookmarks} followed by\+: {\ttfamily kdb set user/sw/elektra/kdb/\#0/current/bookmarks/kdb user/sw/elektra/kdb/\#0/current}

\subsection*{S\+EE A\+L\+SO}


\begin{DoxyItemize}
\item \hyperlink{md_doc_help_kdb_doc_help_kdb_md}{kdb(1)} for how to configure the kdb utility and use the bookmarks.
\item \hyperlink{md_doc_help_elektra-key-names_doc_help_elektra-key-names_md}{elektra-\/key-\/names(7)} for an explanation of key names.
\item \hyperlink{md_doc_help_elektra-values_doc_help_elektra-values_md}{elektra-\/values(7)} for the difference between empty and null values. 
\end{DoxyItemize}