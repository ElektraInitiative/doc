
\begin{DoxyItemize}
\item infos = Information about xmltool plugin is in keys below
\item infos/author = Markus Raab \href{mailto:elektra@libelektra.org}{\tt elektra@libelektra.\+org}
\item infos/licence = B\+SD
\item infos/provides = storage/xml
\item infos/needs =
\item infos/placements = getstorage setstorage
\item infos/status = maintained unittest final memleak unfinished old nodoc
\item infos/description = Storage using libelektratools xml format.
\end{DoxyItemize}

This plugin is a storage plugin allowing Elektra to read and write xml formatted files. It uses the {\ttfamily libelektratools} 0.\+7 xml format.

This plugin can be used for migration of Key Databases from 0.\+7 -\/$>$ 0.\+8. It should not be used otherwise.

\subsection*{Dependencies}


\begin{DoxyItemize}
\item {\ttfamily libxml2-\/dev}
\end{DoxyItemize}

\subsection*{Restrictions}


\begin{DoxyItemize}
\item only supports metadata as defined in Elektra 0.\+7
\item null and empty values are not distinguished
\item exported relative to first key found, not to parent key (ks\+Get\+Common\+Parent\+Name)
\item error messages vague (no difference between error opening file and validation errors)
\end{DoxyItemize}

\subsection*{Examples}

After you have upgraded Elektra, you can import xml files from Elektra 0.\+7\+: \begin{DoxyVerb}kdb import system xmltool < system.xml
kdb import user xmltool < user.xml
\end{DoxyVerb}


Or you can also mount an xml file using {\ttfamily xmltool} (not recommended!)\+: \begin{DoxyVerb}kdb mount /etc/example.xml system/example xmltool\end{DoxyVerb}
 