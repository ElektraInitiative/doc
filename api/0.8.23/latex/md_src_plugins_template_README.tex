
\begin{DoxyItemize}
\item infos = Information about the template plugin is in keys below
\item infos/author = Author Name \href{mailto:elektra@libelektra.org}{\tt elektra@libelektra.\+org}
\item infos/licence = B\+SD
\item infos/needs =
\item infos/provides =
\item infos/recommends =
\item infos/placements = prerollback rollback postrollback getresolver pregetstorage getstorage postgetstorage setresolver presetstorage setstorage precommit commit postcommit
\item infos/status = recommended productive maintained reviewed conformant compatible coverage specific unittest shelltest tested nodep libc configurable final preview memleak experimental difficult unfinished old nodoc concept orphan obsolete discouraged -\/1000000
\item infos/metadata =
\item infos/description = one-\/line description of template
\end{DoxyItemize}

Copy this template if you want to start a new plugin written in C.

\subsection*{Usage}

You can use {\ttfamily scripts/copy-\/template} to automatically rename everything to your plugin name\+: \begin{DoxyVerb}    cd src/plugins
    ../../scripts/copy-template yourplugin
\end{DoxyVerb}


Then update the R\+E\+A\+D\+M\+E.\+md of your newly created plugin\+:


\begin{DoxyItemize}
\item enter your full name+email in {\ttfamily infos/author}
\item make sure {\ttfamily status}, {\ttfamily placements}, and other clauses conform to descriptions in {\ttfamily doc/\+C\+O\+N\+T\+R\+A\+C\+T.\+ini}
\item update the one-\/line description above
\item add your plugin in {\ttfamily src/plugins/\+R\+E\+A\+D\+M\+E.\+md}
\item and rewrite the rest of this {\ttfamily R\+E\+A\+D\+M\+E.\+md} to give a great explanation of what your plugin does
\end{DoxyItemize}

\subsection*{Dependencies}

None.

\subsection*{Examples}


\begin{DoxyCode}
# Backup-and-Restore: user/examples/template

kdb set user/examples/template/key value
#> Create a new key user/examples/template/key with string "value"

kdb get /examples/template/key
#> value
\end{DoxyCode}


\subsection*{Limitations}

None. 