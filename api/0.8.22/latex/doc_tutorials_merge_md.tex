\subsection*{Introduction}

The kdb tool allows users to access and perform functions on the Elektra Key Database from the command line. We added a new command to this very useful tool, the {\ttfamily merge} command. This command allows a user to perform a three-\/way merge of Key\+Sets from the {\ttfamily kdb} tool.

The command to use this tool is\+:


\begin{DoxyCode}
kdb merge [options] ourpath theirpath basepath resultpath
\end{DoxyCode}


The standard naming scheme for a three-\/way merge consists of {\ttfamily ours}, {\ttfamily theirs}, and {\ttfamily base}\+:


\begin{DoxyItemize}
\item {\ttfamily ours} refers to the local copy of a file
\item {\ttfamily theirs} refers to a remote copy
\item {\ttfamily base} refers to their common ancestor.
\end{DoxyItemize}

This works very similarly for Key\+Sets, especially ones that consist of mounted configuration files.

For mounted configuration files\+:


\begin{DoxyItemize}
\item {\ttfamily ours} should be the user\textquotesingle{}s copy
\item {\ttfamily theirs} would be the maintainers copy,
\item {\ttfamily base} would be the previous version of the maintainer\textquotesingle{}s copy.
\end{DoxyItemize}

If the user is just trying to accomplish a three-\/way merge using any two arbitrary keysets that share a base, it doesn\textquotesingle{}t matter which ones are defined as {\ttfamily ours} or {\ttfamily theirs} as long as they use the correct base Key\+Set. In {\ttfamily kdb merge}, {\ttfamily ourpath}, {\ttfamily theirpath}, and {\ttfamily basepath} work just like {\ttfamily ours}, {\ttfamily theirs}, and {\ttfamily base} except each one represents the root of a Key\+Set. The argument {\ttfamily resultpath} is pretty self-\/explanatory, it is just where you want the result of the merge to be saved under. It\textquotesingle{}s worth noting, {\ttfamily resultpath} should be empty before attempting a merge, otherwise there can be unintended consequences.

\subsection*{Options}

As for the options, there are a few basic options\+:


\begin{DoxyItemize}
\item {\ttfamily -\/i}, {\ttfamily -\/-\/interactive}\+: which attempts the merge in an interactive way
\item {\ttfamily -\/t}, {\ttfamily -\/-\/test}\+: which tests the proposed merge and informs you about possible conflicts
\item {\ttfamily -\/f}, {\ttfamily -\/-\/force}\+: which overwrites any Keys in {\ttfamily resultpath}
\end{DoxyItemize}

\subsubsection*{Strategies}

Additionally there is an option to specify a merge strategy, which is very important.

The option for strategy is\+:


\begin{DoxyItemize}
\item {\ttfamily -\/s $<$name$>$}, {\ttfamily -\/-\/strategy $<$name$>$}\+: which is used to specify a strategy to use in case of a conflict
\end{DoxyItemize}

The current list of strategies are\+:


\begin{DoxyItemize}
\item {\ttfamily preserve}\+: the merge will fail if a conflict is detected
\item {\ttfamily ours}\+: the merge will use our version during a conflict
\item {\ttfamily theirs}\+: the merge will use their version during a conflict
\item {\ttfamily base}\+: the merge will use the base version during a conflict
\end{DoxyItemize}

If no strategy is specified, the merge will default to the preserve strategy as to not risk making the wrong decision. If any of the other strategies are specified, when a conflict is detected, merge will use the Key specified by the strategy ({\ttfamily ours}, {\ttfamily theirs}, or {\ttfamily base}) for the resulting Key.

\subsection*{Basic Example}

Basic Usage\+:


\begin{DoxyCode}
kdb merge system/hosts/ours system/hosts/theirs system/hosts/base system/hosts/result
\end{DoxyCode}


\subsection*{Examples Using Strategies}

Here are examples of the same Key\+Sets being merged using different strategies. The Key\+Sets are mounted using the {\ttfamily simpleini} file, the left side of \textquotesingle{}=\textquotesingle{} is the name of the Key, the right side is its string value.

We start with the base Key\+Set, {\ttfamily system/base}\+: \begin{DoxyVerb}    key1=1
    key2=2
    key3=3
    key4=4
    key5=5
\end{DoxyVerb}


Here is our Key\+Set, {\ttfamily system/ours}\+: \begin{DoxyVerb}    key1=apple
    key2=2
    key3=3
    key5=fish
\end{DoxyVerb}


Here is their Key\+Set, {\ttfamily system/theirs}\+: \begin{DoxyVerb}    key1=1
    key2=pie
    key4=banana
    key5=5
\end{DoxyVerb}


Now we will examine the result Key\+Set with the different strategies.

\subsubsection*{Preserve}


\begin{DoxyCode}
kdb merge -s preserve system/ours system/theirs system/base system/result
\end{DoxyCode}


The merge will fail because of a conflict for {\ttfamily key4} since {\ttfamily key4} was deleted in our Key\+Set and edited in their Key\+Set. Since we used preserve, the merge fails and the result Key\+Set is not saved.

\subsubsection*{Ours}


\begin{DoxyCode}
kdb merge -s ours system/ours system/theirs system/base system/result
\end{DoxyCode}


The result Key\+Set, system/result will be\+: \begin{DoxyVerb}    key1=apple
    key2=pie
    key5=fish
\end{DoxyVerb}


Because the conflict of {\ttfamily key4} (it was deleted in {\ttfamily ours} but changed in {\ttfamily theirs}) is solved by using our copy thus deleting the key.

\subsubsection*{Theirs}


\begin{DoxyCode}
kdb merge -s theirs system/ours system/theirs system/base system/result
\end{DoxyCode}


The result Key\+Set, {\ttfamily system/result} will be\+: \begin{DoxyVerb}    key1=apple
    key2=pie
    key4=banana
    key5=fish
\end{DoxyVerb}


Here, the conflict of {\ttfamily key4} is solved by using their copy, thus {\ttfamily key4=banana}.

\subsubsection*{Base}


\begin{DoxyCode}
kdb merge -s base system/ours system/theirs system/base system/result
\end{DoxyCode}


The result Key\+Set, {\ttfamily system/result} will be\+: \begin{DoxyVerb}    key1=apple
    key2=pie
    key4=4
    key5=5
\end{DoxyVerb}


The same conflict is found in {\ttfamily key4}, but here we use the {\ttfamily base} version to solve it so {\ttfamily key4=4}. 