{\ttfamily kdb test $<$path$>$ \mbox{[}$<$test-\/name$>$ ...\mbox{]}}~\newline


Where {\ttfamily path} is the path the user wishes to perform the test under. The option {\ttfamily test-\/name} argument is used to specify which test(s) to run. To run multiple tests, each should be named with a trailing space.~\newline
 If no {\ttfamily test-\/name} is provided, all the tests will be run.~\newline


\subsection*{D\+E\+S\+C\+R\+I\+P\+T\+I\+ON}

This command is used to run part or all of the key database test suite.~\newline
 These tests allow one to user to verify that a backend is capable of storing and retrieving all kinds of configuration keys and values.~\newline


The following tests are available\+: basic string umlauts binary naming meta~\newline


\subsection*{O\+P\+T\+I\+O\+NS}


\begin{DoxyItemize}
\item {\ttfamily -\/H}, {\ttfamily -\/-\/help}\+: Show the man page.
\item {\ttfamily -\/V}, {\ttfamily -\/-\/version}\+: Print version info.
\item {\ttfamily -\/p}, {\ttfamily -\/-\/profile $<$profile$>$}\+: Use a different kdb profile.
\item {\ttfamily -\/C}, {\ttfamily -\/-\/color $<$when$>$}\+: Print never/auto(default)/always colored output.
\end{DoxyItemize}

\subsection*{E\+X\+A\+M\+P\+L\+ES}

To run all tests below the {\ttfamily user/example/tests} key\+:~\newline
 {\ttfamily kdb test user/example/tests}~\newline


To run the {\ttfamily binary} and {\ttfamily naming} tests\+:~\newline
 {\ttfamily kdb test user/example/tests binary naming}~\newline


\subsection*{S\+EE A\+L\+SO}


\begin{DoxyItemize}
\item \hyperlink{md_doc_help_elektra-key-names_doc_help_elektra-key-names_md}{elektra-\/key-\/names(7)} for an explanation of key names. 
\end{DoxyItemize}