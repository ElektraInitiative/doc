Elektra provides a universal and secure framework to store configuration parameters in a global, hierarchical key database.

The core is a small library implemented in C. The plugin-\/based framework fulfills many configuration-\/related tasks to avoid any unnecessary code duplication across applications while it still allows the core to stay without any external dependency. Elektra abstracts from cross-\/platform-\/related issues with an consistent A\+P\+I, and allows applications to be aware of other applications' configurations, leveraging easy application integration.

The man pages can also be viewed online at\+: \href{http://doc.libelektra.org/api/current/html/pages.html}{\tt http\+://doc.\+libelektra.\+org/api/current/html/pages.\+html}

And the page you are currently reading at\+: \href{http://doc.libelektra.org/api/current/html/md_doc_help_kdb.html}{\tt http\+://doc.\+libelektra.\+org/api/current/html/md\+\_\+doc\+\_\+help\+\_\+kdb.\+html}

Concepts are in man page section 7 and are prefixed with {\ttfamily elektra-\/}. You should start reading \hyperlink{md_doc_help_elektra-introduction_doc_help_elektra-introduction_md}{elektra-\/introduction(7)}.

Tools are in man page section 1 and are prefixed with {\ttfamily kdb-\/}. You should start reading \hyperlink{doc_help_kdb-introduction_md}{kdb-\/introduction(1)}.

Documentation of plugins is available using the \hyperlink{md_doc_help_kdb-info_doc_help_kdb-info_md}{kdb-\/info(1)} tool. Run {\ttfamily kdb list} for a list of plugins.

\subsection*{C\+O\+M\+M\+O\+N O\+P\+T\+I\+O\+N\+S}

Every core-\/tool has following options\+:


\begin{DoxyItemize}
\item {\ttfamily -\/\+H}, {\ttfamily -\/-\/help}\+: Show the man page.
\item {\ttfamily -\/\+V}, {\ttfamily -\/-\/version}\+: Print version info.
\item {\ttfamily -\/p}, {\ttfamily -\/-\/profile}=$<$profile$>$\+: Use a different kdb profile, see below.
\end{DoxyItemize}

\subsection*{K\+D\+B}

The {\ttfamily kdb} utility reads its own configuration from three sources within the K\+D\+B (key database)\+:


\begin{DoxyEnumerate}
\item /sw/kdb/$\ast$$\ast$profile$\ast$$\ast$/ (for legacy configuration)
\item /sw/elektra/kdb/\#0/\%/ (for empty profile)
\item /sw/elektra/kdb/\#0/$\ast$$\ast$profile$\ast$$\ast$/ (for current profile)
\end{DoxyEnumerate}

The last source where a configuration value is found, wins.

\subsection*{P\+R\+O\+F\+I\+L\+E\+S}

Profiles allow users to change many/all configuration options of a tool at once. It influences from where the K\+D\+B entries are read. For example if you use\+: {\ttfamily kdb export -\/p admin system}

It will read its format configuration from {\ttfamily /sw/elektra/kdb/\#0/admin/format}.

If you want different configuration per user or directories, you should prefer to use the {\ttfamily user} and {\ttfamily dir} namespaces. Then the correct configuration will be chosen automatically and you do not have to specify the correct {\ttfamily -\/p}.

\subsection*{B\+O\+O\+K\+M\+A\+R\+K\+S}

Elektra recommends \hyperlink{doc_tutorials_application-integration_md}{to use rather long paths} because it ensures flexibility in the future (e.\+g. to use profiles and have a compatible path for new major versions of configuration).

Long paths are, however, cumbersome to enter in the C\+L\+I. Thus one can define bookmarks. Bookmarks are key-\/names that start with {\ttfamily +}. They are only recognized by the {\ttfamily kdb} tool or tools that explicit have support for it. Your applications should not depend on the presence of a bookmark.

Bookmarks are stored below\+: {\ttfamily /sw/elektra/kdb/\#0/current/bookmarks}

For every key found there, a new bookmark will be introduced.

Bookmarks can be used to start key-\/names by using {\ttfamily +} (plus) as first character. The string until the first {\ttfamily /} will be considered as bookmark.

For example, if you set the bookmark kdb\+: {\ttfamily kdb set user/sw/elektra/kdb/\#0/current/bookmarks} {\ttfamily kdb set user/sw/elektra/kdb/\#0/current/bookmarks/kdb user/sw/elektra/kdb/\#0/current}

You are able to use\+: {\ttfamily kdb ls +kdb/bookmarks} {\ttfamily kdb get +kdb/format}

\subsection*{R\+E\+T\+U\+R\+N V\+A\+L\+U\+E\+S}


\begin{DoxyItemize}
\item 0\+: successful.
\item 1\+: Invalid options passed.
\item 2\+: Invalid arguments passed.
\item 3\+: Command terminated unsuccessfully.
\item 4\+: Unknown command.
\item 5\+: K\+D\+B Error, could not read/write from/to K\+D\+B.
\item 7-\/8\+: Unkown errors, wrong exceptions thrown.
\item 9-\/10\+: Reserved error codes.
\end{DoxyItemize}

\subsection*{O\+P\+T\+I\+O\+N\+S}

Commonly used options for all programs\+:


\begin{DoxyItemize}
\item {\ttfamily -\/\+H}, {\ttfamily -\/-\/help}\+: Show the man page.
\item {\ttfamily -\/\+V}, {\ttfamily -\/-\/version}\+: Print version info.
\item {\ttfamily -\/p $<$profile$>$}, {\ttfamily -\/-\/profile $<$profile$>$}\+: Use a different profile instead of current.
\end{DoxyItemize}

\subsection*{S\+E\+E A\+L\+S\+O}


\begin{DoxyItemize}
\item \hyperlink{md_doc_help_elektra-introduction_doc_help_elektra-introduction_md}{elektra-\/introduction(7)}
\item \hyperlink{doc_help_kdb-introduction_md}{kdb-\/introduction(1)} 
\end{DoxyItemize}