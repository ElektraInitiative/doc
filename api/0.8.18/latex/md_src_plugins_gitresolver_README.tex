
\begin{DoxyItemize}
\item infos = Information about the gitresolver plugin is in keys below
\item infos/author = Name \href{mailto:name@libelektra.org}{\tt name@libelektra.\+org}
\item infos/licence = B\+S\+D
\item infos/needs =
\item infos/provides = resolver
\item infos/recommends =
\item infos/placements = rollback getresolver setresolver commit
\item infos/status = recommended productive maintained reviewed conformant compatible coverage specific unittest shelltest tested libc configurable final preview
\item infos/metadata =
\item infos/description =
\end{DoxyItemize}

gitresolver is a resolver that fetches from a local git repository during the get-\/phase and commits them back at the end of the set-\/phase. It operates on a temporary copy of the latest version of your file fetched from the repository. If the temporary copy modified, a new commit with the modified version will be created. Local files won't be touched.

\subsection*{Options}

{\ttfamily branch} defines the branch to work on. Default\+: master {\ttfamily tracking} can be either {\ttfamily object} or {\ttfamily head} (default). if set to {\ttfamily object} a conflict will only occur if the file in the git repository has been updated while you were working on it. {\ttfamily head} will cause a conflict if the {\ttfamily H\+E\+A\+D} commit has been updated.

\subsection*{Limitations}

Currently it only works on already existing files inside existing git repositories.

\subsection*{Examples}

``` kdb mount -\/\+R gitresolver /path/to/my/gitrepo/file.ini system/gittest ini shell execute/set='cd /path/to/my/gitrepo/ \&\& git commit --amend' ``` 