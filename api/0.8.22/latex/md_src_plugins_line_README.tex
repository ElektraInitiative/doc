
\begin{DoxyItemize}
\item infos = Information about line plugin is in keys below
\item infos/author = Ian Donnelly \href{mailto:ian.s.donnelly@gmail.com}{\tt ian.\+s.\+donnelly@gmail.\+com}
\item infos/provides = storage/line
\item infos/licence = B\+SD
\item infos/needs = null
\item infos/placements = getstorage setstorage
\item infos/status = maintained unittest nodep libc final limited
\item infos/description = storage plugin which stores each line from a file
\end{DoxyItemize}

This plugin is useful if you have a file in a format not supported by any other plugin and want to use the Elektra tools to edit individual lines.

This plugin is designed to save each line from a file as a key. The keys form an array. The key names are determined by the line number such as {\ttfamily \#3} or {\ttfamily \#\+\_\+12} for lines 4 and 13. The plugin considers {\ttfamily \#0} to be the first line. The plugin will automatically add {\ttfamily \+\_\+} to the beginning of key names in order to keep them in numerical order (like specified for Elektra arrays).

The value of each key hold the content of the actual file line-\/by-\/line.

\subsection*{Examples}

For example, consider the following content of the file {\ttfamily $\sim$/.config/line} where the numbers on the left represent the line numbers\+: \begin{DoxyVerb}1  setting1 true
2  setting2 false
3  setting3 1000
4  #comment
5
6
7  //some other comment
8
9  setting4 -1
\end{DoxyVerb}


We mount that file by\+: \begin{DoxyVerb}> sudo kdb mount line user/line line
\end{DoxyVerb}


This file would result in the following keyset which is being displayed as {\ttfamily key\+: value}, e.\+g. with\+: \begin{DoxyVerb}> kdb export -c "format=%s: %s" user/line simpleini

#0: setting1 true
#1: setting2 false
#2: setting3 1000
#3: #comment
#4:
#5:
#6: //some other comment
#7:
#8: setting4 -l
\end{DoxyVerb}


\subsubsection*{Creating Files}

``` \section*{Backup-\/and-\/\+Restore\+:/examples/line}

sudo kdb mount line /examples/line line

kdb set /examples/line/add something kdb set /examples/line/ignored huhu kdb set /examples/line ignored \# adding parent key does nothing kdb set /examples/line/add here

cat {\ttfamily kdb file /examples/line} \#$>$ something \#$>$ huhu \#$>$ here

kdb ls /examples/line \#$>$ user/examples/line \#$>$ user/examples/line/\#0 \#$>$ user/examples/line/\#1 \#$>$ user/examples/line/\#2

kdb set /examples/line/\#1 huhu \#$>$ Using name user/examples/line/\#1 \#$>$ Set string to \char`\"{}huhu\char`\"{}

kdb export /examples/line line \#$>$ something \#$>$ huhu \#$>$ here

sudo kdb umount /examples/line 
\begin{DoxyCode}
### Other Tests
\end{DoxyCode}
 \section*{Backup-\/and-\/\+Restore\+:/examples/line}

sudo kdb mount line /examples/line line

\section*{create and initialize testfile}

echo \textquotesingle{}setting1 true\textquotesingle{} $>$ {\ttfamily kdb file /examples/line} echo \textquotesingle{}setting2 false\textquotesingle{} $>$$>$ {\ttfamily kdb file /examples/line} echo \textquotesingle{}setting3 1000\textquotesingle{} $>$$>$ {\ttfamily kdb file /examples/line} echo \textquotesingle{}\#comment\textquotesingle{} $>$$>$ {\ttfamily kdb file /examples/line} echo $>$$>$ {\ttfamily kdb file /examples/line} echo $>$$>$ {\ttfamily kdb file /examples/line} echo \textquotesingle{}//some other comment\textquotesingle{} $>$$>$ {\ttfamily kdb file /examples/line} echo $>$$>$ {\ttfamily kdb file /examples/line} echo \textquotesingle{}setting4 -\/1\textquotesingle{} $>$$>$ {\ttfamily kdb file /examples/line}

\section*{output filecontent and display line numbers}

awk \textquotesingle{}\{print N\+R-\/1 \char`\"{}-\/\char`\"{} \$0\}\textquotesingle{} $<$ {\ttfamily kdb file /examples/line} \#$>$ 0-\/setting1 true \#$>$ 1-\/setting2 false \#$>$ 2-\/setting3 1000 \#$>$ 3-\/\#comment \#$>$ 4-\/ \#$>$ 5-\/ \#$>$ 6-\///some other comment \#$>$ 7-\/ \#$>$ 8-\/setting4 -\/1

\section*{cleanup}

kdb rm -\/r /examples/line sudo kdb umount /examples/line ``` 