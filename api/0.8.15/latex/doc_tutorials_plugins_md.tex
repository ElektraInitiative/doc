This file serves as a tutorial on how to write a storage plugin. Storage plugins are used by Elektra in order to store data in the Elektra Key Database in an intelligent way. They act as a liason between configuration files and the Key Database. Storage plugins are largely responsible for the functionality of Elektra and they allow many of its advanced features to work.

\subsection*{Basics}

First, there are a few basic points to understand about Elektra plugins. This first section will explain the basic layout of a plugin and what various methods exists within one.

\subsubsection*{The Interface}

All plug-\/ins use the same basic interface. This interface consists of five basic functions, \href{http://doc.libelektra.org/api/current/html/group__plugin.html#ga1a72ac76b618943677e00ed7ab50b372}{\tt elektra\+Plugin\+Open}, \href{http://doc.libelektra.org/api/current/html/group__plugin.html#ga2f14a12b205687a31e6fd0645470ec69}{\tt elektra\+Plugin\+Get}, \href{http://doc.libelektra.org/api/current/html/group__plugin.html#ga01dd4018e48c3a091cb03940a7a8341f}{\tt elektra\+Plugin\+Set}, \href{http://doc.libelektra.org/api/current/html/group__plugin.html#gab0f8a88ee9868fb698b4e3040a70e000}{\tt elektra\+Plugin\+Error}, and \href{http://doc.libelektra.org/api/current/html/group__plugin.html#gaed8aeda2b2beab1b8052f8a64c601754}{\tt elektra\+Plugin\+Close}. The developer replaces {\ttfamily Plugin} with the name of their plugin. So in the case of my plugin, the names of these functions would be {\ttfamily elektra\+Line\+Open()}, {\ttfamily elektra\+Line\+Get()}, {\ttfamily elektra\+Line\+Set()}, {\ttfamily elektra\+Line\+Error()}, and {\ttfamily elektra\+Line\+Close()}. Additionally, there is one more function called \href{http://doc.libelektra.org/api/current/html/group__plugin.html#gabe78724d2d477eef39997fd9b85bff16}{\tt E\+L\+E\+K\+T\+R\+A\+\_\+\+P\+L\+U\+G\+I\+N\+\_\+\+E\+X\+P\+O\+R\+T}, where once again {\ttfamily Plugin} should be replaced with the name of the plug-\/in, this time in lower-\/case. So for my line plugin this function would be {\ttfamily E\+L\+E\+K\+T\+R\+A\+\_\+\+P\+L\+U\+G\+I\+N\+\_\+\+E\+X\+P\+O\+R\+T(line)}.

The K\+D\+B relies on the first five functions for interacting with configuration files stored in the key database. Calls for \hyperlink{group__kdb_ga28e385fd9cb7ccfe0b2f1ed2f62453a1}{kdb\+Get()} and \hyperlink{group__kdb_gadb54dc9fda17ee07deb9444df745c96f}{kdb\+Close()} will call the functions elektra\+Plugin\+Get() and elektra\+Plugin\+Close() respectively for the plugin that was used to mount the configuration data. \hyperlink{group__kdb_ga11436b058408f83d303ca5e996832bcf}{kdb\+Set()} calls elektra\+Plugin\+Set() but also elektra\+Plugin\+Error() when an error occurs. \hyperlink{elektra_2plugin_8c_a32a70a7876542c51d153164ac5108a57}{elektra\+Plugin\+Open()} is called before the first call to elektra\+Plugin\+Get() or elektra\+Plugin\+Set(). These functions serve different purposes that allow the plug-\/in to work\+:


\begin{DoxyItemize}
\item \hyperlink{elektra_2plugin_8c_a32a70a7876542c51d153164ac5108a57}{elektra\+Plugin\+Open()} is designed to allow each plug-\/in to do initialization if necessary.
\item elektra\+Plugin\+Get() is designed to turn information from a configuration file into a usable Key\+Set, this is technically the only function that is R\+E\+Q\+U\+I\+R\+E\+D in a plug-\/in.
\item elektra\+Plugin\+Set() is designed to store the information from the keyset back into a configuration file.
\item elektra\+Plugin\+Error() is designed to allow proper rollback of operations if needed and is called if any plugin fails during the set operation. This allows exception-\/safety.
\item elektra\+Plugin\+Close() is used to free resources that might be required for the plug-\/in.
\item E\+L\+E\+K\+T\+R\+A\+\_\+\+P\+L\+U\+G\+I\+N\+\_\+\+E\+X\+P\+O\+R\+T(\+Plugin) simply lets Elektra know that the plug-\/in exists and what the name of the above functions are.
\end{DoxyItemize}

Most simply put\+: most plug-\/ins consist of five major functions, {\ttfamily \hyperlink{elektra_2plugin_8c_a32a70a7876542c51d153164ac5108a57}{elektra\+Plugin\+Open()}}, {\ttfamily elektra\+Plugin\+Close()}, {\ttfamily elektra\+Plugin\+Get()}, {\ttfamily elektra\+Plugin\+Set()}, and {\ttfamily E\+L\+E\+K\+T\+R\+A\+\_\+\+E\+X\+P\+O\+R\+T\+\_\+\+P\+L\+U\+G\+I\+N(\+Plugin)}.

Because remembering all these functions can be cumbersome, we provide a skeleton plugin in order to easily create a new plugin. The skeleton plugin is called \char`\"{}template\char`\"{} and a new plugin can be created by calling the copy-\/template script . For example for my plugin I called {\ttfamily ../../scripts/copy-\/template line} from within the plugins directory. Afterwards two important things are left to be done\+:


\begin{DoxyItemize}
\item remove all functions (and their exports) from the plugin that are not needed. For example not every plugin actually makes use of the {\ttfamily \hyperlink{elektra_2plugin_8c_a32a70a7876542c51d153164ac5108a57}{elektra\+Plugin\+Open()}} function.
\item provide a basic contract as described above
\end{DoxyItemize}

After these two steps your plugin is ready to be compiled, installed and mounted for the first time. Have a look at \href{http://community.libelektra.org/wp/?p=31}{\tt How-\/\+To\+: kdb mount}

\subsection*{Contract}

In Elektra, multiple plugins form a backend. If every plugin would do whatever it likes to do, there would be chaos and backends would be unpredictable.

To avoid this situation, plugins export a so called {\itshape contract}. In this contract the plugin states how nicely it will behave and what other plugins can depend on.

\subsubsection*{Writing a Contract}

Because the contracts also contain information for humans, these parts are written in a R\+E\+A\+D\+M\+E.\+md files of the plugins. To make the contracts machine-\/readable, the following C\+Make command exists\+: \begin{DoxyVerb}    generate_readme(pluginname)
\end{DoxyVerb}


It will generate a readme\+\_\+plugginname.\+c (in the build-\/directory) out of the R\+E\+A\+D\+M\+E.\+md of the plugin''s source directory.

But prefer to use \begin{DoxyVerb}    add_plugin(pluginname)
\end{DoxyVerb}


where the readme (among many other things) are already done for you.

\subsubsection*{Content of R\+E\+A\+D\+M\+E.\+md}

The first lines must look like\+:


\begin{DoxyItemize}
\item infos = Information about Y\+A\+I\+L plugin is in keys below
\item infos/author = Markus Raab \href{mailto:elektra@libelektra.org}{\tt elektra@libelektra.\+org}
\item infos/licence = B\+S\+D
\item infos/needs =
\item infos/provides = storage
\item infos/placements = getstorage setstorage
\item infos/recommends = rebase directoryvalue comment type
\item infos/description = J\+S\+O\+N using Y\+A\+I\+L
\end{DoxyItemize}

The information of these parts are limited to a single line. Only for the description an unlimited amount of lines can be used (until the end of the file).

For the meaning (semantics) of those entries, please refer to \href{/home/markus/Projekte/Elektra/current/doc/CONTRACT.ini}{\tt contract specification}.

The already said generate\+\_\+readme will produce a list of Keys using the information in R\+E\+A\+D\+M\+E.\+md. It would look like (for the third key)\+: \begin{DoxyVerb}            keyNew ("system/elektra/modules/yajl/infos/licence",
                    KEY_VALUE, "BSD", KEY_END),
\end{DoxyVerb}


\subsection*{Including readme\+\_\+pluginname.\+c}

In your plugin, specifically in your elektra\+Plugin\+Get() implementation, you have to return the contract whenever configuration below system/elektra/modules/plugin is requested\+: \begin{DoxyVerb}    if (!strcmp (keyName(parentKey), "system/elektra/modules/plugin"))
    {
            KeySet *moduleConfig = elektraPluginContract();
            ksAppend(returned, moduleConfig);
            ksDel(moduleConfig);
            return 1;
    }
\end{DoxyVerb}


The elektra\+Plugin\+Contract() is a method implemented by the plug-\/in developer containing the parts of the contract not specified in R\+E\+A\+D\+M\+E.\+md. An example of this function (taken from the yajl plugin)\+: \begin{DoxyVerb}    static inline KeySet *elektraYajlContract()
    {
            return ksNew (30,
            keyNew ("system/elektra/modules/yajl",
                    KEY_VALUE, "yajl plugin waits for your orders", KEY_END),
            keyNew ("system/elektra/modules/yajl/exports", KEY_END),
            keyNew ("system/elektra/modules/yajl/exports/get",
                    KEY_FUNC, elektraYajlGet,
                    KEY_END),
            keyNew ("system/elektra/modules/yajl/exports/set",
                    KEY_FUNC, elektraYajlSet,
                    KEY_END),
    #include "readme_yourplugin.c"
            keyNew ("system/elektra/modules/yajl/infos/version",
                    KEY_VALUE, PLUGINVERSION, KEY_END),
            keyNew ("system/elektra/modules/yajl/config", KEY_END),
            keyNew ("system/elektra/modules/yajl/config/",
                    KEY_VALUE, "system",
                    KEY_END),
            keyNew ("system/elektra/modules/yajl/config/below",
                    KEY_VALUE, "user",
                    KEY_END),
            KS_END);
    }
\end{DoxyVerb}


It basically only contains the symbols to be exported (that are dependent on your functions to be available) and the plugin version information that is always defined to the macro P\+L\+U\+G\+I\+N\+V\+E\+R\+S\+I\+O\+N.

As already said, readme\+\_\+yourplugin.\+c is generated in the binary directory, so make sure that your C\+Make\+Lists.\+txt contains (prefer to use add\+\_\+plugin where this is already done correctly)\+: \begin{DoxyVerb}    include_directories (${CMAKE_CURRENT_BINARY_DIR})
\end{DoxyVerb}


\subsection*{Coding}

This section will focus on an overview of the kind of code you would use to develop a plugin. It gives examples from real plugins and should serve as a rough guide of how to write a storage plugin that can read and write configuration data into the Elektra Key\+Set.

\subsubsection*{elektra\+Plugin\+Get}

{\ttfamily elektra\+Plugin\+Get} is the function responsible for turning information from a file into a usable Key\+Set. This function usually differs pretty greatly between each plug-\/in. This function should be of type int, it returns 0 on success or another number on an error. The function will take in a Key, usually called {\ttfamily parent\+Key} which contains a string containing the path to the file that is mounted. For instance, if you run the command {\ttfamily kdb mount /etc/linetest system/linetest line} then {\ttfamily key\+String(parent\+Key)} should be equal to {\ttfamily /etc/linetest}. At this point, you generally want to open the file so you can begin saving it into keys. Here is the trickier part to explain. Basically, at this point you will want to iterate through the file and create keys and store string values inside of them according to what your plug-\/in is supposed to do. I will give a few examples of different plug-\/ins to better explain.

The line plug-\/in was written to read files into a Key\+Set line by line using the newline character as a delimiter and naming the keys by their line number such as {\ttfamily \#1}, {\ttfamily \#2}, .. {\ttfamily \#\textbackslash{}\+\_\+22} for a file with 22 lines. So once I open the file given by {\ttfamily parent\+Key}, every time a I read a line I create a new key, let's call it new\+\_\+key using dup\+Key(parent\+Key). Then I set new\+\_\+keys's name to line\+N\+N (where N\+N is the line number) using {\ttfamily key\+Add\+Base\+Name} and store the string value of the line into the key using {\ttfamily key\+Set\+String}. Once the key is initialized, I append it to the Key\+Set that was passed into the elektra\+Plugin\+Get function, let's call it returned for now, using {\ttfamily ks\+Append\+Key(return, new\+\_\+key)}. Now the Key\+Set will contain {\ttfamily new\+\_\+key} with the name {\ttfamily \#\+N} properly saved where it should be according to the {\ttfamily kdb mount} command (in this case, {\ttfamily system/linetest/\#\+N}), and a string value equal to the contents of that line in the file. The line plug-\/in repeats these steps as long as it hasn't reached end of file, thus saving the whole file into a Key\+Set line by line.

The simpleini plug-\/in works similarly, but it parses for ini files instead of just line-\/by-\/line. At their most simple level, ini files are in the format of {\ttfamily name=value} with each pair taking one line. So for this plug-\/in, it makes a lot of sense to name each Key in the Key\+Set by the string to the left of the {\ttfamily =} sign and store the value into each key as a string. For instance, the name of the key would be {\ttfamily name} and {\ttfamily key\+Get\+String(name)} would return {\ttfamily value}.

As you may have noticed, simpleini and line plug-\/ins work very similarly. However, they just parse the files differently. The simpleini plug-\/in parses the file in a way that is more natural to ini file (setting the key's name to the left side of the equals sign and the value to the right side of the equals sign). The {\ttfamily elektra\+Plugin\+Get} function is the heart of a storage plug-\/in, its what allows Elektra to store configurations in it's database. This function isn't just run when a file is first mounted, but whenever a file gets updated, this function is run to update the Elektra Key Database to match.

\subsubsection*{elektra\+Plugin\+Set}

We also gave a brief overview of {\ttfamily elektra\+Plugin\+Set} function. This function is basically the opposite of {\ttfamily elektra\+Plugin\+Get}. Where {\ttfamily elektra\+Plugin\+Get} reads information from a file into the Elektra Key Database, {\ttfamily elektra\+Plugin\+Set} writes information from the database back into the mounted file.

First have a look at the signature of {\ttfamily elektra\+Line\+Set}\+:

{\ttfamily elektra\+Line\+Set(\+Plugin $\ast$handle E\+L\+E\+K\+T\+R\+A\+\_\+\+U\+N\+U\+S\+E\+D, Key\+Set $\ast$to\+Write, Key $\ast$parent\+Key)}

Lets start with the most important parameters, the Key\+Set and the {\ttfamily parent\+Key}. The Key\+Set supplied is the Key\+Set that is going to be persisted in the file. In our case it would contain the Keys representing the lines. The {\ttfamily parent\+Key} is the topmost Key of the Key\+Set at serves several purposes. First, it contains the filename of the destination file as its value. Second, errors and warnings can be emitted via the parent\+Key. We will discuss error handling in more detail later. The Plugin handle can be used to persist state information in a threadsafe way with {\ttfamily elektra\+Plugin\+Set\+Data}. As our plugin is not stateful and therefore does not use the handle, it is marked as unused in order to suppress compiler warnings.

Basically the implementation of {\ttfamily elektra\+Line\+Set} can be described with the following pseudocode\+: \begin{DoxyVerb}    open the file
    if (error)
    {
            ELEKTRA_SET_ERROR(74, parentKey, keyString(parentKey));
    }
    for each key
    {
            write the key value together with a newline
    }
    close the file
\end{DoxyVerb}


The full-\/blown code can be found at \href{http://libelektra.org/tree/master/src/plugins/line/line.c}{\tt line plugin}

As you can see, all {\ttfamily elektra\+Line\+Set} does is open a file, take each Key from the Key\+Set (remember they are named {\ttfamily \#1}, {\ttfamily \#2} ... {\ttfamily \#\+\_\+22}) in order, and write each key as it's own line in the file. Since we don't care about the name of the Key in this case (other than for order), we just write the value of {\ttfamily key\+String} for each Key as a new line in the file. That's it. Now, each time the mounted Key\+Set is modified, {\ttfamily elektra\+Plugin\+Set} will be called and the mounted file will be updated.

\paragraph*{E\+L\+E\+K\+T\+R\+A\+\_\+\+S\+E\+T\+\_\+\+E\+R\+R\+O\+R}

We haven't discussed {\ttfamily E\+L\+E\+K\+T\+R\+A\+\_\+\+S\+E\+T\+\_\+\+E\+R\+R\+O\+R} yet. Because Elektra is a library, printing errors to stderr wouldn't be a good idea. Instead, errors and warnings can be appended to a key in the form of metadata. This is what {\ttfamily E\+L\+E\+K\+T\+R\+A\+\_\+\+S\+E\+T\+\_\+\+E\+R\+R\+O\+R} does. Because the parent\+Key always exists even if a critical error occurres, we append the error to the {\ttfamily parent\+Key}. The first parameter is an id specifying the general error that occurred. A listing of existing errors together with a short description and a categorization can be found at \href{https://github.com/ElektraInitiative/libelektra/blob/master/src/error/specification}{\tt error specification}. The third parameter can be used to provide additional information about the error. In our case we simply supply the filename of the file that caused the error. The kdb tools will interprete this error and print it in a pretty way. Notice that this can be used in any plugin function where the parent\+Key is available.

\subsubsection*{elektra\+Plugin\+Open and elektra\+Plugin\+Close}

The {\ttfamily elektra\+Plugin\+Open} and {\ttfamily elektra\+Plugin\+Close} functions are not commonly used for storage plug-\/ins, but they can be useful and are worth reviewing. {\ttfamily elektra\+Plugin\+Open} function runs before {\ttfamily elektra\+Plugin\+Get} and is useful to do initialization if necessary for the plug-\/in. On the other hand {\ttfamily elektra\+Plugin\+Close} is run after other functions of the plug-\/in and can be useful for freeing up resources.

\subsubsection*{E\+L\+E\+K\+T\+R\+A\+\_\+\+P\+L\+U\+G\+I\+N\+\_\+\+E\+X\+P\+O\+R\+T}

The last function, one that is always needed in a plug-\/in, is {\ttfamily E\+L\+E\+K\+T\+R\+A\+\_\+\+P\+L\+U\+G\+I\+N\+\_\+\+E\+X\+P\+O\+R\+T}. This functions is responsible for letting Elektra know that the plug-\/in exists and which methods it implements. The code from the line plugin is a good example and pretty self-\/explanatory\+: \begin{DoxyVerb}    Plugin *ELEKTRA_PLUGIN_EXPORT(line)
    {
            return elektraPluginExport("line",
            ELEKTRA_PLUGIN_GET, &amp;elektraLineGet,
            ELEKTRA_PLUGIN_SET, &amp;elektraLineSet,
            ELEKTRA_PLUGIN_END);
    }
\end{DoxyVerb}


For further information see \href{http://doc.libelektra.org/api/current/html/group__plugin.html}{\tt the A\+P\+I documentation}. 