The \hyperlink{doc_tutorials_plugins_md}{plugin tutorial} already covers some of the most interesting parts on how to write a (storage) plugin. This text will tell you a little bit more about how a storage plugin should act. While it is usually relatively easy to create a plugin that stores basic key-\/value pairs, adding advanced features such as support for


\begin{DoxyItemize}
\item \hyperlink{doc_tutorials_arrays_md}{arrays}, and
\item \hyperlink{doc_dev_metadata_md}{metadata},
\end{DoxyItemize}

takes more work. Before you continue with this text, please make sure that you read all of the linked documents above.

One common problem of storage plugins is, that they store too many keys. For example, if the user adds the keys


\begin{DoxyItemize}
\item {\ttfamily user/tests/storage/root} and
\item {\ttfamily user/tests/storage/root/level1/level2/level3},
\end{DoxyItemize}

then your plugin should only store those two keys. {\bfseries Do not} add the keys


\begin{DoxyItemize}
\item {\ttfamily user/tests/storage/root/level1}, or
\item {\ttfamily user/tests/storage/root/level1/level2}
\end{DoxyItemize}

to the key set. One plugin that handles this situation properly is \hyperlink{autotoc_md865_src_plugins_yamlcpp_README_md}{Y\+A\+ML C\+PP}, as the following \href{https://master.libelektra.org/tests/shell/shell_recorder/tutorial_wrapper}{\tt Markdown Shell Recorder} test shows\+:


\begin{DoxyCode}
# Mount plugin
sudo kdb mount config.yaml user/tests/storage yamlcpp

# Add key-value pairs
kdb set user/tests/storage/root 🐓
kdb set user/tests/storage/root/level1/level2/level3 🐣

# Make sure that YAML CPP did not store any additional keys
kdb ls user/tests/storage/root
#> user/tests/storage/root
#> user/tests/storage/root/level1/level2/level3

# Undo modifications to the key database
kdb rm -r user/tests/storage
sudo kdb umount user/tests/storage
\end{DoxyCode}


. For more information on why we allow “holes” in the hierarchy, please take a look \hyperlink{doc_decisions_holes_md}{here}.

Elektra supports both binary and textual values. The main difference between binary and textual data is that textual data always ends with a null byte. Therefore you are not allowed to store the code point {\ttfamily 0} inside textual data. Binary data does not have this limitation.

The simplest textual data is the empty string ({\ttfamily \char`\"{}\char`\"{}} = {\ttfamily 0}) and has length 1, while the simplest binary data stores nothing at all and therefore has length 0. In the {\ttfamily kdb} utility you can disambiguate between these value by checking for the \hyperlink{doc_help_elektra-metadata_md}{metakey `binary`}. The following \href{https://master.libelektra.org/tests/shell/shell_recorder/tutorial_wrapper}{\tt Markdown Shell Recorder} test shows how a storage plugin should handle these values.


\begin{DoxyCode}
# Mount plugin
sudo kdb mount config.yaml user/tests/storage yamlcpp

kdb set user/tests/storage/null
#> Create a new key user/tests/storage/null with null value
kdb get user/tests/storage/null
#>
kdb meta-ls user/tests/storage/null
#> binary

kdb set user/tests/storage/empty ''
#> Create a new key user/tests/storage/empty with string ""
kdb get user/tests/storage/empty
#>
kdb meta-ls user/tests/storage/empty
#>

# Undo modifications to the key database
kdb rm -r user/tests/storage
sudo kdb umount user/tests/storage
\end{DoxyCode}


Elektra uses \hyperlink{doc_decisions_boolean_md}{`0` and `1` to represent binary data}. A storage plugin that uses other values (e.\+g. {\ttfamily false} and {\ttfamily true}) needs to convert these values to {\ttfamily 0} and {\ttfamily 1}. The \href{https://master.libelektra.org/tests/shell/shell_recorder/tutorial_wrapper}{\tt Markdown Shell Recorder} test below shows that \hyperlink{autotoc_md865_src_plugins_yamlcpp_README_md}{Y\+A\+ML C\+PP} handles the conversion from and to \href{https://yaml.org/spec/1.2/spec.html#id2803629}{\tt Y\+A\+M\+L’s boolean type} properly. In the test we also use the \hyperlink{autotoc_md803_src_plugins_type_README_md}{`type` plugin} to makes sure that Y\+A\+ML C\+PP interacts correctly with this essential plugin.


\begin{DoxyCode}
# Mount plugin
sudo kdb mount config.yaml user/tests/storage yamlcpp type

kdb set user/tests/storage/bool/value true
kdb meta-set user/tests/storage/bool/value type boolean
kdb get user/tests/storage/bool/value
#> 1

kdb set user/tests/storage/bool/value 1
kdb get user/tests/storage/bool/value
#> 1

kdb set user/tests/storage/bool/value false
kdb get user/tests/storage/bool/value
#> 0

kdb set user/tests/storage/bool/value 'non boolean'
# RET: 5

kdb get user/tests/storage/bool/value
#> 0

# Undo modifications to the key database
kdb rm -r user/tests/storage
sudo kdb umount user/tests/storage
\end{DoxyCode}


Sometimes the most “natural” mapping of key-\/value pairs to a file format might cause a storage plugin to not be able to store values in so called directory (non-\/leaf) keys.

For example, in a key set that contains the keys\+:


\begin{DoxyCode}
user/directory
user/directory/leaf1
user/directory/leaf2
user/leaf3
\end{DoxyCode}


, all keys at the bottom of the hierarchy\+:


\begin{DoxyCode}
        user
      /      \(\backslash\)
  directory  leaf3
   /     \(\backslash\)
leaf1   leaf2
\end{DoxyCode}


, such as


\begin{DoxyItemize}
\item {\ttfamily user/directory/leaf1}
\item {\ttfamily user/directory/leaf2}
\item {\ttfamily user/leaf3}
\end{DoxyItemize}

are called leaf keys, while {\ttfamily user/directory} is a directory key. Plugins such as \hyperlink{autotoc_md835_src_plugins_yajl_README_md}{Y\+A\+JL} or \hyperlink{autotoc_md865_src_plugins_yamlcpp_README_md}{Y\+A\+ML C\+PP} will not be able to store data in the key with the name {\ttfamily user/directory} directly. To work around this issue these plugin use the \hyperlink{autotoc_md172_src_plugins_directoryvalue_README_md}{Directory Value plugin}. In the Read\+Me of the \href{https://www.libelektra.org/plugins/directoryvalue}{\tt Directory Value plugin} and \href{https://www.libelektra.org/plugins/yamlcpp}{\tt Y\+A\+ML C\+PP} you will find more information about this issue, and how to handle it.

The following Markdown Shell Recorder test shows {\bfseries the proper behavior}\+:


\begin{DoxyCode}
# Mount plugin
sudo kdb mount config.yaml user/tests/storage yamlcpp

# Add key-value pair (leaf key)
kdb set user/tests/storage/root 🐓
# Since we add a key below `user/tests/storage/root`, the key
# `user/tests/storage/root` turns from a leaf key to a directory key.
kdb set user/tests/storage/root/level1/level2/level3 🐣

# Make sure that the directory key still stores the correct value
kdb get user/tests/storage/root
#> 🐓

# Check the value of the leaf key
kdb get user/tests/storage/root/level1/level2/level3
#> 🐣

# Undo modifications to the key database
kdb rm -r user/tests/storage
sudo kdb umount user/tests/storage
\end{DoxyCode}


. To make sure that your storage plugin works correctly, please just replace {\ttfamily yamlcpp} with the name of your plugin and verify that the test above still works.

You already learned about the array syntax and the mandatory {\ttfamily array} metakey in the \hyperlink{doc_tutorials_arrays_md}{array tutorial}. Now it is time to check, if your storage plugin supports array and non-\/array keys properly. Let us look at a concrete example. We use a key set that contains the following keys as example\+:


\begin{DoxyCode}
user/tests/storage/array
user/tests/storage/array/#0
user/tests/storage/array/#1
user/tests/storage/map
user/tests/storage/map/#0
user/tests/storage/map/#1
\end{DoxyCode}


. If we assume that only {\ttfamily user/tests/storage/array} stores the metakey {\ttfamily array}, then the keys


\begin{DoxyItemize}
\item {\ttfamily user/tests/storage/array/\#0}, and
\item {\ttfamily user/tests/storage/array/\#1}
\end{DoxyItemize}

represent array elements, while


\begin{DoxyItemize}
\item {\ttfamily user/tests/storage/map/\#0}, and
\item {\ttfamily user/tests/storage/map/\#1}
\end{DoxyItemize}

are normal key-\/value pairs. The following example shows that the storage plugin \href{https://www.libelektra.org/plugins/yamlcpp}{\tt Y\+A\+ML C\+PP} handles this situation properly\+:


\begin{DoxyCode}
# Mount plugin
sudo kdb mount config.yaml user/tests/storage yamlcpp

# Create an array containing two elements
kdb meta-set user/tests/storage/array array ''
kdb set user/tests/storage/array/#0 one
kdb set user/tests/storage/array/#1 two

# The array parent key stores the basename of the last element
kdb meta-get user/tests/storage/array array
#> #1

# If you do not add the metakey `array`, then keys
# containing array syntax `#0`, `#1`, … will not be
# interpreted as arrays.
kdb set user/tests/storage/map
kdb set user/tests/storage/map/#0
kdb set user/tests/storage/map/#1
kdb meta-get user/tests/storage/map array
# RET: 2

# Undo modifications to the key database
kdb rm -r user/tests/storage
sudo kdb umount user/tests/storage
\end{DoxyCode}
