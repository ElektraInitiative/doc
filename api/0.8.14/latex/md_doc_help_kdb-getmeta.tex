{\ttfamily kdb getmeta $<$key-\/name$>$ $<$meta-\/name$>$}

Where {\ttfamily key-\/name} is the full path to the key and {\ttfamily meta-\/name} is the name of the meta key the user would like to access.

\subsection*{D\+E\+S\+C\+R\+I\+P\+T\+I\+O\+N}

This command is used to print the value of a meta key. A meta key is information stored in a key which describes that key.

The handling of cascading {\ttfamily key-\/name} does not differ to {\ttfamily kdb get}. Make sure to use the namespace {\ttfamily spec}, if you want meta-\/data from there.

\subsection*{R\+E\+T\+U\+R\+N V\+A\+L\+U\+E\+S}

This command will return the following values as an exit status\+:
\begin{DoxyItemize}
\item 0\+: No errors.
\item 1\+: Key not found. (Invalid {\ttfamily path})
\item 2\+: Meta key not found. (Invalid {\ttfamily meta-\/name}).
\end{DoxyItemize}

\subsection*{O\+P\+T\+I\+O\+N\+S}


\begin{DoxyItemize}
\item {\ttfamily -\/\+H}, {\ttfamily -\/-\/help}\+: Show the man page.
\item {\ttfamily -\/\+V}, {\ttfamily -\/-\/version}\+: Print version info.
\item {\ttfamily -\/n}, {\ttfamily -\/-\/no-\/newline}\+: Suppress the newline at the end of the output.
\end{DoxyItemize}

\subsection*{E\+X\+A\+M\+P\+L\+E\+S}

To get the value of a meta key called {\ttfamily description} stored in the key {\ttfamily spec/example/key}\+: {\ttfamily kdb getmeta spec/example/key description}

To get the value of meta key called {\ttfamily override/\#0} stored in the key {\ttfamily spec/example/dir/key}\+: {\ttfamily kdb getmeta spec/example/dir/key \char`\"{}override/\#0\char`\"{}}

\subsection*{S\+E\+E A\+L\+S\+O}


\begin{DoxyItemize}
\item How to set meta data\+: \hyperlink{md_doc_help_kdb-setmeta_doc_help_kdb-setmeta_md}{kdb-\/setmeta(1)}
\item For more about cascading keys see \hyperlink{md_doc_help_elektra-cascading_doc_help_elektra-cascading_md}{elektra-\/cascading(7)}
\item For general information about meta data see \hyperlink{md_doc_help_elektra-meta-data_doc_help_elektra-meta-data_md}{elektra-\/meta-\/data(7)} 
\end{DoxyItemize}