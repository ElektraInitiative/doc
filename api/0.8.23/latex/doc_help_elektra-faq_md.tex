\subsection*{I am stuck. Where can I get help?}

If this F\+AQ does not contain your question, \href{https://git.libelektra.org/issues}{\tt please open an issue}. You can simply remove all template text and it is enough if the issue only contains your question. If you want you can \href{https://git.libelektra.org/issues/labels/question}{\tt label it as question}, but we can also categorize it for you.

Please do not waste too much time to find something out yourself. Information where people get stuck is valuable to improve Elektra and its documentation. Even if you find out directly after you posted the question\+: the pointer can be helpful for other people having the same problem.

\subsection*{Isn\textquotesingle{}t it easier to implement a new parser than to use Elektra?}

No, it is not. And even if it were, the story does not end with implementing a configuration file parser but, at least, you also need\+:


\begin{DoxyItemize}
\item operating-\/system-\/specific code to locate configuration files
\item tools to change the configuration files
\item validation to make such changes user friendly
\end{DoxyItemize}

Every successful project has implemented many features Elektra has. But Elektra has the distinctive advantage that you can pick the features as you need them. Not used plugins do not cause any overhead or dependency. If you need new plugins or bindings, there is a community which can help you. Furthermore, Elektra has a defined A\+PI and you can swap implementations as needed.

So it pays off to use Elektra -- in the short and in the long term.

\subsection*{Why do I need Elektra if I already use configuration management tools?}

Short answer\+: Try \href{https://puppet.libelektra.org}{\tt puppet-\/libelektra} to see how useful it can be.

Longer answer\+:

Elektra abstracts \hyperlink{md_doc_help_elektra-glossary_doc_help_elektra-glossary_md}{configuration settings}, something desperately needed within configuration management. Instead of rewriting complete configuration files, which might create security problems, Elektra operates precisely on the configuration setting you want to change\+: leaving others as chosen by the application or distribution. Furthermore, Elektra also allows us to {\itshape specify} configuration settings, which again brings benefits for configuration management tools.

Elektra is a radical step needed towards better configuration management\+: Let us fix how applications access configuration settings, so that we can properly access them, for example, from configuration management tools.

As an intermediate step, we can \hyperlink{md_doc_help_elektra-mounting_doc_help_elektra-mounting_md}{mount} existing configuration files and operate on them.

\subsection*{Do we retain the old way of configuring things, i.\+e. manually editing a ini file in /etc?}

Absolutely, you can think of libelektra as a small library in C that reads a configuration file and returns a data structure if you do not use any of its advanced features.

\subsection*{Do we retain the old way reloading/restarting the system service?}

Elektra does not interfere with restarting. It is a passive library. It provides some techniques for reloading but they are optional (but we recommend that you keep the in-\/memory and persistent configuration in sync via notification).

\subsection*{Is this an actual problem of Elektra or is it just me?}

In case of doubt \href{https://git.libelektra.org/issues}{\tt please open an issue}. If the question was already answered or is already in the documentation, we will simply point it out to you.

So do not worry too much, do not hesitate to ask any question. We welcome feedback, only then we can improve the documentation such as this F\+A\+Q!

\subsection*{What should I do if I found a bug?}

Please check the \href{https://git.libelektra.org/issues}{\tt issue tracker} if it has already been reported. If it has not, please \href{https://git.libelektra.org/issues/new}{\tt fill out the template}. If you are in doubt, please report it.

\subsection*{How can I contribute to Elektra?}

Due to the modular architecture we can accept nearly all contributions as plugins. Please only make sure that the R\+E\+A\+D\+M\+E.\+md clearly states the purpose and quality of the plugin. If you need, e.\+g., a plugin that crashes the process make sure that you tag ({\ttfamily infos/status}) it with {\ttfamily discouraged}.

Please start by reading here.

\subsection*{What is the Elektra’s license?}

New B\+SD license which allows us to have plugins link against G\+PL and G\+P\+L-\/incompatible libraries. If you compile Elektra, e.\+g., with G\+PL plugins, the result is G\+PL.

\subsection*{Who are the authors?}

\hyperlink{doc_AUTHORS_md}{List of authors}.

\subsection*{How can I advertise Elektra?}

If questions about configuration come up, point users to \href{https://www.libelektra.org}{\tt https\+://www.\+libelektra.\+org} Display the S\+VG logos found at \href{https://master.libelektra.org/doc/images/logo}{\tt https\+://master.\+libelektra.\+org/doc/images/logo} and already rastered logos at \href{https://github.com/ElektraInitiative/blobs/tree/master/images/logos}{\tt https\+://github.\+com/\+Elektra\+Initiative/blobs/tree/master/images/logos} Distribute the flyer found at \href{https://github.com/ElektraInitiative/blobs/raw/master/flyers/flyer.odt}{\tt https\+://github.\+com/\+Elektra\+Initiative/blobs/raw/master/flyers/flyer.\+odt} And of course\+: talk about it! 