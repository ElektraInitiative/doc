\subsection*{Issue}

Calling exported functions is a primary way for coordinating plugins in Elektra. Since some plugins encapsulate other plugins, exported functions from encapsulated plugins become unavailable.

Since encapsulating plugins cannot implement and forward calls for every function exported by encapsulated plugins a generic mechanism for function calls to these encapsulated plugins is needed.

Since some plugins also lazy-\/load encapsulated plugins the call mechanism is required to be able to defer these calls until the plugins are loaded.

For example when setting I/O bindings with {\ttfamily \hyperlink{group__kdbio_ga187345483bdfbb404919c6797bc2db77}{elektra\+Io\+Set\+Binding()}} the exported function {\ttfamily set\+Io\+Binding} is called for all globally mounted plugins. Since global mounting is implemented using the \char`\"{}list\char`\"{} plugin which uses lazy-\/loading for its plugins the exported functions from the plugins are unavailable.

Other examples are the \char`\"{}dini\char`\"{} and \char`\"{}multifile\char`\"{} plugins which use multiple plugins to support different file formats. These plugins also \char`\"{}hide\char`\"{} functions exported by encapsulated plugins.

\subsection*{Constraints}


\begin{DoxyEnumerate}
\item Plugins encapsulating other plugins shall be able to lazy-\/load them.
\item Callers shall not be entangled in encapsulating plugin details.
\item The encapsulation of the plugins shall not be broken.
\item The mechanism shall be generic.
\end{DoxyEnumerate}

\subsection*{Assumptions}


\begin{DoxyEnumerate}
\item The called functions do not return a value (e.\+g. {\ttfamily set}, {\ttfamily open}, {\ttfamily close}, ...). Callbacks can be used as return channel (see \char`\"{}\+Implications\char`\"{})
\end{DoxyEnumerate}

\subsection*{Considered Alternatives}


\begin{DoxyItemize}
\item Encapsulating plugins export a function called {\ttfamily get\+Encapsulated\+Plugins}. This would break encapsulation and break lazy-\/loading since to return the encapsulated plugin handles they have to be loaded.
\item Compatible plugins export a function called {\ttfamily get\+Interfaces}. It returns a Key\+Set with well-\/known interface names like \char`\"{}notification\char`\"{} or \char`\"{}io\+Binding\char`\"{} and the required functions (e.\+g. {\ttfamily open} \& {\ttfamily close} for \char`\"{}notification\char`\"{} or {\ttfamily set} for \char`\"{}io\+Binding\char`\"{}). While normal plugins simply return their interfaces, encapsulating plugins collect interfaces from its plugins and combine them into a single Key\+Set.

The example below shows a Key\+Set from a plugin encapsulating two plugins \char`\"{}\+A\char`\"{} and \char`\"{}\+B\char`\"{}. Both plugins implement the \char`\"{}notification\char`\"{} interface, plugin \char`\"{}\+B\char`\"{} also implements the \char`\"{}io\+Binding\char`\"{} interface.
\begin{DoxyItemize}
\item {\ttfamily /notification/\#0/open}\+: address of {\ttfamily open\+Notification} of plugin \char`\"{}\+A\char`\"{}
\item {\ttfamily /notification/\#0/close}\+: address of {\ttfamily close\+Notification} of plugin \char`\"{}\+A\char`\"{}
\item {\ttfamily /notification/\#1/open}\+: address of {\ttfamily open\+Notification} of plugin \char`\"{}\+B\char`\"{}
\item {\ttfamily /notification/\#1/close}\+: address of {\ttfamily close\+Notification} of plugin \char`\"{}\+B\char`\"{}
\item {\ttfamily /io\+Binding/\#0/set}\+: address of {\ttfamily set\+Io\+Binding} of plugin \char`\"{}\+B\char`\"{}
\end{DoxyItemize}

The upside of this approach is that it makes encapsulating plugins transparent to the caller -\/ it does not known whether the plugin encapsulates other plugins. This approach breaks lazy-\/loading since for combining all interfaces the plugins have to be loaded and their {\ttfamily get\+Interfaces} functions have to be called.
\end{DoxyItemize}

\subsection*{Decision}

Encapsulating plugins export a function called {\ttfamily deferred\+Call} with the declaration {\ttfamily void elektra\+Deferred\+Call (Plugin $\ast$ plugin, char $\ast$ name, Key\+Set $\ast$ parameters)}. Encapsulating plugins shall save multiple deferred calls and call the exported functions specified by {\ttfamily name} passing the {\ttfamily parameters} Key\+Set when a plugin is initialized in the same order as received.

Plugins that support deferred calls shall have the following declaration for their functions {\ttfamily void some\+Plugin\+Function (Plugin $\ast$ plugin, Key\+Set $\ast$ parameters)}. The calling developer is responsible for ensuring that the called functions have a compatible declaration. Encapsulated plugins that do not export a specified function name are omitted.

\subsection*{Argument}

The solution allows to change encapsulating plugin implementations without breaking callers.

\subsection*{Implications}

The called function receive their parameters via a Key\+Set.

While called functions could return data using the {\ttfamily parameters} Key\+Set (or a separate Key\+Set) there is no defined moment when the data can be collected. Defining such a moment would break the lazy-\/loading constraint. It is recommended to use callbacks passed as {\ttfamily parameters}. Callback function declarations are not limited by this decision.

\subsection*{Related decisions}


\begin{DoxyItemize}
\item Elektra\textquotesingle{}s invoke functionality will be extended to also allow us to use deferred calls with new functions\+:
\item {\ttfamily int elektra\+Invoke\+Function\+Deferred (Elektra\+Invoke\+Handle $\ast$ handle, const char $\ast$ elektra\+Plugin\+Function\+Name, Key\+Set $\ast$ ks)} which defers a call if the plugin exports {\ttfamily deferred\+Call}.
\item {\ttfamily void elektra\+Invoke\+Execute\+Deferred\+Calls (Elektra\+Invoke\+Handle $\ast$ handle, Elektra\+Deferred\+Call\+List $\ast$ list)} which executes deferred calls for a encapsulated plugin loaded with invoke.
\item Functions supporting deferred calls should allow for multiple calls (i.\+e. they should be idempotent). This leaves state at affected plugins and does avoid duplicating state (e.\+g. \char`\"{}was this function called for this plugin before?\char`\"{}) in encapsulating plugins.
\end{DoxyItemize}

\subsection*{Notes}

Utility functions that help with managing deferred calls would be nice\+:


\begin{DoxyItemize}
\item {\ttfamily Elektra\+Deferred\+Call\+List $\ast$ elektra\+Deferred\+Call\+Create\+List (void)}
\item {\ttfamily void elektra\+Deferred\+Call\+Delete\+List (Elektra\+Deferred\+Call\+List $\ast$ list)}
\item {\ttfamily int elektra\+Deferred\+Call\+Add (Elektra\+Deferred\+Call\+List $\ast$ list, char $\ast$ name, Key\+Set $\ast$ parameters)}
\item {\ttfamily void elektra\+Deferred\+Calls\+Execute (Plugin $\ast$ plugin, Elektra\+Deferred\+Call\+List $\ast$ list)} 
\end{DoxyItemize}