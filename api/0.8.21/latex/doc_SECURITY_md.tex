Security is a very important point in libraries. In most use cases there is nearly no point of danger in using Elektra. But some a very security related, especially when you use a daemon or some kind of distributed configuration.

\subsection*{Access Permissions}

We only use access permissions from the kernel, we do not add an additional layer (or daemon). So configuration file access is as secure as with direct access to configuration files.

\subsection*{Namespaces}

Elektra by default guarantees that configuration from specific namespaces come from respective paths in the file system\+:


\begin{DoxyItemize}
\item {\ttfamily dir}-\/namespace\+: from current working directory
\item {\ttfamily user}-\/namespace\+: from users home directory
\item {\ttfamily system} or {\ttfamily spec}-\/namespace\+: no restrictions
\end{DoxyItemize}

\subsection*{Environment Variables}

Environment variables are usually avoided, but instead Elektra itself is used to configure Elektra. The core is not allowed to use any environment variables.

For some plugins, however, Environment variables are used for better integration in systems. This might be a security risk.

\subsection*{Compiler Options}

Can be changed using standard C\+Make ways. Some hints\+:

\href{http://wiki.debian.org/Hardening}{\tt http\+://wiki.\+debian.\+org/\+Hardening}

\subsection*{Memory Leaks}

We use Valgrind ({\ttfamily -\/-\/tool=memcheck}) to make sure that Elektra does not suffer memory leaks and incorrect memory handling. 