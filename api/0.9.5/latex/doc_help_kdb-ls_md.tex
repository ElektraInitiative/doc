{\ttfamily kdb ls $<$path$>$}

Where {\ttfamily path} is the path in which the user would like to list keys below.

This command will list the name of all keys below a given path.


\begin{DoxyItemize}
\item {\ttfamily -\/H}, {\ttfamily -\/-\/help}\+: Show the man page.
\item {\ttfamily -\/V}, {\ttfamily -\/-\/version}\+: Print version info.
\item {\ttfamily -\/p}, {\ttfamily -\/-\/profile $<$profile$>$}\+: Use a different kdb profile.
\item {\ttfamily -\/C}, {\ttfamily -\/-\/color $<$when$>$}\+: Print never/auto(default)/always colored output.
\item {\ttfamily -\/m}, {\ttfamily -\/-\/min-\/depth $<$min-\/depth$>$}\+: Specify the minimum path depth of the output (0 by default), exclusive and relative to the name to list.
\item {\ttfamily -\/M}, {\ttfamily -\/-\/max-\/depth $<$max-\/depth$>$}\+: Specify the maximum path depth of the output (unlimited by default, 1 to show only the next level), inclusive and relative to the name to list.
\item {\ttfamily -\/v}, {\ttfamily -\/-\/verbose}\+: Explain what is happening. Prints additional information in case of errors/warnings.
\item {\ttfamily -\/d}, {\ttfamily -\/-\/debug}\+: Give debug information. Prints additional debug information in case of errors/warnings.
\item {\ttfamily -\/0}, {\ttfamily -\/-\/null}\+: Use binary 0 termination.
\end{DoxyItemize}


\begin{DoxyCode}
# Backup-and-Restore: user:/tests/examples

# We use `dump` as storage format here, since storage plugins such as INI
# automatically add keys between levels (e.g. `user:/tests/examples/kdb-ls/test/foo`).
sudo kdb mount ls.ecf user:/tests/examples dump

# Create the keys we use for the examples
kdb set user:/tests/examples/kdb-ls/test val1
kdb set user:/tests/examples/kdb-ls/test/foo/bar val2
kdb set user:/tests/examples/kdb-ls/test/fizz/buzz fizzbuzz
kdb set user:/tests/examples/kdb-ls/tost val3
kdb set user:/tests/examples/kdb-ls/tost/level lvl

# list all keys below /tests/examples/kdb-ls
kdb ls /tests/examples/kdb-ls
#> user:/tests/examples/kdb-ls/test
#> user:/tests/examples/kdb-ls/test/fizz/buzz
#> user:/tests/examples/kdb-ls/test/foo/bar
#> user:/tests/examples/kdb-ls/tost
#> user:/tests/examples/kdb-ls/tost/level

# list the next level of keys below /tests/examples/kdb-ls
# note that if the search key ends with a /, it lists the next level
kdb ls /tests/examples/kdb-ls/ --max-depth=1
#> user:/tests/examples/kdb-ls/test
#> user:/tests/examples/kdb-ls/tost

# list the current level of keys below /tests/examples/kdb-ls
# note the difference to the previous example
kdb ls /tests/examples/kdb-ls --max-depth=1
# this yields no output as /tests/examples/kdb-ls is not a key

# list all keys below /tests/examples/kdb-ls with are minimum 1 level (inclusive) away from that key
# and maximum 2 levels away (exclusive)
kdb ls /tests/examples/kdb-ls --min-depth=1 --max-depth=2
#> user:/tests/examples/kdb-ls/test
#> user:/tests/examples/kdb-ls/tost

# list all keys below /tests/examples/kdb-ls/test
kdb ls /tests/examples/kdb-ls/test
#> user:/tests/examples/kdb-ls/test
#> user:/tests/examples/kdb-ls/test/fizz/buzz
#> user:/tests/examples/kdb-ls/test/foo/bar

# list all keys under /tests/examples/kdb-ls in verbose mode
kdb ls /tests/examples/kdb-ls/ -v
#> size of all keys in mount point: 5
#> size of requested keys: 5
#> user:/tests/examples/kdb-ls/test
#> user:/tests/examples/kdb-ls/test/fizz/buzz
#> user:/tests/examples/kdb-ls/test/foo/bar
#> user:/tests/examples/kdb-ls/tost
#> user:/tests/examples/kdb-ls/tost/level

kdb rm -r user:/tests/examples
sudo kdb umount user:/tests/examples
\end{DoxyCode}



\begin{DoxyItemize}
\item If the user would also like to see the values of the keys below {\ttfamily path} then you should consider the \hyperlink{doc_help_kdb-export_md}{kdb-\/export(1)} command.
\item \hyperlink{doc_help_elektra-key-names_md}{elektra-\/key-\/names(7)} for an explanation of key names. 
\end{DoxyItemize}