\subsection*{Problem}

Too verbose error message Currently for every error, 9 lines are shown in which most of them are not relevant to end users/administrators. One goal is to reduce the verbosity of such messages and let users/administrators see only information they need.

\subsection*{Constraints}


\begin{DoxyItemize}
\item Supporting multiple programming languages
\item Plugin System
\item Error Code should be preserved
\end{DoxyItemize}

\subsection*{Assumptions}

\subsection*{Considered Alternatives}

Possible variations on what message should be displayed, e.\+g., to keep the mountpoint information or on how wordings should be (with or without \char`\"{}\+Sorry, ...\char`\"{}, coloring of certain parts of a message, etc.)

Examples would be to


\begin{DoxyItemize}
\item Leave out the \char`\"{}\+Sorry\char`\"{} in the error message or leave the introduction sentence completely
\item Drop {\ttfamily At}, {\ttfamily Mountpoint}, {\ttfamily Configfile}, {\ttfamily Module}. This information though yields useful information or was even added as a request
\item Show mountpoint, configfile, module, etc in beneath the general introduction message. Eg. {\ttfamily The command kdb set failed while accessing the key database on mountpoint (...) with the info}
\item Incorporating the description in another ways\+: {\ttfamily Reason\+: Validation of key \char`\"{}$<$key$>$\char`\"{} with string \char`\"{}$<$value$>$\char`\"{} failed. (validation failed)}
\item Use one command line option to show all additional info which gets hidden per default from now on instead of two
\item Color the main message differently compared to the general introduction message
\item Do not color messages as it might confuse users with overwhelming many colors
\item Do not print out the error code. It is useful though for googling
\end{DoxyItemize}

\subsection*{Decision}

The error message has the current format\+:


\begin{DoxyCode}
The command kdb set failed while accessing the key database with the info:
Sorry, the error (#121) occurred ;(
Description: validation failed
Reason: Validation of key "<key>" with string "<value>" failed.
Ingroup: plugin
Module: enum
At: ....../src/plugins/enum/enum.c:218
Mountpoint: <parentKey>
Configfile: ...../<file>.25676:1549919217.284067.tmp
\end{DoxyCode}


The new default message will look like this\+:


\begin{DoxyCode}
Sorry, module `MODULE` issued [error|warning] `NR`:
`ERROR\_CODE\_DESCRIPTION`: Validation of key "<key>" with string "<value>" failed.
\end{DoxyCode}


The {\ttfamily NR} will be the color red in case of an error or yellow in case of a warning while {\ttfamily M\+O\+D\+U\+LE} will be the color blue.

Optionally a third line indicating a solution can be added. Eg. for a permission related error there would be a third line\+:


\begin{DoxyCode}
Possible Solution: Retry the command as sudo (sudo !!)
\end{DoxyCode}


To avoid losing information, the user can use the command line argument {\ttfamily -\/v} (verbose) to show {\ttfamily Mountpoint}, {\ttfamily Configfile} in addition to the current error message. Furthermore a developer can use the command line argument {\ttfamily -\/d} (debug) to show {\ttfamily At} for debugging purposes.

\subsection*{Rationale}

The new error message is much more succinct which gives end users more relevant information. Furthermore the solution approach still holds all necessary information if requested by users.

\subsection*{Implications}

{\ttfamily Description} will be incorporated into {\ttfamily Reason} whereas the {\ttfamily Module} will be incorporated into the general sentence starting the error message.

\subsection*{Related Decisions}


\begin{DoxyItemize}
\item \hyperlink{doc_decisions_error_codes_md}{Error Codes} Shows how the new error codes are meant to be
\item \hyperlink{doc_decisions_ingroup_removal_md}{Ingroup Removal} Shows the decision of why the {\ttfamily ingoup} field has been removed
\end{DoxyItemize}

\subsection*{Notes}