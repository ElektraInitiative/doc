Elektra has three different main targets.

Elektra allows applications to read and write from a global configuration tree. We miss a specification (schema) so that these configuration values can be shared (integrated).

Known users\+:


\begin{DoxyItemize}
\item \href{https://oyranos.org}{\texttt{ Oyranos}}
\item \href{http://lcdproc.omnipotent.net/}{\texttt{ L\+C\+Dproc}} (in progress)
\item \href{https://kde.org}{\texttt{ K\+DE}} (in progress)
\item \href{https://www.openhab.org/}{\texttt{ Open\+H\+AB}} (in progress)
\end{DoxyItemize}

Elektra is on the frontier for embedded systems because of its tiny core and the many possibilities with its plugins.

Known users\+:


\begin{DoxyItemize}
\item Open\+W\+RT (distribution)
\item Broadcom (blue-\/ray devices)
\item Kapsch (cameras)
\item Toshiba (T\+Vs)
\end{DoxyItemize}

Elektra is ideal suited for a local configuration storage by mounting existing configuration files into the global tree. Nodes using Elektra can be connected by already existing configuration management tools.

Known users\+:


\begin{DoxyItemize}
\item Allianz
\item TU Wien
\item Other Universities
\end{DoxyItemize}


\begin{DoxyItemize}
\item Continue reading\+: \mbox{\hyperlink{doc_WHY_md}{Why should I use Elektra?}} 
\end{DoxyItemize}