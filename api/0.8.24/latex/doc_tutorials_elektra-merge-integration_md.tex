We assume that you know what \href{https://packages.debian.org/sid/ucf}{\tt ucf} is and have some \href{https://wiki.debian.org/ConfigPackages}{\tt general knowledge about configuration file handling in Debian}.

This guide explains how to use ucf\textquotesingle{}s new {\ttfamily -\/-\/three-\/way-\/merge-\/command} functionality in conjunction with Elektra in order to utilize Elektra’s powerful tools in order to allow automatic three-\/way merges of your package\textquotesingle{}s configuration during upgrades in a way that is more reliable than a diff3 merge. This guide assumes that you are familiar with ucf already and are just trying to implement the {\ttfamily -\/-\/three-\/way-\/merge-\/command} option using Elektra.

\subsection*{The New Option}

The addition of the {\ttfamily -\/-\/three-\/way-\/merge-\/command} option was a part of my Google Summer of Code Project. This option takes the form\+: --three-\/way-\/merge-\/command command $<$\+New file$>$=\char`\"{}\char`\"{}$>$ $<$\+Destination$>$

Where {\ttfamily command} is the command you would like to use for the merge. {\ttfamily New File} and {\ttfamily Destination} are the same as always.

\subsection*{elektra-\/merge}

We added a new script to Elektra called elektra-\/merge for use with this new option in ucf. This script acts as a liaison between ucf and Elektra, allowing a regular ucf command to run a {\ttfamily kdb merge} even though ucf commands only pass {\ttfamily New File} and {\ttfamily Destination} whereas kdb merge requires {\ttfamily ourpath}, {\ttfamily theirpath}, {\ttfamily basepath}, and {\ttfamily resultpath}. Since ucf already performs a three-\/way merge, it keeps track of all the necessary files to do so, even though it only takes in {\ttfamily New File} and {\ttfamily Destination}.

In order to use {\ttfamily elektra-\/merge}, the current configuration file must be mounted to K\+DB to serve as {\ttfamily ours} in the merge. The script automatically mounts {\ttfamily theirs}, {\ttfamily base}, and {\ttfamily result} using the {\ttfamily kdb remount} command in order to use the same backend as {\ttfamily ours} (since all versions of the same file should use the same backend anyway) and this way users don\textquotesingle{}t need to worry about specifying the backend for each version of the file. Then the script attempts a merge on the newly mounted Key\+Sets. Once this is finished, either with success or not, the script finishes by unmounting all but {\ttfamily our} copy of the file to cleanup K\+DB. Then, if the merge was successful ucf will replace {\ttfamily ours} with the result providing the package with an automatically merged configuration which will also be updated in K\+DB itself.

Additionally, we added two other scripts, {\ttfamily elektra-\/mount} and {\ttfamily elektra-\/umount} which act as simple wrappers for {\ttfamily kdb mount} and {\ttfamily kdb umount}. They work identically but are more script friendly.

\subsection*{The Full Command}

The full command to use {\ttfamily elektra-\/merge} to perform a three-\/way merge on a file managed by ucf is\+: ucf --three-\/way --threeway-\/merge-\/command elektra-\/merge $<$\+New file$>$=\char`\"{}\char`\"{}$>$ $<$\+Destination$>$

That\textquotesingle{}s it! As described above, {\ttfamily elektra-\/merge} is smart enough to run the whole merge off of the information from that command and utilizes the new {\ttfamily kdb remount} command to do so.

\subsection*{How-\/\+To Integrate}

Integrating {\ttfamily elektra-\/merge} into a package that already uses ucf is very easy! In {\ttfamily postinst} you should have a line similar to\+: ucf $<$\+New file$>$=\char`\"{}\char`\"{}$>$ $<$\+Destination$>$

or perhaps\+: ucf --three-\/way $<$\+New file$>$=\char`\"{}\char`\"{}$>$ $<$\+Destination$>$

All you must do is in {\ttfamily postinst}, when run with the {\ttfamily configure} option you must mount the config file to Elektra\+: kdb elektra-\/mount $<$\+New file$>$=\char`\"{}\char`\"{}$>$ $<$\+Mounting destination$>$=\char`\"{}\char`\"{}$>$ $<$\+Backend$>$

Next, you must update the line containing {\ttfamily ucf} with the options {\ttfamily -\/-\/three-\/way} and {\ttfamily -\/-\/threeway-\/merge-\/command} like so\+: ucf --three-\/way --threeway-\/merge-\/command elektra-\/merge $<$\+New file$>$=\char`\"{}\char`\"{}$>$ $<$\+Destination$>$

Then, in your {\ttfamily postrm} script, during a purge, you must unmount the config file before deleting it\+: kdb elektra-\/umount $<$name$>$

That\textquotesingle{}s it! With those small changes you can use Elektra to perform automatic three-\/way merges on any files that your package uses ucf to handle!

\subsection*{Example}

Below is a diff representing the changes we made to the samba-\/common package in order to allow automatic configuration merging for {\ttfamily smb.\+conf} using Elektra. We chose this package because it already uses ucf to handle {\ttfamily smb.\+conf} but it frequently requires users to manually merge changes across versions. Here is the patch showing what we changed\+:


\begin{DoxyCode}
diff samba\_orig/samba-3.6.6/debian/samba-common.postinst samba/samba-3.6.6/debian/samba-common.postinst
92c92,93
< ucf --three-way --debconf-ok "$NEWFILE" "$CONFIG"
---
> kdb elektra-mount "$CONFIG" system/samba/smb ini
> ucf --three-way --threeway-merge-command elektra-merge --debconf-ok "$NEWFILE" "$CONFIG"
Only in samba/samba-3.6.6/debian/: samba-common.postinst~
diff samba\_orig/samba-3.6.6/debian/samba-common.postrm samba/samba-3.6.6/debian/samba-common.postrm
4a5
>       kdb elektra-umount system/samba/smb
\end{DoxyCode}


As you can see, all we had to do was add the line to mount {\ttfamily smb.\+conf} during install, update the ucf command to include the new {\ttfamily -\/-\/threeway-\/merge-\/command} option, and unmount {\ttfamily system/samba/smb} during a purge. It really is that easy! 